\RequirePackage[l2tabu, orthodox]{nag}
\documentclass[version=last, pagesize, twoside=off, bibliography=totoc, DIV=calc, fontsize=12pt, a4paper, french, english]{scrartcl}
%Permits to copy eg x ⪰ y ⇔ v(x) ≥ v(y) from PDF to unicode data, and to search. From pdfTeX users manual. See https://tex.stackexchange.com/posts/comments/1203887.
	\input glyphtounicode
	\pdfgentounicode=1
%Latin Modern has more glyphs than Computer Modern, such as diacritical characters. fntguide commands to load the font before fontenc, to prevent default loading of cmr.
	\usepackage{lmodern}
%Encode resulting accented characters correctly in resulting PDF, permits copy from PDF.
	\usepackage[T1]{fontenc}
%UTF8 seems to be the default in recent TeX installations, but not all, see https://tex.stackexchange.com/a/370280.
	\usepackage[utf8]{inputenc}
%Provides \newunicodechar for easy definition of supplementary UTF8 characters such as → or ≤ for use in source code.
	\usepackage{newunicodechar}
%Text Companion fonts, much used together with CM-like fonts. Provides \texteuro and commands for text mode characters such as \textminus, \textrightarrow, \textlbrackdbl.
	\usepackage{textcomp}
%Solves bug in lmodern, https://tex.stackexchange.com/a/261188; probably useful only for unusually big font sizes; and probably better to use exscale instead. Note that the authors of exscale write against this trick.
	%\DeclareFontShape{OMX}{cmex}{m}{n}{
		%<-7.5> cmex7
		%<7.5-8.5> cmex8
		%<8.5-9.5> cmex9
		%<9.5-> cmex10
	%}{}
	%\SetSymbolFont{largesymbols}{normal}{OMX}{cmex}{m}{n}
%More symbols (such as \sum) available in bold version, see https://github.com/latex3/latex2e/issues/71.
	\DeclareFontShape{OMX}{cmex}{bx}{n}{%
	   <->sfixed*cmexb10%
	   }{}
	\SetSymbolFont{largesymbols}{bold}{OMX}{cmex}{bx}{n}
%For small caps also in italics, see https://tex.stackexchange.com/questions/32942/italic-shape-needed-in-small-caps-fonts, https://tex.stackexchange.com/questions/284338/italic-small-caps-not-working.
	\usepackage{slantsc}
	\AtBeginDocument{%
		%“Since nearly no font family will contain real italic small caps variants, the best approach is to substitute them by slanted variants.” -- slantsc doc
		%\DeclareFontShape{T1}{lmr}{m}{scit}{<->ssub*lmr/m/scsl}{}%
		%There’s no bold small caps in Latin Modern, we switch to Computer Modern for bold small caps, see https://tex.stackexchange.com/a/22241
		%\DeclareFontShape{T1}{lmr}{bx}{sc}{<->ssub*cmr/bx/sc}{}%
		%\DeclareFontShape{T1}{lmr}{bx}{scit}{<->ssub*cmr/bx/scsl}{}%
	}
%Warn about missing characters.
	\tracinglostchars=2
%Nicer tables: provides \toprule, \midrule, \bottomrule.
	%\usepackage{booktabs}
%For new column type X which stretches; can be used together with booktabs, see https://tex.stackexchange.com/a/97137. “tabularx modifies the widths of the columns, whereas tabular* modifies the widths of the inter-column spaces.” Loads array.
	%\usepackage{tabularx}
%math-mode version of "l" column type. Requires \usepackage{array}.
	%\usepackage{array}
	%\newcolumntype{L}{>{$}l<{$}}
%Provides \xpretocmd and loads etoolbox which provides \apptocmd, \patchcmd, \newtoggle… Also loads xparse, which provides \NewDocumentCommand and similar commands intended as replacement of \newcommand in LaTeX3 for defining commands (see https://tex.stackexchange.com/q/98152 and https://github.com/latex3/latex2e/issues/89).
	\usepackage{xpatch}
%ntheorem doc says: “empheq provides an enhanced vertical placement of the endmarks”; must be loaded before ntheorem. Loads the mathtools package, which loads and fixes some bugs in amsmath and provides \DeclarePairedDelimiter. amsmath is considered a basic, mandatory package nowadays (Grätzer, More Math Into LaTeX).
	\usepackage[ntheorem]{empheq}
%Package frenchb asks to load natbib before babel-french. Package hyperref asks to load natbib before hyperref.
	\usepackage{natbib}

\newtoggle{LCpres}
	\newtoggle{LCart}
	\newtoggle{LCposter}
	\makeatletter
	\@ifclassloaded{beamer}{
		\toggletrue{LCpres}
		\togglefalse{LCart}
		\togglefalse{LCposter}
		\wlog{Presentation mode}
	}{
		\@ifclassloaded{tikzposter}{
			\toggletrue{LCposter}
			\togglefalse{LCpres}
			\togglefalse{LCart}
			\wlog{Poster mode}
		}{
			\toggletrue{LCart}
			\togglefalse{LCpres}
			\togglefalse{LCposter}
			\wlog{Article mode}
		}
	}
	\makeatother%

%Language options ([french, english]) should be on the document level (last is main); except with tikzposter: put [french, english] options next to \usepackage{babel} to avoid warning. beamer uses the \translate command for the appendix: omitting babel results in a warning, see https://github.com/josephwright/beamer/issues/449. Babel also seems required for \refname.
%	\iftoggle{LCpres}{
		\usepackage{babel}
%	}{
%	}
	%\frenchbsetup{AutoSpacePunctuation=false}
%listings (1.7) does not allow multi-byte encodings. listingsutf8 works around this only for characters that can be represented in a known one-byte encoding and only for \lstinputlisting. Other workarounds: use literate mechanism; or escape to LaTeX (but breaks alignment).
	%\usepackage{listings}
	%\lstset{tabsize=2, basicstyle=\ttfamily, escapechar=§, literate={é}{{\'e}}1}
%I favor acro over acronym because the former is more recently updated (2018 VS 2015 at time of writing); has a longer user manual (about 40 pages VS 6 pages if not counting the example and implementation parts); has a command for capitalization; and acronym suffers a nasty bug when ac used in section, see https://tex.stackexchange.com/q/103483 (though this might be the fault of the silence package and might be solved in more recent versions, I do not know) and from a bug when used with cleveref, see https://tex.stackexchange.com/q/71364. However, loading it makes compilation time (one pass on this template) go from 0.6 to 1.4 seconds, see https://bitbucket.org/cgnieder/acro/issues/115. Option short-format not usable in the package options as it is fragile, see https://tex.stackexchange.com/q/466882.
	%\usepackage[single]{acro}
	%\acsetup{short-format = {\scshape}}
	%\DeclareAcronym{AMCD}{short=AMCD, long={Aide Multicritère à la Décision}}
\DeclareAcronym{AHP}{short=AHP, long={Analytic Hierarchy Process}}
\DeclareAcronym{AR}{short=AR, long={Argumentative Recommender}}
\DeclareAcronym{DA}{short=DA, long={Decision Analysis}}
\DeclareAcronym{DJ}{short=DJ, long={Deliberated Judgment}}
\DeclareAcronym{DM}{short=DM, long={Decision Maker}}
\DeclareAcronym{DP}{short=DP, long={Deliberated Preference}}
\DeclareAcronym{MAVT}{short=MAVT, long={Multiple Attribute Value Theory}}
\DeclareAcronym{MCDA}{short=MCDA, long={Multicriteria Decision Aid}}
\DeclareAcronym{MIP}{short=MIP, long={Mixed Integer Program}}
\DeclareAcronym{SEU}{short=SEU, long={Subjective Expected Utility}}


\iftoggle{LCpres}{
	%I favor fmtcount over nth because it is loaded by datetime anyway; and fmtcount warns about possible conflicts when loaded after nth.
	\usepackage{fmtcount}
	%For nice input of date of presentation. Must be loaded after the babel package. Has possible problems with srcletter: https://golatex.de/verwendung-von-babel-und-datetime-in-scrlttr2-schlaegt-fehlt-t14779.html.
	\usepackage[nodayofweek]{datetime}
}{
}
%For presentations, Beamer implicitely uses the pdfusetitle option. ntheorem doc says to load hyperref “before the first use of \newtheorem”. autonum doc mandates option hypertexnames=false. I want to highlight links only if necessary for the reader to recognize it as a link, to reduce distraction. In presentations, this is already taken care of by beamer (https://tex.stackexchange.com/a/262014). If using colorlinks=true in a presentation, see https://tex.stackexchange.com/q/203056. Crashes the first compilation with tikzposter, just compile again and the problem disappears, see https://tex.stackexchange.com/q/254257.
\makeatletter
\iftoggle{LCpres}{
	\usepackage{hyperref}
}{
	\usepackage[hypertexnames=false, pdfusetitle, linkbordercolor={1 1 1}, citebordercolor={1 1 1}, urlbordercolor={1 1 1}]{hyperref}
	%https://tex.stackexchange.com/a/466235
	\pdfstringdefDisableCommands{%
		\let\thanks\@gobble
	}
}
\makeatother
%urlbordercolor is used both for \url and \doi, which I think shouldn’t be colored, and for \href, thus might want to color manually when required. Requires xcolor.
	\NewDocumentCommand{\hrefblue}{mm}{\textcolor{blue}{\href{#1}{#2}}}
%hyperref doc says: “Package bookmark replaces hyperref’s bookmark organization by a new algorithm (...) Therefore I recommend using this package”.
	\usepackage{bookmark}
%Need to invoke hyperref explicitly to link to line numbers: \hyperlink{lintarget:mylinelabel}{\ref*{lin:mylinelabel}}, with \ref* to disable automatic link. Also see https://tex.stackexchange.com/q/428656 for referencing lines from another document.
	%\usepackage{lineno}
	%\NewDocumentCommand{\llabel}{m}{\hypertarget{lintarget:#1}{}\linelabel{lin:#1}}
	%\setlength\linenumbersep{9mm}
%For complex authors blocks. Seems like authblk wants to be later than hyperref, but sooner than silence. See https://tex.stackexchange.com/q/475513 for the patch to hyperref pdfauthor.
	\ExplSyntaxOn
	\seq_new:N \g_oc_hrauthor_seq
	\NewDocumentCommand{\addhrauthor}{m}{
		\seq_gput_right:Nn \g_oc_hrauthor_seq { #1 }
	}
	%Should be \NewExpandableDocumentCommand, but this is not yet provided by my version of xparse
	\DeclareExpandableDocumentCommand{\hrauthor}{}{
		\seq_use:Nn \g_oc_hrauthor_seq {,~}
	}
	\ExplSyntaxOff
	{
		\catcode`#=11\relax
		\gdef\fixauthor{\xpretocmd{\author}{\addhrauthor{#2}}{}{}}%
	}
	\iftoggle{LCart}{
		\usepackage{authblk}
		\renewcommand\Affilfont{\small}
		\fixauthor
		\AtBeginDocument{
		    \hypersetup{pdfauthor={\hrauthor}}
		}
	}{
	}
%I do not use floatrow, because it requires an ugly hack for proper functioning with KOMA script (see scrhack doc). Instead, the following command centers all floats (using \centering, as the center environment adds space, http://texblog.net/latex-archive/layout/center-centering/), and I manually place my table captions above and figure captions below their contents (https://tex.stackexchange.com/a/3253).
	\makeatletter
	\g@addto@macro\@floatboxreset\centering
	\makeatother
%Permits to customize enumeration display and references
	%\nottoggle{LCpres}{
		%\usepackage{enumitem} %follow list environments by a string to customize enumeration, example: \begin{description}[itemindent=8em, labelwidth=!] or \begin{enumerate}[label=({\roman*}), ref={\roman*}].
	%}{
	%}
%Provides \Cen­ter­ing, \RaggedLeft, and \RaggedRight and en­vi­ron­ments Cen­ter, FlushLeft, and FlushRight, which al­low hy­phen­ation. With tikzposter, seems to cause 1=1 to be printed in the middle of the poster.
	%\usepackage{ragged2e}
%To typeset units by closely following the “official” rules.
	%\usepackage[strict]{siunitx}
%Turns the doi provided by some bibliography styles into URLs. However, uses old-style dx.doi url (see 3.8 DOI system Proxy Server technical details, “Users may resolve DOI names that are structured to use the DOI system Proxy Server (https://doi.org (current, preferred) or earlier syntax http://dx.doi.org).”, https://www.doi.org/doi_handbook/3_Resolution.html). The patch solves this.
	\usepackage{doi}
	\makeatletter
	\patchcmd{\@doi}{http://dx.doi.org}{https://doi.org}{}{}
	\makeatother
%Makes sure upper case greek letters are italic as well.
	\usepackage{fixmath}
%Provides \mathbb; obsoletes latexsym (see http://tug.ctan.org/macros/latex/base/latexsym.dtx). Relatedly, \usepackage{eucal} to change the mathcal font and \usepackage[mathscr]{eucal} (apparently equivalent to \usepackage[mathscr]{euscript}) to supplement \mathcal with \mathscr. This last option is not very useful as both fonts are similar, and the intent of the authors of eucal was to provide a replacement to mathcal (see doc euscript). Also provides \mathfrak for supplementary letters.
	\usepackage{amsfonts}
%Provides a beautiful (IMHO) \mathscr and really different than \mathcal, for supplementary uppercase letters. But there is no bold version. Alternative: mathrsfs (more slanted), but when used with tikzposter, it warns about size substitution, see https://tex.stackexchange.com/q/495167.
	\usepackage[scr]{rsfso}
%Multiple means to produce bold math: \mathbf, \boldmath (defined to be \mathversion{bold}, see fntguide), \pmb, \boldsymbol (all legacy, from LaTeX base and AMS), \bm (the most recommended one), \mathbold from package fixmath (I don’t see its advantage over \boldsymbol).
%“The \boldsymbol command is obtained preferably by using the bm package, which provides a newer, more powerful version than the one provided by the amsmath package. Generally speaking, it is ill-advised to apply \boldsymbol to more than one symbol at a time.” — AMS Short math guide. “If no bold font appears to be available for a particular symbol, \bm will use ‘poor man’s bold’” — bm. It is “best to load the package after any packages that define new symbol fonts” – bm. bm defines \boldsymbol as synonym to \bm. \boldmath accesses the correct font if it exists; it is used by \bm when appropriate. See https://tex.stackexchange.com/a/10643 and https://github.com/latex3/latex2e/issues/71 for some difficulties with \bm.
	\usepackage{bm}
	\nottoggle{LCpres}{
	%https://ctan.org/pkg/amsmath recommends ntheorem, which supersedes amsthm, which corrects the spacing of proclamations and allows for theoremstyle. Option standard loads amssymb and latexsym. Must be loaded after amsmath (from ntheorem doc). From cleveref doc, “ntheorem is fully supported and even recommended”; says to load cleveref after ntheorem. When used with tikzposter, warns about size substitution for the lasy (latexsym) font when using \url, because ntheorem loads latexsym; relatedly (but not directly related to ntheorem), size substitution warning with the cmex font happens when loading amsmath and using \url.
		\usepackage[thmmarks, amsmath, standard, hyperref]{ntheorem}
		%empheq doc says to do this after loading ntheorem
		\usetagform{default}
	%Provides \cref. Unfortunately, cref fails when the language is French and referring to a label whose name contains a colon (https://tex.stackexchange.com/q/83798). Use \cref{sec\string:intro} to work around this. cleveref should go “laster” than hyperref.
		\usepackage{cleveref}
	}{
	}
	\nottoggle{LCposter}{
	%Equations get numbers iff they are referenced. Loading order should be “amsmath → hyperref → cleveref → autonum”, according to autonum doc. Use this in preference to the showonlyrefs option from mathtools, see https://tex.stackexchange.com/q/459918 and autonum doc. See https://tex.stackexchange.com/a/285953 for the etex line. Incompatible with my version of tikzposter (produces “! Improper \prevdepth”).
		\expandafter\def\csname ver@etex.sty\endcsname{3000/12/31}\let\globcount\newcount
		\usepackage{autonum}
	}{
	}
%Also loaded by tikz.
	\usepackage{xcolor}
\iftoggle{LCpres}{
	\usepackage{tikz}
	%\usetikzlibrary{babel, matrix, fit, plotmarks, calc, trees, shapes.geometric, positioning, plothandlers, arrows, shapes.multipart}
}{
}
%Vizualization, on top of TikZ
	%\usepackage{pgfplots}
	%\pgfplotsset{compat=1.14}
\usepackage{graphicx}
	\graphicspath{{graphics/}}

%Provides \print­length{length}, useful for debugging.
	%\usepackage{printlen}
	%\uselengthunit{mm}

\iftoggle{LCpres}{
	\usepackage{appendixnumberbeamer}
	%I have yet to see anyone actually use these navigation symbols; let’s disable them
	\setbeamertemplate{navigation symbols}{} 
	\usepackage{preamble/beamerthemeParisFrance}
	\setcounter{tocdepth}{10}
}{
}

%Do not use the displaymath environment: use equation. Do not use the eqnarray or eqnarray* environments: use align(*). This improves spacing. (See l2tabu or amsldoc.)


\newcommand{\R}{ℝ}
\newcommand{\N}{ℕ}
\newcommand{\Z}{ℤ}
\newcommand{\card}[1]{\lvert{#1}\rvert}
\newcommand{\powerset}[1]{\mathscr{P}(#1)}%\mathscr rather than \mathcal: scr is rounder than cal (at least in XITS Math).
\newcommand{\suchthat}{\;\ifnum\currentgrouptype=16 \middle\fi|\;}
%\newcommand{\Rplus}{\reels^+\xspace}

\AtBeginDocument{%
	\renewcommand{\epsilon}{\varepsilon}
% we want straight form of \phi for mathematics, as recommended in UTR #25: Unicode support for mathematics.
%	\renewcommand{\phi}{\varphi}
}

% with amssymb, but I don’t want to use amssymb just for that.
% \newcommand{\restr}[2]{{#1}_{\restriction #2}}
%\newcommand{\restr}[2]{{#1\upharpoonright}_{#2}}
\newcommand{\restr}[2]{{#1|}_{#2}}%sometimes typed out incorrectly within \set.
%\newcommand{\restr}[2]{{#1}_{\vert #2}}%\vert errors when used within \Set and is typed out incorrectly within \set.
\DeclareMathOperator*{\argmax}{arg\,max}
\DeclareMathOperator*{\argmin}{arg\,min}


\NewDocumentCommand{\range}{}{R}

%Decision Theory (MCDA and SC)
\NewDocumentCommand{\allalts}{}{\mathscr{X}}
\NewDocumentCommand{\allcrits}{}{\mathscr{C}}
\NewDocumentCommand{\alts}{}{X}
\NewDocumentCommand{\alt}{}{x}
\NewDocumentCommand{\altp}{}{y}%alt prime, another alt
\NewDocumentCommand{\dm}{}{i}
\NewDocumentCommand{\allF}{}{\mathscr{F}}
\NewDocumentCommand{\allvoters}{}{\mathscr{N}}
\NewDocumentCommand{\voters}{}{N}
\NewDocumentCommand{\allprofs}{}{\boldsymbol{\mathcal{R}}}
\NewDocumentCommand{\prof}{}{\boldsymbol{R}}
\NewDocumentCommand{\linors}{}{\mathscr{L}(\allalts)}
%Thanks to https://tex.stackexchange.com/q/154549
	%\makeatletter
	%\def\@myRgood@#1#2{\mathrel{R^X_{#2}}}
	%\def\myRgood{\@ifnextchar_{\@myRgood@}{\mathrel{R^X}}}
	%\makeatother
\NewDocumentCommand{\ind}{}{\sim}
\NewDocumentCommand{\peq}{}{\succeq}
\NewDocumentCommand{\pst}{}{\succ}
\NewDocumentCommand{\npeq}{}{\nsucceq}
\NewDocumentCommand{\npst}{}{\nsucc}

%Deliberated Judgment
%%Normative theory
\NewDocumentCommand{\allargs}{}{\mathscr{A}}
\NewDocumentCommand{\args}{}{A}
\NewDocumentCommand{\ard}{O{}}{a^\mathit{d}_{#1}}
\NewDocumentCommand{\ardp}{O{}}{a^{\mathit{d}\prime}_{#1}}
\NewDocumentCommand{\ar}{o}{%
	\IfValueTF{#1}{%
		a^{(#1)}%
	}{%
		a%
	}%
}
\NewDocumentCommand{\zar}{}{\mathbf{0}}%zero, or empty, argument
\NewDocumentCommand{\allhist}{}{\mathscr{A}^*}
\NewDocumentCommand{\hist}{}{α}
\NewDocumentCommand{\histp}{}{α^{\prime}}
\NewDocumentCommand{\histpp}{}{α^{\prime\prime}}
\NewDocumentCommand{\histend}{o}{%
	\IfValueTF{#1}{%
		α^{#1}_\mathit{end}%
	}{%
		α_\mathit{end}%
	}%
}
\NewDocumentCommand{\histpend}{}{α^{\prime}_\mathit{end}}
\NewDocumentCommand{\histppend}{}{α^{\prime\prime}_\mathit{end}}
\NewDocumentCommand{\allprops}{}{\Phi}
\NewDocumentCommand{\prop}{}{φ}
\NewDocumentCommand{\propbar}{}{φ'}%\overline
\NewDocumentCommand{\incompat}{}{\Phi^\mathit{incompat}}
%%Empirical theory
\NewDocumentCommand{\gC}{}{C_γ}
\NewDocumentCommand{\gPhi}{}{\Phi_γ}
\NewDocumentCommand{\gpropse}{O{γ}}{{\hookrightarrow_{#1}}(\allargs)}%e for explicit
\NewDocumentCommand{\gprops}{O{γ}}{\Phi_{#1}}
\NewDocumentCommand{\dargs}{O{}}{A^\mathit{d}_{#1}}
\NewDocumentCommand{\alldargs}{}{\mathscr{A}^d}
\NewDocumentCommand{\gargs}{O{φ}}{A^{#1}_{γ, i}}
\NewDocumentCommand{\gargsmu}{}{A^{φ}_{μ, i}}
\NewDocumentCommand{\gargsnu}{}{A^{φ'}_{ν, i}}
\NewDocumentCommand{\gargsgamma}{}{A^{φ}_{γ, i}}
\NewDocumentCommand{\gargsdelta}{}{A^{φ'}_{δ, i}}
\NewDocumentCommand{\gleadsto}{O{γ}}{\hookrightarrow_{#1}}
\NewDocumentCommand{\gleadstoinv}{O{γ}}{{\hookrightarrow^{-1}_{#1}}}
\NewDocumentCommand{\gbeats}{O{γ}}{⊳^\mathit{t}_{#1}}
\NewDocumentCommand{\gbeatsinv}{O{γ}}{{(⊳^\mathit{t}_{#1})^{-1}}}
\NewDocumentCommand{\ngbeats}{O{γ}}{\not⊳^\mathit{t}_{#1}}
\NewDocumentCommand{\dbeats}{O{γ}}{⊳^\mathit{d}_{#1}}
\NewDocumentCommand{\dbeatsinv}{O{γ}}{{(⊳^\mathit{d}_{#1})^{-1}}}
\NewDocumentCommand{\df}{O{γ}}{\mathit{def}_{#1}}
\NewDocumentCommand{\dfp}{O{γ}}{\mathit{def}_{#1}^+}
\NewDocumentCommand{\dg}{O{γ}}{d_{#1}}
\NewDocumentCommand{\dgip}{O{γ, i}}{d^\phi_{#1}}
%%%DP
\NewDocumentCommand{\choices}{}{\mathscr{C}}
\NewDocumentCommand{\gind}{O{}}{\sim_\gamma^{#1}}
\NewDocumentCommand{\gpeq}{}{\succeq_\gamma}
\NewDocumentCommand{\gpst}{}{\succ_\gamma}
\NewDocumentCommand{\ngpeq}{}{\nsucceq_\gamma}
\NewDocumentCommand{\ngpst}{}{\nsucc_\gamma}

%%i
\NewDocumentCommand{\iprops}{}{\Phi_i}
\NewDocumentCommand{\allleadsto}{}{⇝}%Or \dashrightarrow?
\NewDocumentCommand{\ileadsto}{O{i}}{⇝_{#1}}
\NewDocumentCommand{\nileadsto}{O{i}}{\not⇝_{#1}}
\NewDocumentCommand{\ileadstoe}{O{i}}{⇝_{#1}^\exists}
\NewDocumentCommand{\nileadstoe}{O{i}}{\not⇝_{#1}^\exists}
\NewDocumentCommand{\ileadstost}{}{\hookrightarrow_i}
\NewDocumentCommand{\nileadstost}{}{\not\hookrightarrow_i}
\NewDocumentCommand{\di}{}{c^φ_{γ, i}}
\NewDocumentCommand{\dip}{}{d^{φ +}_{γ, i}}
\NewDocumentCommand{\ibeats}{}{⊳^\text{\sout{\ensuremath{φ}}}_{γ, i}}%Or: \usepackage[normalem]{ulem} \text{\sout{\ensuremath t}}
\NewDocumentCommand{\nibeats}{}{⋫^\text{\sout{\ensuremath{φ}}}_{γ, i}}
%%%Deliberated Preference
\NewDocumentCommand{\ipeq}{}{\succeq_i}
\NewDocumentCommand{\ipst}{}{\succ_i}


\definecolor{darkgreen}{rgb}{0,0.6,0}
\newcommand{\commentOC}[1]{{\small\color{blue}{\selectlanguage{french}$\big[$OC: #1$\big]$}}}
%\newcommand{\commentOC}[1]{{\selectlanguage{french}{\todo{OC : #1}}}}
%Or: \todo[color=green!40]
\newcommand{\innote}[1]{{\scriptsize{#1}}}

%this probably requires outdated float package, see doc KomaScript for an alternative.
% \newfloat{program}{t}{lop}
% \floatname{program}{PM}

%definition, theorem, lemma, example environments, qed trickery are only needed in article mode (not Beamer)
\nottoggle{LCpres}{
%style is plain by default (italic text)
	\newtheorem{definition}{Definition}
	\newtheorem{theorem}{Theorem}
%no italic: expected.
%http://tex.stackexchange.com/questions/144653/italicizing-of-amsthm-package
	\newtheorem{lemma}{Lemma}
%\crefname{axiom}{axiom}{axioms}%might be needed for workaround bug in cref when defining new theorems?

%\ifdefined\theorem\else
%\newtheorem{theorem}{\iflanguage{english}{Theorem}{Théorème}}
%\fi

\theoremstyle{remark}
	\newtheorem{examplex}{Example}
	\newtheorem{remarkx}{Remark}

%trickery allowing use of \qedhere and the like.
\newenvironment{example}{
	\pushQED{\qed}\renewcommand{\qedsymbol}{$\triangle$}\examplex
}{
	\popQED\endexamplex
}
\newenvironment{remark}{
	\pushQED{\qed}\renewcommand{\qedsymbol}{$\triangle$}\remarkx
}{
	\popQED\endremarkx
}
}{
}
\crefname{examplex}{example}{examples}% I wonder why this is unnecessary in case of singular

%which line breaks are chosen: accept worse lines, therefore reducing risk of overfull lines. Default = 200
\tolerance=2000
%accept overfull hbox up to...
\hfuzz=2cm
%reduces verbosity about the bad line breaks
\hbadness 5000
%reduces verbosity about the underful vboxes
\vbadness=1300
%sloppy sets tolerance to 9999
\apptocmd{\sloppy}{\hbadness 10000\relax}{}{}

\bibliographystyle{abbrvnat}
%or \bibliographystyle{apalike} for presentations?

%doi package uses old-style dx.doi url, see 3.8 DOI system Proxy Server technical details, “Users may resolve DOI names that are structured to use the DOI system Proxy Server (http://doi.org (preferred) or http://dx.doi.org).”, https://www.doi.org/doi_handbook/3_Resolution.html
\makeatletter
\patchcmd{\@doi}{dx.doi.org}{doi.org}{}{}
\makeatother

% WRITING
%\newcommand{\ie}{i.e.\@\xspace}%to try
%\newcommand{\eg}{e.g.\@\xspace}
%\newcommand{\etal}{et al.\@\xspace}
\newcommand{\ie}{i.e.\ }
\newcommand{\eg}{e.g.\ }
\newcommand{\mkkOK}{\checkmark}%\color{green}{\checkmark}
\newcommand{\mkkREQ}{\ding{53}}%requires pifont?%\color{green}{\checkmark}
\newcommand{\mkkNO}{}%\text{\color{red}{\textsf{X}}}

\newlength{\xdescwd}
\makeatletter
\NewEnviron{xdesc}{%
  \setbox0=\vbox{\hbadness=\@M \global\xdescwd=0pt
    \def\item[##1]{%
      \settowidth\@tempdima{\textbf{##1}:}%
      \ifdim\@tempdima>\xdescwd \global\xdescwd=\@tempdima\fi}
  \BODY}
  \begin{description}[leftmargin=\dimexpr\xdescwd+.5em\relax,
    labelindent=0pt,labelsep=.5em,
    labelwidth=\xdescwd,align=left]\BODY\end{description}}
\makeatother

\makeatletter
\newcommand{\boldor}[2]{%
	\ifnum\strcmp{\f@series}{bx}=\z@
		#1%
	\else
		#2%
	\fi
}
\newcommand{\textstyleElProm}[1]{\boldor{\MakeUppercase{#1}}{\textsc{#1}}}
\makeatother
\newcommand{\electre}{\textstyleElProm{Électre}\xspace}
\newcommand{\electreIv}{\textstyleElProm{Électre Iv}\xspace}
\newcommand{\electreIV}{\textstyleElProm{Électre IV}\xspace}
\newcommand{\electreIII}{\textstyleElProm{Électre III}\xspace}
\newcommand{\electreTRI}{\textstyleElProm{Électre Tri}\xspace}
% \newcommand{\utadis}{\texorpdfstring{\textstyleElProm{utadis}\xspace}{UTADIS}}
% \newcommand{\utadisI}{\texorpdfstring{\textstyleElProm{utadis i}\xspace}{UTADIS I}}

%TODO
% \newcommand{\textstyleElProm}[1]{{\rmfamily\textsc{#1}}} 


\usepackage[normalem]{ulem}
%\NewDocumentCommand{\tikzmark}{m}{%
	\tikz[overlay, remember picture, baseline=(#1.base)] \node (#1) {};%
}

\newlength{\GraphsDNodeSep}
\setlength{\GraphsDNodeSep}{7mm}
\tikzset{/GraphsD/dot/.style={
	shape=circle, fill=black, inner sep=0, minimum size=1mm
}}

% MCDA Drawing Sorting
\newlength{\MCDSCatHeight}
\setlength{\MCDSCatHeight}{6mm}
\newlength{\MCDSAltHeight}
\setlength{\MCDSAltHeight}{4mm}
%separation between two vertical alts
\newlength{\MCDSAltSep}
\setlength{\MCDSAltSep}{2mm}
\newlength{\MCDSCatWidth}
\setlength{\MCDSCatWidth}{3cm}
\newlength{\MCDSAltWidth}
\setlength{\MCDSAltWidth}{2.5cm}
\newlength{\MCDSEvalRowHeight}
\setlength{\MCDSEvalRowHeight}{6mm}
\newlength{\MCDSAltsToCatsSep}
\setlength{\MCDSAltsToCatsSep}{1.5cm}
\newcounter{MCDSNbAlts}
\newcounter{MCDSNbCats}
\newlength{\MCDSArrowDownOffset}
\setlength{\MCDSArrowDownOffset}{0mm}
\tikzset{/MCD/S/alt/.style={
	shape=rectangle, draw=black, inner sep=0, minimum height=\MCDSAltHeight, minimum width=\MCDSAltWidth
}}
\tikzset{/MCD/S/pref/.style={
	shape=ellipse, draw=gray, thick
}}
\tikzset{/MCD/S/cat/.style={
	shape=rectangle, draw=black, inner sep=0, minimum height=\MCDSCatHeight, minimum width=\MCDSCatWidth
}}
\tikzset{/MCD/S/evals matrix/.style={
	matrix, row sep=-\pgflinewidth, column sep=-\pgflinewidth, nodes={shape=rectangle, draw=black, inner sep=0mm, text depth=0.5ex, text height=1em, minimum height=\MCDSEvalRowHeight, minimum width=12mm}, nodes in empty cells, matrix of nodes, inner sep=0mm, outer sep=0mm, row 1/.style={nodes={draw=none, minimum height=0em, text height=, inner ysep=1mm}}
}}

%Git
\newlength{\GitDCommitSep}
\setlength{\GitDCommitSep}{13mm}
\tikzset{/GitD/commit/.style={
	shape=rectangle, draw, minimum width=4em, minimum height=0.6cm
}}
\tikzset{/GitD/branch/.style={
	shape=ellipse, draw, red
}}
\tikzset{/GitD/head/.style={
	shape=ellipse, draw, fill=yellow
}}

%Social Choice
\tikzset{/SCD/profile matrix/.style={
	matrix of math nodes, column sep=3mm, row sep=2mm, nodes={inner sep=0.5mm, anchor=base}
}}
\tikzset{/SCD/rank-profile matrix/.style={
	matrix of math nodes, column sep=3mm, row sep=2mm, nodes={anchor=base}, column 1/.style={nodes={inner sep=0.5mm}}, row 1/.style={nodes={inner sep=0.5mm}}
}}
\tikzset{/SCD/rank-vector/.style={
	draw, rectangle, inner sep=0, outer sep=1mm
}}
\tikzset{/SCD/isolated rank-vector/.style={
	draw, matrix of math nodes, column sep=3mm, inner sep=0, matrix anchor=base, nodes={anchor=base, inner sep=.33em}, ampersand replacement=\&
}}

% GUI
\tikzset{/GUID/button/.style={
	rectangle, very thick, rounded corners, draw=black, fill=black!40%, top color=black!70, bottom color=white
}}

% Logger objects
\tikzset{/loggerD/main/.style={
	shape=rectangle, draw=black, inner sep=1ex, minimum height=7mm
}}
\tikzset{/loggerD/helper/.style={
	shape=rectangle, draw=black, dashed, minimum height=7mm
}}
\tikzset{/loggerD/helper line/.style={
	<->, draw, dotted
}}

% Beliefs
\tikzset{/BeliefsD/attacker/.style={
	shape=rectangle, draw, minimum size=8mm
}}
\tikzset{/BeliefsD/supporter/.style={
	shape=circle, draw
}}


%\DeclareAcronym{AMCD}{short=AMCD, long={Aide Multicritère à la Décision}}
\DeclareAcronym{AHP}{short=AHP, long={Analytic Hierarchy Process}}
\DeclareAcronym{AR}{short=AR, long={Argumentative Recommender}}
\DeclareAcronym{DA}{short=DA, long={Decision Analysis}}
\DeclareAcronym{DJ}{short=DJ, long={Deliberated Judgment}}
\DeclareAcronym{DM}{short=DM, long={Decision Maker}}
\DeclareAcronym{DP}{short=DP, long={Deliberated Preference}}
\DeclareAcronym{MAVT}{short=MAVT, long={Multiple Attribute Value Theory}}
\DeclareAcronym{MCDA}{short=MCDA, long={Multicriteria Decision Aid}}
\DeclareAcronym{MIP}{short=MIP, long={Mixed Integer Program}}
\DeclareAcronym{SEU}{short=SEU, long={Subjective Expected Utility}}


\addtokomafont{labelinglabel}{\sffamily\bfseries}
\renewcommand{\phi}{\varphi}%TODO
\DeclareMathAlphabet{\mathup}{OT1}{\familydefault}{m}{n}

%I find these settings useful in draft mode. Should be removed for final versions.
	%Which line breaks are chosen: accept worse lines, therefore reducing risk of overfull lines. Default = 200.
		\tolerance=2000
	%Accept overfull hbox up to...
		\hfuzz=2cm
	%Reduces verbosity about the bad line breaks.
		\hbadness 5000
	%Reduces verbosity about the underful vboxes.
		\vbadness=1300

\begin{document}
\title{Toward theories of deliberated preferences}
\author{Olivier Cailloux}
\affil{Université Paris-Dauphine, PSL Research University, CNRS, LAMSADE, 75016 PARIS, FRANCE\\
	\href{mailto:olivier.cailloux@dauphine.fr}{olivier.cailloux@dauphine.fr}
}
\makeatletter
	\hypersetup{
		pdfsubject={Epistemology},
		pdfkeywords={Decision aiding, Decision making, Argumentation}
	}
\makeatother
\maketitle

\begin{abstract}
	TODO: abstract
\end{abstract}

\section{Motivation} 
\TODO{“Before asking whether agents want to revise their choices or not, both rival theories should be proposed.” Guala 2000. My approach can be qualified as a debiasing technique (Mitchell, Libertarian paternalism is an oxymoron. Nw UL Rev 99:1245–1817, 2005). Mitchell observes that “in many of those surveys that governments use in order to inform social policy there are not enough resources (time and money) in order to reach the stage where preferences are not labile” (exact quote?).} 

It has escaped nobody’s attention that 
studying reflective preferences, in supplement to shallow preferences, is an important endeavor for economics: we want to recommend to individual decision makers; we want to adopt optimization criteria for society, and this ought not to merely faithfully reproduce unreflectful behavior. This is one important difference between the normative and the (inappropriately called, as will be seen) “descriptive” approach in economics. As \citet{hausman_preference_1994} illustrate with an example of a person drinking a fatal poison while erroneously thinking that it is water, shallow preferences are not sufficient to inform normative economics.
Yet, classical methods to obtain preference models in economics are either formally defined on the basis of observable data that do not describe reflective preferences, or are obtained by other means than observable preferential data and systematic reproducible formal procedures. 

This article is interested in defining theories of decision making based on another kind of observable data, namely, the reactions of decision makers to arguments. This permits to overcome a current important limitation of normative approaches, that of being not empirical (in a sense that will be developed below).

By reflective preferences (of an individual), I mean here preferences that represent informed, pondered decisions, that would be taken after due consideration by the individual. Reflective preferences can be contrasted to shallow preferences, those that \citet[p.\ 16]{von_neumann_theory_2004} describe as “intuitive”, as originating from an “immediate sensation” (cited in \citet{fishburn_retrospective_1989}).
This distinction draws on our capacity, as human beings, to think about our shallow preferences and consider some of them as mistaken \citep{frankfurt_freedom_1971}. A famous example of such a situation is the one of \citet[pp.\ 101--103]{savage_foundations_1972} having realized, after due consideration, that his initial declared preference in a decision problem presented by \citet{allais_so-called_1979} was mistaken.

It is well known that shallow preferences do not always match reflective ones.
Indeed, forming reflective preferences is recognised as a difficult problem in many decision problems, contrary, by definition, to shallow preferences. For example, \citet[p.\ 6]{colell_microeconomic_1995} write, when presenting the preference relation that is the basis for most of their famous book about Microeconomic theory: “Introspection quickly reveals how hard it is to evaluate alternatives that are far from the realm of common experience. It takes work and serious reflection to find out one’s own preferences.” 
It can be expected that some behavior highlighted by psychologists \citet{lichtenstein_construction_2006}, such as framing effects, would be rejected by the individuals themselves when pondering their decisions. 

That reflective preferences are relevant to economic theory, in supplement to shallow preferences, is also illustrated by the general requirement of transitivity of preferences for economic theory, which is reasonably expected from reflective rather than from shallow preferences. As an illustration, \citet[p.\ 7]{colell_microeconomic_1995} go on writing: “the transitivity assumption can be hard to satisfy when evaluating alternatives far from common experience. As compared to the completeness property, however, it is also more fundamental in the sense that substantial portions of economic theory would not survive if economic agents could not be assumed to have transitive preferences.” 

Let me state how the approach proposed here differs from those proposed in the literature, and what problems it aims at solving.

\subsection{Existing approaches for capturing reflective preferences}
\subsubsection{Laundering preferences}
An approach much discussed recently is the one of laundering preferences, also called preference purification \citep{bernheim_beyond_2009, rubinstein_eliciting_2011, infante_preference_2016}: having collected shallow preference data (from the field or from interviews), laundering preferences consists in transforming them, using a formal process, in such a way that they satisfy axioms generally considered as required by rationality, or keeping only preference data collected in some situations considered appropriate to determine welfare-relevant preferences. For example, \citet{bleichrodt_making_2001} transform observed preferences in order to obtain a classical Expected Utility model; \citet{bernheim_beyond_2009} propose to exclude situations where the individual seems (by objective criteria) unable to process information considered relevant. 

This creates a risk of mis-representing the reflective preferences of the individual: not only is it not known whether the individual would reflectively want to adhere to these axioms and, if so, how she would reflectively adapt her preferences when her shallow preferences exhibit related inconsistencies; but it is not even discussed how in principle this could be known. As a result, a theory postulating constraints on individual’s behavior on the basis of rational requirements is subject to the criticism addressed by \citet[p.\ 228]{dold_toward_2018} to libertarian paternalism, in that it “effectively denies the possibility that deviations from rational choice might be something to be acknowledged as reasonable”. 
That this is an important methodological problem is sometimes recognized by authors adopting this approach. \Citet{bleichrodt_making_2001}, for example, state that laundering require controversial assumptions (p. 1500), and recommend it only as a last resort, recommending discussing with the decision-maker whenever possible to elicit her reflective preferences (p. 1499). Note that nudging \citep{thaler_nudge_2009} creates a similar problem: we have to guess what the individual would like if he thought hard about the decision problem, was well-informed, and actively worked against the psychological features he himself does not consider adequate for good decision making. \TODO{Comment about akrasia: we may want to correct for features that the individual herself can’t correct.}  

Some axioms arguably seem innocuous. For example, Pareto-dominance requires that $a$ be weakly preferred to $b$ if $a$ is at least as good as $b$ on every point of view that matter. However, as has been amply remarked, what is required is not merely the correctness of the axiom in the abstract, but its correct application to the decision situation. Has a point of view that the decision maker considers relevant been forgotten, leading the analyst to mistakenly think that $a$ Pareto-dominates $b$, and the seemingly impeccable conclusion that $a$ ought to be preferred to $b$ turns incorrect. \citet{sen_maximisation_1997} illustrated this reasoning (with another axiom) with his well-known example of a person picking her second most-preferred fruit from a basket out of politeness reasons, which could be considered irrational by an analyst blindly applying WARP. As \citet[p.\ 40]{lecouteux_reconciling_2015} puts it (citing \citet[p.\ 13]{bacharach_beyond_2006} for the part in italics), and as analyzed by \citet{sen_information_1986, sen_internal_1993}, “whether a decision-theoretic principle of rationality has been violated \emph{depends on how we, the theorists, ‘cut up the world’}” \TODO{Find that book}.

\subsubsection{Informal deliberation}
Another path to obtain reflectful preferences has been followed: the one of deliberation, an umbrella term by which I designate processes, either, involving discussions between an analyst and a decision-maker while querying her preferences in view of building a formal model of it \citep{raiffa_back_1985, keeney_decisions_1993, roy_multicriteria_1996, belton_multiple_2002}, or involving group discussions, with presentations from experts to an audience and internal deliberation phases \citep{fishkin_when_2011}, aiming to obtain well-informed preferences. 

A major problem of these techniques is that they are not easily reproducible, and that the resulting preferences elicited from the subjects might reflect properties of the personalities of the intervening experts or group-dynamic effects rather than true reflective preferences of the individuals \commentOCf{Yves, as-tu une référence pour moi ?}. This problem is exacerbated by the fact that psychologists have shown that apparent coherence can appear, that is, preferences that do not violate axioms usually considered as representing rationality, although some framing effect may be at play that result in the apparently coherent preferences to arguably not represent reflective preferences \citep{ariely_predictably_2010}. It also remains to be clarified how to check on a principled, systematic basis, whether all relevant arguments have been presented faithfully. Comfronted with (possibly self-designated) experts with enough assertiveness, a decision-maker can be convinced to follow principles of decision-making that several experts in the field of decision-theory consider incorrect. As an illustration, after discussing the weaknesses of the \ac{AHP}, \citet[p.\ 53]{howard_foundations_2007} write: “Why, then, do inferior processes find favor with decision makers? The answer is that they do not force you to think very hard or to think in new ways.”
Beyond the controversial \ac{AHP}, famous decision analysts have voiced disagreements with normative theories of decision making \citep{ellsberg_risk_1961, allais_so-called_1979} that are however generally applied in the field. 
\Citet[p.\ 669]{ellsberg_risk_1961} concluded his influential article as follows: “[In some situations, for some people,] the Bayesian or Savage approach gives [...] bad advice. [Some people] act in conflict with the axioms deliberately, without apology, because it seems to them the sensible way to behave. Are they clearly mistaken?” 
This rethorical question illustrates the need of systematic procedures able to determine whether decision-makers, when confronted with the relevant arguments, accept the soundness of the principles on whose basis the recommendation rests (this has been somewhat explored, but not systematically, for current lack of a systematic way of confronting decision makers to arguments \citep{slovic_who_1974, stanovich_discrepancies_1999}). 

Different analysts, using different methods and different axioms as their favorite conception of rational decision-making, will obtain possibly contradictory conclusions, “elicited” from the decision-makers who hire them. Some analysts admit this possibility and do not consider it to be a problem.
For example, \citet[p.\ 214]{roy_comparison_1995} write: “In our view, these inevitable disagreements do not imply that decision-aid is useless but simply that a single problem may have several valid responses. Given that two different decision-aid models cannot be implemented in the same decision process, the decision-maker must be conscious of the qualitative choices implied by the different models – often conveying the analysts’ own ethical choices – before coming to personal conclusions on the choice to be made. In this domain, the many different approaches reflect in our view the complexity of the researcher's task much more than a scientific weakness.”
%Some analysts claim that the value of such interactions between decision-makers and analysts is the fact that it made decision-makers think hard about the problem, and not the resulting preference model. 
Although being an important insight about the (likely vain) hope of determining a single correct recommendation in many complex decision situations (about which more below), this answer does not address the problem that some principles of decision making may be wrong guides in some situations, possibly including some principles that are commonly applied in the field, hence the need of principled ways of discriminating them.

The prescriptive approach to decision making \citep{bell_decision_1988} has been proposed (in supplement to the normative one and the descriptive one), which insists on incorporating the findings of psychologists about decision maker’s weaknesses of reasoning in the procedures leading to recommendations. I do not discuss this approach specifically here, because its methods can be considered to fall into the two categories I have described here above when it aims at obtaining reflective preference models. When considering normative economics broadly, as I do here, as the methods aiming for valid prescription for an individual or a group, prescriptive approaches are a part of normative economics.

\TODO{Argumentation theory inspired; but they consider arguments are given. Pursues another article of mine.}

\subsection{This approach}
The proposal of this article to capture reflective preferences while avoiding the just described problems is to define normative theories on the basis of a low demanding axiom complemented with principles of reasonings; that allow decision makers to “reconstruct” the reasoning proposed by the principles so as to be able to check that they apply to their situation; therefore permitting to falsify the theories using observable data representing the rejection by the decision makers of these principles. Such theories are therefore both normative and empirical: normative, in the sense of claiming to capture reflective preferences, which provide a better basis for recommendations or policy making than shallow preferences; and empirical, in the sense of the principles of reasoning being exposed to empirical falsification.

The proposed approach is based on a concept of preference that I call deliberated preference, a formal refinement of the vague notion of reflective preferences. Those are the preferences of an individual that are stable facing counter-arguments. The arguments are presented to the decision maker, and her reaction is observed, for example, in the form of a choice between some alternatives. This is the basic observable of interest to the theories here defined.

It is not an ambition of this article to actually propose concrete theories of deliberated preferences, for this is not achievable in a single article; and the theories as defined here admittedly still suffer important weaknesses. The ambition of this article is accordingly limited to present a novel approach of capturing reflective preferences in simple enough settings.
%This requires to define precisely what such theories should aim to capture; and how they should be compared, which is the sole ambition of this article. 
Some challenges that must be overcomed in order to build concrete theories applicable in realistic settings will be discussed in conclusion.

One long-term goal of this research program is to give individuals the means to reject theories that they find unconvincing after due consideration of all arguments; and accordingly, give incentives to decision theorists to build defenses of these theories able to convince individuals whose preferences are modeled of the soundness of these theories. This might match \possessivecite{hausman_debate_2010} suggestion that we should use rational persuasion rather than nudge individuals, arguing that it would promote autonomy.

\section{Informal presentation}
Although the final aim is about the concept of deliberated preference, which permits to compare alternatives (options in a decision problem), this article will mostly discuss \acp{DJ}. That is because, once the concept of \ac{DJ} is clearly established, it is a simple matter to define deliberated preferences, and will be done in a later section, by viewing a preference instance as a judgment about a pair of alternatives.

For a clear separation of concern, this article defines first normative theories, then empirical theories for these normative theories. A normative theory makes normative claims only, which have to be accepted or rejected on the basis of the apparent soundness of the axiom on which it rests. An empirical theory for a given normative theory, by contrast, makes empirical claims, hence, can be falsified by data. What I call here an empirical theory, for simplicity, is not however purely empirical, but can also be considered normative, for two reasons. First, its conclusions inevitably rest on the normative validity of the normative theory for which it is built. Second, it claims to capture preferences that provide an appropriate basis for recommendation. Empirical theories of \ac{DJ} could thus be called empirical normative theories of \ac{DJ}.

\subsection{Normative theories}
As \citet[p.\ 16]{von_neumann_theory_2004} aptly summarize, “It is clear that every measurement – or rather every claim of measurability – must ultimately be based on some immediate sensation, which possibly cannot and certainly need not be analyzed any futher”. The fundamental element of a normative theory of deliberated preferences is the reaction of an individual, whose preferences are to be captured, to arguments. 

Normative theories define an agenda: a set of propositions that we are interested in knowing the deliberated judgment of individuals about; a set of individuals that the theory claims applies to; a set of arguments; and a protocol of observation of the judgments an individual adopts towards a proposition after exposure to various arguments. Normative theories also define a negation operation on the set of propositions, which this discussion neglects for the sake of simplicity.

Examples of propositions in the agenda include: “I ought to pay at most \$100 for purchasing bundle $b$”, “I ought to acquire alternative $a$ rather than $b$ if it is available in the same conditions”; “I ought not to perform action $a$ because it is immoral”.

The set of individuals in a normative theory represents the situations, socio-economic conditions, acquired diplomas, or other observable features of individuals that must be satisfied for the theory to apply to them. (More about this in conclusion. \TODO{})

Arguments are defined here as anything that can be presented to an individual that may possibly influence her judgment. As an example, an argument can be a text presenting the features of an alternative, or arguing for the (im)morality of an action. The term argument evokes a debate, a metaphor appropriate for the comparison of theories that will be described later; but that should be taken with a grain of salt: an image, a tag price, a poem can be an argument in this article. 

The protocol of observation in a normative theory defines how the theory proposes to present a sequence of arguments to an individual, and how to observe her resulting judgment about a proposition of interest. This protocol permits to determine whether a sequence of arguments “leads” a given individual to a given proposition. If the arguments are texts and propositions are of the form “I ought to acquire alternative $a$ rather than $b$”, an example of such a protocol is: present each argument (each text) in the sequence to the individual, asking her to read it carefully; then give the individual a choice between $a$ and $b$; the sequence of arguments is considered to having led the individual to the proposition “I ought to acquire alternative $a$ rather than $b$” iff he chose $a$ rather than $b$.
In order to allow for repeated observations with a given individual without modifying his endowment, this example protocol should either be content with purely hypothetical choices, or can make these choices binding by letting the individual know that one of his choices will be picked randomly at the end of the experiment and honored.

The judgment of an individual towards propositions may evolve during her exposition to sequences of arguments. Of interest to theories of \ac{DJ} are the cases where a so-called decisive argument exists for a given proposition: an argument is decisive when its presence at the end of the sequence of arguments systematically leads the individual to the proposition, whatever the sequence of arguments. An argument is at the end of a sequence iff it is the last or before last argument of the sequence. Therefore, an argument is decisive when, intuitively, it is able to resist every counter-arguments. 

Theories of \ac{DJ} attempt to capture the \emph{\acl{DJ}} of the individuals it deals with, towards some of the propositions in the agenda. 
The \ac{DJ} of an individual is conceived here as existing independently of any theories of decision.
The \emph{normative adequacy} axiom, the sole axiom on which a normative theory rests, defines a proposition as being in the deliberated judgment of an individual whenever some argument is decisive for that proposition.
Accepting this axiom require to accept that decisive arguments, in the sense defined here, reveal something about the \ac{DJ} of an individual. 

That the \ac{DJ} of an individual exists independently of any theory of decision making might be considered a strong assumption, by analogy to shallow preferences: psychologists have shown that shallow preferences often depend on a priori irrelevant features pertaining to the description of the decision problem or the context of interrogation. Preferences are often said to be constructed, thus, to not exist independently of the framing and context of study.

With no intent of reducing the importance of these findings, it is important to qualify such statements. In supplement to them being not necessarily applicable to reflective preferences, the findings of psychologists do not, evidently, show that all preference statements are labile. As \citet[p.\ xvi]{lichtenstein_construction_2006} write, “To be sure, there are limits to the process of construction. Some televisions just won’t sell; most of the time, people are unlikely to want 80\% of their wages to go into savings. It would be an overstatement to say of preferences, as Gertrude Stein said of Oakland, that “there is no there there.” 

This remark points towards the importance of allowing for incomplete theories of reflective preferences. As \citet{mandler_difficult_2001} discusses in details, it would be unrealistic to postulate that individuals would, after due reflection, in non trivial decision problems, adopt a well specified judgment about any possible proposition of interest.
From this does not follow that \acp{DJ} do not exist, but that someone’s \ac{DJ} should not be considered to be necessarily complete. 
\TODO{Check Mandler.}
The definitions of the theories presented here accordingly tolerate both theory incompleteness, which permits the theory to stay silent on some of the propositions in the agenda, and, more fundamentally, deliberation incompleteness, which permits to tolerate that for some propositions, neither it nor its negation figure in someone’s \ac{DJ}. 
%\cref{sec:incompl} defines formally the notions of theory incompleteness and deliberation incompleteness, and briefly analyzes their relationships.

\subsection{Empirical theories}
A normative theory defines how to determine that an argument is not decisive for a given proposition: suffices to find a sequence of arguments that does not lead to that proposition. This, in turn, under condition of accepting its normative validity, permits to know something about the \ac{DJ} of an individual. But this gives no concrete means to know someone’s \ac{DJ}. The set of potentially relevant arguments will generally be very large, and can’t be explored thoroughfully. More importantly, only some sequences of arguments can be tested for a given individual. That is because, first, the individual’s patience is limited, and second, once a sequence has been presented to the individual, this cannot necessarily be undone: she might remember some of the arguments, and they may have a lasting influence on her perspective. Postulating that the individual forgets after some time and separating the tests in time might mitigate this problem, but not necessarily cancel it. 

%As a result, it should be possible to determine someone’s \ac{DJ} using only a small increasing set of sequences of arguments. A set of sequences of arguments is increasing iff the sequences can be ordered completely such that each sequence following another one “continues” the other one: its first arguments are those constituting the sequence that it follows.

These two problems are tackled by introducing empirical theories of \ac{DJ}. An empirical theory for a given normative theory defines a subset of propositions of the agenda, that it claims are contained in the \acp{DJ} of individuals concerned by the normative theory. They are called the proposition that the theory supports. The theory also claims that it is able to argue in favor of the proposition it supports in such a way that individuals will be convinced of the quality of these arguments. This is accomplished by three binary relations, part of the definition of empiricial theories: one represents its initial arguments, one plans the possible attacks, and one defines its counter-arguments. (An empirical theory also defines a concatenation operation able to dynamically build arguments from fragments of arguments, an aspect which is not discussed in this informal presentation.)

\TODO{Pursue here.}
More precisely, the argumentation requirements are as follows. Given an individual, the empirical theory must be able to present an argument that is, at least initially, convincing to the individual (as judged thanks to the observation protocol of the normative theory). It is not mandated that this argument be decisive, as only a weaker requirement is enough to ensure the validity of its claims. Instead, the theory must declare which arguments it knows how to answer. Such arguments are called the planned attackers. Whether an argument effectively attacks another one (from the point of view of the individual) can, again, be defined in terms of the observation protocol of the normative theory. Requiring that the theory declares in advance its attackers mean that if an argument effectively attacks one of its argument, then this attack must have been planned. 

More precisely, the theory must, for each sequence of arguments leading the individual to a proposition contrary to what the theory supports, be able to exhibit an argument that, when appended to that sequence, leads the 

exhibit a decisive argument for that proposition, as judged from the individual’s point of view.

\section{Normative theories}
\subsection{Definition}
One of the primitive elements of a normative theory is a set of arguments $\allargs$.
Given a set of arguments $\allargs$, define the set of finite sequences of elements in $\allargs$ (including the empty sequence) as $\allhist = \bigcup_{k \in \N} \allargs^{\intvl{1, k}}$.
Throughout the article, $\N$ includes zero and $\N^* = \N \setminus \set{0}$. Given $j, k \in \N$, the notation $\intvl{j, k}$ represents $\set{l \suchthat j ≤ l ≤ k}$. Note that $\intvl{1, 0} = \emptyset$, and given any set $S$, the notation $S^\emptyset$ represents the empty set. A generic element of $\allhist$ is denoted by $\hist$ and called a path.

Two supplementary primitive elements of a normative theory are its agenda $\allprops$, whose elements are called propositions; and a set of individuals $I$ that the theory applies to.
Given a set of individuals, a set of arguments and an agenda, an observation protocol ${\allleadsto}: I → \Phi^{\allhist}$ maps individuals and paths to propositions. 
Equivalently, given any individual $i \in I$, ${\ileadsto}: \allhist → \Phi$ maps paths to propositions for this individual. 
Given an individual $i \in I$, a path $\hist$ and a proposition $\phi$, the path is said to lead to the proposition for this individual iff $\hist \ileadsto \phi$.

A normative theory also defines an involution operator $¬$ on $\allprops$, thus a function $¬: \allprops → \allprops$ satisfying $\forall \prop \in \allprops: ¬¬\prop = \prop$.

A normative theory is a tuple $(\allprops, ¬, \allargs, I, {\allleadsto})$ 
%with $\allprops$ an agenda, $¬$ a negation operator on $\allprops$, $\allargs$ a set of arguments, $I$ a set of individuals, and ${\allleadsto}$ an observation protocol, 
satisfying the constraints described here above.
%where $\Phi = \allprops × \set{\mathit{poss}, \mathit{sure}}$.

This article also uses this symbol as a negation operator on $\Phi$, defined as follows, $\forall \prop \in \allprops$: $¬\propposs = (¬\prop, \mathit{sure})$ and $¬\propsure = (¬\prop, \mathit{poss})$. Observe that $¬¬\phi = \phi$. (The symbol $¬$ is also used in this article to negate classical mathematical propositions; this should create no confusion.)
As usual with binary relations, $\hist \nileadsto \phi$ means $(\hist, \phi) \notin {\ileadsto}$.

Given $i \in I, \phi \in \Phi, \hist \in \allhist$, say that $\hist$ leads strongly to $\phi$ iff $\hist \ileadsto \phi \land \hist \nileadsto ¬\phi$.
Define $\hist \ileadstost \phi$ iff $[\hist \text{ leads strongly to } \phi] \land [\phi = \propsure ⇒ \hist \text{ leads strongly to } \propposs]$.
(The visual proximity of the symbols $\ileadsto$ and $\ileadstost$ creates a small risk of confusion, but its usage seems justified by the fact that $\ileadsto$ is not used in the rest of this article, except in the proof of \cref{th:protcoh}.)

Given $\hist \in \allhist$, if $l$ denotes its length, $\hist(\intvl{1, l})$ denotes the range of $\hist$, that is, all arguments contained in the sequence $\hist$.
Let $\histend = \hist(\intvl{\max(1, l-1), l})$ denote the set containing the last two arguments of the sequence $\hist$ if the sequence has at least two elements; and all arguments of the sequence if it has exactly one element or is empty.

Given $i \in I, \phi \in \Phi, \ar \in \allargs$, define $\ar \ileadstost \phi$ iff $\forall \hist \in \allhist \suchthat \ar \in \histend: \hist \ileadstost \phi$.

\subsection{Classify propositions}
\begin{axiom}[Normative adequacy]
	\label{ax:norm}
	$\forall i \in I, \phi \in \Phi: 
		[(\exists \ar \in \allargs \suchthat \ar \ileadstost \phi) \land ¬(\exists \ar \in \allargs \suchthat \ar \ileadstost ¬\phi)] ⇒ \phi \in \iPhi \land ¬\phi \notin \iPhi.$
\end{axiom}

\Cref{ax:norm} is actually equivalent to the following property.
\begin{property}[Simple normative adequacy]
	\label{ax:snorm}
	$\forall i \in I, \phi \in \Phi: 
		[\exists \ar \in \allargs \suchthat \ar \ileadstost \phi] ⇒ \phi \in \iPhi \land ¬\phi \notin \iPhi.$
\end{property}

To see why this hold, consider the following definition and proposition.
\begin{definition}[Protocol coherence]
	\label{def:protcoh}
	$\forall i \in I, \phi \in \Phi$:
	\begin{equation}
		[\exists \ar \in \allargs \suchthat \ar \ileadstost \phi] ⇒ [\nexists \ar \in \allargs \suchthat \ar \ileadstost ¬\phi].
	\end{equation}
\end{definition}
\begin{proposition}
	\label{th:protcoh}
	Any normative theory satisfies Protocol coherence.
\end{proposition}
\begin{proof}
	Protocol coherence is satisfied by definition of $\ileadstost$. Indeed, if for some $\ar \in \allargs$, $\ar \ileadstost \phi$, then $\forall \ar_1 \in \allargs: (\ar_1, \ar) \nileadsto ¬\phi$; thus, for any $\ar' \in \allargs$, $\ar' \ileadstost ¬\phi$ is false, because it requires that $\forall \ar_1 \in \allargs: (\ar', \ar_1) \ileadsto ¬\phi$, and thus in particular, that $(\ar', \ar) \ileadsto ¬\phi$.
\end{proof}

That \cref{ax:norm} is equivalent to \cref{ax:snorm} immediately follows from \cref{th:protcoh}.
\begin{proposition}
	\label{th:normeqsnorm}
	Any normative theory satisfies Normative adequacy iff it satisfies simple normative adequacy.
\end{proposition}

To further analyze this and related matters, given a normative theory and $i \in I$, say that a proposition $\phi \in \Phi$ is \emph{$i$-decidable} iff $\exists \ar \in \allargs \suchthat \ar \ileadstost \phi \lor \ar \ileadstost ¬\phi$.
The $i$-decidable propositions are those on which the normative theory permits to take a position. It will generally be unknown in practical applications, and may be empty.
The next property ensures an unambiguous interpretation of the content of $\iPhi$ for all $i$-decidable propositions. In this definition, $\oplus$ denotes the exclusive disjunction; thus $\phi \in \iPhi \oplus ¬\phi \in \iPhi$ is equivalent to: $\phi \in \iPhi ⇔ ¬\phi \notin \iPhi$. It can be seen to hold by using \cref{th:normeqsnorm}.

\begin{property}[Restricted interpretability]
	\label{def:restrinterpr}
	$\forall i \in I, \phi \in \Phi: [\exists \ar \in \allargs \suchthat \ar \ileadstost \phi \lor \ar \ileadstost ¬\phi] ⇒ [\phi \in \iPhi \oplus ¬\phi \in \iPhi]$.
\end{property}

\subsection{Non-completeness}
\label{sec:incompl}
\begin{definition}[Theory completeness]
	$\forall \phi \in \Phi: [\exists \ar \in \allargs \suchthat \ar \ileadstost \phi] \lor [\exists \ar \in \allargs \suchthat \ar \ileadstost ¬\phi]$.
\end{definition}
\begin{definition}[Unrestricted interpretability]
	$\forall \phi \in \Phi: \phi \in \iPhi \oplus ¬\phi \in \iPhi$.
\end{definition}
\begin{definition}[Deliberation completeness]
	$\forall \phi \in \Phi: \phi \in \iPhi \lor ¬\phi \in \iPhi$.
\end{definition}
The following proposition directly follows from the definitions, together with \cref{th:protcoh}.
\begin{proposition}
	1) Under Normative adequacy, Theory completeness implies Deliberation completeness. 2) Unrestricted interpretability implies Deliberation completeness.
\end{proposition}
Theories as defined here do not mandate Theory completeness. This is good for two reasons. First, even the weaker Deliberation completeness can be considered too strong. Second, it is dubious that, for non trivial decision problems, theories can be found for which both Normative adequacy is normatively compelling and Theory completeness holds.

\section{Empirical theories}
\subsection{Definition}
In this section, a normative theory is supposed given.

An empirical theory is constituted of the following elements.

An initial argumentation relation ${\gleadsto} \subseteq \allargs × \allprops$ relates some arguments to some propositions. Its role in an empirical theory is to define the propositions that the empirical theory supports (corresponding to $\gPhi$, the range of ${\gleadsto}$) and, for each such proposition, to define the arguments that the theory wants to use as initial arguments to argue in favor of this proposition. The set of initial arguments is $\gleadstoinv(\Phi) = \set{\ar \in \allargs \suchthat \exists \phi \in \Phi \suchthat \ar \gleadsto \phi}$.

A planned attacks relation ${\gbeats} \subseteq \allargs × \allargs$ is an irreflexive binary relation on the arguments. An argument $\ar_1$ is said to be a planned attack against an argument $\ar_0$ iff $\ar_1 \gbeats \ar_0$. Given  an argument $\ar_0$, $\gbeatsinv(\ar_0) = \set{\ar_1 \in \allargs \suchthat \ar_1 \gbeats \ar_0}$ denotes the set of planned attacks against $\ar_0$; and $\gbeatsinv(\allargs) = \set{\ar_1 \in \allargs \suchthat \exists \ar_0 \in \allargs \suchthat \ar_1 \gbeats \ar_0}$ denotes the set of arguments that are planned attacks against at least one argument.

An empirical theory defines a set of argument fragments $\alldargs$. They are primitive elements that can be added to selected other fragments and arguments to form new arguments. They are used to dynamically build defensive arguments against the planned attacks. 

A defense relation ${\dbeats} \subseteq \alldargs × \gbeatsinv(\allargs)$ is an injective and surjective binary relation over $\alldargs × \gbeatsinv(\allargs)$. Thus, $\forall \ar_1 \suchthat [\exists \ar_0 \suchthat \ar_1 \gbeats \ar_0]$: $\dbeatsinv(\ar_1)$ is a singleton. 
Note that $\dbeats$ need not be a bijection as it is generally not a function. 
Given $\ar_1 \in \gbeatsinv(\allargs)$, $\dbeatsinv(\ar_1)$ represents the fragment that the theory $\gamma$ plans to use as a defense against the planned attack $\ar_1$.

A fragment is said to defend an argument iff it is used as a defense against a planned attack against that argument. Formally, given any set $S$, define $\powersetz{S}$ as the set of subsets of $S$ without the empty set.
Define $\df: \allargs → \powersetz{\allargs}$ as $\df = \dbeatsinv \circ \gbeatsinv$.
Given $\ar \in \allargs$, $\df(\ar)$ denotes the \emph{fragments defending $\ar$}. 
In other words, given $\ard \in \dargs$ and $\ar_0 \in \allargs$, $\ard \in \df(\ar_0)$ iff $\exists \ar_1 \in \allargs \suchthat \ard \dbeats \ar_1 \gbeats \ar_0$.

A \emph{coalition of fragments defending $\ar$} is any element of $\powersetz{\df(\ar)}$.
Such a coalition can be combined with the argument it defends to form a new argument.
An empirical theory defines, to that effect, a function $+: \bigcup_{\ar \in \allargs}(\set{s} × \powersetz{\df(\ar)}) → \allargs$.
Given any $\ar \in \allargs$ and $\emptyset ≠ \dargs \subseteq \df(\ar)$, the notation $\ar + \dargs$ stands for $+(\ar, \dargs)$. 

This finally permits to define the defenders of an argument: they are all the arguments that can be formed by adding the argument to a coalition of fragments that defends it.
Given any $\ar \in \allargs$, define $\dfp(\ar) \subseteq \allargs$, the \emph{defenders of $\ar$}, as $\dfp(\ar) = \set{\ar + \dargs \suchthat \emptyset ≠ \dargs \subseteq \df(\ar)}$.

This mechanism permits to define defenders of defenders of an argument, $\dfp(\dfp(\ar))$, defenders of defenders of defenders of an argument, … An empirical theory must however be a priori restricted in the maximal “depth” of its argumentation strategy, in the following sense.
Given any set $S$ and function $f: S → S$, let $f^0$ represent the identity function, and given $k \in \N$, let $f^{k + 1} = f \circ f^k$.
An empirical theory must satisfy
$\exists k \in \N \suchthat (\dfp)^k(\gleadstoinv(\Phi)) = \emptyset$.
This does not imply that the relations $\gbeats$ or $\dbeats$ should be finite, however.

An empirical theory (for a normative theory) can now be defined as a tuple $(\gleadsto, \gbeats, \alldargs, \dbeats, +)$ satisfying the constraints described here above.

An example permits to ensure that the constraint of finite depth does not mandate finiteness of the relations $\gbeats$ or $\dbeats$.
\begin{proposition}[Possibility of infinite relations]
	An empirical theory may involve infinite relations $\gbeats$ and $\dbeats$.
\end{proposition}
\begin{proof}
	Consider an agenda containing two propositions, $\allprops = \set{\phi, ¬\phi}$, an infinite set of arguments $\allargs = \set{\ar_k \suchthat k \in \N}$ with $\forall j, k \in \N: \ar_j ≠ \ar_k$, and consider any set of individuals $I$ and observation protocol $\allleadsto$. 
	Define $2\N^* = \set{k \suchthat k / 2 \in \N^*}$ as the set of non-zero even integers.
	Define ${\gleadsto} = \set{(\ar_0, \phi)}$.
	Define ${\gbeats} = \set{(\ar_{k - 1}, \ar_0) \suchthat k \in 2\N^*}$ so that all odd numbered arguments are planned attacks against $\ar_0$. 
	Consider an infinite set of fragments $\alldargs = \set{\ard[k] \suchthat k \in 2\N^*}$ with $\forall j, k \in 2\N^*: \ard[j] ≠ \ard[k]$ so that fragments correspond to non-zero even integers, and define ${\dbeats} = \set{(\ard[k], \ar_{k - 1}) \suchthat k \in 2\N^*}$ so that each (even numbered) defense fragment attacks exactly one (odd numbered) planned attacker of $\ar_0$. 
	Given any coalition of fragments $\dargs$ defending $\ar_0$, consider the set of indices corresponding to the fragments in the coalition, $K_{\dargs} = \set{k \in 2\N^* \suchthat \exists \ard[k] \in \dargs}$; 
	define $j_{\dargs} = \sum_{k \in K_{\dargs}}$ as the sum of these indices;
	and define $\ar_0 + \dargs = \ar_{j_{\dargs}}$. 
%	$\ard[k] \in \alldargs$, define $\ar_0 + \set{\ard[k]} = \ar_k$. 
	Consider an empirical theory $(\gleadsto, \gbeats, \alldargs, \dbeats, +)$ for the normative theory $(\allprops, ¬, \allargs, I, {\allleadsto})$. That this is indeed a legal empirical theory can be confirmed by checking that all constraints are satisfied, in particular, $(\dfp)^2(\gleadstoinv(\Phi)) = \dfp(\dfp(\ar_0)) = \dfp(\set{\ar_k \suchthat k \in 2\N^*}) = \emptyset$ because there is no planned attack against any argument in $\set{\ar_k \suchthat k \in 2\N^*}$. This theory involves infinite relations $\gbeats$ and $\dbeats$.
\end{proof}

\subsection{Falsifiability}
Given $\args \subseteq \allargs$, $\hist \in \allhist$, and letting $l$ denote the length of $\hist$,
the set $\args \cap \hist(\intvl{1, l})$ denotes the arguments that are both in $\args$ and contained in the sequence $\hist$.
Abusing notation, I write $\args \cap \hist$ to designate $\args \cap \hist(\intvl{1, l})$, therefore using simply $\hist$ in that expression to denote the \emph{range} of $\hist$.

Given $i \in I, \phi \in \Phi, \ar \in \allargs$ and $\args \subseteq \gbeatsinv(\ar)$, define $\args \ibeats \ar$ iff $\exists \hist \in \allhist \suchthat \args = \gbeatsinv(\ar) \cap \hist \land \ar \in \histend \land \hist \nileadstost \phi$. 
Such a set $\args$, when not empty, represents an effective set of planned attackers of $\ar$ against $\phi$. 
The notation uses a striked through $\phi$ in superscript to suggest that when $\args \ibeats \ar$, the coalition $\args$ argues against $\phi$, from the point of view of $i$.
I write that $\ar$ \emph{$i$-defends} $\phi$ to mean that $\emptyset \nibeats \ar$. 

Given $i \in I, \phi \in \gPhi, \ar \in \allargs$, define $\di(\ar) \subseteq \powersetz{\alldargs}$ as $\di(\ar) = \set{\dbeatsinv(\args) \suchthat \emptyset ≠ \args \subseteq \gbeatsinv(\ar) \land \args \ibeats \ar}$. It represents the coalitions of fragments defending $\ar$ (when arguing for $\phi$), from the point of view of $i$, against its effective planned attackers. 
The following proposition shows that these coalitions can be added to $\ar$, and will be useful for other results.
\begin{proposition}
	\label{th:indf}
	$\forall i \in I, \phi \in \gPhi, \ar \in \allargs: \dargs \in \di(\ar) ⇒ \emptyset ≠ \dargs \subseteq \df(\ar)$.
\end{proposition}
\begin{proof}
	I will prove that $\forall \ar \in \allargs: \set{\dbeatsinv(\args) \suchthat \emptyset ≠ \args \subseteq \gbeatsinv(\ar)} = \powersetz{\df(\ar)}$, from which the conclusion follows, and which can be useful to clarify the relationship between $\di(\ar)$ and $\df(\ar)$. Note that by definition of $\df(\ar)$, $\dargs \in \powersetz{\df(\ar)}$ iff $\emptyset ≠ \dargs \subseteq \dbeatsinv(\gbeatsinv(\ar))$. 
	Thus, we have to prove that
	$\forall \dargs \subseteq \allargs$:
	\begin{equation}
		\exists \emptyset ≠ \args \subseteq \gbeatsinv(\ar) \suchthat \dbeatsinv(\args) = \dargs ⇔ \emptyset ≠ \dargs \subseteq \dbeatsinv(\gbeatsinv(\ar)).
	\end{equation}

	First (from left to right), given any $\args \suchthat \emptyset ≠ \args \subseteq \gbeatsinv(\ar_0) \land \dargs = \dbeatsinv(\args)$, let us show that $\dargs ≠ \emptyset$ and that $\dargs \subseteq \dbeatsinv(\gbeatsinv(\ar))$. 
	Indeed, as $\args ≠ \emptyset$, we can pick some $\ar_1 \in \args$, from which follows that $\ar_1 \gbeats \ar_0$, hence, for some $\ard \in \alldargs$, $\ard \dbeats \ar_1$ (as mandated by ${\dbeats}$), thus $\ard \in \dargs$, hence $\dargs ≠ \emptyset$. Also, because $\args \subseteq \gbeatsinv(\ar)$, $\dbeatsinv(\args) \subseteq \dbeatsinv(\gbeatsinv(\ar))$. 
	
	Second (from right to left), let us show that $\emptyset ≠ \dargs \subseteq \dbeatsinv(\gbeatsinv(\ar)) ⇒ \exists \emptyset ≠ \args \subseteq \gbeatsinv(\ar_0) \suchthat \dbeatsinv(\args) = \dargs$. 
	Defining $\args = {\dbeats}(\dargs) \cap \gbeatsinv(\ar_0)$, we see that $\dbeatsinv(\args) = \dargs$: considering any $\ard \in \dargs$, $\ard \dbeats \ar_1 \gbeats \ar_0$ for some $\ar_1 \in \allargs$, thus, $\ar_1 \in \args$, thus, $\ard \in \dbeatsinv(\args)$; and considering any $\ar' \in \dbeatsinv(\args)$, $\ar' \dbeats \ar_1$ for some $\ar_1 \in \args$, thus $\ar_1 \gbeats \ar_0$, thus, $\ar_1 \in {\dbeats}(\dargs)$, thus $\ard \dbeats \ar_1$ for some $\ard \in \dargs$, and by injectivity  of $\dbeats$, $\ar' = \ard$, whence $\ar' \in \dargs$. 
	That $\args ≠ \emptyset$ follows from $\dargs ≠ \emptyset$.
\end{proof}

Given $\ar \in \allargs$, define $\dip(\ar) = \set{(\ar + \dargs) \suchthat \dargs \in \di(\ar)}$, representing the subset of defenders of $\ar$ required for convincing $i$ of $\phi$.
Given $i \in I, \phi \in \gPhi$, define $\gargs \subseteq \allargs$ as the transitive closure of ${\gleadstoinv}(\phi)$ under $\dip$: $\gargs = \cup_{k \in \N} (\dip)^k({\gleadstoinv}(\phi))$.

\begin{definition}[Convincingness]
	An empirical theory is \emph{convincing} iff $\forall i \in I, \phi \in \gPhi, \ar \in \gargs: \ar$ $i$-defends $\phi$.
\end{definition}

A \emph{falsification instance} of an empirical theory is a tuple $(i, \phi, \hist, \ar) \in I × \gPhi × \allhist × \gargs$ such that $\gbeatsinv(\ar) \cap \hist = \emptyset \land \ar \in \histend \land \hist \nileadstost \phi$. 
I also write that $(i, \phi, \hist, \ar)$ falsifies $\gamma$ to mean that it is a falsification instance of $\gamma$.

\begin{theorem}
	An empirical theory is convincing iff it has no falsification instances.
\end{theorem}
\begin{proof}
	For all $i \in I, \phi \in \gPhi, \ar \in \gargs$: $\ar$ does not $i$-defend $\phi$ iff $\exists \hist \in \allhist \suchthat \left[\gbeatsinv(\ar) \cap \hist = \emptyset \land \ar \in \histend \land \hist \nileadstost \phi\right]$ iff $\exists \hist \in \allhist$ such that $(i, \phi, \hist, \ar)$ falsifies $\gamma$. Therefore, $\gamma$ is not convincing iff $\exists (i, \phi, \ar) \suchthat \ar$ does not $i$-defend $\phi$ iff $\exists \hist \suchthat (i, \phi, \hist, \ar)$ falsifies $\gamma$ iff $\gamma$ has a falsification instance.
\end{proof}

\begin{definition}[Validity]
	An empirical theory is \emph{valid} iff $\forall i \in I, \phi \in \gPhi: [\phi \in \iPhi \land ¬\phi \notin \iPhi]$.
\end{definition}

\begin{lemma}
	\label{th:produce}
	Given $\phi \in \gPhi$, $\ar_0 \in \gargs$ that $i$-defends $\phi$, and $\hist \nileadstost \phi$ with $\ar_0 \in \histend$, define $\args =  \gbeatsinv(\ar_0) \cap \hist$, $\dargs = \dbeatsinv(\args)$, and $\ar_2 = \ar_0 + \dargs$. Then, $\ar_2 \in \gargs \cap \dfp(\ar_0)$.
\end{lemma}
\begin{proof}
	By definition of $\ibeats$, $\args \ibeats \ar_0$.
	Also, because $\ar_0$ $i$-defends $\phi$, $\args ≠ \emptyset$. 
	Therefore, $\dargs \in \di(\ar_0)$.
	It follows that $\dargs \in \df(\ar_0)$, by \cref{th:indf}, and that $\ar_2 \in \dfp(\ar_0)$, by definition of $\dfp$. It also follows that $\ar_2 \in \dip(\ar_0)$, hence, that $\ar_2 \in \gargs$.
\end{proof}

\begin{theorem}
	A normatively adequate and convincing empirical theory is valid.
\end{theorem}
\begin{proof}
	Given any $i \in I$ and $\phi \in \gPhi$, suffices to show that for some $\ar \in \allargs$, $\ar \ileadstost \phi$; the conclusion then follows thanks to simple normative adequacy (that \cref{th:normeqsnorm} shows to be equivalent to normative adequacy). 
	
	Lemma: $[\ar_0 \in \gargs \land \ar_0 \nileadstost \phi] ⇒ \exists \ar_2 \in \gargs \cap \dfp(\ar_0)$. That is because $\ar_0 \in \gargs$ implies that $\ar_0$ $i$-defends $\phi$, by convincingness, and,
	by definition, $\ar_0 \nileadstost \phi$ means that $\exists \hist \in \allhist \suchthat \ar_0 \in \histend \land \hist \nileadstost \phi$; \cref{th:produce} thus applies.

	To prove the theorem, build a finite sequence of arguments in $\gargs$, starting with any $\ar_0 \in \gleadstoinv(\phi)$. Observe that $\ar_0 \in \gargs$. Given $k \in 2\N$, if $\ar_k \nileadstost \phi$, define $\ar_{k + 2}$ using the above lemma, obtaining $\ar_{k + 2} \in \gargs \cap (\dfp)^\frac{k + 2}{2}(\ar_0)$. Continue applying the lemma and producing further arguments as long as $\ar_k \nileadstost \phi$.
	By definition of an empirical theory, $\exists k \in \N \suchthat (\dfp)^k(\ar_0) = \emptyset$, thus, this process ends with some $\ar$ such that $\ar \ileadstost \phi$.
\end{proof}

\subsection{Effectively comparing models}
Given $\phi \in \Phi$ and two empirical theories $\gamma$ and $\delta$ for a given normative theory, say that they disagree about $\phi$ iff $\phi \in \gPhi[\gamma] \land ¬\phi \in \gPhi[\delta]$. Say that they are incompatible iff there is at least one $\phi \in \Phi$ that they disagree about.

Given $\phi \in \Phi$, $\hist \in \allhist$ and $\ar_\gamma \in \gargs$, say that $(\hist, \ar_\gamma)$ challenges $\phi$ iff $\hist \nileadstost \phi$ and $\ar_\gamma \in \histend$.
Given $\phi \in \Phi, \ar_\gamma \in \gargs, \ar_\delta \in \gargsdelta$, say that $(\hist, \ar_\gamma, \ar_\delta)$ is a $\phi/¬\phi$-challenge iff $(\hist, \ar_\gamma)$ challenges $(\gamma, \phi)$ or $(\hist, \ar_\delta)$ challenges $(\delta, ¬\phi)$.

\begin{lemma}
	\label{th:endchal}
	Given $\ar_\gamma \in \gargs$, $\ar_\delta \in \gargsdelta$ and $\hist \in \allhist$ such that $\set{\ar_\gamma, \ar_\delta} \subseteq \histend$, $(\hist, \ar_\gamma, \ar_\delta)$ is a $\phi/¬\phi$-challenge.
\end{lemma}
\begin{proof}
	Either $\hist \nileadstost \phi$, and $(\hist, \ar_\gamma)$ challenges $\phi$, or $\hist \ileadstost \phi$, hence, by \cref{th:protcoh}, $\hist \nileadstost ¬\phi$, and $(\hist, \ar_\delta)$ challenges $¬\phi$. 
\end{proof}

\begin{lemma}
	\label{th:nextchal}
	Given $(\hist, \ar_\gamma)$ challenging $\phi$, with $\ar_\gamma \in \gargs \cap (\dfp)^k(\gleadstoinv(\phi))$, and $\ar_\delta \in \gargsdelta$: either $(i, \phi, \hist, \ar_\gamma)$ falsifies $\gamma$, or there exists a $\phi/¬\phi$-challenge $(\histp, \ar'_\gamma, \ar_\delta)$, with $\ar'_\gamma \in \gargs \cap (\dfp)^{k + 1}(\gleadstoinv(\phi))$.
\end{lemma}
\begin{proof}
	If $\ar_\gamma$ does not $i$-defend $\phi$, then $(i, \phi, \hist, \ar_\gamma)$ falsifies $\gamma$, which ends the proof. 
	Thus, assume that $\ar_\gamma$ $i$-defends $\phi$. 

	\Cref{th:produce} applies and yields $\ar'_\gamma = \ar_\gamma + \dbeatsinv(\gbeatsinv(\ar_\gamma)) \in \gargs \cap \dfp(\ar_\gamma)$, thus $\ar'_\gamma \in (\dfp)^{(k + 1)}(\gleadstoinv(\phi))$. 
	Now, define $\histpp = (\hist, \ar'_\gamma)$ and test whether $\histpp \nileadstost \phi$. 
	
	If $\histpp \nileadstost \phi$, define $\histp = \histpp$, and observe that $(\histp, \ar'_\gamma, \ar_\delta)$ challenges $\phi$, which concludes the proof. 
	Thus, assume that $\histpp \ileadstost \phi$. 

	Let us now define $\histp$ such that $\set{\ar'_\gamma, \ar_\delta} \subseteq \histpend$, so that \cref{th:endchal} applies on $(\histp, \ar'_\gamma, \ar_\delta)$, which will conclude the proof. 
	
	If $\ar_\delta \in \histppend$, this is done by defining $\histp = \histpp$. Otherwise, define $\histp = (\histpp, \ar_\delta)$ and test whether $\histp \nileadstost \phi$. 
\end{proof}

%Given a normative theory and two empirical theories for that normative theory and $i \in I$, 
%a \emph{test procedure with $i$} is a finite sequence 
\commentOC{Need the test procedure to be a tree. Atom: $(k, \phi)$ meaning whether ${\hist}^k \ileadstost \phi$. Branch label: atoms linked using or and and operators. From a node $n = (\hist, \ar, \phi)$, let branches depart, such that their labels do not overlap. From a node, if no branch label is true, the node must falsify a theory. In the case of the procedure below, it is not a list because we may query several times the same theory or switch to the other one, depending on the last answer. But if simplifying the theory (single-try), we may then simplify the definition of a test procedure.}

A path $\histp$ starts with a path $\hist$ of length $l$ iff $\histp$ has length at least $l$ and $\forall k \in \intvl{1, l}: \histp_k = \hist_k$.
An increasing sequence of paths is a finite sequence $({\hist}^k)_{k \in \intvl{1, N}}$, for some $N \in \N^*$, of paths, 
such that $\forall k \in \intvl{2, N}, {\hist}^k$ starts with ${\hist}^{k-1}$. 

Given two incompatible empirical theories for a given normative theory, $\alpha = (\gleadsto[\alpha], \gbeats[\alpha], \alldargs_\alpha, \dbeats[\alpha], +_\alpha)$ and $\beta = (\gleadsto[\beta], \gbeats[\beta], \alldargs_\beta, \dbeats[\beta], +_\beta)$ and given any $i \in I$, the following procedure falsifies one of these theories, queries only $i$, and queries $i$ only about some increasing sequence of paths.
\begin{procedure}
	Pick any $\phi \in \Phi$ about which $\alpha$ and $\beta$ disagree.
	
	Given $k_\alpha, k_\beta \in \N, \hist \in \allhist, \ar_\alpha \in \gargsalpha, \ar_\beta \in \gargsbeta$, 
	say that 
	$(\hist, \ar_\alpha, \ar_\beta)$ is a challenge of complexity $(k_\alpha, k_\beta)$ iff 
	$(\hist, \ar_\alpha, \ar_\beta)$ is a $\phi/¬\phi$-challenge, $\ar_\alpha \in (\dfp[\alpha])^{k_\alpha}(\gleadstoinv[\alpha](\phi))$, and $\ar_\beta \in (\dfp[\beta])^{k_\beta}(\gleadstoinv[\beta](¬\phi))$.
	
	Pick any $\ar_{-1} \in \allargs \suchthat \ar_{-1} \gleadsto[\beta] ¬\phi$. Observe that $\ar_{-1} \in \gargsbeta \cap (\dfp[\beta])^0(\gleadstoinv[\beta](¬\phi))$. Test whether $(\ar_{-1}) \nileadstost \phi$. If so, $(i, \phi, (\ar_{-1}), \ar_{-1})$ falsifies $\beta$, which ends the procedure. 
	Thus, assume that $(\ar_{-1}) \ileadstost \phi$. 
	
	Pick any $\ar_0 \in \allargs \suchthat \ar_0 \gleadsto[\alpha] \phi$. Observe that $\ar_0 \in \gargsalpha \cap (\dfp[\alpha])^0(\gleadstoinv[\alpha](\phi))$.  Define $\hist = (\ar_{-1}, \ar_0)$ and test whether $\hist \nileadstost \phi$. By \cref{th:endchal}, $(\hist, \ar_0, \ar_{-1})$ is a $\phi/¬\phi$-challenge, thus, $(\hist, \ar_0, \ar_{-1})$ is a challenge of complexity $(0, 0)$.
	
	Suffices now to apply \cref{th:nextchal} repetitively, until finding a falsification instance of $\alpha$ or $\beta$. This is because, given a challenge $(\hist, \ar_\alpha, \ar_\beta)$ of complexity $(k_\alpha, k_\beta)$, \cref{th:nextchal} applies, yielding either a falsification instance, or a new challenge of higher complexity. 
Indeed, if $(\hist, \ar_\alpha)$ challenges $\phi$, apply \cref{th:nextchal} with $\alpha$ for $\gamma$, $\beta$ for $\delta$, and obtain either that $\hist$ falsifies $\alpha$, or a new tuple $(\histp, \ar'_\alpha, \ar_\beta)$ that is a challenge of complexity $(k_\alpha + 1, k_\beta)$.
And if $(\hist, \ar_\beta)$ challenges $¬\phi$, apply \cref{th:nextchal} with $¬\phi$ for $\phi$, $\beta$ for $\gamma$, $\alpha$ for $\delta$, and obtain either that $\hist$ falsifies $\beta$, or a new tuple $(\histp, \ar'_\beta, \ar_\alpha)$ such that $(\histp, \ar_\alpha, \ar'_\beta)$ is a challenge of complexity $(k_\alpha, k_\beta + 1)$.
	
	Given that $\exists K_\alpha \suchthat (\dfp[\alpha])^{K_\alpha}(\gleadstoinv[\alpha](\phi)) = \emptyset$ and $K_\beta \suchthat (\dfp[\beta])^{K_\beta}(\gleadstoinv[\beta](¬\phi)) = \emptyset$ by definition of empirical theories, this process must end with a falsifying instance of $\alpha$ or $\beta$.
\end{procedure}

\subsection{TODO, about comparisons}
Count the number of tests; assume simpler theory.

Indicate sufficient hypothesis for ensuring that if a model resisted every attack, it is valid. Thus, that searching for falsification instances in the tested ones is enough.

\section{Preferences}
\subsection{Preferences, better}
Let $\allalts$ be a set of alternatives. 

A choice prediction $C_\gamma: \powersetz{\allalts} → \powersetz{\powersetz{\allalts}}$ is a mapping from each possible bundle $\emptyset ≠ B \subseteq \allalts$ to a non-empty set $\emptyset ≠ C_\gamma(B) = \mathcal{C} \subseteq \powersetz{B}$ of non-empty sets of alternatives included in the bundle. 
Thus, a generic element $\alts \in C_\gamma(B)$ satisfies $\emptyset ≠ \alts \subseteq B$. 
The choice prediction given a bundle $B$ predicts that $i$ will, given $B$, choose some indifference set that includes one of the elements $\alts \in C_\gamma(B)$.

Define $\Phi_\allalts = \set{\phi_{B, a \ind b} \suchthat \emptyset ≠ B \subseteq \allalts, a ≠ b \in B} \cup \set{\phi_{B, ¬a} \suchthat \emptyset ≠ B \subseteq \allalts, a \in B}$.

Given $\emptyset ≠ B \subseteq \allalts$, define $\emptyset ≠ B_{\hist, i} \subseteq B$ as the subset that $i$ chooses as her indifference choice after exposure to $\hist$.
Define $\hist \ileadsto \phi_{B, a \ind b}$ iff $[a \in B_{\hist, i} ⇔ b \in B_{\hist, i}]$.
Define $\hist \ileadsto \phi_{B, ¬a}$ iff $a \notin B_{\hist, i}$.

Given $a, b \in \allalts$, define $a \peq b$ iff $[a = b] \lor [\forall \set{a, b} \subseteq B: \phi_{B, a \ind b} \in \gPhi \lor \phi_{B, ¬b} \in \gPhi]$.

In order to be as resolute as possible, define an arbitrary connected ordering $>$ on $\allalts$ (a transitive, asymmetric and connected relation, connected meaning that $\forall a ≠ b \in \allalts:  [a > b \lor b > a]$), used to break indifference. Define $a \pst b$ iff $[a \peq b \land (b \npeq a \lor a > b)]$.

To prove. 1) $\peq$ is transitive. 2) If $a \pst b$, then $b$ is never chosen by $i$ when $a$ is present: $\forall \phi_{B_k} \in \iPhi: $ no $B_k$ contain $b$.

\subsection{Preferences}
Let $\allalts$ be a set of alternatives. Define $\Phi_\allalts = \{\phi_{a \pst b} \suchthat a ≠ b \in \allalts\} \cup \{\phi_{a \peq b} \suchthat a ≠ b \in \allalts\}$, with $\phi_{a \pst b}$ intuitively representing a strict preference of $i$ for $a$ against $b$, and $\phi_{a \peq b}$ intuitively representing a weak preference of $i$ for $a$ against $b$. Define $¬\phi_{a \pst b} = \phi_{b \peq a}$ and $¬\phi_{a \peq b} = \phi_{b \pst a}$.

A normative theory of preferences about $\allalts$ is a theory about $\Phi_\allalts$: a tuple $(\allprops, ¬, \allargs, I, (\ileadsto)_{i \in I})$, with $\allprops$ defined as $\allprops_\allalts$, $¬$ defined as above.

Define $\peqi$, the deliberated preference of $i$, as $a \peqi b$ iff $a = b \lor \phi_{a \peq b} \in \iPhi$. 

Given an empirical theory for a normative theory of preferences, say that it claims a complete preference iff $\forall a ≠ b \in \allalts: \phi_{a \peq b} \in \gPhi \lor \phi_{b \peq a} \in \gPhi$. Similarly, it claims a transitive preference iff $\forall a, b, c \in \allalts: \phi_{a \peq b} \in \gPhi \land \phi_{b \peq c} \in \gPhi ⇒ \phi_{a \peq c} \in \gPhi$. Observe that establishing whether an empirical theory claims a complete or transitive preference can be determined a priori (without empirical observations).

Given a bundle $B \subseteq \allalts$ and $i \in I$, $b \in B$ is a deliberated preferred choice for $i$ among $B$ iff $\forall b' \in B: b \peqi b'$.

The next result follows from these definitions and from classical results about choice functions and complete, transitive preferences. \TODO{Be more precise.}
 \begin{theorem}
 	If an empirical theory for a normatively adequate normative theory of preference claims a complete and transitive preference and is valid, then, given any bundle $\emptyset ≠ B \subseteq \allalts$, some $b \in B$ is a deliberated preferred choice for $i$ among $B$.
 \end{theorem}
 
Given $i$ and $\phi_{a > b}$, $\hist \ileadsto \phi$ iff $i$, after having been presented $\ar$, is asked to choose among $a$, $b$, or declare she is indifferent if permitted; and $i$ chooses $a$. And $\hist \ileadsto \phi_{a ≥ b}$ iff $i$ does not choose $b$ (thus chooses $a$ or declare she is indifferent if permitted). Note that $\hist \ileadsto \phi ⇔ \hist \nileadsto ¬\phi$.

\subsection{Satisfaction}
Consider a normative theory of preferences about a given set of alternatives $\allalts$. Let $\gamma$ be the following empirical theory for that normative theory.

Define $C$ a set of criteria, functions $c: \allalts → \R$ measuring the performances of the alternatives according to some point of view. Let $t: \R^C$ be a set of thresholds, thus, consisting of one real threshold $t_c \in \R$ per criterion. Say that $a \in \allalts$ is satisficing iff $\forall c \in C: c(a) ≥ t_c$.

The theory $\gamma$ says that $i$ is a satisficer using some defined criteria $C$ and thresholds $t = (t_c)_{c \in C}$. The set $\gPhi$ is defined as $\set{\phi_{a > b} \suchthat a \text{ is satisficing and $b$ is not}}$. Given $a, b \in \allalts$, with $a$ satisficing and $b$ not, define $\ar_{a > b}$ as the text: “$a$ has performance $c_1(a)$ on criterion $c_1$, and it is reasonable to consider that having at least $t_{c_1}$ is satisfying enough from the point of view captured by $c_1$, (… same phrase for the other criteria…), therefore, it is a fully satisfactory alternative; whereas $b$ is not satisfactory on criteria (… enumeration of the criteria such that $c(b) < t_{c}$).” 
Define ${\gleadsto} = \set{\ar_{a > b}, \phi_{a > b}) \suchthat a \text{ is satisficing and $b$ is not}}$.
Define $\alldargs = \gbeats = \dbeats = {+} = \emptyset$: the theory does not consider any counter-argument as potentially damaging. 
This defines the theory $(\gleadsto, \gbeats, \alldargs, \dbeats, +)$.

Observe that $\gamma$ does not claim a complete preference, but, whenever there is exactly one satisficing alternative in $\allalts$, it still determines a deliberated preferred choice for $i$ among $\allalts$.

\subsection{LS}
%Idea: argue for P > $.
%4 ways:
%- choose
%- rate attractivity: b1 → +; b2 → ++ thus b2 > b1
%- bids to buy (B-bid): I’d pay max $4 to buy and play b1; I’d pay max $6 to play b2 thus b2 > b1
%- bids to sell (S-bid): I’d sell b1 for at least $3; I’d sell b2 for at least $5 thus b2 > b1.
%
%First two: focus on probs. Last two: focus on amounts.
%
%Compare P = (.99, +$4; .01, −$0.1) VS $ = (.33, +$16; .67, −$2). Typically, choose P and bid more for $.
%E[P] ≈ $4; E[$] ≈ 5+1/3 − 4/3 = $4.
%
%XP1. Choice VS S-bid. No gambling. 6 P bets, 7 $ bets. 12 pairs, of which 6 presented in the table: (P, $), choose one, and indicate the strength of your preference (slight to very strong). Then 19 bets using S-big, 6 practice (other bets) then 13 were analyzed.
%
%E[P] and E[$] (of the example above) are similar, but bids “greatly exceeding the expected value of $3.94 are common”.
%
%XP2. Choice VS B-bid.
%
%XP3. With effective gambling and iterated thinking to thing about their preference.
Define $P = (.99, +\$4; .01, −\$0.1)$ and $D = (.33, +\$16; .67, −\$2)$.

Theory claims that all $i$’s prefer deliberately $P$ to $D$, in the sense of betting at least as much for.

Define $\phi$ as: $i$ deliberately prefers $P$ to $D$, in the sense of bidding at least as much for $P$ than for $D$ after deliberation. Define $¬\phi$ similarly, but bidding more for $D$ than for $P$. 

Define $\ileadstost$ as follows. Given $\hist = (…, \ar)$, $\hist \ileadstost \phi$ iff, in a situation of $i$ having been exposed to the prefix …, saying: “Considering the argument $\ar$, would you please indicate your maximum selling price for $P$, and for $D$” leads $i$ to act accordig to $\phi$. Note that $\hist \ileadstost \phi$ iff $\hist \nileadstost ¬\phi$.

Thus, the normative axiom is satisfied iff, whenever there is a argument that, facing any counter-argument, leads $i$ to bid at least as much for $P$ than for $D$, we are ready to accept that $i$ deliberately prefers $P$ to $D$, in the bidding sense.

Define $\ar_0$ as arguing that the bid for both P and D should equal their expected revenue, namely, $.99 × (\$4) − 0.01 × \$0.1$ or etc. “Imagine you play this bet 100 times. On average, you will win 99 times and lose 1 time, which means gaining 99 × \$4 = \$396 and losing 1 × \$0.10 = \$0.10, hence a total net gain of \$395.90. Thus, your average net gain for the $P$ bet would be \$3.9590. Etc.”

Define $\ar_1$ as arguing that the utility of money should matter, not the revenue, and arguing that a loss hurts more than a gain. Define $\ar_2$ as arguing that the goal is to maximise average revenue, and thus, losses should count equally to gains. Define $\ar'_1$ as arguing by considering only what is to be gained and not the probabilities, and thus, arguing that $D$ is worth much more. Define $\ar'_2$ as arguing that also the probabilities should be taken into account.
Define $+$ as the simple concatenation of its arguments.

Then, define $\gbeats = \set{(\ar_1, \ar_0), (\ar'_1, \ar_0), (\ar'_1, \ar_2), (\ar_1, \ar'_2)}$, $\dbeats = \set{(\ar_2, \ar_1), (\ar'_2, \ar'_1)}$.

\TODO{Consider using only $\phi$ being sure, and dropping everything related to possibility.}

\section{Conclusion}
Back To Bentham? Explorations Of Experienced Utility, Kahneman Wakker, Sarin, propose to measure hedonimeter total for normative.

\citet[p.\ 100]{raiffa_back_1985} (about SEU): “Give therapy to deviators”.
\citet[p.\ 108]{raiffa_back_1985}: “In experiments with my students, my ingenious protestations did not always prove to be ingenious enough.
Yes, they were dazzled (and perhaps confused) but many remained unconvinced; if the chips were down, they would behave just as before.”

\section{Next}
Define a way of saying “we never need this argument”: either the model can prove phi without s0, or if it can’t, then phi does not hold, because I can play s1 against it. 

What about a model for only one situation? Then we need to accept that $q$ can’t be falsified, and accept that it defines the individual (or just assume without proof that it does not).

Assume $i$, given $\ar$ and $\ar'$, picks repeatedly (in various circumstances) nonempty subsets of $\set{\prop, ¬\prop}$. Define $(\ar', \ar) \ileadstoe (\prop, \text{sure})$ iff she sometimes chooses $\set{\prop}$ and $(\ar', \ar) \ileadstoe (\prop, \text{poss})$ iff she sometimes chooses $\set{\prop}$ or $\set{\prop, ¬\prop}$. Then, $(\ar', \ar) \ileadstost (\prop, \text{sure})$ iff she always chooses $\set{\prop}$ and $(\ar', \ar) \ileadstost (\prop, \text{poss})$ iff she always chooses either $\set{\prop}$ or $\set{\prop, ¬\prop}$.

Discuss variant with an empirical claim: $\hist \ileadsto \phi ⇒ \hist \nileadsto ¬\phi$; and change \cref{ax:norm}: $[\forall \hist, \hist \ileadsto \propposs] ⇒ \propposs \in \iPhi$, … This is essentially what will be done in the following, I suppose.

If we do not have $\propsure$ implies $\propposs$, then we can have $\propsure$ and $\notpropsure$ both accepted. But all propositions of the normative analysis would still hold, it seems. We could have a property that says: $\propsure \in \iPhi ⇒ \propposs \in \iPhi \land \notpropsure \notin \iPhi \land \notpropposs \notin \iPhi$.

Show that if the claim of the model is of a different form, then the model is not falsifiable, as suggested here below for some form of claims.

Extend the definition of additive single-answer theories to non-additive single-answer theories.

An additive theory need not have freedom for defining planned attacks on additive terms: this should be a consequence on the planned attacks on defenses and on initial arguments only. This makes the theory easier to define, as only attacks on partial arguments are to be defined.

General theory says that $\ar_1$ is replaceable (suspects that it’s good for some people): $(\ar_1, \ar_0) \ileadstoe ¬\phi ⇒ (\ar'_1, \ar_0) \ileadstoe ¬\phi$. (This will fail if some $i$ thinks $\ar'_1$ is bad arg because of $\ar_0$ but $\ar_1$ is bad because of $\ar_2$.) It may say that $\ar_1$ is bad. General theory prepares potentially good arg $\ar_0$ and claims that if $\ar_0 \ileadstoe \phi$, then $\ar_0 \ileadstost \phi$ (this is Answerability). So that the model only has to prove the weaker existential claim.

It is important to remark that when the query protocol does not admit observation of “both”, thus, when $\ileadstost$ does not distinguish sure from possible, it does \emph{not} follow that $i$ is sure of either $\prop$ or $¬\prop$ for every pair of proposition: the “undecided” case will hold whenever $i$ considers no argument as sufficiently strong to determine a decisive preference (assuming that in that case, $i$ at least sometimes alternates his choice). Admittedly, it might also happen that $i$ is really indifferent between $\prop$ and $¬\prop$ but still systematically chooses, say, $\prop$ over $¬\prop$, in which case it will be (somewhat) erroneously concluded that $\prop \in \iPhi$. This is the price to pay for using protocols that do not allow to observe indifference. Observe however that (depending on the application) this error might be considered not harmful, as the obtained model will be faithful to $i$’s behavior after deliberation.

\subsection{Example model}
\begin{itemize}
	\item $\ar_0 = \text{“Veg bec of ethics and health”}$
	\item $\ar_1 = \text{“Ethics is not defined”}$
	\item $\ar_2 = \text{“It is since Aristotle at least”}$
	\item $\ar_1' = \text{“Health isn’t important”}$
	\item $\ar_2' = \text{“Ask my grandmother”}$
	\item $\ar_1 \gbeats \ar_0$, $\ar_1' \gbeats \ar_0$
	\item $\ar_2 \gbeats \ar_1$, $\ar_2' \gbeats \ar_1'$
	\item $\ar_0 + \ar_2 \gbeats \ar_1$
	\item $\ar_1' \gbeats \ar_0 + \ar_2$
	\item $\ar_0 + \ar_2' \gbeats \ar_1'$
	\item $\ar_1 \gbeats \ar_0 + \ar_2'$
	\item $\ar_0 + \ar_2 + \ar_2' \gbeats \ar_1$
	\item $\ar_0 + \ar_2 + \ar_2' \gbeats \ar_1'$
\end{itemize}

\subsection{Interpreting Answerability}
At a given stage in operationalizing the protocol, define $f(\ar_0)$ as the arguments that are considered as having a chance to attack $\ar_0$ (without backtracking). Thus, this will not include $\ar_1$ if indeed $\ar_2$ is considered by $i$ as a satisfying answer to $\ar_1$. We can interpret Answerability in the following ways. Strong answ: $\exists \ar_1 \in f(\ar_0) \suchthat \ar_1 \gbeats \ar_0$. Weak answ: $f(\ar_0) ≠ \emptyset ⇒ \exists \ar_1' \suchthat \ar_1' \gbeats \ar_0$. Or suppress Answ and replace Op. val. by Strong op. val.: $\forall \phi \in \gPhi, \ar_0 \in \gargs[\phi], \ar_1 \in \allargs: (\ar_1, \ar_0) \ileadstoe ¬\phi ⇒ \exists \ar_2 \in {\gbeatsinv}(\ar_1)$.

A model can be allowed to try only once each answer. If $\ar_2 \gbeats \ar_1 \land \ar_1 \in f(\ar_0 + \ar_2)$, the model has lost.

Thus, four variants, given $\ar_2 \gbeats \ar_1$.
\begin{itemize}
	\item Strong answ with single try: $\ar_1 \notin f(\ar_0 + \ar_2) \land \forall \ar_1' \in f(\ar_0 + \ar_2): \ar_1' \gbeats \ar_0 + \ar_2$.
	\item Strong answ with multiple tries: $\forall \ar_1' \in f(\ar_0 + \ar_2): \ar_1' \gbeats \ar_0 + \ar_2$. (Example: $\ar_1 = \text{“for ethical reasons”}$, and $\eta$ knows two c-a to this, from different angles.) 	\item Weak answ with single try: $\ar_1 \notin f(\ar_0 + \ar_2) \land [f(\ar_0 + \ar_2) ≠ \emptyset ⇒ \exists \ar_1' \suchthat \ar_1' \gbeats \ar_0]$.
	\item Weak answ with multiple tries: $f(\ar_0 + \ar_2) ≠ \emptyset ⇒ \exists \ar_1' \suchthat \ar_1' \gbeats \ar_0 + \ar_2$.
\end{itemize}

Goal: leave addition free so as to admit multiple tries, and prove that Strong answ with multiple tries is the right one for a general approach: suitable restriction to the model permits single try (with same conditions); and weak answ actually does not permit more models.

When $f$ is available and is single-valued, question: should we use Strong or weak answ? Right answer: use Strong answ. It is stricly more general. A model that wants weak answ can define its internal attacks suitable (to be proven); and a model that wants to claim more can.

When $f$ is not available, only remains: single try (just give a decisive argument) or multiple tries (try more and more arguments, blindly).

Suppose that a model also defines $f_\eta$, meaning: $f_\eta(\ar_1) = \ar_2$ the argument the model plays against $\ar_1$. Thus $f_\eta$, like $\gbeats$, is a binary relation over $\allargs$, but $f_\eta$ in supplement returns a single argument or none. Also, $\ar_2 = f(\ar_1) ⇒ \ar_2 \gbeats \ar_1$. When a reason $\ar_1$ for rejecting $\ar_0$ is not known, $f_\eta(\ar_0)$ instead gives the argument to be played next. Note that $f_\eta$ is only useful when (in the known c-a variant) for some $\ar_1$ there are multiple candidates $\ar_2$ or when (in the unknown c-a variant) for some $\ar_0$ here are multiple candidates $\ar_2$. In fact, this is more complicated: $f_\eta(\ar_1)$ can give first $\ar_2$, then, if that is not enough, $\ar_4$ (thus it’s rather $f_\eta(\ar_1, \ar_0)$).

Under Strong anws, we ask that $f(\ar_0) \cap {\gbeatsinv}(\ar_0) ≠ \emptyset$, and (inevitably) that $f_\eta(\ar_1) ≠ \emptyset$. Under weak answ, we do not expect anticipation and only demand that $f_\eta(\ar_1) ≠ \emptyset$. But this does not allow supplementary models. TODO prove this.

\subsection{Transitive, complete preference theories}
Given a set of alternatives $\allalts$. Define $\allprops_\allalts = \set{\ppst, a ≠ b \in \allalts} \cup \set{\ppstba, a ≠ b \in \allalts}$. Define $¬_\allalts$ such that $\forall a, b \in \allalts: ¬_\allalts\ppst = \ppstba, ¬_\allalts\ppstba = \ppst$. Define $\Phi_\allalts = \props_\allalts × \set{\mathit{sure}, \mathit{poss}}$.

Let $I$ represent a given set of individuals; $\allalts$ a set of products; 
and $\allargs$ the set of all strings.
Given $i \in I$, define $\ileadstost$ as follows. 
Given $\hist \in \allhist$ and $\phi \in \Phi_\allalts$, with $\phi \in \set{\ppst} × \set{\mathit{sure}, \mathit{poss}}$ for some $a ≠ b \in \allalts$, the individual is presented the arguments in $\hist$, and has then to choose a product from the set $\set{a, b}$. Define $\hist \ileadstost \phi$ iff $i$ chooses $a$. \commentOC{This is inadequate: if $i$ picks $b$ from $\set{a, b, c}$, the current definition considers it compatible with $\ppstsure \in \iPhi$, thus, $\ppst$ does not have a meaningful content. Should rather define $\ppstsure$ so that $a$ is picked rather than $b$ whatever the choice set. This requires to generalize $\ileadstost$.}

Any normative theory of the form $(\allprops_\allalts, ¬_\allalts, \allargs, I, (\ileadsto)_{i \in I})$, for some sets $\allargs$ and $I$, is called a normative theory about $\allalts$.

Given a set $\domc \subseteq \powersetz{\allalts}$, where $\powersetz{\allalts}$ designates the set of subsets of $\allalts$ without the empty set, a function $c: \domc → \allalts$ is a choice function iff $\forall B \in \domc: c(B) \subseteq B$.

Given a normative theory about $\allalts$, an individual $i \in I$, a set $\domc \subseteq \powersetz{\allalts}$ and a choice function $c$, say that $c$ represents a \emph{deliberated choice function} for $i$ iff $\forall B \in \domc: c(B) \subseteq \set{a \in B \suchthat \forall b \in B: \ppstsure[b][a] \notin \iPhi}$.

A normative theory about $\allalts$ is transitive iff $\forall a ≠ b ≠ c ≠ a \in \allalts: [\exists \ar_{a \pst b} \in \allargs \suchthat \ar_{a \pst b} \ileadstost \ppstsure] \land [\exists \ar_{b \pst c} \in \allargs \suchthat \ar_{b \pst c} \ileadstost \ppstsure[b][c]] ⇒ [\exists \ar_{a \pst c} \in \allargs \suchthat \ar_{a \pst c} \ileadstost \ppstsure[a][c]]$. 

A set of propositions $P \subseteq \Phi_\allalts$ is transitive iff $\forall a ≠ b ≠ c ≠ a \in \allalts: \ppstsure \in P \land \ppstsure[b][c] \in P ⇒ \ppstsure[a][c] \in P$.

A transitive empirical preference theory about $\allalts$ is an empirical theory for a normative theory about $\allalts$ such that $\gPhi$ is transitive.

Associate to such a theory a preference ordering $\succ$ on $\allalts$ as follows: $a \succ b$ iff $\ppstsure \in \gPhi$.

The deliberated preference of $i$ under a normative theory 
Then, observe $a$ incomp to $z$ using a good argument for $a > z$ and one for $z > a$. Observe also $b$ incomp to $z$. And observe $a > b$. This proves incompleteness as $z$ indifferent to the rest is impossible. BUT this requires to observe possibility.

\subsubsection{Transitive arguments}
Given a normative theory and a function $\tau: \allargs × \allargs → \allargs$, say that $\tau$ is a transitive argument builder iff $\forall a ≠ b ≠ c ≠ a \in \allalts, \ar_{a \pst b}, \ar_{b \pst c} \in \allargs \suchthat \ar_{a \pst b} \ileadstost \ppstsure \land \ar_{b \pst c} \ileadstost \ppstsure[b][c]: \tau(\ar_{a \pst b}, \ar_{b \pst c}) \ileadstost \ppstsure[a][c]$.

Observe that if a normative theory admits a transitive argument builder, then it is transitive.

TODO It is possible to determine whether a given function $\tau$ is a transitive argument builder empiricially. But not to prove empirically that a normative theory is not transitive.

\section{Decision situation}
\NewDocumentCommand{\ileadstoprop}{}{⇝^\prop}
\NewDocumentCommand{\ileadstonprop}{}{⇝^{¬\prop}}
\NewDocumentCommand{\ileadstosts}{}{⇝_*}
\NewDocumentCommand{\ileadstos}{}{\overset{s}{⇝_i}}
\NewDocumentCommand{\ileadstostw}{O{}}{⇝^\mathit{w}_\forall}
\NewDocumentCommand{\ileadstostp}{O{}}{⇝^\mathit{pos}_\forall}
\NewDocumentCommand{\nileadstosts}{}{\not⇝_*}
\NewDocumentCommand{\Ti}{}{T_i}
\NewDocumentCommand{\ibeatse}{O{}}{⊳^{#1}_\exists}
\NewDocumentCommand \ibeatseinv { o }{
	\IfValueTF{#1}{%
		{⊳_\exists^{#1}}^{-1}%
	}{%
		⊳_\exists^{-1}%
	}%
}
\NewDocumentCommand{\mPhi}{}{\Phi_\gamma}
\NewDocumentCommand{\argsd}{}{S^\mathit{d}}
\NewDocumentCommand{\ileadstoall}{O{}}{⇝^{#1}_\forall}
\NewDocumentCommand{\nileadstoall}{O{}}{\not⇝^{#1}_\forall}

The object of study in a given decision situation is the deliberated preferences of a given individual $i$. 
A decision situation concerns a topic $\allprops$, given a set of arguments $\allargs$, given a certain query protocol, and over a certain time frame. The relation $\ileadstoall$, to be defined shortly, models the reaction of $i$ to arguments about the topic. The decision situation will be defined as a triple $(\allprops, \allargs, \ileadstoall)$ satisfying some conditions to be defined after having presented those three fundamental elements.

The decision situation relates to a topic denoted by $\allprops$, which is a set of propositions $\prop \in \allprops$ about which we are interested of knowing the deliberated preferences of $i$. Propositions are not described further, but are supposedly understandable by $i$ (for example, they could be sentences in some natural language). Alternatively, in the case of an analyst helping $i$ to make a decision, $\allprops$ is a set of propositions about which $i$ is interested to know his own deliberated preferences. I assume $\allprops$ is closed under negation and introduce a symbol $¬$ for negating a proposition: if $\prop \in \allprops$, then $¬\prop \in \allprops$, and $¬(¬\prop) = \prop$. 

With the decision situation comes also a set of all arguments $\allargs$. This set contains all the arguments that may possibly be considered relevant by $i$ for his decision problem, and possibly more.
Importantly, the notion of deliberated preferences does not require to constrain a priori $\allargs$ to some set of arguments that would fit some precise notion of relevancy, coherence, or even well-formedness. This permits to avoid introducing inadequate normative principles into the deliberated preferences: only the individual $i$ is then considered legitimate to dictate what is a relevant argument. Under that view, $\allargs$ may contain anything that can possibly be considered as an argument by anyone, under the widest possible conception of an argument. The notion of deliberated judgment can however also be applied when considering a restricted set of arguments, for example, the arguments that have been put forward by some specific set of experts when talking about the topic $\allprops$. 

The set $\allargs$ may be infinite or very large: as we will see, thanks to our falsificationist approach, there is no need to be able to explore it entirely. The set of arguments contains at least the empty argument, denoted by $\zar$.

\begin{example}
	Consider a decision problem where a set of alternatives $\allalts$ is given, containing food products among which $i$ would like to form a deliberated preference (taking into account the effects of food on health, price, pleasure, morality issues related to the production process, and so on). The topic could be defined as $\allprops$ = $\set{\prop_\alt, ¬\prop_\alt, \forall \alt \in \allalts}$, where $\prop_\alt$ is the proposition according to which $\alt$ is one of the best alternatives among $\allalts$ (there is no strictly better alternatives among $\allalts$, from $i$’s point of view), and $¬\prop_\alt$ is the proposition according to which $\alt$ is not one of the best alternatives among $\allalts$ (some $\alt' \in \allalts$ is a strictly better alternative, from $i$’s point of view). 
	The set $\allargs$ is defined as the set of all strings, because in this example it is considered acceptable to restrict arguments to those that can take a textual form.
\end{example}

Define $\Phi = \allprops × \set{\text{possible}, \text{sure}}$. Elements in $\Phi$ will also be called propositions, which should create no ambiguity. 
The relation $\ileadstoall \subseteq (\allargs × \allargs) × \Phi$, pronounced “always leads to”, has the following semantics.
Whenever $(\ar, \ar') \ileadstoall (\prop, \text{sure})$, $i$ always considers that $\prop$ is sure (meaning that $¬\prop$ is excluded), when presented with these two arguments, and whenever $(\ar, \ar') \ileadstoall (\prop, \text{possible})$, $i$ always considers that $\prop$ is “at least” possible (meaning that $¬\prop$ may be possible as well), when presented with these two arguments. In both cases, the term “always” means that the consideration is stable over time, including after having presented other arguments to $i$. 
Two query protocols presented below will illustrate how these semantics might be satisfied. 
However, I voluntarily leave the querying protocol not precisely specified in general, as multiple reasonable choices are possible. The results of this article depend only on assumptions about $\ileadstoall$. 

Define the negation operator $¬$ over $\Phi$ as: $¬(\prop, \text{possible}) = (¬\prop, \text{sure})$ and $¬(\prop, \text{sure}) = (¬\prop, \text{possible})$. It follows that $¬¬\phi = \phi$. Write $(\ar, \ar') \nileadstoall \phi$ for $¬[(\ar, \ar') \ileadstoall \phi]$ and define $(\ar, \ar') \ileadstoe \phi$ as equivalent to $(\ar, \ar') \nileadstoall ¬\phi$. 
The relation $\ileadstoe$ is pronounced “sometimes leads to”.
A proposition $\phi$ is about $\prop$ iff $\phi \in \{\prop, ¬\prop\} × \{\text{possible}, \text{sure}\}$. 

The pair $(\ar, \ar')$ is to be understood as unordered (thus $(\ar, \ar') \ileadstoall \prop ⇔ (\ar', \ar) \ileadstoall \prop$).

\begin{definition}[Decision situation]
	A decision situation is a topic $\allprops$ closed under negation, a set of arguments $\allargs$, an “always leads to” relation $\ileadstoall \subseteq (\allargs × \allargs) × \Phi$ satisfying (A1) and (A2) and using unordered pairs of arguments, $\Phi$ being the set of propositions $\allprops × \set{\text{sure}, \text{possible}}$.
\end{definition}

\subsection{Query protocols}
As vNM aptly summarize, “It is clear that every measurement – or rather every claim of measurability – must ultimately be based on some immediate sensation, which possibly cannot and certainly need not be analyzed any futher.” The fundamental element here is the reaction of $i$ to arguments, which will serve as the informational basis to define $\ileadstoall$. Given a proposition $\prop$ and two arguments $\ar, \ar' \in \allargs$, a query consists in presenting both arguments to $i$ and observing which proposition $i$ considers valid in her current state of mind, thus using both arguments and possibly other arguments she has in mind: is it that $\prop$ holds, that $¬\prop$ holds, or that both are possible? The third possibility allows for the case where no argument appears more decisive than the other one. 

It is also necessary to capture the evolution of the position of $i$ towards arguments over time: a repeated query using a given pair of arguments may yield different answers, for example because the repeated question has been interleaved with another query containing other arguments (which $i$ may still have in mind when answering the second repetition), or for any other known or unknown reason. The important information for us is the \emph{set} of answers that $i$ could give to a query involving two given arguments and a given topic, among the set $\{\prop, ¬\prop, \text{both}\}$. The set designates all possible answers over time and over different order of asking the queries.
Hence, given $\prop \in \allprops$, define a relation $\ileadstoprop \subseteq (\allargs × \allargs) × \powersetz{\{\prop, ¬\prop, \text{both}\}}$, where $\powersetz{B}$ designates the subsets of $B$ excluding the emptyset, with the semantics that $(\ar, \ar') \ileadstoprop C$, with $C \subseteq \set{\prop, ¬\prop, \text{both}}$, iff $\forall c \in C$, it is observable at least once that $i$, when presented with $(\ar, \ar')$, considers $c$ valid in her current state of mind. The querying protocol is supposedly made so that for any $\prop \in \allprops$, $\ileadstoprop = \ileadstonprop$.

This information is only indirectly observable: for example, if $(\ar, \ar') \ileadstoprop \{\prop\}$, a querier will never know more than $(\ar, \ar') \ileadstoprop C$ with $\prop \in C \subseteq \set{\prop, ¬\prop, \text{both}}$. 
Also, it may be only possible to test one ordering of the queries (assuming that $i$ forgets over time might permit to test different orderings, but this assumption might be unrealistic). In some cases, it may even be only possible to test a single query with a given individual.
This limitation of our observation will not be a problem for the results of this article. Intuitively, this bears on the fact that, first, theories apply to multiple individuals, and second, if a theory claims that $(\ar, \ar') \ileadstoprop \{\prop\}$, then either this claim is correct, or it is falsifiable, in the sense that there is a possibility of observing a contradiction to its claim.

To be more concrete, one possible choice of a query protocol, when arguments and propositions are strings, is to read both arguments to $i$, together with $\prop$, and ask $i$ to verbally report the choice he opts for among the three possibilities. Another is to present both arguments and ask $i$ to pick an item among some choice set and make the choice correspond to a validation of $\prop$ or $¬\prop$, possibly even letting the choice engage $i$, for example telling $i$ that she may keep the object of choice (this is illustrated in \cref{ex:pick}). This second possibility is especially interesting as it makes the protocol partially observable in the restricted sense usually appreciated in revealed preferences approaches. Note that depending on the protocol, the choice “both” is not necessarily observable (more about this below).
%In that case, the semantics of $\prop$ and $¬\prop$ relatedly change: instead of meaning that $\prop$ is definitely the 
%Indeed, \citet{tversky_intransitivity_1969} note that individuals “are not perfectly consistent in their choices. When faced with repeated choices between x and y, people often choose x in some instances and y in others. Furthermore, such inconsistencies are observed even in the absence of systematic changes in the decision maker’s taste which might be due to learning or sequential effects. It seems, therefore, that the observed inconsistencies reflect inherent variability or momentary fluctuation in the evaluative process.” This is an accepted fact of experimental psychology \citep{luce_utility_2000}. However, it is reasonable to suspect that fluctuations occur less often, if at all, for some choices. For example, facing a choice between tea or coffee, I would virtually never pick tea (all other things being equal). We are interested in capturing precisely those kind of preferences.

One can also use $\zar$ as one or both of the arguments, in which case the protocol queries the preference of $i$ given only one or given no argument.

\begin{example}[cont.]
	\label{ex:pick}
	Continuing the example, define the querying protocol as, given $\prop_\alt$, presenting two arguments to $i$ and let $i$ choose an item among $\allalts$. $i$ may keep the chosen item. 
	If $i$ chooses $\alt$, it is considered that $i$ has validated $\prop_\alt$, otherwise, $¬\prop_\alt$. The option “both” is not observable.
%	Queries are separated by one week at least, which (as supposed in this example) is enough to ensure that no effect of pleasure for variability enter into play. 
\end{example}

Define $\ileadstoall$ from $\ileadstoprop$ as follows, for any $\prop \in \allprops$: $(\ar, \ar') \ileadstoall (\prop, \text{possible}) ⇔ (\ar, \ar') \ileadstoprop C$ with $C \subseteq \set{\prop, \text{both}}$; and $(\ar, \ar') \ileadstoall (\prop, \text{sure}) ⇔ (\ar, \ar') \ileadstoprop \set{\prop}$. 

Note that (A1) and (A2) hold.

When “both” is not observable, $\ileadstoall$ is defined as: $(\ar, \ar') \ileadstoall (\prop, \text{possible}) ⇔ (\ar, \ar') \ileadstoall (\prop, \text{sure}) ⇔ (\ar, \ar') \ileadstoprop \set{\prop}$. (A1) and (A2) still hold. We say in this case that $\ileadstoall$ does not distinguish sure from possible. Thus, elements $\phi \in \Phi$ may be considered as simply equal to $\prop$ or $¬\prop$, as $(\prop, \text{sure})$ is treated exactly as $(\prop, \text{possible})$.

\subsection{Example}
\begin{example}[(cont.)]
	Returning to the previous example, a theory $T$ could claim that, for all individuals that belong to some given socio-economic situation and that have some given degree (described by the theory in sufficient details that it is possible to determine precisely to which kind of individuals it applies), $(\prop_a, \text{sure})$ is in the deliberated preferences of the individuals, for a given alternative $\alt$ also described by $T$. 
The theory also suggests some argument $\ar$, a text that (according to $T$) will convince any individual (fitting the description) that $\prop_\alt$ holds, thus, that $\alt$ is a “good” food product for him. 
The theory can be put to the test by picking an individual fitting the description, presenting the argument together with any other argument (for example, proposed by another theory), and observing whether $i$ is convinced.
$T$ resists to such a falsification test if $i$ is convinced. According to a previous theorem, $T$ tells the truth if it would resist to any possible falsification test. (This is impossible to definitely make sure of.)
\end{example}
This example illustrates a difference between this proposal and the classical revealed approach: it could be that $T$ tells the truth even though its claim does not correspond to the revealed preference of some individuals, thus that, given no argument, some individuals would consider $a$ as dominated by another alternative in the set considered. It also illustrates a difference between this proposal and persuasion: the goal of $T$ is not merely to convince $i$ (possibly by using its lack of knowledge of any counter-argument to what $T$ says), but to resist any falsification attempt. Thus, $T$ should be confronted to arguments coming from as varied perspectives as possible, in order to give confidence that it tells the truth.

\section{Empirical theories}
\subsection{Simple to complex}
From an additive single-answer empirical theory, we can define a function: $\forall \ar_1 \gbeats \ar_0: f(\ar_1, \ar_0) = \ar_0 + {\gbeatsinv}(\ar_1) = \set{\ar_0 + \ard \suchthat \ard \in {\gbeatsinv}(\ar_1)}$, and $f(\ar_1, \ar_0) = \emptyset$ otherwise. Thus, $f = \set{((\ar_1, \ar_0), \ar_2) \suchthat \exists \ard \in \allargs \suchthat \ard \gbeats \ar_1 \gbeats \ar_0 \land \ar_2 = \ar_0 + \ard}$.
Also, from the previous definition, $g(\ar_0)(\ar_0 + \argsd_2) = dom(f^{-1}(\ar_0 + \argsd_2) \cap (\allargs × \set{\ar_0})) = {\gbeats}(\argsd_2) \cap {\gbeatsinv}(\ar_0)$ because $(\ar_1, \ar_0) \in f^{-1}(\ar_0 + \argsd_2) ⇔ \exists \ar_2 \in \ar_0 + \argsd_2 \suchthat \ar_2 \in f(\ar_1, \ar_0) ⇔ \ar_1 \gbeats \ar_0 \land \exists \ard \in \argsd_2 \suchthat \ard \gbeats \ar_1$.
\commentOC{Could be false.}

Note that $f^{-1}(\ar_0 + \ar_2) ≠ {\gbeats}(\ar_2) × \set{\ar_0}$: if ${\gbeats} = \set{(\ard, \ar_1), (\ar_1, \ar_0), (\ard[6], \ar_5), (\ar_5, \ar_4)}$, and $\ar_0 + \ard = \ar_4 + \ard[6]$, then $f^{-1}(\ar_0 + \ard) = f^{-1}(\ar_4 + \ard[6]) = \set{(\ar_1, \ar_0), (\ar_5, \ar_4)}$ and ${\gbeats}(\ard) × \set{\ar_0} = \set{(\ar_1, \ar_0)}$.
Also, note that $g(\ar_0)(\ar_0 + \argsd_2) ≠ {\gbeats}(\argsd_2) \cap {\gbeatsinv}(\ar_0)$. Consider ${\gbeats} = \set{(\ard, \ar_1), (\ar_1, \ar_0)}, \ar_2 = \ar_0 + \ard = \ar_0 + \ard$. Then, $(\ar_1, \ar_0) \in f^{-1}(\ar_0 + \ard)$, hence, $\ar_1 \in g(\ar_0)(\ar_0 + \ard)$, but ${\gbeats}(\ard) = \emptyset$.

\subsection{Falsifiability}
\commentOC{I should define $\Box(\hist \ileadsto \phi)$ iff it has been observed that $\hist \ileadsto \phi$. Then, axiom Correct observations (that is self-evident): $\Box(\hist \ileadsto \phi) ⇒ \hist \ileadsto \phi$ (what has been observed is). Supplementary axiom Falsification is enough (that is not self-evident): $\not\Box(\hist \ileadsto \phi) ⇒ \hist \nileadsto \phi$ (what has not been observed is not).}

\commentOC{I want: An empirical theory for a normative theory is something like a normative and falsifiable claim. It is also configurable: it must be possible to guarantee that the falsification be easy (if the claim is false) depending on the kind of claim.}

Given $\hist \in \allhist$ and $i \in I$, the claim $\hist \ileadstoe[i] \phi$ may not be verifiable, and is too specific to be falsifiable. However, given $\hist \in \allhist$, a broader claim is falsifiable: that for a given set of individuals $Q \subseteq I, \forall i \in Q, \hist \ileadstoe[i] \phi$ (provided $Q$ is big enough). Indeed, that claim it is equivalent to $\nexists i \in Q \suchthat \hist \nileadstoe[i] \phi$; exhibiting an $i \in Q$ such that $\hist \nileadstoe[i] \phi$ thus falsifies the broader claim.

Given any $(\allprops, ¬, \allargs, I)$, an \emph{observation protocol} is a binary tree whose non-leaf nodes are among $I × \allhist × \Phi$ and whose two edges out of any non leaf node are labeled by “yes” and “no”; the leaf nodes being simply $\emptyset$. The observation protocol is said to assume memory iff for each non-leaf node $n = (i, \hist, \phi)$ about $i$, any node $n' = (i, {\hist}', \phi')$ about $i$ that is further down the tree continues the history, that is, the sequence ${\hist}'$ starts with $\hist$. Given an observation protocol and observables $(\ileadstoe[i])_{i \in I}$, the \emph{observed path} refers to the path in the tree defined recursively as consisting of the root node and, for each non-leaf node $n = (i, \hist, \phi)$ in the path, the child node along the branch “yes” if $\hist \ileadstoe[i] \phi$ and the child node along the branch “no” otherwise. The \emph{observations} refer to the set of nodes along the observed path.

Given any $(\allprops, ¬, \allargs, I)$, a \emph{normative claim} about $(\allprops, ¬, \allargs, I)$ (or about $\allargs$, when the other defined objects are left implicit) is any first-order logic proposition involving the atoms $[(\hist, \phi) \in {\ileadstoe[i]}], \hist \in \allhist, \phi \in \Phi$ which, together with the axioms, permits to deduce that $\phi \in \iPhi$ for some $i \in I$ and $\phi \in \Phi$. The claim is said to be \emph{about} the given set $\allargs$. An observation protocol, together with observables $(\ileadstoe[i])_{i \in I}$, \emph{falsifies} a normative claim iff the falsity of the claim is deducible from the axioms and the observations. A normative claim is \emph{falsifiable} iff there exists a finite observation protocol such that for any observables $(\ileadstoe[i])_{i \in I} \in (\allhist × \Phi)^I$, if the claim is false given these observables, then it is possible to deduce it from the axioms and the observations given the observables.

\commentOC{Falsifiable may not be the right word. Perhaps “provable“, meaning “tautological or Popper-falsifiable”. “Falsifiable” suggests that if a claim is not falsifiable, then a weaker claim is not falsifiable either, which does not hold. My falsifiability criterion demands that if a claim is false, we can see it; whether Popper’s demands that the claim be possibly false. Perhaps I should demand that we can pick a set of consequences of the claim and show it wrong it at least one possible state of the universe.}

A normative claim is \emph{trivial} iff it is about a singleton $\allargs = \set{\ar}$, and has the form: $\ar \ileadstosts \phi$, for some $i \in I$. A normative claim is minimal iff it has the form: $\exists \ar \in \allargs \suchthat \ar \ileadstosts \phi$, for some $i \in I$ and $\phi \in \Phi$. Note that given a minimal claim, $\card{\allargs} = 1$ is equivalent to the claim being trivial.

\begin{proposition}[Trivial claims are not falsifiable]
	Given any $(\allprops, ¬, \allargs ≠ \emptyset, I)$, no minimal normative claim is falsifiable.
\end{proposition}
\begin{proof}
	Consider any claim about some $i$ and some $\phi$ and any finite observation protocol. We have to show that for some observables, the claim is false, but it is impossible to observe it (meaning, to deduce it from the axioms and observations).
	
	Given $k \in \N$, define $\ileadstosts[k]$ as $\ar^k \ileadstosts \phi$ and $\ar^{k+1} \ileadstosts ¬\phi$ (where $\ar^k$ designates a finite sequence repeating $k$ times $\ar$, for $\ar \in \allargs$). I claim that for some $k$, the observables that include this relation makes the claim false, but not observably so. This is because the observation protocol is finite, thus, only permits to observe some finite number of instances of the relation $\ileadsto$. Suffices to define $k$ as that number.
\end{proof}

There is another reason for non falsifiability, which does not involve the problem of finiteness.
\begin{proposition}[Minimal non-trivial claims are not falsifiable]
	Given any $(\allprops, ¬, \allargs, I), \card{\allargs} ≥ 2$, no minimal normative claim is falsifiable.
\end{proposition}
\begin{proof}
	Define $\ar_1$ as the first element of the sequence $\hist$ of the root node of the observation protocol (if the observation protocol is empty, suffices to define any observables that make the claim false). Define the observables so that $\forall \hist \in \allhist: \ar_2, \hist \ileadsto ¬\phi$. This proves that no argument is decisive, hence $\phi \notin \Box\Phi$, and the claim is false. But the observables may also be defined so as to satisfy the protocol (defined as: when going through the protocol, the observations can’t fail the claim).
\end{proof}

\subsection{Definition}
A general empirical theory for a normative theory $(\allprops, ¬, \allargs, I, (\ileadstoe[i])_{i \in I})$ is a tuple $({\gleadsto}, f)$, where ${\gleadsto} \subseteq \allargs × \Phi$ and $f \subseteq (\allargs × \allargs) × \allargs$ with $f(\allargs × \allargs)$ being finite.

Define $g(\ar_0) \subseteq \allargs × \allargs$ as $g(\ar_0) = \set{(\ar_2, \ar_1) \suchthat \ar_2 \in f(\ar_1, \ar_0)} = \bigcup_{\ar_1 \in \allargs} (f(\ar_1, \ar_0) × \set{\ar_1})$. The set $g(\ar_0)(\allargs)$ contains the planned attacks on $\ar_0$; the set $g(\ar_0)(\ar_2)$ contains the planned attacks on $\ar_0$ for which $\ar_2$ is a planned response; and the set $g(\ar_0)^{-1}(\ar_1) = f(\ar_1, \ar_0)$ contains the planned protections of $\ar_0$ against $\ar_1$. 

Given $\ar_0 \in \allargs$ and $\args_2 \subseteq \allargs$, define the set $\args_1 \subseteq \allargs$ of arguments that $\args_2$ protects $\ar_0$ from as $\args_1 = [f^{-1}(\args_2)]^{-1}(\ar_0) = \set{\ar_1 \in \allargs \suchthat f(\ar_1, \ar_0) \cap \args_2 ≠ \emptyset} = g(\ar_0)(\args_2) = dom(f^{-1}(\args_2) \cap \allargs × \set{\ar_0})$.
 
\subsection{A general condition sufficient for validity and (hopefully) falsifiability}
It should be possible to define this condition as a claim that does not involve $f$, so that it can be done before defining an empirical theory. Then, show that a normative and falsifiable claim must have the form of a convincingness claim.

\begin{remark}
	In fact, the model is tasked with finding a stable point after interrogating $i$. The process of trying counter-arguments $\ar_1, …$ is $q(i)$. When done, the model claims that $\ar_0 + \ar_2 + …$ is decisive. The model also claims that another process of interrogation does not lead to different answers. This is falsifiable.
	When claiming that $(\ar_1, \ar_0) \ileadstoall[i] \phi$, the model really claims that $\nexists p, i' \in q^{-1}(q(i)) \suchthat p, \ar_1, \ar_0 \ileadsto[i'] ¬\phi \lor p, \ar_0, \ar_1 \ileadsto[i'] ¬\phi$, where $i' \in q^{-1}(q(i))$ iff $q(i) = q(i')$.
	
	To say that $\phi \in \iPhi$, need to not depend on choice of $q$, in the following sense: if with $q$, $\phi$ seems stable, but with $q'$, $¬\phi$ seems stable, then $\phi$ is not proved as being in $\iPhi$. But if with $q$, $\phi$ seems stable, but with $q'$, nothing can be shown to be stable, then $\phi$ is proved (temporarily) as being in $\iPhi$. More precisely, after $q'$, the model must be able to stabilize $i$ with its $q$.
\end{remark}

$\ar_0$ $i$-defends $\phi$ when ignoring $S_1$ iff $\forall \hist \in \allhist: [\args_1 \cap \hist = \emptyset] ⇒ (\hist, \ar_0) \ileadstosts \phi$.
$\ar_0$ $i$-defends $\phi$ under protection of $S_2$ iff $\ar_0$ $i$-defends $\phi$ when ignoring $g(\ar_0)(\args_2)$.

\begin{definition}[Convincingness]
	$\forall i \in I, \phi \in \gPhi, \exists k \in 2\N$ and a finite sequence $(\args_j)_{j \in 2\N, j ≤ k} \suchthat \emptyset ≠ \args_0 \subseteq {\gleadstoinv}(\phi)$ and $\forall j \in 2\N, j ≤ k, \forall \ar_j \in S_j: \ar_j$ $i$-defends $\phi$ under protection of $S_{j + 2}$ (defining $S_{k + 2} = \emptyset$) and $S_j \subseteq \cup_{\ar_{j - 2} \in \args_{j - 2}, \ar_{j - 1} \in \allargs} f(\ar_{j - 1}, \ar_{j - 2})$.
\end{definition}
Note that requiring that $\args_0$ be a singleton does not modify the condition, in the sense that any theory satisfying the condition also satisfies its strenghtening. Similarly, requiring that $S_k ≠ \emptyset$ does not modify the condition (because if some $k$ satisfy the condition with $S_k = \emptyset$, then picking $k' = k-2$ with the same sets $S_j$ also satisfy the condition, and because $S_0 ≠ \emptyset$, this recursion must end with a suitable $k$).

\begin{remark}
	The claim must have the form: some structure is adequate for all individuals. Here, the structure is given by $f$. The claim should claim that looking at the attacks planned by $f$ is enough; and looking at the defenses planned by $f$ is enough.

	This may be adequate as if $\ar_0$ seems suitable but some unplanned attacker $\ar_1$ attacks it, then use $\ar_1 \in \hist$ and observe that $¬(\hist, \ar_0) \ileadstoe \phi$ to deny the claim.
	
	Perhaps this could be phrased as follows. Say that $\ar_2$ is replaceable by $\args_2$ iff what $\ar_2$ protects is either protected by $\args_2$ or replaceable (by what?). Say that a set $\args$ is sufficient iff its complement is replaceable by $\args$. Then $g$ claims (among others) that $g(\ar_0)^{-1}(\ar_1)$ is sufficient.
	
	The claim can (hopefully) be given the form: $\forall i \suchthat …, \ar$ decisive. Given $i$, either $\ar_0$ is decisive, or $\ar_1$ attacks it but then $\ar_2$ is decisive, …
\end{remark}

\begin{theorem}
	If an empirical theory is convincing, and its normative theory satisfies the axioms, then it is valid.
\end{theorem}
\begin{proof}
	Consider $i \in I$ and $\phi \in \gPhi$. 
	By hypothesis, $\exists k$ such that some $\ar_k \in \args_k$ defends $\phi$ under protection of $\emptyset$. Therefore, $\ar_k \ileadstosts \phi$. Using \cref{ax:norm}, validity follows.
\end{proof}

\begin{theorem}
	The convincingness claim of any empirical theory is falsifiable: given an ET, there exists a finite obs prot with memory such that for any observables, if the claim is false given these observables, then it is possible to deduce it from the axioms and observations.
\end{theorem}

\hbadness 10000
\bibliography{simple}
\end{document}
