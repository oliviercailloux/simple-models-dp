\RequirePackage[l2tabu, orthodox]{nag}
\documentclass[version=last, pagesize, twoside=off, bibliography=totoc, DIV=calc, fontsize=12pt, a4paper, french, english]{scrartcl}
%Permits to copy eg x ⪰ y ⇔ v(x) ≥ v(y) from PDF to unicode data, and to search. From pdfTeX users manual. See https://tex.stackexchange.com/posts/comments/1203887.
	\input glyphtounicode
	\pdfgentounicode=1
%Latin Modern has more glyphs than Computer Modern, such as diacritical characters. fntguide commands to load the font before fontenc, to prevent default loading of cmr.
	\usepackage{lmodern}
%Encode resulting accented characters correctly in resulting PDF, permits copy from PDF.
	\usepackage[T1]{fontenc}
%UTF8 seems to be the default in recent TeX installations, but not all, see https://tex.stackexchange.com/a/370280.
	\usepackage[utf8]{inputenc}
%Provides \newunicodechar for easy definition of supplementary UTF8 characters such as → or ≤ for use in source code.
	\usepackage{newunicodechar}
%Text Companion fonts, much used together with CM-like fonts. Provides \texteuro and commands for text mode characters such as \textminus, \textrightarrow, \textlbrackdbl.
	\usepackage{textcomp}
%Solves bug in lmodern, https://tex.stackexchange.com/a/261188; probably useful only for unusually big font sizes; and probably better to use exscale instead. Note that the authors of exscale write against this trick.
	%\DeclareFontShape{OMX}{cmex}{m}{n}{
		%<-7.5> cmex7
		%<7.5-8.5> cmex8
		%<8.5-9.5> cmex9
		%<9.5-> cmex10
	%}{}
	%\SetSymbolFont{largesymbols}{normal}{OMX}{cmex}{m}{n}
%More symbols (such as \sum) available in bold version, see https://github.com/latex3/latex2e/issues/71.
	\DeclareFontShape{OMX}{cmex}{bx}{n}{%
	   <->sfixed*cmexb10%
	   }{}
	\SetSymbolFont{largesymbols}{bold}{OMX}{cmex}{bx}{n}
%For small caps also in italics, see https://tex.stackexchange.com/questions/32942/italic-shape-needed-in-small-caps-fonts, https://tex.stackexchange.com/questions/284338/italic-small-caps-not-working.
	\usepackage{slantsc}
	\AtBeginDocument{%
		%“Since nearly no font family will contain real italic small caps variants, the best approach is to substitute them by slanted variants.” -- slantsc doc
		%\DeclareFontShape{T1}{lmr}{m}{scit}{<->ssub*lmr/m/scsl}{}%
		%There’s no bold small caps in Latin Modern, we switch to Computer Modern for bold small caps, see https://tex.stackexchange.com/a/22241
		%\DeclareFontShape{T1}{lmr}{bx}{sc}{<->ssub*cmr/bx/sc}{}%
		%\DeclareFontShape{T1}{lmr}{bx}{scit}{<->ssub*cmr/bx/scsl}{}%
	}
%Warn about missing characters.
	\tracinglostchars=2
%Nicer tables: provides \toprule, \midrule, \bottomrule.
	%\usepackage{booktabs}
%For new column type X which stretches; can be used together with booktabs, see https://tex.stackexchange.com/a/97137. “tabularx modifies the widths of the columns, whereas tabular* modifies the widths of the inter-column spaces.” Loads array.
	%\usepackage{tabularx}
%math-mode version of "l" column type. Requires \usepackage{array}.
	%\usepackage{array}
	%\newcolumntype{L}{>{$}l<{$}}
%Provides \xpretocmd and loads etoolbox which provides \apptocmd, \patchcmd, \newtoggle… Also loads xparse, which provides \NewDocumentCommand and similar commands intended as replacement of \newcommand in LaTeX3 for defining commands (see https://tex.stackexchange.com/q/98152 and https://github.com/latex3/latex2e/issues/89).
	\usepackage{xpatch}
%ntheorem doc says: “empheq provides an enhanced vertical placement of the endmarks”; must be loaded before ntheorem. Loads the mathtools package, which loads and fixes some bugs in amsmath and provides \DeclarePairedDelimiter. amsmath is considered a basic, mandatory package nowadays (Grätzer, More Math Into LaTeX).
	\usepackage[ntheorem]{empheq}
%Package frenchb asks to load natbib before babel-french. Package hyperref asks to load natbib before hyperref.
	\usepackage{natbib}

\newtoggle{LCpres}
	\newtoggle{LCart}
	\newtoggle{LCposter}
	\makeatletter
	\@ifclassloaded{beamer}{
		\toggletrue{LCpres}
		\togglefalse{LCart}
		\togglefalse{LCposter}
		\wlog{Presentation mode}
	}{
		\@ifclassloaded{tikzposter}{
			\toggletrue{LCposter}
			\togglefalse{LCpres}
			\togglefalse{LCart}
			\wlog{Poster mode}
		}{
			\toggletrue{LCart}
			\togglefalse{LCpres}
			\togglefalse{LCposter}
			\wlog{Article mode}
		}
	}
	\makeatother%

%Language options ([french, english]) should be on the document level (last is main); except with tikzposter: put [french, english] options next to \usepackage{babel} to avoid warning. beamer uses the \translate command for the appendix: omitting babel results in a warning, see https://github.com/josephwright/beamer/issues/449. Babel also seems required for \refname.
%	\iftoggle{LCpres}{
		\usepackage{babel}
%	}{
%	}
	%\frenchbsetup{AutoSpacePunctuation=false}
%listings (1.7) does not allow multi-byte encodings. listingsutf8 works around this only for characters that can be represented in a known one-byte encoding and only for \lstinputlisting. Other workarounds: use literate mechanism; or escape to LaTeX (but breaks alignment).
	%\usepackage{listings}
	%\lstset{tabsize=2, basicstyle=\ttfamily, escapechar=§, literate={é}{{\'e}}1}
%I favor acro over acronym because the former is more recently updated (2018 VS 2015 at time of writing); has a longer user manual (about 40 pages VS 6 pages if not counting the example and implementation parts); has a command for capitalization; and acronym suffers a nasty bug when ac used in section, see https://tex.stackexchange.com/q/103483 (though this might be the fault of the silence package and might be solved in more recent versions, I do not know) and from a bug when used with cleveref, see https://tex.stackexchange.com/q/71364. However, loading it makes compilation time (one pass on this template) go from 0.6 to 1.4 seconds, see https://bitbucket.org/cgnieder/acro/issues/115. Option short-format not usable in the package options as it is fragile, see https://tex.stackexchange.com/q/466882.
	%\usepackage[single]{acro}
	%\acsetup{short-format = {\scshape}}
	%\DeclareAcronym{AMCD}{short=AMCD, long={Aide Multicritère à la Décision}}
\DeclareAcronym{AHP}{short=AHP, long={Analytic Hierarchy Process}}
\DeclareAcronym{AR}{short=AR, long={Argumentative Recommender}}
\DeclareAcronym{DA}{short=DA, long={Decision Analysis}}
\DeclareAcronym{DJ}{short=DJ, long={Deliberated Judgment}}
\DeclareAcronym{DM}{short=DM, long={Decision Maker}}
\DeclareAcronym{DP}{short=DP, long={Deliberated Preference}}
\DeclareAcronym{MAVT}{short=MAVT, long={Multiple Attribute Value Theory}}
\DeclareAcronym{MCDA}{short=MCDA, long={Multicriteria Decision Aid}}
\DeclareAcronym{MIP}{short=MIP, long={Mixed Integer Program}}
\DeclareAcronym{SEU}{short=SEU, long={Subjective Expected Utility}}


\iftoggle{LCpres}{
	%I favor fmtcount over nth because it is loaded by datetime anyway; and fmtcount warns about possible conflicts when loaded after nth.
	\usepackage{fmtcount}
	%For nice input of date of presentation. Must be loaded after the babel package. Has possible problems with srcletter: https://golatex.de/verwendung-von-babel-und-datetime-in-scrlttr2-schlaegt-fehlt-t14779.html.
	\usepackage[nodayofweek]{datetime}
}{
}
%For presentations, Beamer implicitely uses the pdfusetitle option. ntheorem doc says to load hyperref “before the first use of \newtheorem”. autonum doc mandates option hypertexnames=false. I want to highlight links only if necessary for the reader to recognize it as a link, to reduce distraction. In presentations, this is already taken care of by beamer (https://tex.stackexchange.com/a/262014). If using colorlinks=true in a presentation, see https://tex.stackexchange.com/q/203056. Crashes the first compilation with tikzposter, just compile again and the problem disappears, see https://tex.stackexchange.com/q/254257.
\makeatletter
\iftoggle{LCpres}{
	\usepackage{hyperref}
}{
	\usepackage[hypertexnames=false, pdfusetitle, linkbordercolor={1 1 1}, citebordercolor={1 1 1}, urlbordercolor={1 1 1}]{hyperref}
	%https://tex.stackexchange.com/a/466235
	\pdfstringdefDisableCommands{%
		\let\thanks\@gobble
	}
}
\makeatother
%urlbordercolor is used both for \url and \doi, which I think shouldn’t be colored, and for \href, thus might want to color manually when required. Requires xcolor.
	\NewDocumentCommand{\hrefblue}{mm}{\textcolor{blue}{\href{#1}{#2}}}
%hyperref doc says: “Package bookmark replaces hyperref’s bookmark organization by a new algorithm (...) Therefore I recommend using this package”.
	\usepackage{bookmark}
%Need to invoke hyperref explicitly to link to line numbers: \hyperlink{lintarget:mylinelabel}{\ref*{lin:mylinelabel}}, with \ref* to disable automatic link. Also see https://tex.stackexchange.com/q/428656 for referencing lines from another document.
	%\usepackage{lineno}
	%\NewDocumentCommand{\llabel}{m}{\hypertarget{lintarget:#1}{}\linelabel{lin:#1}}
	%\setlength\linenumbersep{9mm}
%For complex authors blocks. Seems like authblk wants to be later than hyperref, but sooner than silence. See https://tex.stackexchange.com/q/475513 for the patch to hyperref pdfauthor.
	\ExplSyntaxOn
	\seq_new:N \g_oc_hrauthor_seq
	\NewDocumentCommand{\addhrauthor}{m}{
		\seq_gput_right:Nn \g_oc_hrauthor_seq { #1 }
	}
	%Should be \NewExpandableDocumentCommand, but this is not yet provided by my version of xparse
	\DeclareExpandableDocumentCommand{\hrauthor}{}{
		\seq_use:Nn \g_oc_hrauthor_seq {,~}
	}
	\ExplSyntaxOff
	{
		\catcode`#=11\relax
		\gdef\fixauthor{\xpretocmd{\author}{\addhrauthor{#2}}{}{}}%
	}
	\iftoggle{LCart}{
		\usepackage{authblk}
		\renewcommand\Affilfont{\small}
		\fixauthor
		\AtBeginDocument{
		    \hypersetup{pdfauthor={\hrauthor}}
		}
	}{
	}
%I do not use floatrow, because it requires an ugly hack for proper functioning with KOMA script (see scrhack doc). Instead, the following command centers all floats (using \centering, as the center environment adds space, http://texblog.net/latex-archive/layout/center-centering/), and I manually place my table captions above and figure captions below their contents (https://tex.stackexchange.com/a/3253).
	\makeatletter
	\g@addto@macro\@floatboxreset\centering
	\makeatother
%Permits to customize enumeration display and references
	%\nottoggle{LCpres}{
		%\usepackage{enumitem} %follow list environments by a string to customize enumeration, example: \begin{description}[itemindent=8em, labelwidth=!] or \begin{enumerate}[label=({\roman*}), ref={\roman*}].
	%}{
	%}
%Provides \Cen­ter­ing, \RaggedLeft, and \RaggedRight and en­vi­ron­ments Cen­ter, FlushLeft, and FlushRight, which al­low hy­phen­ation. With tikzposter, seems to cause 1=1 to be printed in the middle of the poster.
	%\usepackage{ragged2e}
%To typeset units by closely following the “official” rules.
	%\usepackage[strict]{siunitx}
%Turns the doi provided by some bibliography styles into URLs. However, uses old-style dx.doi url (see 3.8 DOI system Proxy Server technical details, “Users may resolve DOI names that are structured to use the DOI system Proxy Server (https://doi.org (current, preferred) or earlier syntax http://dx.doi.org).”, https://www.doi.org/doi_handbook/3_Resolution.html). The patch solves this.
	\usepackage{doi}
	\makeatletter
	\patchcmd{\@doi}{http://dx.doi.org}{https://doi.org}{}{}
	\makeatother
%Makes sure upper case greek letters are italic as well.
	\usepackage{fixmath}
%Provides \mathbb; obsoletes latexsym (see http://tug.ctan.org/macros/latex/base/latexsym.dtx). Relatedly, \usepackage{eucal} to change the mathcal font and \usepackage[mathscr]{eucal} (apparently equivalent to \usepackage[mathscr]{euscript}) to supplement \mathcal with \mathscr. This last option is not very useful as both fonts are similar, and the intent of the authors of eucal was to provide a replacement to mathcal (see doc euscript). Also provides \mathfrak for supplementary letters.
	\usepackage{amsfonts}
%Provides a beautiful (IMHO) \mathscr and really different than \mathcal, for supplementary uppercase letters. But there is no bold version. Alternative: mathrsfs (more slanted), but when used with tikzposter, it warns about size substitution, see https://tex.stackexchange.com/q/495167.
	\usepackage[scr]{rsfso}
%Multiple means to produce bold math: \mathbf, \boldmath (defined to be \mathversion{bold}, see fntguide), \pmb, \boldsymbol (all legacy, from LaTeX base and AMS), \bm (the most recommended one), \mathbold from package fixmath (I don’t see its advantage over \boldsymbol).
%“The \boldsymbol command is obtained preferably by using the bm package, which provides a newer, more powerful version than the one provided by the amsmath package. Generally speaking, it is ill-advised to apply \boldsymbol to more than one symbol at a time.” — AMS Short math guide. “If no bold font appears to be available for a particular symbol, \bm will use ‘poor man’s bold’” — bm. It is “best to load the package after any packages that define new symbol fonts” – bm. bm defines \boldsymbol as synonym to \bm. \boldmath accesses the correct font if it exists; it is used by \bm when appropriate. See https://tex.stackexchange.com/a/10643 and https://github.com/latex3/latex2e/issues/71 for some difficulties with \bm.
	\usepackage{bm}
	\nottoggle{LCpres}{
	%https://ctan.org/pkg/amsmath recommends ntheorem, which supersedes amsthm, which corrects the spacing of proclamations and allows for theoremstyle. Option standard loads amssymb and latexsym. Must be loaded after amsmath (from ntheorem doc). From cleveref doc, “ntheorem is fully supported and even recommended”; says to load cleveref after ntheorem. When used with tikzposter, warns about size substitution for the lasy (latexsym) font when using \url, because ntheorem loads latexsym; relatedly (but not directly related to ntheorem), size substitution warning with the cmex font happens when loading amsmath and using \url.
		\usepackage[thmmarks, amsmath, standard, hyperref]{ntheorem}
		%empheq doc says to do this after loading ntheorem
		\usetagform{default}
	%Provides \cref. Unfortunately, cref fails when the language is French and referring to a label whose name contains a colon (https://tex.stackexchange.com/q/83798). Use \cref{sec\string:intro} to work around this. cleveref should go “laster” than hyperref.
		\usepackage{cleveref}
	}{
	}
	\nottoggle{LCposter}{
	%Equations get numbers iff they are referenced. Loading order should be “amsmath → hyperref → cleveref → autonum”, according to autonum doc. Use this in preference to the showonlyrefs option from mathtools, see https://tex.stackexchange.com/q/459918 and autonum doc. See https://tex.stackexchange.com/a/285953 for the etex line. Incompatible with my version of tikzposter (produces “! Improper \prevdepth”).
		\expandafter\def\csname ver@etex.sty\endcsname{3000/12/31}\let\globcount\newcount
		\usepackage{autonum}
	}{
	}
%Also loaded by tikz.
	\usepackage{xcolor}
\iftoggle{LCpres}{
	\usepackage{tikz}
	%\usetikzlibrary{babel, matrix, fit, plotmarks, calc, trees, shapes.geometric, positioning, plothandlers, arrows, shapes.multipart}
}{
}
%Vizualization, on top of TikZ
	%\usepackage{pgfplots}
	%\pgfplotsset{compat=1.14}
\usepackage{graphicx}
	\graphicspath{{graphics/}}

%Provides \print­length{length}, useful for debugging.
	%\usepackage{printlen}
	%\uselengthunit{mm}

\iftoggle{LCpres}{
	\usepackage{appendixnumberbeamer}
	%I have yet to see anyone actually use these navigation symbols; let’s disable them
	\setbeamertemplate{navigation symbols}{} 
	\usepackage{preamble/beamerthemeParisFrance}
	\setcounter{tocdepth}{10}
}{
}

%Do not use the displaymath environment: use equation. Do not use the eqnarray or eqnarray* environments: use align(*). This improves spacing. (See l2tabu or amsldoc.)


\newcommand{\R}{ℝ}
\newcommand{\N}{ℕ}
\newcommand{\Z}{ℤ}
\newcommand{\card}[1]{\lvert{#1}\rvert}
\newcommand{\powerset}[1]{\mathscr{P}(#1)}%\mathscr rather than \mathcal: scr is rounder than cal (at least in XITS Math).
\newcommand{\suchthat}{\;\ifnum\currentgrouptype=16 \middle\fi|\;}
%\newcommand{\Rplus}{\reels^+\xspace}

\AtBeginDocument{%
	\renewcommand{\epsilon}{\varepsilon}
% we want straight form of \phi for mathematics, as recommended in UTR #25: Unicode support for mathematics.
%	\renewcommand{\phi}{\varphi}
}

% with amssymb, but I don’t want to use amssymb just for that.
% \newcommand{\restr}[2]{{#1}_{\restriction #2}}
%\newcommand{\restr}[2]{{#1\upharpoonright}_{#2}}
\newcommand{\restr}[2]{{#1|}_{#2}}%sometimes typed out incorrectly within \set.
%\newcommand{\restr}[2]{{#1}_{\vert #2}}%\vert errors when used within \Set and is typed out incorrectly within \set.
\DeclareMathOperator*{\argmax}{arg\,max}
\DeclareMathOperator*{\argmin}{arg\,min}


\NewDocumentCommand{\range}{}{R}

%Decision Theory (MCDA and SC)
\NewDocumentCommand{\allalts}{}{\mathscr{X}}
\NewDocumentCommand{\allcrits}{}{\mathscr{C}}
\NewDocumentCommand{\alts}{}{X}
\NewDocumentCommand{\alt}{}{x}
\NewDocumentCommand{\altp}{}{y}%alt prime, another alt
\NewDocumentCommand{\dm}{}{i}
\NewDocumentCommand{\allF}{}{\mathscr{F}}
\NewDocumentCommand{\allvoters}{}{\mathscr{N}}
\NewDocumentCommand{\voters}{}{N}
\NewDocumentCommand{\allprofs}{}{\boldsymbol{\mathcal{R}}}
\NewDocumentCommand{\prof}{}{\boldsymbol{R}}
\NewDocumentCommand{\linors}{}{\mathscr{L}(\allalts)}
%Thanks to https://tex.stackexchange.com/q/154549
	%\makeatletter
	%\def\@myRgood@#1#2{\mathrel{R^X_{#2}}}
	%\def\myRgood{\@ifnextchar_{\@myRgood@}{\mathrel{R^X}}}
	%\makeatother
\NewDocumentCommand{\ind}{}{\sim}
\NewDocumentCommand{\peq}{}{\succeq}
\NewDocumentCommand{\pst}{}{\succ}
\NewDocumentCommand{\npeq}{}{\nsucceq}
\NewDocumentCommand{\npst}{}{\nsucc}

%Deliberated Judgment
%%Normative theory
\NewDocumentCommand{\allargs}{}{\mathscr{A}}
\NewDocumentCommand{\args}{}{A}
\NewDocumentCommand{\ard}{O{}}{a^\mathit{d}_{#1}}
\NewDocumentCommand{\ardp}{O{}}{a^{\mathit{d}\prime}_{#1}}
\NewDocumentCommand{\ar}{o}{%
	\IfValueTF{#1}{%
		a^{(#1)}%
	}{%
		a%
	}%
}
\NewDocumentCommand{\zar}{}{\mathbf{0}}%zero, or empty, argument
\NewDocumentCommand{\allhist}{}{\mathscr{A}^*}
\NewDocumentCommand{\hist}{}{α}
\NewDocumentCommand{\histp}{}{α^{\prime}}
\NewDocumentCommand{\histpp}{}{α^{\prime\prime}}
\NewDocumentCommand{\histend}{o}{%
	\IfValueTF{#1}{%
		α^{#1}_\mathit{end}%
	}{%
		α_\mathit{end}%
	}%
}
\NewDocumentCommand{\histpend}{}{α^{\prime}_\mathit{end}}
\NewDocumentCommand{\histppend}{}{α^{\prime\prime}_\mathit{end}}
\NewDocumentCommand{\allprops}{}{\Phi}
\NewDocumentCommand{\prop}{}{φ}
\NewDocumentCommand{\propbar}{}{φ'}%\overline
\NewDocumentCommand{\incompat}{}{\Phi^\mathit{incompat}}
%%Empirical theory
\NewDocumentCommand{\gC}{}{C_γ}
\NewDocumentCommand{\gPhi}{}{\Phi_γ}
\NewDocumentCommand{\gpropse}{O{γ}}{{\hookrightarrow_{#1}}(\allargs)}%e for explicit
\NewDocumentCommand{\gprops}{O{γ}}{\Phi_{#1}}
\NewDocumentCommand{\dargs}{O{}}{A^\mathit{d}_{#1}}
\NewDocumentCommand{\alldargs}{}{\mathscr{A}^d}
\NewDocumentCommand{\gargs}{O{φ}}{A^{#1}_{γ, i}}
\NewDocumentCommand{\gargsmu}{}{A^{φ}_{μ, i}}
\NewDocumentCommand{\gargsnu}{}{A^{φ'}_{ν, i}}
\NewDocumentCommand{\gargsgamma}{}{A^{φ}_{γ, i}}
\NewDocumentCommand{\gargsdelta}{}{A^{φ'}_{δ, i}}
\NewDocumentCommand{\gleadsto}{O{γ}}{\hookrightarrow_{#1}}
\NewDocumentCommand{\gleadstoinv}{O{γ}}{{\hookrightarrow^{-1}_{#1}}}
\NewDocumentCommand{\gbeats}{O{γ}}{⊳^\mathit{t}_{#1}}
\NewDocumentCommand{\gbeatsinv}{O{γ}}{{(⊳^\mathit{t}_{#1})^{-1}}}
\NewDocumentCommand{\ngbeats}{O{γ}}{\not⊳^\mathit{t}_{#1}}
\NewDocumentCommand{\dbeats}{O{γ}}{⊳^\mathit{d}_{#1}}
\NewDocumentCommand{\dbeatsinv}{O{γ}}{{(⊳^\mathit{d}_{#1})^{-1}}}
\NewDocumentCommand{\df}{O{γ}}{\mathit{def}_{#1}}
\NewDocumentCommand{\dfp}{O{γ}}{\mathit{def}_{#1}^+}
\NewDocumentCommand{\dg}{O{γ}}{d_{#1}}
\NewDocumentCommand{\dgip}{O{γ, i}}{d^\phi_{#1}}
%%%DP
\NewDocumentCommand{\choices}{}{\mathscr{C}}
\NewDocumentCommand{\gind}{O{}}{\sim_\gamma^{#1}}
\NewDocumentCommand{\gpeq}{}{\succeq_\gamma}
\NewDocumentCommand{\gpst}{}{\succ_\gamma}
\NewDocumentCommand{\ngpeq}{}{\nsucceq_\gamma}
\NewDocumentCommand{\ngpst}{}{\nsucc_\gamma}

%%i
\NewDocumentCommand{\iprops}{}{\Phi_i}
\NewDocumentCommand{\allleadsto}{}{⇝}%Or \dashrightarrow?
\NewDocumentCommand{\ileadsto}{O{i}}{⇝_{#1}}
\NewDocumentCommand{\nileadsto}{O{i}}{\not⇝_{#1}}
\NewDocumentCommand{\ileadstoe}{O{i}}{⇝_{#1}^\exists}
\NewDocumentCommand{\nileadstoe}{O{i}}{\not⇝_{#1}^\exists}
\NewDocumentCommand{\ileadstost}{}{\hookrightarrow_i}
\NewDocumentCommand{\nileadstost}{}{\not\hookrightarrow_i}
\NewDocumentCommand{\di}{}{c^φ_{γ, i}}
\NewDocumentCommand{\dip}{}{d^{φ +}_{γ, i}}
\NewDocumentCommand{\ibeats}{}{⊳^\text{\sout{\ensuremath{φ}}}_{γ, i}}%Or: \usepackage[normalem]{ulem} \text{\sout{\ensuremath t}}
\NewDocumentCommand{\nibeats}{}{⋫^\text{\sout{\ensuremath{φ}}}_{γ, i}}
%%%Deliberated Preference
\NewDocumentCommand{\ipeq}{}{\succeq_i}
\NewDocumentCommand{\ipst}{}{\succ_i}


\definecolor{darkgreen}{rgb}{0,0.6,0}
\newcommand{\commentOC}[1]{{\small\color{blue}{\selectlanguage{french}$\big[$OC: #1$\big]$}}}
%\newcommand{\commentOC}[1]{{\selectlanguage{french}{\todo{OC : #1}}}}
%Or: \todo[color=green!40]
\newcommand{\innote}[1]{{\scriptsize{#1}}}

%this probably requires outdated float package, see doc KomaScript for an alternative.
% \newfloat{program}{t}{lop}
% \floatname{program}{PM}

%definition, theorem, lemma, example environments, qed trickery are only needed in article mode (not Beamer)
\nottoggle{LCpres}{
%style is plain by default (italic text)
	\newtheorem{definition}{Definition}
	\newtheorem{theorem}{Theorem}
%no italic: expected.
%http://tex.stackexchange.com/questions/144653/italicizing-of-amsthm-package
	\newtheorem{lemma}{Lemma}
%\crefname{axiom}{axiom}{axioms}%might be needed for workaround bug in cref when defining new theorems?

%\ifdefined\theorem\else
%\newtheorem{theorem}{\iflanguage{english}{Theorem}{Théorème}}
%\fi

\theoremstyle{remark}
	\newtheorem{examplex}{Example}
	\newtheorem{remarkx}{Remark}

%trickery allowing use of \qedhere and the like.
\newenvironment{example}{
	\pushQED{\qed}\renewcommand{\qedsymbol}{$\triangle$}\examplex
}{
	\popQED\endexamplex
}
\newenvironment{remark}{
	\pushQED{\qed}\renewcommand{\qedsymbol}{$\triangle$}\remarkx
}{
	\popQED\endremarkx
}
}{
}
\crefname{examplex}{example}{examples}% I wonder why this is unnecessary in case of singular

%which line breaks are chosen: accept worse lines, therefore reducing risk of overfull lines. Default = 200
\tolerance=2000
%accept overfull hbox up to...
\hfuzz=2cm
%reduces verbosity about the bad line breaks
\hbadness 5000
%reduces verbosity about the underful vboxes
\vbadness=1300
%sloppy sets tolerance to 9999
\apptocmd{\sloppy}{\hbadness 10000\relax}{}{}

\bibliographystyle{abbrvnat}
%or \bibliographystyle{apalike} for presentations?

%doi package uses old-style dx.doi url, see 3.8 DOI system Proxy Server technical details, “Users may resolve DOI names that are structured to use the DOI system Proxy Server (http://doi.org (preferred) or http://dx.doi.org).”, https://www.doi.org/doi_handbook/3_Resolution.html
\makeatletter
\patchcmd{\@doi}{dx.doi.org}{doi.org}{}{}
\makeatother

% WRITING
%\newcommand{\ie}{i.e.\@\xspace}%to try
%\newcommand{\eg}{e.g.\@\xspace}
%\newcommand{\etal}{et al.\@\xspace}
\newcommand{\ie}{i.e.\ }
\newcommand{\eg}{e.g.\ }
\newcommand{\mkkOK}{\checkmark}%\color{green}{\checkmark}
\newcommand{\mkkREQ}{\ding{53}}%requires pifont?%\color{green}{\checkmark}
\newcommand{\mkkNO}{}%\text{\color{red}{\textsf{X}}}

\newlength{\xdescwd}
\makeatletter
\NewEnviron{xdesc}{%
  \setbox0=\vbox{\hbadness=\@M \global\xdescwd=0pt
    \def\item[##1]{%
      \settowidth\@tempdima{\textbf{##1}:}%
      \ifdim\@tempdima>\xdescwd \global\xdescwd=\@tempdima\fi}
  \BODY}
  \begin{description}[leftmargin=\dimexpr\xdescwd+.5em\relax,
    labelindent=0pt,labelsep=.5em,
    labelwidth=\xdescwd,align=left]\BODY\end{description}}
\makeatother

\makeatletter
\newcommand{\boldor}[2]{%
	\ifnum\strcmp{\f@series}{bx}=\z@
		#1%
	\else
		#2%
	\fi
}
\newcommand{\textstyleElProm}[1]{\boldor{\MakeUppercase{#1}}{\textsc{#1}}}
\makeatother
\newcommand{\electre}{\textstyleElProm{Électre}\xspace}
\newcommand{\electreIv}{\textstyleElProm{Électre Iv}\xspace}
\newcommand{\electreIV}{\textstyleElProm{Électre IV}\xspace}
\newcommand{\electreIII}{\textstyleElProm{Électre III}\xspace}
\newcommand{\electreTRI}{\textstyleElProm{Électre Tri}\xspace}
% \newcommand{\utadis}{\texorpdfstring{\textstyleElProm{utadis}\xspace}{UTADIS}}
% \newcommand{\utadisI}{\texorpdfstring{\textstyleElProm{utadis i}\xspace}{UTADIS I}}

%TODO
% \newcommand{\textstyleElProm}[1]{{\rmfamily\textsc{#1}}} 


\usepackage[normalem]{ulem}
%\NewDocumentCommand{\tikzmark}{m}{%
	\tikz[overlay, remember picture, baseline=(#1.base)] \node (#1) {};%
}

\newlength{\GraphsDNodeSep}
\setlength{\GraphsDNodeSep}{7mm}
\tikzset{/GraphsD/dot/.style={
	shape=circle, fill=black, inner sep=0, minimum size=1mm
}}

% MCDA Drawing Sorting
\newlength{\MCDSCatHeight}
\setlength{\MCDSCatHeight}{6mm}
\newlength{\MCDSAltHeight}
\setlength{\MCDSAltHeight}{4mm}
%separation between two vertical alts
\newlength{\MCDSAltSep}
\setlength{\MCDSAltSep}{2mm}
\newlength{\MCDSCatWidth}
\setlength{\MCDSCatWidth}{3cm}
\newlength{\MCDSAltWidth}
\setlength{\MCDSAltWidth}{2.5cm}
\newlength{\MCDSEvalRowHeight}
\setlength{\MCDSEvalRowHeight}{6mm}
\newlength{\MCDSAltsToCatsSep}
\setlength{\MCDSAltsToCatsSep}{1.5cm}
\newcounter{MCDSNbAlts}
\newcounter{MCDSNbCats}
\newlength{\MCDSArrowDownOffset}
\setlength{\MCDSArrowDownOffset}{0mm}
\tikzset{/MCD/S/alt/.style={
	shape=rectangle, draw=black, inner sep=0, minimum height=\MCDSAltHeight, minimum width=\MCDSAltWidth
}}
\tikzset{/MCD/S/pref/.style={
	shape=ellipse, draw=gray, thick
}}
\tikzset{/MCD/S/cat/.style={
	shape=rectangle, draw=black, inner sep=0, minimum height=\MCDSCatHeight, minimum width=\MCDSCatWidth
}}
\tikzset{/MCD/S/evals matrix/.style={
	matrix, row sep=-\pgflinewidth, column sep=-\pgflinewidth, nodes={shape=rectangle, draw=black, inner sep=0mm, text depth=0.5ex, text height=1em, minimum height=\MCDSEvalRowHeight, minimum width=12mm}, nodes in empty cells, matrix of nodes, inner sep=0mm, outer sep=0mm, row 1/.style={nodes={draw=none, minimum height=0em, text height=, inner ysep=1mm}}
}}

%Git
\newlength{\GitDCommitSep}
\setlength{\GitDCommitSep}{13mm}
\tikzset{/GitD/commit/.style={
	shape=rectangle, draw, minimum width=4em, minimum height=0.6cm
}}
\tikzset{/GitD/branch/.style={
	shape=ellipse, draw, red
}}
\tikzset{/GitD/head/.style={
	shape=ellipse, draw, fill=yellow
}}

%Social Choice
\tikzset{/SCD/profile matrix/.style={
	matrix of math nodes, column sep=3mm, row sep=2mm, nodes={inner sep=0.5mm, anchor=base}
}}
\tikzset{/SCD/rank-profile matrix/.style={
	matrix of math nodes, column sep=3mm, row sep=2mm, nodes={anchor=base}, column 1/.style={nodes={inner sep=0.5mm}}, row 1/.style={nodes={inner sep=0.5mm}}
}}
\tikzset{/SCD/rank-vector/.style={
	draw, rectangle, inner sep=0, outer sep=1mm
}}
\tikzset{/SCD/isolated rank-vector/.style={
	draw, matrix of math nodes, column sep=3mm, inner sep=0, matrix anchor=base, nodes={anchor=base, inner sep=.33em}, ampersand replacement=\&
}}

% GUI
\tikzset{/GUID/button/.style={
	rectangle, very thick, rounded corners, draw=black, fill=black!40%, top color=black!70, bottom color=white
}}

% Logger objects
\tikzset{/loggerD/main/.style={
	shape=rectangle, draw=black, inner sep=1ex, minimum height=7mm
}}
\tikzset{/loggerD/helper/.style={
	shape=rectangle, draw=black, dashed, minimum height=7mm
}}
\tikzset{/loggerD/helper line/.style={
	<->, draw, dotted
}}

% Beliefs
\tikzset{/BeliefsD/attacker/.style={
	shape=rectangle, draw, minimum size=8mm
}}
\tikzset{/BeliefsD/supporter/.style={
	shape=circle, draw
}}


%\DeclareAcronym{AMCD}{short=AMCD, long={Aide Multicritère à la Décision}}
\DeclareAcronym{AHP}{short=AHP, long={Analytic Hierarchy Process}}
\DeclareAcronym{AR}{short=AR, long={Argumentative Recommender}}
\DeclareAcronym{DA}{short=DA, long={Decision Analysis}}
\DeclareAcronym{DJ}{short=DJ, long={Deliberated Judgment}}
\DeclareAcronym{DM}{short=DM, long={Decision Maker}}
\DeclareAcronym{DP}{short=DP, long={Deliberated Preference}}
\DeclareAcronym{MAVT}{short=MAVT, long={Multiple Attribute Value Theory}}
\DeclareAcronym{MCDA}{short=MCDA, long={Multicriteria Decision Aid}}
\DeclareAcronym{MIP}{short=MIP, long={Mixed Integer Program}}
\DeclareAcronym{SEU}{short=SEU, long={Subjective Expected Utility}}


\addtokomafont{labelinglabel}{\sffamily\bfseries}
\DeclareMathAlphabet{\mathup}{OT1}{\familydefault}{m}{n}

%I find these settings useful in draft mode. Should be removed for final versions.
	%Which line breaks are chosen: accept worse lines, therefore reducing risk of overfull lines. Default = 200.
		\tolerance=2000
	%Accept overfull hbox up to...
		\hfuzz=2cm
	%Reduces verbosity about the bad line breaks.
		\hbadness 5000
	%Reduces verbosity about the underful vboxes.
		\vbadness=1300

\begin{document}
\title{Theories of deliberated judgment}
\author{Olivier Cailloux}
\affil{Université Paris-Dauphine, PSL Research University, CNRS, LAMSADE, 75016 PARIS, FRANCE\\
	\href{mailto:olivier.cailloux@dauphine.fr}{olivier.cailloux@dauphine.fr}
}
\makeatletter
	\hypersetup{
		pdfsubject={Epistemology},
		pdfkeywords={Decision aiding, Decision making, Argumentation}
	}
\makeatother
\maketitle

\begin{abstract}
	TODO: abstract
\end{abstract}

\section{Introduction} 
Observing subjects performing acts of choice is the basis of much work in econometrics. Such acts have been used, among others, to give an observable basis to the concept of preference, thanks in part to the seminal work of Samuelson. Acts of choice are considered here in the broad sense of choice between courses of actions, encompassing the simple setting of choosing between various elementary consumption bundles as well as settings involving more abstract acts such as choosing between defection or cooperation in a game-theoretical setting. 

Most of these approaches consider only acts of choice that are performed spontaneously in the normal course of action of an individual, or that are given a specific frame depending on the psychological aspect that is to be studied. As an example of the former kind, typical willingness to pay studies asks the subject whether she prefers an object at some price or another object at another price, describing the objects in a way that is considered immediately understandable by the subject (or simply showing the objects). Examples of the latter kind include the celebrated work of Kahneman and Tversky.

On the other hand, since at least the seminal work of Fishkin, another sort of judgment involving subjective desirability is emerging as also worth studying. These are judgments given by subjects after arguments in favor and against different possible stance have been considered. Such judgments may differ from immediate judgments of desirability given in natural conditions. Indeed, individuals may change opinion on the desirability of some course of action, and therefore revise their choice, while learning about the properties of objects, the consequences of some acts (or empirical knowledge useful to estimate the likelihood of some consequences), or the logical or empirical impossibilities between consequences. This is particularly likely to happen when choosing among non everyday objects or acts whose consequences are not well known by the subject. 

While numerous articles have discussed the concept of deliberation in the last decades, it remains unsettled, to the best of my knowledge, how to observe a deliberated judgment, or, correspondingly, how to define this concept in a way that makes it amenable to empirical study, sufficiently precisely that any disagreement about someone’s deliberated judgment in a given context could be solved, at least in principle, by an empirical experiment. %This article is interested in defining the concept of deliberated judgementof a subject, as opposed to the classical one, that will henceforth be called her natural judgment, that is, the way that she chooses between acts when she does not face any arguments, i.e., 

While the concept of deliberation in the deliberative democracy literature usually refers to multiple individuals debating, this article considers deliberation as related to a single individual at a time: an individual deliberates (in the sense of: carefully weights) the stance she considers most appropriate while being confronted to arguments. The deliberated judgment of an individual is thus defined with no requirement that the arguments be uttered by peers in a debate; they can be given by an automata, for example. In this sense, deliberated judgments come close to what Rawls calls reflective equilibrium (referring to an idea of Goodman), a concept that has much inspired the present proposal.

To define deliberated judgments, one needs to start with the set of arguments under consideration, and the way the subject gets to know them, called hereafter the “exposure protocol”. As a result, we need always talk about the deliberated judgment \emph{given some exposure protocol}. The notion of argument invoked here is an extremely large one, including texts, images, sounds, experiences (a hiking trip), and so on: anything that can be transmitted to a subject qualifies as a possible argument. Such an “argument” need not even have the logical structure of what is considered a proper argument in argumentation theory, which permits to avoid having to take position on what is a correct argument, and permits to study wide enough conceptions of deliberated judgment, including those involving no paternalism: anything that influences the judgment of a subject may a priori be considered worth including in at least some conceptions of deliberated judgment, thus including whatever some could consider “bad reasoning” or “bad reasons”.

The dependence of the deliberated judgment concept on the exposure protocol yields a first qualification about the concept of deliberated judgments defined here when one desires to use it normatively: it is only as reasonable to call the resulting judgments “deliberate” in a normative sense as the exposure protocol is normatively relevant to deliberation. A sufficiently restricted, or even purposefully chosen, set of arguments may yield judgments that are only deliberate in the technical sense used here, but that pertain more to brain-washing than to reflective judgments. While it is indeed a long-term goal of this research program to yield a normatively relevant conception of preference (appropriate, for example, for recommendation), it is not a claim of this article that anything that corresponds to deliberated judgments as defined here qualify as being appropriate for recommendation (or for any other normative use): much depends on the chosen exposure protocol. The goal of this article is, more modestly, one of a conceptual and practical clarification. It proposes a sharp distinction between the empirical content and the normative content of the concept of deliberated judgments: the definition proposed here transfers the full normative weight on the choice of the exposure protocol; and leaves the rest of the disagreements to empirical settling. This is methodologically akin, mutatis mutandis, to the axiomatic study of justice such as done in social choice, only replacing mathematical deduction with empirical claims. In social choice, the choice of axioms bears the full normative weight; once axioms are chosen, it is a matter of deduction which social choice rules (if any) correspond to that choice. In our case, once an exposure protocol is chosen, it is a pure empirical matter which deliberated judgments result.
In supplement to a possibly helpful conceptual clarification, such a sharp separation may reveal practically useful for separation of concern as it permits to study deliberated judgments empirically under various exposure protocols independently of possible persistent disagreements about which norms are more appropriate for deliberation in which circumstances.

The deliberated judgment of a subject is said to depend on the exposure protocol, that is, on the set of arguments and the way they are submitted to the subject, and not only on the set of arguments, because the very way that they are transmitted to the individual may affect her judgment differently. For example, reading an image description and showing an image to a subject may impress them differently.

The exposure protocol defines a set of argument and, importantly, how arguments are sent to the subject and her subsequent choice observed. There is, as need be, no requirement on the internal process that occurs in the subject’s mind while she (possibly) processes the arguments, as it is not desirable to assume that the internal process could be observed. We consider subjects as black boxes and only require a capacity of observation of their choices after exposure to arguments. Depending on the exposure protocol and the subject, it is of course possible for the arguments to never affect any judgment of the subject, a conclusion that may be interesting in its own right.

The deliberated judgment of a subject is defined as a function of the exposure protocol, as accepting every claims that resists every counter-arguments. The notion of resistance is itself defined according to the exposure protocol.

Studying judgments resulting from exposure to arguments includes a difficulty that is not present with natural choice experiments: exposing an individual to arguments may change the individual’s stance, an effect that cannot be erased to submit the individual (considered “clean” again) to another sequence of arguments. The observer thus cannot in general assume that multiple unrelated sequences of arguments can be tested on a given individual. This difficulty will be worked around by considering general theories of deliberated judgments, that apply to sets of individuals. This is methodologically similar to defining theories of how objects break. One can test only one way of breaking a given object (assuming it cannot be repaired), so such theories must apply to sets of objects if one wants to be able to test multiple ways of falsifying the theories.

For conceptual clarity, I will focus on the case of a single binary decision to take, akin to a choice situation facing a bundle of two objects among which exactly one must be adopted.

\section{The model}
Let $\Phi = \set{\phi, ¬\phi}$ represent the decision at stake.

\subsection{Exposure protocol}
An exposure protocol is a tuple $(I, \allargs, {\allleadsto})$, where $I$ is a set of individuals, $\allargs$ is a set of arguments, and ${\allleadsto}$ is the behavior of the individuals as a function of the arguments they have been exposed to.

While $I$ and $\allargs$ are fundamental sets whose elements are considered elementary, a few notations and concepts are required to define ${\allleadsto}$.
Throughout the article, $\N$ includes zero. %and $\N^* = \N \setminus \set{0}$. 
Given $j, l \in \N$, the notation $\intvl{j, l}$ represents $\set{k \in \N \suchthat j ≤ k ≤ l}$ (thus $\forall j \in \N: \intvl{j, 0} = \emptyset$).
Define $\allhist = \bigcup_{k \in \N} \allargs^{\intvl{1, k}}$ as the finite sequences of arguments, including the empty sequence. 
A generic element of $\allhist$ (a sequence of arguments) is denoted by $\hist$ and called an argumentative path, or simply a path when this raises no ambiguity. Its length is denoted by $\card{\hist} \in \N$.
Thus, given an argumentative path $\hist \in \allhist$, $\forall k \in \intvl{1, \card{\hist}}$, $\hist_k \in \allargs$, and $\hist = \emptyset ⇔ \card{\hist} = 0$.

The function ${\allleadsto}: I × \allhist → \powersetz{\Phi}$ associates to each individual and path a non-empty subset of $\Phi$, ${\allleadsto}(i, \alpha)$ representing the choice of $i$ following exposure to the arguemnts of $\alpha$, in order. The case ${\allleadsto}(i, \alpha) = \Phi$ is used to represent indifferent choice, in case this is considered observable.
If $\phi \in {\allleadsto}(i, \alpha)$, we say that $\phi$ is considered as one of the best options by $i$ considering $\alpha$. 
It is also denoted by $\alpha \ileadsto \phi$, considering ${\ileadsto} \subseteq \allhist × \Phi$ as a binary relation.
As usual with binary relations, $\hist \nileadsto \prop$ means $(\hist, \prop) \notin {\ileadsto}$ or equivalently $\phi \notin {\allleadsto}(i, \alpha)$.

\subsection{Deliberated choice}
We can now define the deliberated choice of $i$ given the exposure protocol $(I, \allargs, {\allleadsto})$.
It is denoted $\Phi_i$ and is a subset of $\Phi$, with the semantics that $\phi \in \Phi_i$ means that $\phi$ is considered by $i$, after deliberation, one of the best options, given the exposure protocol. In other words, if $\Phi_i = \Phi$, the individual considers both $\phi$ and $¬\phi$ equally worth after deliberation and if $\Phi_i = \set{\phi}$, the individual considers $\phi$ the strictly best option among the proposed set, after deliberation (and similarly for $\Phi_i = \set{¬\phi}$).

Given $\hist \in \allhist$, $\range(\hist) = \hist(\intvl{1, \card{\hist}})$ denotes the range of $\hist$, that is, all arguments contained in the sequence $\hist$.
Let $\histend = \hist(\intvl{\max(1, \card{\hist} - 1), \card{\hist}})$ denote the set containing the last two arguments of the sequence $\hist$ if the sequence has at least two elements; the unique argument of the sequence if it has exactly one element; and the empty set if it is empty.

Given $i \in I, \prop \in \allprops, \ar \in \allargs$, define $\ar \ileadstost \prop$ iff $\forall \hist \in \allhist \suchthat \ar \in \histend: \hist \ileadsto \prop$. 
When $\ar \ileadstost \prop$, $\ar$ is said to be a decisive argument for $\prop$ from $i$’s point of view. 
Given any $i \in I$, when some argument is decisive for some proposition $\prop$, thus, when $\exists \ar \in \allargs \suchthat \ar \ileadstost \prop$, the proposition is said to be acceptable, from $i$’s point of view.

The following axiom permits the theory to have bite on what should be recommended to $i$. It says that it is sufficient that a proposition be acceptable to be in $i$’s \ac{DJ}.
\begin{axiom}[Normative adequacy]
%	\label{ax:norm}
	$\forall i \in I, \prop \in \allprops: 
		[(\exists \ar \in \allargs \suchthat \ar \ileadstost \prop)] ⇒ \prop \in \iprops.$
\end{axiom}

\appendix
\section{Old introduction} 
Decision theoretic approaches in economy are classically divided into normative and descriptive approaches. Normative approaches aim at prescribing decision behavior; descriptive approaches aim at describing decision behavior. These two goals are generally considered incompatible: what is normative is not to be judged on descriptive grounds, and conversely. (Some authors consider prescription as the art of using descriptive insights to inform recommendation \citep{bell_decision_1988}, a view that also considers the normative aim as distinct to the descriptive aim.)

As a result, normative approaches are difficult to validate: being considered as non-descriptive in their goal, normative claims cannot be submitted to empirical tests that would permit to verify or falsify them. Observing a behavior that deviates from a prescribed norm can be taken as a sign that the norm is useful: a decision maker does not hire analysts just to be prescribed the choice that she would have naturally opted for.
Instead, the validity of a normative approach is generally considered to be determined by examining the convincingness power of the normative axioms on which it rests.
\commentYM{par qui ? dans le cadre de l'analyse ou en amont ? et pourquoi sur la seule base de la soundness de l'axiome ? pourquoi pas sur la base de ses implications dans des cas concrets ?}
This validity criterion is unsatisfactory, as it is unable to settle disputes between experts when they arise. In supplement, agreement between experts in decision theory is arguably not enough to ground the practical applicability of some normative axioms: one must ensure that decision makers to which the prescription is addressed agree with the particular instanciation of these axioms in the specific decision problem under study. But no systematic, formally defined procedure has been proposed to proceed with this check, to the best of my knowledge. 
In practical applications, an expert may informally discuss the axioms with the decision maker in order to check endorsement, either in general, or by discussing violations on a case by case basis when they are observed during an interactive elicitation session. 
The resulting state of conviction arguably depends in part on the rethorical ability of the expert, not solely on the normative strength of the axioms. It may lead to a decision maker accepting a decision procedure considered unsound by many experts in the field, such as the \ac{AHP} \citep{smith_anniversary_2004}. 
This problem is compounded by the enormous constraining power gained when axioms interact, as the celebrated theorem of Arrow shows dramatically: axioms that appear innocuous when taken alone may precipitate undesirable results when combined.

By contrast, this difficulty of validation is much alleviated with descriptive approaches. Such approaches can be validated by confronting their predictions to empirical data. When two descriptive theories disagree on their predictions, it is in principle possible to show that at least one of them is wrong with a suitable experiment. 
\commentYM{quelle différence avec les implications empiriques d'une théorie normative ?}
\commentOC{Est-ce plus clair maintenant ?}
Alàs, descriptive approaches as currently existing in decision theory are not suitable to obtain valid prescriptions, as we do not merely want to prescribe natural, unreflectful behavior.

This article defines decision theories, called theories of \ac{DJ}, that are both normative and descriptive: they claim to be able to yield valid prescriptions, and these prescriptions are claimed to predict reflective behavior. Reflective behavior as defined in this article is observable, by means of observing the reaction of individuals to arguments. This allows to confront theories of \ac{DJ} to empirical data.

This article defines theories of \ac{DJ} in general, not a specific theory of \ac{DJ}. Specific theories of \ac{DJ} are tailored to a set of situations and yield claims relevant to those situations. For example, a specific theory of \ac{DJ} may be tailored to situations of choice between lotteries involving money. This article adopts a more general point of view. It defines a general form for theories of \ac{DJ}, and proves claims that hold for any specific theory of \ac{DJ}, showing that such theories are indeed normative and descriptive in some technical sense to be defined.

Here is an example opposing two simplistic theories of \ac{DJ}, meant to give a flavor of the sort of theories this article aims at defining.

\begin{example}[Two disagreeing theories]
	\label{ex:two}
	Consider a set $\allalts$ of items evaluated on two criteria, say, price and speed, using functions $g_1$ and $g_2$, where $g_i: \allalts → \R$ measures the performance of items on the criterion $i$, and a set of individuals $I$ whose reflective choice among $\allalts$ we are interested in. 
	Consider the following the theories of \ac{DJ}.
	
	The \emph{satisfycing} theory claims that any individual $i \in I$ ought to choose, and will reflectively choose, whichever item is satisfying (and be indifferent if they are both satisfying or if none of them is). An item $\alt$ is satisfying iff $g_1(\alt) < t_1 \land g_2(\alt) > t_2$, where $t_1$ and $t_2$ are two real numbers determined by the theorist.

	The \emph{additive} theory claims that any individual $i \in I$ ought to choose, and will reflectively choose, whichever item $\alt$ maximises $v(\alt) = v_1(g_1(\alt)) + v_2(g_2(\alt))$, where $v_i: \R → \R$ are functions  chosen by the theorist.
	
	Consider a set $\alts = \set{\alt_1, \alt_2} \subset \allalts$ of two items, and assume the two theories of \ac{DJ} are opposed on their prediction of the reflective choice of individuals having to choose one item among $\alts$. Say, $\alt_1$ is satisfying and $\alt_2$ is not, failing on the speed criterion; while $v(\alt_2) > v(\alt_1)$.
	
	To determine which theory is right, it is insufficient to simply observe what an individual $i \in I$ effectively chooses. This would merely represent her unreflective behavior. The approach adopted in this article to capture \acp{DJ} is to declare that the winning theory (if any) is the one that exhibit arguments able to systematically convince individuals that their proposed choice is to be adopted.
	The satisfycing theory could argue as follows: “$\alt_1$ is better than $\alt_2$, because it has a satisfactory price (lower than $t_1$, an acceptable price) and a satisfactory speed (higher than $t_2$, an acceptable performance); whereas $\alt_2$ is not satisfactory on the speed criterion”. 
	The additive theory could use the following argument: “$\alt_2$ is better than $\alt_1$, because $\alt_2$ is cheaper than $\alt_1$, and this more than compensates for its weaker performance in terms of speed”.
	
	The basic intuition on which this article is based is that determining which theory is right requires to present both arguments to $i$, then let her choose one item in $\alts$, to see which argument she finds more convincing.
\end{example}

The proposed approach starts with this intuition that the reaction of individuals to arguments should matter, and presents a general solution to the problem of capturing reflective judgments or choices. Contrary to the simplistic situation of \cref{ex:two}, this solution must allow to confront more than just two arguments or two theories. It should define precisely what is required for a judgment to be called deliberated, and determine formally which hypothesis are required for the prescriptions of a theory of \ac{DJ} to be valid.

\section{Normative theories (focused)}
Define a normative axiom as a formal constraint on behavior that constrains behavior so as to yield valid prescription. 
Define a descriptive axiom as a formal constraint on behavior interpreted as an empirical prediction, that is, as a claim that behavior (in some a priori described circumstances) satisfies the constraint.

A normative axiom has to be accepted on the basis of its apparent adequacy to capture what is intuitively an acceptable requirement for a prescription to a \ac{DM}. In other words, it is not a hypothesis on the observable world: that the requirement holds cannot be proven by observation. Examples are given by the axioms of classical decision theories, when they are applied for prescription. Consider for example Savage’s sure thing principle. One can observe whether individuals unreflectively behave according to Savage’s sure thing principle, but even when they fail to, it can be said that this is because of a mistake: the normative status of an axiom is not hit by violations due to unreflective behavior. Such axioms are not falsifiable by behavioral observation. Conversely, observing that some axiom perfectly matches observed behavior, that empirical observation alone does not yield any validity to prescription based on that constraint: accepting the validity of a prescription rests on accepting the soundness of its reasoning, not on its ability to reflect natural behavior.

Let us assume that some natural behavior is not to be prescribed.

At least one such axiom is required to ground any prescription. Hence, no theory can be fully descriptive, and theories of \ac{DJ} make no exception. The proposed path out of this conundrum is to restrict normative axioms to a self-evident claim, and leave most of the constraining power to descriptive claims. 

The pursued path is as follows.
This article introduces first a normative axiom that mandates that whatever propositions the \ac{DM} accepts when reasoning carefully can be prescribed. This notion of careful acceptability is defined formally on the basis of an observable relation that captures the reactions of the \ac{DM} to arguments.

Second, this article defines descriptive theories of acceptable propositions. Acceptable here is in the sense of the careful acceptability just defined informally. Descriptive here means that these theories can predict at least some of the propositions that will be considered acceptable by the \ac{DM}.
This is arguably the more important and difficult part of the current research program: it has to convince that it is practically possible to define theories that predict such an a priori volatile thing as the way individuals reason. It is plain that mandating that theories be able to predict the reaction of individuals to every possible sequence of arguments they are exposed to would be plainly irrealistic. Some weaker criterion is required. How weak can it become? As announced, the proposed theories must be purely descriptive, apart from the required acceptance of the normative adequacy axiom described in the above paragraph. In other words, their conclusions must hold on the sole basis of the unique normative axiom, aided by claims that can be proven false experimentally. Equivalently, if the theory is invalid, in the sense of failing its predictions about the acceptability of propositions, then it must be falsifiable: there must exist a set of observable reactions of the \ac{DM} to arguments that contradict the claims of the theory. Accordingly, this article claims and proves that empirical theories of \ac{DJ} as defined here are such that under normative adequacy, if the theory is not falsified, then it is valid. I admit readily that this might not suffice to convince that building such concrete theories is feasible. It is only one of the difficulties to tackle.

A possible objection to the usefulness of this approach should perhaps be lifted before delving into the formal descriptions of this rough sketch.
It may be argued that because this proposition needs a normative axiom, which cannot be validated empirically, it does not fare better than the alternative propositions existing in the literature. 
%\ac{SEU}, for example, also can be considered to claim to obtain reflective preferences on the basis of normative axioms.
I grant that it is impossible to get completely rid of normative axioms, because prescriptions must be linked to descriptions, and this cannot be done in a complete empirical way.
However, it is good to lighten as much as possible the content of normative axioms, and defer as much content as possible to descriptive claims. The descriptive parts of theories can be confronted to observations, a natural way to lift disagreements that has often proved useful in the history of science.
I postulate that the normative axiom proposed here will be consensual, at least for some non trivial concrete instanciations of the proposed approach, among economists, practitioners, external observers and \acp{DM}. Furthermore, its simplicity and a priori obviousness are strong points in its favor. It has, in some intuitive sense that cannot be described formally, less “content”, in the sense of that word used when economists argue that their theories should not have content about the preferences of the individuals (for fear of being paternalistic or dictatorial). Of course, any decision theory has content in a wider sense of imposing some structure to preferences, otherwise, the theory is void. The question is: are the restrictions imposed of a kind that can be judged acceptable with respect to the considered applications of the theory? Favoring judgments that resist counter-arguments, as done here, seems to me to capture exactly what is desired in a prescription context.
%Arguably, the normative axiom proposed here is much weaker than those of a theory such as \ac{SEU}, in the sense of having less substance. (This claim can only be valid at an intuitive level: because \ac{SEU} does not use arguments, its axioms cannot be compared formally to mine.) 
The lack of concrete substance in this normative axiom does not prevent theories of \ac{DJ} to propose substantive, non-trivial claims about reflective preferences. It simply moves the debate from the world of discussions between experts about the validity of the axioms to the world of observations of acceptability of arguments by the persons to which these theories claim to be applicable.
It is also worth noting that the proposed axiom has a specific form that may permit \acp{DM} to reconstruct the reasoning proposed by the theory so as to be able to check that they apply to their situation. Detailed exploration of this aspect is deferred to future work.

The rest of this article is structured as follows.
\Cref{sec:approaches} defines more precisely what is meant here with normative and descriptive theories. These definitions permit to readily see why the apparent impossibility of a claim being both normative and descriptive is an illusion, thereby enabling a solution to what could be called the “normative against descriptive” gap. It also reviews related works in the literature, and briefly sketches the current proposal.
\Cref{sec:normative,sec:descriptive} define the theories of \ac{DJ} and prove that these theories are normative and descriptive in the sense defined in \cref{sec:approaches}. \Cref{sec:preferences} define theories of \ac{DP} on the basis of theories of \ac{DJ}, and \cref{sec:conclusion} concludes.

\section{Eliciting reflective preferences}
\label{sec:approaches}
\subsection{Species of theories}
For this article, a \emph{normative decision theory} is a theory that, when applied in the situations it self-describes as applicable to, claims to yield valid prescriptions. I consider a prescription in the wide sense: recommending against some behavior, or recommending conditionally, also counts as prescribing. 

A \emph{descriptive decision theory} is a theory that, when correctly applied in the situations it self-describes as applicable to, claims to predict the decision behavior of a given set of individuals. For this article, a claim is a prediction only when it is in principle possible to show that it is false using empirical data. 

I will simply write normative theory and descriptive theory from now on to mean normative and descriptive decision theories.

For example, the \ac{SEU} theory of Savage is a normative theory. It may be used to recommend conditionally some behavior: “if you choose this lottery rather than that lottery, then you ought to \emph{not} choose this other lottery rather than that other lottery”.
Examples of descriptive theories include the many variants of Prospect theory.

These definitions avoid to define the concept of theory itself, for defining precisely such a fundamental concept raises serious difficulties. Is Prospect theory in general a theory, or only some of its variants, sufficiently precisely defined? Can a theory be extremely specific, or is some level of generality required to deserve being called a theory? Must a theory consist in a non-trivial assembly of laws, some deducible from others? It is not necessary for the purpose of the present discussion to take position on these issues, rather, the focus is on the distinction between normative and descriptive approaches. An intuitive understanding of the concept of theory will suffice.
Similarly, I sidestep insuperable difficulties pertaining to capturing in a short definition the concept of a descriptive theory in general: some descriptive theories in a more general sense may propose to describe without predicting, for example (historical decisions or trends come to mind); such endeavor simply does not enter the scope of this discussion. For similar reasons, I do not claim that the term normative is unanimously used in the literature as defined here. The present definition encompasses theories used in approaches that do not necessarily self-describe as normative \citep{bell_decision_1988, meinard_rationality_2019}; conversely, some theories considered as normative in the literature, such as some social choice theories, may not fall into the definition proposed here, depending on how they are interpreted.

I say that a normative theory is also a descriptive theory when the prescriptions of the normative theory are also claimed to predict the decision behavior of a given set of individuals.

Another contrast can now be introduced: the contrast between shallow and reflective behavior. \emph{Shallow behavior} is defined as the everyday, usual behavior of an individual. 
This is the behavior whose observation defines what \citet[p.\ 16]{von_neumann_theory_2004} describe as the “intuitive” preferences of an individual, those originating from an “immediate sensation” (the quotes from their article used here are cited by \citet{fishburn_retrospective_1989}).
\emph{Reflective behavior}, by contrast, is defined as the behavior that enacts informed, pondered decisions, those that would be taken after due consideration.
% an individual has consciously adopted after having pondered everything that is possibly relevant to the decision. 
%This pair of definitions is, again, vague and unsatisfactory for a deep philosophical discussion, but hopefully sufficient for the present one. 
%What is meant by “pondering about everything that is possibly relevant to the decision” will be defined more precisely later; an intuitive understanding of this phrase will do at this stage.
This distinction draws on our capacity, as human beings, to think about our shallow preferences and consider some of them as mistaken \citep{frankfurt_freedom_1971}. 
A famous example of such a situation is the one of \citet[pp.\ 101--103]{savage_foundations_1972} having realized, after due consideration, that his initial declared preference in a decision problem presented by \citet{allais_so-called_1979} was mistaken.

This contrast raises a new distinction, which appears by simple combination of the just given definitions: a \emph{shallow descriptive theory} is a descriptive theory that claims to predict the shallow behavior of individuals; a \emph{reflective descriptive theory} is one that claims to predict the reflective behavior of individuals. 

The descriptive theories proposed in decision theory, in the sense defined here, are mostly shallow descriptive theories, to the best of my knowledge (\cref{sec:purification} discusses the particular case of “preference purification” approaches).
%This need not be so. Indeed, this article introduces descriptive theories about reflective behavior.
This observation permits to understand what could be called the “normative versus descriptive gap”, referring to the belief, implicit in the decision theoretic literature, that a theory cannot be both normative and descriptive. Because normative theories aim to prescribe, they are required to not yield recommendations that precisely, systematically match shallow behavior. Given the numerous results in experimental psychology showing that shallow behavior is undefensible in many situations, a normative theory whose prescriptions would systematically match shallow behavior would be suspicious, to say the least. It would also be of low practical interest.
It is an immediate consequence of this minimal requirement that a normative theory cannot also be a shallow descriptive theory. 
In other words, as \citet{hausman_preference_1994} illustrate with an example of a person drinking a fatal poison while erroneously thinking that it is water, shallow behavior is not sufficient to inform normative economics.
The “normative versus descriptive gap” comes from an implicit identification of descriptive theories with shallow descriptive theories. 

This gap can thus be overcome by considering reflective descriptive theories, instead of shallow descriptive theories. This is the path this article adopts.

%This is done by defining on the one hand a non-observable set, containing those propositions that are valid prescriptions; and on the other hand an observable set, containing those propositions decisively defended by arguments, as judged by a given individual. This latter set is called the individual’s \ac{DJ}.
%A normative axiom, that any theory of \ac{DJ} rests on, links these two sets by stating that what is in someone’s \ac{DJ} is a valid prescription.
%This justifies the normative aspect of the theories. 
%The theory yields descriptive claims thanks to the 

\subsection{The proposed approach}
The proposal elaborated in this article defines theories that are both normative, and descriptive of reflective behavior.
The kind of behavior of interest here is choice behavior, although the proposal is more general. 

In fact, the main object of study of this article is judgments about propositions. The main goal is accordingly to study reflective adherence to propositions.
This will then be applied to determining reflective choice behavior. Choice behavior is obtained when considering adherence to propositions of the form: “I ought to choose alternative $\alt$ rather than $\alt'$ if both are available”. The alternatives can be concrete objects such as consumable goods.

One of our main preoccupation will be to clarify, using formal and precise definitions, the intuitive notion of reflective adherence to propositions. This is done using the concept of \emph{deliberated} judgments (a normative concept) and of \emph{acceptable} propositions (a descriptive concept).

A theory of \ac{DJ} applies to a set of individuals and a set of propositions. The \ac{DJ} of an individual $i$ is the set of propositions $\prop$ such that it is legitimate to recommend to $i$ adherence to $\prop$. Someone’s \ac{DJ} is not directly observable.
However, we can observe the reactions of $i$ to arguments in favor and against the propositions.
A theory of \ac{DJ} assumes that it knows a set of arguments such that exploring the reactions to these arguments “sufficiently” permits to obtain knowledge about someone’s \ac{DJ}.
Thus, one important task dealt with in \cref{sec:normative} will be to define the notion of acceptability, that serves to capture formally this notion of “sufficient” exploration. Intuitively, a proposition is acceptable when some argument makes $i$ seem (according to the observations) to adhere to the proposition whatever the counter-arguments given to $i$, and whatever the sequence of arguments given to $i$ previously. 

A theory of \ac{DJ} must use an axiom that relates the descriptive notion of acceptability to the normative notion of being deliberated. This axiom simply mandates that whatever is acceptable, is deliberated. \Cref{sec:normative} presents it formally.

This defines theories of \ac{DJ} that can be considered normative, as accepting the axiom is sufficient to determine unambiguously a set of legitimate prescriptions to be given to individuals. 
This is done by linking \acp{DJ} to acceptable propositions, a notion whose major interest is that it can be confronted to empirical observations. But at this stage, it is still unclear how to obtain knowledge of someone’s \ac{DJ}, and accordingly, such theories cannot still be considered as descriptive. In particular, they are not falsifiable, as they take no position on the reactions of individuals to arguments. \Cref{sec:descriptive} accordingly defines descriptive theories of acceptable propositions, assuming given a normative theory of \ac{DJ}. 

Such descriptive theories can take multiple forms, and some of them would permit to see clearly that the theory is descriptive in the required sense, that is, that its claims can be falsified given observational data. 
But the task we face, for such theories to be useful, is more complex than this minimal requirement: we have to find forms of theories that are not hugely demanding on the part of the theorist. In particular, we do not want to oblige the theorist to predict the whole of the argumentative attitude of individuals if this is not required to obtain information about her \ac{DJ}. Thus, \cref{sec:descriptive} defines descriptive theories of \ac{DJ} that only claim to be able to answer counter-arguments, in a sense that will be made precise. The desire that its claim be falsifiable is also defined formally. More formally, the result that is obtained in that section, and that is the central result of this article, is that if a theory resists all empirical falsification attempts, then it is valid, assuming acceptance of the normative axiom described here above informally (and formally in \cref{sec:normative}). A theory is valid if the propositions that it claims are in someone’s \ac{DJ} are indeed in that person’s \ac{DJ}. This central result thus amounts to prove the main claim of this article, namely, that accepting a weak normative axiom, supplemented with empirical claims, is enough to capture someone’s reflective preferences.

One long-term goal of this research program is to give individuals the means to reject theories that they find unconvincing after due consideration of all arguments; and accordingly, give incentives to decision theorists to build defenses of these theories able to convince individuals of the soundness of these theories. \commentYM{phrase trop longue}This might match \possessivecite{hausman_debate_2010} suggestion that we should use rational persuasion rather than nudge individuals, arguing that it would promote autonomy.
It also follows \possessivecite[p.\ 72]{guala_logic_2000} suggestion (discussing the Allais paradox) about obtaining relevant data to compare normative behavioral theories: “Before asking whether agents want to revise their choices or not, \emph{both} rival theories should be proposed” (emphasis in the original text).

Let us now examine how the approach proposed here to capture reflective preferences differs from those proposed in the literature.

The next two sections review existing approaches aiming at obtaining a model of the reflective preferences of an individual. They distinguish two kinds of approaches. \Cref{sec:purification} mentions formal techniques, that can be applied by following an explicit procedure, but that do not consider the reaction of individuals to arguments; while \cref{sec:informal} reviews studies that involve informal discussions with individuals, which do consider arguments, but require involving experts. By contrast, this article presents theories that can be confronted to empirical data in a formal, systematic way, and that consider the reactions of individuals to arguments.

Before delving into these two categories of approaches, let me mention a few others that, though related to the topic discussed here, fall outside the scope of this article.

Some authors have proposed to base normative theories on observable aspects of nature and individuals; for example, \citet{kahneman_back_1997} propose to measure “experienced utility”, considered as a hedonic quality. 
To the best of my knowledge, these proposals never aim at observing reflective preferences as considered here and thus fall outside the scope of this article.

The field of formal argumentation theory has developed technical approaches to deal with arguments. Such endeavor starts with an “attack” relation, which indicates which arguments argue against which. Various criteria can then be mathematically defined, in the form of a function selecting subsets of arguments given an attack relation, typically representing sets of arguments that “best resist” their counter-arguments. This field has greatly inspired this proposal, in particular, the notion of attack and of reinstatement. This field however does not view the attack relation as observable, but considers it as given, and its methodology is therefore not empirical in the sense seeked here. (\Citet{cailloux_formal_2019} elaborates somewhat on this difference of objective.) 
\commentYM{ici aussi, je trouve la mention de ces approches maladroite, car elles sont mises en exergue comme si elles étaient des manières de traiter le problème qui t'intéresse, ce qu'elles ne sont pas}

The field of Empirical social choice \citep{gaertner_empirical_2012} investigates individuals’s judgments about fairness: in a typical laboratory experiment, subjects are described some setting requiring to distribute resources among a set of individuals, and are asked to choose a distribution that they consider fair. These interesting approaches permit to study empirically the sense of fairness of real individuals, and compare them to social choice theories. They are distinct from the currently proposed approach as they seek to obtain their a priori sense of fairness, rather than aim at observing how subjects would appreciate (possibly counter-intuitive) claims about justice when they are backed by (possibly elaborate) arguments; and do not specifically aim to prescribe to an individual.

The phrase “\acl{DP}” has been used by \citet{orchard-webb_deliberative_2016, kenter_deliberative_2017} in the context of group deliberation. As used in this article, the concept of deliberation does not necessarily refer to a group deliberation: this approach is agnostic as to the provenance of the arguments.

The idea of using axiomatized decision theories to systematically generate explanations targeted to non experts has received much interest recently \citep{labreuche_general_2011, belahcene_explaining_2017, belahcene_accountable_2018, belahcene_comparing_2019, cailloux_arguing_2016, boixel_automated_2020, peters_explainable_2020, procaccia_axioms_2019}. These works are most interesting from the point of view of producing concrete theories of \ac{DJ}, but do not discuss how to systematically exploit decision makers’s judgments about the strengths of texts promoting different alternatives. They should therefore be viewed as complementary to this proposition.

\subsection{Preference purification}
\label{sec:purification}
An approach much discussed recently is the one of preference purification \citep{bernheim_beyond_2009, rubinstein_eliciting_2011, infante_preference_2016}.
Having collected shallow preference data (from the field or from interviews), purifying preferences consists in transforming them, using a formal process, in such a way that they satisfy axioms generally considered as required by rationality, or keeping only preference data collected in some situations considered appropriate to determine welfare-relevant preferences. 
For example, \citet{bleichrodt_making_2001} transform observed preferences in order to obtain a classical Expected Utility model; \citet[Section VII]{bernheim_beyond_2009} propose to exclude situations where the individual seems (by objective criteria) unable to process information considered (by the theorist) as relevant information.%; \citet{rubinstein_eliciting_2011} study the possible “rational” preference relations that are coherent with the observed data when assuming some distortion process relating (shallow, observed) and (reflective, unobserved) behavior. 
\commentYM{les deux mécanismes évoqués sont très différents, et ont des enjeux différents. Je pense qu'il serait utile de les distinguer, et je pense qu'il faudrait en dire plus sur la nature de la transformation et du filtrage, sans quoi la discussion des ces approches est trop superficielle}
\commentOC{Je ne vois pas comment faire ça sans en faire une digression énorme. Il faudrait introduire des notations, des définitions, … Je vois à peu près comment les présenter sous un cadre unifié et les comparer axiomatiquement aux théories du DJ, mais cela devrait faire l’objet d’un autre article, il me semble.}

These proposals leave unspecified how to confirm or disconfirm them empirically, thus, in particular, how to empirically investigate disagreements between theories: if two theories, on the basis of the same observed data, postulate that they correspond to different purified preferences, the disagreement may at best be arbitrated using ad-hoc, case by case criteria to assess for their relative plausibility. By contrast, the present proposal indicates a clear, systematic empirical way to arbitrate disagreements.
%These proposals deserve a detailed comparison to the current one, but this requires a treatment that cannot be given in this article. Let me simply highlight one major difference. which can be done with a high level description of these approaches using a variant of the framework that \citet{rubinstein_eliciting_2011} describe.
%Consider a set of possible observations $S$, describing shallow behavior, and a set of possible “rational” preference relations $R$. The set $S$ may contain observations of choice instances, for example, given the bundle $B$, the individual chose $b \in B$; or may contain all possible linear orders over some set of alternatives. The set $R$ contains all relations that satisfy some set of axioms of rationality, representing what the modeler considers as an acceptably purified relation. \Citet{bleichrodt_making_2001}, for example, use for $R$ the preference relations satisfying \ac{SEU}.

%The purification process amounts to define a function $f: \powerset{S} \rightarrow \powerset{R}$, where $\powerset{A}$ represents the set of subsets of a given set $A$. The function $f$ relates a set of observations to their possible purified preference relations. The argument of the function represents observations obtained using different frames. se could be preference orders 
%This \commentYM{what? ici on voit qu'on gagnerait à savoir concrètement quelle est la manip déployée par les auteurs ci-dessus}creates a risk of mis-representing the reflective preferences of the individual: not only is it not known whether the individual would reflectively want to adhere to these axioms 
%\commentYM{quels axiomes ? besoin d'expliquer + en détail ce que font ces auteurs ; et tu sembles pressupposer ici qu'adéhrer aux axiomes et s'approprier une transformation ou un filtre basé sur des axiomes, c'est la même chose, ce qui ne me semble pas évident}
%and, if so, how she would reflectively adapt her preferences when her shallow preferences exhibit related inconsistencies; but it is not even discussed how in principle this could be known \commentYM{phrase en + super longue : arrivée à la fin, je suis perdu}. As a result, a theory postulating constraints on individual’s behavior \commentYM{je ne comprends pas de quelles contraintes sur le comportement il est question ici. Les théories des laundered preferences que je connais n'imposent aucune contrainte au COMPORTEMENT. Cela souligne encore une fois qu'il faudrait être plus prolixe sur la description des approches dont tu parles}on the basis of rational requirements is subject to the criticism addressed by \citet[p.\ 228]{dold_toward_2018} to libertarian paternalism, in that it “effectively denies the possibility that deviations from rational choice might be something to be acknowledged as reasonable”. \commentYM{phrase bien alambiquée, qui fait référence à un concept ("libertarian paternalism") que je pense abscond pour bien des gens (dont moi). Je pense que tu devrais reformuler, donner l'argument plus directement, et si tu y tiens mettre entre parenthèse que cet arguement est utilisé dans un autre débat par x et y quand ils parlent de LP}
In other words, it is left unclear how, from a given observation of shallow behavior, one may obtain empirically information about reflective (or purified, in the terminology of this literature) preferences, or about the axioms they should satisfy. Authors endorsing this approach sometimes recognize that this is an important methodological problem. \Citet{bleichrodt_making_2001}, for example, state that their proposal require controversial assumptions (p.\ 1500), and recommend it only as a last resort, recommending discussing with the decision-maker whenever possible to elicit her reflective preferences (p.\ 1499). Note that nudging \citep{thaler_nudge_2009} creates a similar problem: we have to guess what the individual would like if he thought hard about the decision problem, was well-informed, and actively worked against the psychological features he himself does not consider adequate for good decision making.

Some axioms arguably seem innocuous. For example, Pareto-dominance requires that $a$ be weakly preferred to $b$ if $a$ is at least as good as $b$ on every point of view that matter. However, as has been amply remarked, what is required is not merely the correctness of the axiom in the abstract, but its correct application to the decision situation. Has a point of view that the decision maker considers relevant been forgotten, leading the analyst to mistakenly think that $a$ Pareto-dominates $b$, and the seemingly impeccable conclusion that $a$ ought to be preferred to $b$ turns incorrect. \citet{sen_maximisation_1997} illustrated this reasoning (with another axiom) with his well-known example of a person picking her second most-preferred fruit from a basket out of politeness reasons, which could be considered irrational by an analyst blindly applying WARP. As \citet[p.\ 40]{lecouteux_reconciling_2015} puts it (citing \citet[p.\ 13]{bacharach_beyond_2006} for the part in italics), and as analyzed by \citet{sen_information_1986, sen_internal_1993}, “whether a decision-theoretic principle of rationality has been violated \emph{depends on how we, the theorists, ‘cut up the world’}” \TODO{Find that book}.\commentYM{Je § tout entier me semble mal placé ici. Les considérations évoquées sont relatives au choix d'un "bon" axiome, problématique classique dans les discussions des théories du choix rational, mais dont le lien avec le laundering des préférences ne crève pas les yeux. Il y a des moyens de faire le lien, mais là tu n'expliques aucun lien. Donc ton § tombe comme un cheveux sur la soupe et le lecteur est embrouillé, ne se rappelle même plus quel était l'object de la sous-sous section qu'il vient de lire}

\subsection{Informal deliberation}
\label{sec:informal}
Another path to obtain reflective preferences has been followed: the one of deliberation, an umbrella term by which I designate processes, either, involving discussions between an analyst and a decision-maker while querying her preferences in view of building a formal model of them \citep{raiffa_back_1985, keeney_decisions_1993, roy_multicriteria_1996, belton_multiple_2002}, or involving group discussions, with presentations from experts to an audience and internal deliberation phases \citep{fishkin_when_2011, chappell_deliberative_2012}, aiming to obtain well-informed preferences. 
The term as used here also includes the informal use of debiasing techniques: techniques that aim at encouraging individuals to think more about their decision problem in order to reduce occurrence of behavioral “mistakes”. Examples include suggesting the individuals to consider alternative points of view, or asking them to give reasons for their choices (an insightful article by \citet{mitchell_libertarian_2004} cite many such studies). 
\commentYM{il me semble nécessaire ici de mentionner aussi tous les dispositifs participatifs et de concertation employés en routine dans les prises de décisions relatives aux politiques publiques et leur mise en oeuvre}
\commentOC{Je m’interroge. Ça me semble de plus en plus éloigné du scope, car je m’intéresse aux préférences d’un individu donné, hors questions d’agrégation.}

A major problem of these techniques is that they are not easily reproducible \commentYM{il faudrait expliquer en quel sens}, because they require informal interactions with experts. Furthermore, the resulting preferences elicited from the subjects might reflect properties of the personalities of the intervening experts or group-dynamic effects rather than true reflective preferences of the individuals. 

This problem is compounded by the phenomenon of path dependency, or arbitrary coherence, exhibited by psychologists \citep{ariely_predictably_2010}.  This term denotes situations where individuals form an apparently coherent system of preference that depends on the path taken to elicit it. 
\commentYM{phrase beaucoup trop longue pour être compréhensible, qui fait par ailleurs référence au concept de rationalité sans le définir, ce qui est acceptable en début d'intro, mais me semble problématique à ce stade dans un papier qui cherche à éclairer les débats flous par des clarifications formelles et conceptuelles}. 

It also remains to be clarified how to check on a principled, systematic basis, whether all relevant arguments have been presented faithfully. Comfronted with 
%(possibly self-designated \commentYM{ceci est un tout autre problème, dont la mention ici brouille le propos}) 
experts with enough assertiveness, a decision-maker can be convinced to follow principles of decision-making that several experts in the field of decision-theory would consider incorrect. 
%As an illustration, after discussing the weaknesses of the \ac{AHP}, \citet[p.\ 53]{howard_foundations_2007} write \commentYM{s}: “Why, then, do inferior processes find favor with decision makers? The answer is that they do not force you to think very hard or to think in new ways.” \commentYM{je ne vois pas en quoi ceci est censé illustrer le propos précédent (ou alors je n'ai rien compris à la citation). J'ai l'impression que la citation dit que, parfois, il y a des forces psychologiques qui s'opposent à l'adoption d'un processus de décision, quand bien même celui-ci aurait des qualités indéniables}
Beyond the controversial \ac{AHP} mentioned in the introduction, strongly criticized, for example, by \citet[p.\ 53]{howard_foundations_2007}, famous decision analysts have voiced disagreements with normative theories of decision making \citep{ellsberg_risk_1961, allais_so-called_1979} that are however generally applied in the field. \commentYM{il me semble ici intenable de faire l'économie de définir ce que tu appelles "normative theories"} 
\Citet[p.\ 669]{ellsberg_risk_1961} concluded his influential article as follows: “[In some situations, for some people,] the Bayesian or Savage approach gives [...] bad advice. [Some people] act in conflict with the axioms deliberately, without apology, because it seems to them the sensible way to behave. Are they clearly mistaken?” \commentYM{d'une manière générale, dans toute la section 1, l'avalanche de citation me semble nuire au propos. Cela induit des ruptures dans le style, et force le lecteur à se pencher sur des questions d'interprétation des citations qui l'empêchent de se focaliser sur ton propos à toi}
This rethorical question illustrates the need of systematic procedures able to determine whether decision-makers, when confronted with the relevant arguments, accept the soundness of the principles that the recommendation rests on (this has been somewhat explored, but not systematically, for lack of a systematic way of confronting decision makers to arguments \citep{slovic_who_1974, stanovich_discrepancies_1999}). 

Another way to view this problem is that different analysts, using different methods and different axioms as their favorite conception of rational decision-making, will obtain possibly contradictory conclusions, “elicited” from the decision-makers who hire them. Some analysts admit this possibility and do not consider it to be a problem. \commentYM{il me semble que tu soulèves là encore un nouveau problème : comme la notion de rationalité n'est pas claire, il y a débat sur les axiomes qu'on peut vouloir lui associer, et du coup différentes approches, choisissant différents axiomes, vont éliciter des préférences différentes. Je te rejoins sur le fait que c'est un problème important, mais il est différent de ce qui a été discuté juste avant, et tu ne l'explicites pas. Il me semble qu'il faudrait bien distinguer les problèmes, et bien les formuler, et il faudrait que le plan soit plus clair et lisible. Là, en lisant, je ne sais pas si tu considères ce que tu abordes dans ce § comme un nouveau problème, ou comme un aspect ou une implication, du problème exploré précédemment}
\commentOCf{C’est discutable si c’est un autre problème ou un peu le même problème ou carrément le même problème vu autrement. Du point de vue logique, toutes les propositions fausses sont équivalentes. En l’absence d’une analyse comparative approfondie de toutes ces approches et de leurs propriétés, je ne vois pas à ce stade comment faire beaucoup plus clair. Mais je serais heureux d’en discuter si tu vois une manière de corriger le problème raisonnablement facilement.}
For example, \citet[p.\ 214]{roy_comparison_1995} write: “In our view, these inevitable disagreements do not imply that decision-aid is useless but simply that a single problem may have several valid responses. Given that two different decision-aid models cannot be implemented in the same decision process, the decision-maker must be conscious of the qualitative choices implied by the different models – often conveying the analysts’ own ethical choices – before coming to personal conclusions on the choice to be made. In this domain, the many different approaches reflect in our view the complexity of the researcher's task much more than a scientific weakness.”\commentYM{ici encore, je fais une overdose de citations. Et je ne trouve pas celle-ci particulièrement éclairante pour ton propos. Certes il y a un lien entre ce que tu dis et ce que dit cette citation, mais elle n'est pas vraiment une illustration de ce que tu dis. Du coup elle brouille le message}
\commentOCf{J’ai un peu réduit les citations, j’espère que ça va mieux. Concernant celle-ci en particulier, elle me semble très utile. Je trouve stupéfiant qu’on puisse trouver ces propos dans la littérature, écrits noir sur blanc. Si je ne les cite pas, un lecteur pourrait (ou devrait) ne pas me croire quand j’affirme que des gens trouvent que deux études soi-disant scientifiques puissent arriver à des conclusions apparemment contradictoires et être toutes les deux valides. La citation étendue permet de comprendre, je crois, leur point de vue, et de se faire une idée de ce qu’ils veulent dire par là.}
Although being an important insight about the (likely vain) hope of determining a single correct recommendation in many complex decision situations (about which more below), this insight does not address the problem that some principles of decision making may be wrong guides in some situations, possibly including some principles that are commonly applied in the field, hence the need of principled ways of discriminating them. \commentYM{est-ce vraiment ton projet ? as-tu l'intention de trouver la meilleure des axiomatiques de décision ? si c'est ton projet, il faut le dire clairement. Ce que je comprennais jusqu'ici était que tu cherchais un moyen de connaître les "preferences that represent informed, pondered decisions, that would be taken after due consideration by the individual", ce qui me semble être un tout autre projet}
\commentOCf{En gros, je propose de trancher entre approches normatives en utilisant des données empiriques, plutôt que simplement la force de conviction des axiomes (qui parlent différemment à différentes personnes et peuvent conserver les désaccords intacts). Je ne sais pas si c’est plus clair maintenant.}

As a consequence of these techniques not being completely formalised and reproducible, each application requires manual intervention \commentYM{quelles sont ces "interventions manuelles" ? à expliquer}of (very skilled) experts in order to elicit reflective preferences interactively. Eliciting reflective preferences therefore requires enormous resources. \Citet[p.\ 572]{pinto_normative_2012} observe that “in many of those surveys that governments use in order to inform social policy there are not enough resources (time and money) in order to reach the stage where preferences are not labile”. There is, to reduce this cost, a need for theories that can be iteratively improved upon and yield reusable knowledge about reflective preferences. \commentYM{Cette idée de limiter le coût est intéressante, et je me dis qu'elle peut faire mouche avec certains lecteurs. Mais en l'état ça ne fonctionne pas car tu ne expliques pas quels sont les coûts impliqués dans les approaches "manuelles"}
\commentOC{J’ai reformulé légèrement, j’espère que ça suffit. Puis-je m’en tenir à l’idée qui me semble évidente que faire intervenir toute sorte d’experts pour discuter avec des gens chaque fois qu’on veut leur préférence coute cher ?}

\subsection{Previous proposal to study \aclp{DJ}}
The current proposal pursues a path started with \citet{cailloux_formal_2019}, an article that defines the notion of \ac{DJ} and describe how \acp{DJ} can be observed. 

That previous proposal postulates that an individual $i$, whose reflective preferences are being modeled, can adopt multiple perspectives. The proposal postulates that the modeler can observe two relations by interacting with $i$, called support and trump. They constitute the argumentative disposition of an individual $i$ and are defined as follows. 

An argument $\ar$ supports a proposition $\prop$ iff $i$ considers that $\ar$ is in favor of $\prop$. This notion is conditional, in the sense that it does not say anything about the validity of $\ar$; only that if $\ar$ holds, then $i$ should endorse $\prop$.

An argument $\ar_1$ trumps another argument $\ar_2$ iff there is at least one perspective within which $i$ considers that $\ar_2$ turns $\ar_1$ into an ineffective argument. (This notion is formally decomposed into two relations, “trump in some perspective” and “does not trump in some perspective”, but these details do not need to be discussed here.)

That proposal then defines a number of notions, including the central notion of the \ac{DJ} of $i$, on the basis of her argumentative disposition. An individual’s \ac{DJ} is the set of propositions that resist counter-arguments, in some technical sense defined on the basis of these relations.

The current article improves on several aspects over this previous proposal.

First, the previous proposal requires the individual to declaratively take position on notions that may be unclear to him, as they include some possibly counterfactual situations, and as the concept of turning an argument into an uneffective argument may not be understood in the same way by everybody. 
In order for the conclusions about $i$’s \ac{DJ} to hold, we must be sure that $i$ understands the semantics of the relations on which it is based as analysts do. It is therefore desirable that these notions have unambiguous semantics. It is undoubtedly an important strength of \citet{samuelson_foundations_1983}, \citet{von_neumann_theory_2004} or \citet{savage_foundations_1972}’s theories that they use as primitive notions observable entities whose semantics is unambiguous. This is not to say that these proposals cannot be criticized. For example, the claim that the notion of preference can be disposed of entirely has been much criticized \citep{hausman_preference_2011, wong_foundations_2006}, with good reasons; and observations of shallow behavior, though having a clear semantics, are insufficient to deduce normative claims, as has been amply remarked and as argued in this very article. This is also not to say that declarations should never be used and that only choice acts should matter to any theory in economics.
Nonetheless, as \citet[p.\ 16]{von_neumann_theory_2004} aptly summarize, “It is clear that every measurement – or rather every claim of measurability – must ultimately be based on some immediate sensation, which possibly cannot and certainly need not be analyzed any futher”. A weakness of the previous proposal for studying \acp{DJ} is that the observable relations it uses may be considered inadequate primitive notions. This article improves on this aspect by postulating a non-ambiguous observable relation, that does not require $i$ to take position declaratively on relative qualities of arguments; but rather to choose (by declaration of preference or by act) among bundles, or to declare whether $i$ adheres to a given proposition in her current perspective. (In the latter case, the clarity of the semantics of the observable relation will depend on the clarity of the propositions.)

Another central aspect of the study of \acp{DJ} is that individuals may change perspective, thus, change opinion about whether an argument trumps another one. An important case of change of perspective is when the current perspective depends on which arguments $i$ has seen so far. The previous proposal studies these changes of perspectives through the trump relation, without modeling explicitly the arguments seen so far. This has the advantage of simplicity, but the price to pay is a lack of ability to reason explicitly about path dependency mechanisms: individuals may form stable opinions that differ according to, for example, the first argument they have seen. In such a case, although their opinion may stabilize (in the sense of resisting counter-arguments), it is still arbitrary. This article models paths (sequences of arguments) explicitly and prohibits such situations of arbitrariness to count as forming \acp{DJ}.

This article also clarifies the normative and descriptive status of the theories of \acp{DJ}. It introduces an explicit axiom which gives them their normative strength. This axiom is supplemented by claims, which are shown to be falsifiable by empirical data. The previous proposal defines conditions and discusses several possible interpretations of them, including empirical ones, but leaves open the question of how to test these conditions systematically using the observable relations.

\commentYM{la vision vNM est très réductrice, et ne peut pas être considérée, me semble-t-il, comme valable pour toutes les théories normatives, car l'idée qu'il doit y avoir des sensations, atomiques et immédiates relève de l'empirisme logique qui sous-tend leur approche, et cette épistémologie, ou ontologie à ce stade, est loin d'être partagée par toutes les approches à même de développer des théories normatives. Ta phrase qui suit risque de n'être pas beaucoup plus convaincantes pour les lecteurs, je le crains. Imaginons par exemple une théorie marxiste, elle s'appuiera sur des intérêts matériels, et se contre fout des réactions des individus aux arguments. On peut envisager par ailleurs des théories supra-individualistes, qui se foutent des réactions des individus. Mais bien entendu la pertinence de ce que je dis là dépend de ta réponse à ma première question dans l'intro du 2. Si ce dont tu parles ici, ce n'est pas des théories normatives existantes ou possibles, mais uniquement des théories normatives QUE TU ACCEPTES COMME ELEMENTS DE TA CONSTRUCTION, alors le problème est différent : il faut justifier TON adhérence à l'empirisme logique, ce qui n'est pas une mince affaire étant donné le discrédit qui lui est aujourd'hui attaché} 
\commentOCf{Je soupçonne que le problème était dû à une formulation maladroite de ma part (\og{}The fundamental element of a normative theory of deliberated preferences is the reaction of an individual, whose preferences are to be captured, to arguments\fg). Si ma reformulation ne résoud pas le problème, je suis intéressé qu’on en discute.}

\commentYM{je ne pense pas qu'on puisse tenir pour acquis que le lectorat connait par coeur l'article delibenv. Il faut du coup donner plus d'explication. Par ailleurs, il est impératif que tu expliques pourquoi tu considères qu'il n'est pas raisonnable de considérer que trump et support sont observables} 
\commentYM{je ne pense pas qu'on puisse dire que l'approche RP est privilégiée en économie. Elle est aussi très critiquée. Si tu la considère particulièrement solide, il faut expliquer pourquoi}
\commentYM{il n'y QU'UN axiome au final ?}

Finally, the current proposal links \acp{DJ} and \acp{DP}. \commentYM{indispensable de dire explicitement quelle est la différence entre les deux dès maintenant}

\Citet{meinard_justification_2020} discuss some requirements on justification of recommendations, which gives further arguments in favor of studying \acp{DJ}.

\commentYM{arrivé à la fin de la section 1, je suis perplexe. Si ce n'est pas une introduction (ce que ça ne semble pas être, car ça ne s'appelle pas "introduction", qu'est-ce que c'est, et pourquoi n'y a-t-il pas d'introduction ? Si c'est une introduction, il faut l'appeler comme ça, c'est très long, il est inhabituel d'avoir des découpages en sous-sections en intro, il n'y a pas d'annonce de plan (ce qui manque cruellement), il me semble que la problématique n'apparaît pas clairement, le lien avec l'article delibenv devrait être plus clair. Par ailleurs, il me semble qu'il y a quelques trous importants dans la revue de la littérature. Je pense en particulier à R.E. Goodin, « Laundering preferences », in J. Elster \& A. Hylland (eds.), Foundations of
Social Choice Theory, Cambridge University Press, 1986, p. 75-102 ; J.C. Harsanyi, Rational Behavior and Bargaining Equilibrium in Games and Social
Situations, Cambridge University Press, 1977 ; S.-C. Kolm, Macrojustice, Cambridge University Press, 2005, chap. 19.; ainsi qu'à l'économie expérimentale à base de jeux comme N. Frohlich \& J. Oppenheimer, Choosing Justice. An Experimental
Approach to Ethical Theory, University of California Press, 1992 ; V. Clément \& D. Serra,
« Égalitarisme et responsabilité, une investigation expérimentale », Revue d’économie politique
111, 2001, p. 173-193.}

\section{Normative theories}
\label{sec:normative}
\subsection{Definition}
A normative theory of \ac{DJ} is a tuple $(\allprops, \allargs, I, {\allleadsto})$, where $\allprops$ is a set of propositions, $\allargs$ is a set of arguments, $I$ represents a set of individuals, and ${\allleadsto}$ is an observation protocol. Intuitively speaking, the theory claims that the proposed observation protocol, observing the reactions of individuals from $I$ to arguments from $\allargs$, is normatively appropriate to yield recommendations to individuals about the propositions from $\allprops$. These concepts are detailed in this section.
\commentYM{pourquoi le protocole d'obs est-il déjà évoqué ici ? j'aurais cru que ce protocole relève de la théorie empirique qui se greffe sur la théorie normative en question. Par ailleurs, je croyais que c'était l'axiome de la théorie normative qui devait être mis à l'épreuve, pas une instantiation sur une propositon. De l'axiome, il n'est plus question. Où est-il ?}.
\commentYM{il manque dans la "theorie normative" ce que j'aurais considéré comme le plus important : un modèle de la force de conviction des arguments et de leurs liens avec les propositions. J'aurais dit que l'axiome de la "théorie" se cacherait là-dedans}

The set $\allprops$ is called the agenda of the theory. Propositions composing the agenda are not further detailed, they are simply atomic elements. A proposition must be an element such that one can observe whether an individual adheres to it, or whether an individual’s behavior is consistent with it.
\commentYM{pourquoi est-ce que cela fait partie de la théorie ? c'est plutôt un des éléments du contexte d'une de SES APPLICATIONS à mon sens. L'expression  "normative model" serait selon moi bien plus appropriée si on veut que les détails du cas d'application fassent partie de l'objet. "Theory" a de très fortes connotations de généralité} 
Examples of propositions include: “I ought to pay at most \$100 for purchasing bundle $b$”, “I ought to acquire alternative $a$ rather than $b$ if it is available in the same conditions”; “I ought not to perform action $a$ because it is immoral”.

Another primitive elements of a normative theory is the set of arguments $\allargs$.
Arguments are defined here as anything that can be presented to an individual that may possibly influence her judgment. As an example, an argument can be a text presenting the features of an alternative 
\commentYM{donc un texte qui ne prend pas de position sur l'alternative, qui est purement descriptif, est un argument ? cela me semble raisonnable, mais ça mérite discussion}, or arguing for the (im)morality of an action. An argument may be purely descriptive or may forcefully argue in favor of the truth or falsity of one or several propositions.
The term argument generally evokes a debate, a metaphor appropriate for the comparison of theories that will be described later; but that should be taken with a grain of salt: an image, a tag price, a poem can be an argument in this article. 

The set of individuals in a normative theory represents the situations, socio-economic conditions, acquired diplomas, or other observable features of individuals that must be satisfied for the theory to apply to them \commentYM{cela ressemble plus à un domaine d'application, qu'à un ensemble d'individus auxquels ça s'applique. Dans ce cas pourquoi ne pas le dire clairement ?}.

The protocol of observation in a normative theory defines how the theory proposes to present a sequence \commentYM{qu'est-ce qu'une séquence ? y a-t-il une notion de temporalité ? de lien logique entre les arguments ? une nécessité que chaque élément de la "séquence" attaque le précédent ? les embranchements sont-ils exclus ? les cycles ? bref, de quoi parles-tu ?}
\commentOCf{C’est le \hrefblue{https://en.wikipedia.org/wiki/Sequence}{\underline{terme habituel}} en mathématiques, à ma connaissance.}
of arguments to an individual, and how to observe her resulting judgment about a proposition of interest. This protocol permits to determine whether a sequence of arguments “leads” a given individual to a given proposition. If the arguments are texts and propositions are of the form “I ought to acquire alternative $a$ rather than $b$”, an example of such a protocol is: present each argument (each text) in the sequence to the individual, asking her to read it carefully; then give the individual a choice between $a$ and $b$; the sequence of arguments is considered to having led the individual to the proposition “I ought to acquire alternative $a$ rather than $b$” iff he chose $a$ rather than $b$.

To define formally the notion of an observation protocol, the concept of a sequence of arguments is required.
Define the set of finite sequences of elements in $\allargs$ \commentYM{qu'est-ce qui lie les uns aux autres les éléments d'une séquence ? pourquoi est-ce que ça s'appelle une "séquence" ? une connotation temporelle ? une idée de succession ?}
(including the empty sequence) as $\allhist = \bigcup_{k \in \N} \allargs^{\intvl{1, k}}$.\commentYM{la notation $\allhist$ rend les choses inutilement difficiles à lire. Pourquoi noter ça comme on noterait $\allargs$ sans le zéro, si ça voulait dire quelque chose ? J'aurais inventé un symbole du genre $ \allargs^{\intvl{}}$ ; par ailleurs je trouve l'explication relative à l'empty sequence particulièrement alambiquée}
\commentOCf{Il me semble assez naturel de, étant donné un ensemble de caractères $S$, définir l’ensemble des chaines de caractères par $S^*$ (représentant le fait de prendre toutes les puissances possibles de $S$), mais c’est peut-être un tropisme d’informaticien qui ne sera donc pas compris par des lecteurs économistes, et effectivement l’étoile est aussi utilisée pour exclure zéro, ce qui n’a rien à voir. J’ai vu la notation $Seq(\allargs)$ être utilisée, mais elle me semble un peu longue, vu l’usage intensif de ce symbole ici. Oui, peut-être en viendrai-je à l’invention que tu proposes, mais inventer ses propres symboles n’est légitime que si aucun n’existe déjà. Je vais essayer d’interroger d’autres gens, dis-moi en attendant si tu penses qu’on peut s’y habituer ou si c’est vraiment très dérangeant.} 
Throughout the article, $\N$ includes zero. %and $\N^* = \N \setminus \set{0}$. 
A generic element of $\allhist$ is denoted by $\hist$ and called an argumentative path, or simply a path when this raises no ambiguity. Its length is denoted by $\card{\hist} \in \N$.
Given $j, l \in \N$, the notation $\intvl{j, l}$ represents $\set{k \in \N \suchthat j ≤ k ≤ l}$.
Thus, given an argumentative path $\hist \in \allhist$, $\forall k \in \intvl{1, \card{\hist}}$, $\hist_k \in A$, and $\hist = \emptyset ⇔ \card{\hist} = 0$.
%Note that $\intvl{1, 0} = \emptyset$, and given any set $S$, the notation $S^\emptyset$ represents the empty set. 

An observation protocol ${\allleadsto}: I → \allprops^{\allhist}$ maps individuals and paths to propositions. 
Equivalently, given any individual $i \in I$, ${\ileadsto}: \allhist → \allprops$ maps paths to propositions for this individual. 
Given an individual $i \in I$, a path $\hist$ and a proposition $\prop$, the path is said to lead to the proposition for this individual iff $\hist \ileadsto \prop$.\commentYM{la correspondance entre cette définition et la définition du protocole donné dans la partie informelle est pour le moins difficile à voir. Où cette défintion formelle spécifie-t-elle comment on observe quoi ? Il me semble que ça ressemble + à une définition formelle de ce que je disais plus haut manquer dans ta définition}
\commentOCf{Je ne comprends pas le problème, il faut qu’on en discute.}

This defines a normative theory of \ac{DJ}.

\begin{example}[A normative theory about two lotteries]
	\label{ex:twolott}
	In a famous article about preference reversals, \citet{lichtenstein_reversals_2006} discuss an example involving two lotteries. The lottery $P$ makes you win four dollars with 99\% chance and lose ten cents with 1\% chance; in summary, $P = (.99, +\$4; .01, −\$0.1)$. The lottery $D$ is, using the same notations: $(.33, +\$16; .67, −\$2)$. The letter $P$ stands for probability: one salient aspect of $P$ is the very high chance of winning. The letter $D$ stands for dollar: one salient aspect of $D$ is the high gain it gives (compared to $P$) in dollar amount if the player is lucky.

This example is interesting because when asked to bid for lotteries, the theory discussed by the authors predicts, and their experiments confirm, that subjects tend to bid more for $D$ than for $P$ (because the bidding question tend to make individuals focus on dollar amounts) whereas when asked to choose one lottery, the lotteries being presented in pairs, subjects tend to choose $P$ (giving a greater attention to probabilities) among the pair $\set{P, D}$.

Define $\allprops = \set{P ≥ D, P > D, D ≥ P, D > P}$, where $P ≥ D$ represents the proposition that $i$ should choose $P$ or declare to be indifferent when choosing among $\set{P, D}$; $P > D$, the proposition that $i$ should choose $P$ and not $D$ when choosing among $\set{P, D}$; and similarly for $D ≥ P$ and $D > P$.

Define $\ar_0$ as a text arguing that the choice between $P$ and $D$ should be according to their expected revenue, namely, $\$3.959$ for $P$ and $\$4$ for $D$. Here is a possible text for $\ar_0$. “Imagine you play this bet 100 times. On average, you would win 99 times and lose once, which means gaining 99 × \$4 = \$396 and losing 1 × \$0.10 = \$0.10, hence a total net gain of \$395.90. Thus, your average net gain for the $P$ bet would be \$3.9590. [Similar text for $D$.] Therefore, we suggest you choose $D$ for a higher expected revenue.”
Define $\ar_1$ as a text arguing that the utility of money should matter, not the revenue, and arguing that a loss hurts more than a gain. 
Omitting many other relevant arguments for now (such as arguing that utility should be linear for small amounts of money, for example), define $\allargs = \set{\ar_0, \ar_1}$.

Define $I$ as a set of individuals facing a choice between $P$ and $D$ in a laboratory experiment.

Define $\ileadsto$ as follows. Given an argumentative path $\hist$, $\hist \ileadsto (P ≥ D)$ iff, in a situation of $i$ having been exposed, in order, to the arguments in $\hist$, letting $i$ choose between $P$ and $D$ or declare indifference leads $i$ to choose $P$, or declare indifference. Here is a possible text presented to $i$ to ask for a choice: “Considering these arguments, would you please choose either to play $P$, or play $D$, or declare to be indifferent (in which case one will be picked randomly for you)”. 
Similarly, define $\hist \ileadsto (P > D)$ iff showing the arguments in $\hist$ to $i$ and letting $i$ choose between $P$ and $D$ or declare indifference leads $i$ to choose $P$ and not $D$. 
Define similarly $\hist \ileadsto \prop$ for $\prop \in \set{D ≥ P, D > P}$.

The tuple $(\allprops, \allargs, I, {\allleadsto})$ is an example of a normative theory of \ac{DJ}.
\end{example}

We now need an axiom that links the observable reactions of the individuals to their \ac{DJ}.
This will be done through the notion of a decisive argument, that is, intuitively, an argument that is able to convince an individual of the acceptability of a proposition whatever the argumentative path submitted to the individual. For formal definitions, some supplementary notations must be defined.

\subsection{Normative adequacy}
The \ac{DJ} of an individual $i$ is denoted by $\iprops \subseteq \allprops$. It represents the (a priori unknown) set of propositions that can legitimately be recommended to $i$. It is independent of any normative theory. A normative theory of \ac{DJ} claims that it knows part of the \ac{DJ} of the individuals $i \in I$.

As usual with binary relations, $\hist \nileadsto \prop$ means $(\hist, \prop) \notin {\ileadsto}$.

The judgment of an individual towards propositions may evolve during her exposition to sequences of arguments. Of interest to theories of \ac{DJ} are the cases where a so-called decisive argument exists for a given proposition: an argument is decisive, informally, when its presence towards the end of the sequence of arguments systematically leads the individual to accepting the proposition, whatever the sequence of arguments. 
Therefore, an argument is decisive when, intuitively, it is able to resist every counter-arguments. 

Given $\hist \in \allhist$, $\range(\hist) = \hist(\intvl{1, \card{\hist}})$ denotes the range of $\hist$, that is, all arguments contained in the sequence $\hist$.
Let $\histend = \hist(\intvl{\max(1, \card{\hist} - 1), \card{\hist}})$ denote the set containing the last two arguments of the sequence $\hist$ if the sequence has at least two elements; the unique argument of the sequence if it has exactly one element; and the empty set if it is empty.

Given $i \in I, \prop \in \allprops, \ar \in \allargs$, define $\ar \ileadstost \prop$ iff $\forall \hist \in \allhist \suchthat \ar \in \histend: \hist \ileadsto \prop$. 
When $\ar \ileadstost \prop$, $\ar$ is said to be a decisive argument for $\prop$ from $i$’s point of view. 
This definition mandates that $\ar$ be effective (in the sense of leading $i$ to adhere to $\prop$, as determined by the observation protocol) whatever the sequence of arguments presented to $i$, so long as $\ar$ is presented “towards the end” of the sequence, that is, as a last or before last argument.
Given any $i \in I$, when some argument is decisive for some proposition $\prop$, thus, when $\exists \ar \in \allargs \suchthat \ar \ileadstost \prop$, the proposition is said to be acceptable, from $i$’s point of view.

The following axiom permits the theory to have bite on what should be recommended to $i$. It says that it is sufficient that a proposition be acceptable to be in $i$’s \ac{DJ}.
\begin{axiom}[Normative adequacy]
	\label{ax:norm}
	$\forall i \in I, \prop \in \allprops: 
		[(\exists \ar \in \allargs \suchthat \ar \ileadstost \prop)] ⇒ \prop \in \iprops.$
\end{axiom}
\commentYM{où est défini $\iprops$ ? Par ailleurs, je pense qu'il faut expliquer le sens des axiomes, sinon l'article ne peut pas être convaincant ; enfin, pourquoi cet axiome s'appelle-t-il ainsi ? cela ressemble plus à un critère de validité empirique quà un critère normatif. Mais peut-être mon incompréhension vient-elle du fait que je ne comprend pas ce que dénote $\iprops$}

\begin{example}[Two lotteries (cont.)]
	Continuing \cref{ex:twolott}, the normative adequacy axiom is satisfied for that normative theory iff, for all $i \in I$, if there is a argument $\ar \in \allargs$ that leads $i$ to never choose $D$, facing any counter-argument and in the context of any preceding argumentative path, under condition that $\ar$ be included towards the end of the path, then we are ready to accept that $i$ deliberately judges that he should choose $P$ rather than $D$, and similarly for the other propositions $\prop \in \allprops$.
	
	As described in \cref{ex:twolott}, the set $\allargs$ is not wide enough to make this axiom reasonably acceptable: normative adequacy can only be considered a reasonable requirement when the set of arguments includes all arguments that can possibly change the opinion of at least one individual $i \in I$ about the considered topic $\allprops$.
\end{example}

To end the presentation of the normative theories of \ac{DJ}, it may be useful to analyze the status of contradictory propositions. \Cref{th:protcoh} shows that two incompatible propositions cannot both be acceptable. In supplement, it may be interesting to observe that theories of \ac{DJ} do not require to postulate that $i$’s \ac{DJ} is complete, in the sense that when the agenda is closed under negation, it must not necessarily be the case that either the proposition or its negation is in $i$’s \ac{DJ}. This justifies that the normative axiom asks only for an implication between acceptability and \ac{DJ}, instead of an equivalence.

\subsection{Coherence and completeness}
\label{sec:coh}
Two propositions $\prop, \prop' \in \allprops$ are said to be incompatible iff $\forall \hist \in \allhist: \hist \ileadsto \prop ⇒ \hist \nileadsto \prop'$. Let $\incompat \subseteq \Phi × \Phi$ denote the set of incompatible pairs of propositions.

The following proposition shows that \cref{ax:norm} cannot lead to incompatible propositions being considered deliberated. 
\begin{proposition}[Coherence of acceptability]
	\label{th:protcoh}
	Given any normative theory,
	$\forall i \in I, (\prop, \propbar) \in \incompat$:
	\begin{equation}
		[\exists \ar \in \allargs \suchthat \ar \ileadstost \prop] ⇒ [\nexists \ar \in \allargs \suchthat \ar \ileadstost \propbar].
	\end{equation}
\end{proposition}
\commentYM{de même que je trouve très ténu et elliptique le lien entre le protocole dans sa description informelle et dans sa définition formelle, de même je trouve très dérangeant de mettre de terme "protocole" dans le nom de ce genre de proposition. Que l'usage de ce terme ait été choisi par provocation ou non, il me semble indispensable de le discuter}
\begin{proof}
	If for some $\ar \in \allargs$, $\ar \ileadstost \prop$, then by definition of $\ileadstost$, $\forall \ar_1 \in \allargs: (\ar_1, \ar) \ileadsto \prop$, thus by definition of incompatibility, $\forall \ar_1 \in \allargs: (\ar_1, \ar) \nileadsto \propbar$; thus, for any $\ar' \in \allargs$, $\ar' \ileadstost \propbar$ is false, because it requires that $\forall \ar_1 \in \allargs: (\ar', \ar_1) \ileadsto \propbar$, and thus in particular, that $(\ar', \ar) \ileadsto \propbar$.
\end{proof}

To further analyze coherence matters, 
\commentYM{analyser quoi ? qu'y a-t-il à analyser et pourquoi voudrait-on l'analyser ?}
consider defined a negation operator $¬$ on $\allprops$. $¬$ is a negation operator iff it is an involution operator (thus a function $¬: \allprops → \allprops$ satisfying $\forall \prop \in \allprops: ¬¬\prop = \prop$) and furthermore satisfying, $\forall \hist \in \allhist, \phi \in \allprops: \hist \ileadsto \prop ⇒ \hist \nileadsto ¬\prop$.
Given any $i \in I$, say that a proposition $\prop \in \allprops$ is \emph{$i$-decidable} iff $\exists \ar \in \allargs \suchthat \ar \ileadstost \prop \lor \ar \ileadstost ¬\prop$.
The $i$-decidable propositions are those on which the normative theory permits, in principle, to take a position. 
%It will generally be unknown in practical applications, and may be empty.
When a proposition $\prop$ is such that $\prop \in \iprops ⇔ ¬\prop \notin \iprops$, I say that this proposition admits an unambiguous interpretation in terms of $i$’s \ac{DJ}.
The next proposition ensures an unambiguous interpretation of all $i$-decidable propositions.
%The symbol $\oplus$ denotes the exclusive disjunction; thus $\prop \in \iprops \oplus ¬\prop \in \iprops$ is equivalent to: $\prop \in \iprops ⇔ ¬\prop \notin \iprops$ \commentYM{au moins dans les propositions et définitons de cette section, l'introduction  de cet opérateur n'apporte rien, et rend la lecture inutilement difficile}. 
It can be seen to hold by using \cref{th:protcoh}.

\begin{proposition}[Restricted interpretability]
	\label{th:restrinterpr}
	Any normative theory with a negation operator $¬$ satisfy,
	$\forall i \in I, \prop \in \allprops: [\exists \ar \in \allargs \suchthat \ar \ileadstost \prop \lor \ar \ileadstost ¬\prop] ⇒ [\prop \in \iprops ⇔ ¬\prop \notin \iprops]$.
\end{proposition}

This interpretability is however restricted to the $i$-decidable propositions. The following proposition and discussion (assuming a normative theory and a negation operator $¬$ defined) argue 
\commentYM{"suggest" ou "show", mais je ne pense pas qu'on puisse dire que des "definitions" ou "propositons" "argue" quoi que ce soit}
that going further is not desirable.

\begin{definition}[Theory completeness]
	$\forall \prop \in \allprops: [\exists \ar \in \allargs \suchthat \ar \ileadstost \prop] \lor [\exists \ar \in \allargs \suchthat \ar \ileadstost ¬\prop]$.
\end{definition}
\begin{definition}[Unrestricted interpretability]
	$\forall \prop \in \allprops: \prop \in \iprops ⇔ ¬\prop \notin \iprops$.
\end{definition}
\begin{definition}[Deliberation completeness]
	$\forall \prop \in \allprops: \prop \in \iprops \lor ¬\prop \in \iprops$.
\end{definition}
	
The following proposition directly follows from the definitions, together with \cref{th:protcoh}.
\begin{proposition}
	1) Under Normative adequacy, Theory completeness implies Deliberation completeness. 2) Unrestricted interpretability implies Deliberation completeness.
\end{proposition}

\begin{example}[Two lotteries: completeness]
	Applying these definitions and results to \cref{ex:twolott}, we see that $\set{(P ≥ D, D > P), (D ≥ P, P > D)} \subseteq \incompat$, and we can define a negation operator such that $¬(P ≥ D) = D > P$ and $¬(D ≥ P) = P > D$. We see that $P ≥ D$ is $i$-decidable iff $D > P$ is $i$-decidable iff there exists a decisive argument $\ar \in \allargs$ for $P ≥ D$ or for $D > P$. It follows from \cref{th:restrinterpr} that at most one of $P ≥ D$ and $D > P$ are in $\iprops$.
	Theory completeness holds iff there exists a decisive argument for $P ≥ D$ or for $D > P$ and a decisive argument for $D ≥ P$ or for $P > D$. In such a case, and assuming Normative Adequacy holds, Deliberation completeness holds as well, thus for every $i$ and every proposition, either it or its contrary is in $i$’s \ac{DJ}.
\end{example}

Theories as defined here do not mandate Theory completeness.
This is good for two reasons. First, even the weaker Deliberation completeness can be considered too strong. Second, it is dubious that, for non trivial decision problems, theories can be found for which both Normative adequacy is normatively compelling and Theory completeness holds.

It might be useful to exemplify those remarks in the case of our running example before discussing them further.
\begin{example}[Discussion about completeness]
\label{ex:incompl}
	It should be remarked that there is no logical contradiction in assuming that $i$ could, for example, deliberately judge that $P ≥ D$, but have no deliberated judgment about whether $D ≥ P$ or rather $P > D$ holds, thus, whether these two lotteries have equal worth or whether $P$ is strictly more valuable.  In such a case, Deliberation completeness does not hold.
	More pragmatically, there may be a decisive argument in $\allargs$ for $P ≥ D$, but no argument in $\allargs$ strong enough to convince $i$ to pick $P$ rather than $D$ against all counter-arguments, and no argument strong enough to convince $i$ that $P$ and $D$ are equally preferable. For a formal example where this would happen, assume $\ileadsto$ is such that $i$ is lead to choose only $P$ for some of the paths including $\ar_0$ towards their end, but to be indifferent for the other ones. Thus: $\forall \hist \in \allhist \suchthat \ar_0 \in \histend: \hist \ileadsto P ≥ D$, $\exists \hist \suchthat \ar_0 \in \histend \land \hist \ileadsto P > D$ and $\exists \hist \suchthat \ar_0 \in \histend \land \hist \ileadsto D ≥ P$.
\end{example}

Not mandating Theory completeness is a first important element to allow hope that effective theories of \ac{DJ} can be built. 
To understand why, an analogy with the case of shallow preferences may be useful.
Psychologists have shown that shallow preferences often depend on a priori irrelevant features pertaining to the description of the decision problem or the context of interrogation. Preferences are often said to be constructed, thus, to not exist independently of the framing and context of study.
But, with no intent of reducing the importance of these findings, it is important to qualify such statements. 
The findings of psychologists do not, evidently, show that all preference statements are labile. As \citet[p.\ xvi]{lichtenstein_construction_2006} write, “To be sure, there are limits to the process of construction. Some televisions just won’t sell; most of the time, people are unlikely to want 80\% of their wages to go into savings. It would be an overstatement to say of preferences, as Gertrude Stein said of Oakland, that ‘there is no there there’.”
\commentYM{ce § et le précédent me semble mériter plus que le reste une réécriture. Par ailleurs, je ne vois pas à quoi ils servent dans cette sous-section}
This observation justifies propositions that obtain only partial shallow preference models: a shallow statement preference would be considered to exist only when it stays intact under all framings considered relevant. The proposal of \citet{bernheim_beyond_2009} may be understood under that light.

This remark points towards the importance of allowing for incomplete theories of reflective preferences. \commentYM{définir ce qu'il faut entendre ici par "incomplete"} 
As \citet{mandler_difficult_2001} discusses in details, it would be unrealistic to postulate that individuals would, after due reflection, in non trivial decision problems, adopt a well specified judgment about any possible proposition of interest.
From this does not follow that \acp{DJ} do not exist, but that someone’s \ac{DJ} should not be considered to be necessarily complete. 
\TODO{Check Mandler.}
This observation also offers a way of reconciling the remark of \citet{roy_comparison_1995} cited in \cref{sec:informal} and the objectivity of the conclusions of a decision theoretic approach: if, for some propositions, both the proposition and its contrary proposition can be held, with no argument being strong enough to impose one of them, let neither these propositions nor their contrary be part of $i$’s \ac{DJ}; and let the theorist focus on the subset of propositions for which argumentative stability can be found.

The definitions of the theories presented here accordingly tolerate both theory incompleteness, which permits the theory to stay silent on some of the propositions in the agenda, and, more fundamentally, deliberation incompleteness, which permits to tolerate that for some propositions, neither it nor its negation figure in someone’s \ac{DJ}. 

\commentYM{dans toute cette section, il me semble qu'il faudrait soit des exemples, soit des explications verbales}

\section{Descriptive theories}
\label{sec:descriptive}
\subsection{Informal presentation}
\commentOCf{Je ne me suis pas encore décidé : descriptive theory or empirical theory ? Pour le moment, j’utilise les deux expressions de façon interchangeable.}
A normative theory defines when a given argument is decisive for a given proposition and a given individual: whenever every sequence of arguments having that argument towards the end of the sequence lead the individual to adhere to the proposition. This, in turn, under condition of accepting its normative validity \commentYM{mais c'est quoi cette "normative validiy" ? de quelles conditions parles-tu ?}, permits in principle to know something about the \ac{DJ} of an individual. But this gives no concrete means to know anyone’s \ac{DJ}. The set of finite sequence of arguments, thus of argumentative paths, is infinite. The definition of decisiveness could admittedly be modified to account for an a priori bound on the maximal length of sequences, but this would not solve the practical problem that this set would be too large to be explored thoroughly 
\commentYM{certain que ce terme existe ?}
\commentOCf{Non, thoroughfully n’existe pas, en effet}. 
Furthermore, only some argumentative paths can be tested for a given individual: the individual’s patience is limited. 

Achieving practical knowledge of \acp{DJ} is tackled by introducing descriptive theories of \ac{DJ}, the subject of the next section.

\subsection{Definition}
\label{sec:descrdef}
In this section, a normative theory of \ac{DJ} $(\allprops, \allargs, I, {\allleadsto})$ is supposed given.
A descriptive theory $\gamma$ for a normative theory of \ac{DJ} is defined as a tuple $(\gC, \gPhi)$, where $\gC \subseteq \powerset{I × \allhist × \Phi}$ is a falsifiable claim, $\gPhi \subseteq \Phi$ are the propositions that the theory claims are in the \acp{DJ} of the individuals, and such that its falsifiable claim implies its convincingness.
\commentYM{dans la mesure où, à ce stade, je ne comprends pas bien ce que contient une "normative theory", la séparation avec l' "empirical theory" n'est pas claire pour moi} 
These concepts are now defined formally.
\commentYM{puisque la théorie empirique est une théorie empirique D'UNE CERTAINE THEORIE NORMATIVE, je m'attendrais  à ce qu'on voit apparaître la théorie normative dans la formule de la théorie empirique, non ?}

A claim $\gC \subseteq \powerset{I × \allhist × \Phi}$ represents the theory’s claim about the relation $\leadsto$: the theory claims that ${\leadsto} \in \gC$. Such a claim is required to be falsifiable. Defining falsification formally require to consider the relationship between the real relation $\leadsto$ and what one can know about it through observations. 

Recall that $\leadsto$ will in general not be known in its entirety, but can be queried through observations. Observations of the relation $\leadsto$ can in general include positive statements, of the form $\hist \ileadsto \phi$, and negative ones, of the form $\hist \nileadsto \phi$. (One way to obtain such negative observations is to use an incompatible proposition $\phi'$ and deduce from $\hist \ileadsto \phi'$ that $\hist \nileadsto \phi$.) 
Assuming (possibly counterfactually) that $R \subseteq I × \allhist × \Phi$ is the “real” relation $\leadsto$, the possible observations are $(O, O') \suchthat O, O' \subseteq I × \allhist × \Phi \land [O \subseteq R] \land [R \cap O' = \emptyset]$; and conversely, once we have gathered some observations $(O, O')$, we know that ${\leadsto} \in \set{R \subseteq I × \allhist × \Phi \suchthat O \subseteq R \land R \cap O' = \emptyset}$.

A claim that ${\leadsto}$ is included in some given set is said to be falsifiable iff, assuming the claim is false, and whatever the real relation is that makes it false, it is possible to prove that the claim is false using a finite set of observations. Formally, the claim $\gC \subseteq \powerset{I × \allhist × \Phi}$ is falsifiable iff $\forall R \notin \gC: \exists \text{ finite sets } O, O' \subseteq I × \allhist × \Phi \suchthat \forall R' \subseteq I × \allhist × \Phi \suchthat O \subseteq R' \land R' \cap O' = \emptyset: R' \notin \gC$. 
In such a case, no possible relation $R'$ compatible with the observations $O, O'$ are in $\gC$; in other words, the observations $O, O'$ prove that the claim that ${\leadsto} \in \gC$ is false.

%Distinguish this condition from: $\exists R, finite O such that C is proven false$: the claim "a is dec for phi AND some argument is decisive for phi'" (or "$\alpha \ileadsto \phi$ and some argument is decisive for phi") satisfies this weaker condition, but it should not be considered falsifiable as it is not necessarily possible to distinguish it empirically from a disagreeing theory: if indeed $\alpha \ileadsto \phi$ but NO arg dec, it is impossible to observe the falsity of the claim. Note that the weaker requirement is satisfied by BR but not the stronger one.

Finally, a descriptive theory must ensure that its falsifiable claim implies it being convincing, it the sense of being able to show the existence of decisive arguments in favor of its supported propositions, whatever the true relation is.
So far, $\ar$ being a decisive argument for $\phi$ from the point of view of $i$ is denoted $\ar \ileadstost \phi$, this being defined on the basis of the real relation ${\leadsto}$. We need to extend this definition and notation: assuming the observable relation would be any $R \subseteq I × \allhist × \Phi$ (and not specifically $\leadsto$), let $\ar \ileadstost^R \phi$ denote the fact that $\ar$ is a decisive argument for $\phi$ from the point of view of $i$ when the observable relation is $R$.
\begin{definition}[Conditional convincingness]
	A pair $(\gC, \gPhi)$ is conditionally convincing iff $\forall R \in \gC, i \in I, \phi \in \gPhi: \exists \ar \in \allargs \suchthat \ar \ileadstost^R \phi$.
\end{definition}
TODO use notation: $a \suchthat \set{\hist \in \allhist \suchthat \ar \in \histend} R_i \phi$.

The notion of validity permits to relate the falsifiable claim of a theory, $\gC$, and the set of propositions it supports, $\gPhi$.
\begin{definition}[Validity]
	An empirical theory is \emph{valid} iff $\forall i \in I, \gPhi \subseteq \iprops$.
\end{definition}
The condition that the falsifiable claim entails convincingness is equivalent to the requirement that, assuming normative acceptability, if ${\leadsto} \in \gC$, then the theory is valid.

\begin{example}[Two claims]
	Given a normative theory and $\phi \in \Phi$, the claim $C_1 = \set{{\leadsto} \subseteq I × \allhist × \Phi \suchthat \forall i \in I, \exists \ar \in \allargs \suchthat \ar \ileadstost \phi}$ corresponds to claiming that for all individuals $i \in I$, some argument is $i$-decisive for $\phi$. If the set of arguments is infinite, $C_1$ is not falsifiable.  
	Given $\ar \in \allargs$ and $\phi \in \Phi$, the set $C_2 = \set{{\leadsto} \subseteq I × \allhist × \Phi \suchthat \forall i \in I, \ar \ileadstost \phi}$ denotes the claim that all individuals will consider the argument $\ar$ as decisive for $\phi$. This claim is falsified by any observation such that $\hist \nileadsto \phi$ for some $\hist \in \allhist$ with $\ar \in \histend$, thus, $C_2$ is falsifiable.
	
	Consider the pairs $(C_1, \set{\phi})$ and $(C_2, \set{\phi})$: only the second one constitutes a descriptive theory. Both pairs satisfy the requirement that if $\leadsto \in \gC$, then $\forall i \in I: \phi \in \iprops$ (with $\gC = C_1$ or $\gC = C_2$), but $C_1$ fails to be falsifiable.
\end{example}

Returning to our running scenario permits to give more concrete examples. Here is an example of a theory that simply claims that a given argument is decisive for everybody.
\begin{example}[Constant argumentation]
	Given the normative theory of \cref{ex:twolott}, and the argument $\ar_0$ as defined there, define $\gC = \set{{\leadsto} \subseteq I × \allhist × \Phi \suchthat \forall i \in I: \ar_0 \ileadstost D > P}$ and $\gPhi = \set{D > P}$. The pair $(\gC, \gPhi)$ is a descriptive theory for that normative theory.
\end{example}

The following example illustrates that a descriptive theory need not claim that a given argument will convince everybody.

\begin{example}[A theory for $P$]
	\label{ex:twolottstrings}
	Define a normative theory as in \cref{ex:twolott}, with an extended set of arguments: define $\allargs$ as the set of all strings. (This correspondingly changes $\allhist$ and thus $\ileadsto$, which must be extended to its new domain.) 
	Define $\ar_0$ as in \cref{ex:twolott}.
	Define $\ar_\text{for$P$}$ as the text “We recommend to adopt $P$: it has a much higher probability of winning.” 
	
	The goal of this example is to define a theory that argues in favor of $P$ using two arguments. Following the intuition described just above, the theory will first attempt to use $\ar_\text{for$P$}$, a short argument that may be sufficient to convince a subset of individuals. But some may be left unconvinced because of the counter-argument $\ar_0$ in favor of $D$. In that case, the theory is ready to use another argument, more subtle, designed to counter $\ar_0$. The theory claims that at least one of these two arguments is decisive.
	In order to define a counter-argument to $\ar_0$, define $\ar_\text{non-linear}$ as the text “$D$ is better than $P$ in terms of expected gain, but it is in general absurd to reason in terms of expected gain. For example, consider a choice between winning $10^6$ dollars with probability $0.8$ and winning $2 × 10^6$ dollars with probability $0.4$. These two lotteries have the same expected gain, but the first one is clearly preferrable. Therefore, we recommend to rather pick $P$ as it presents a much better chance of earning money”.
	
	Define $\gC = \set{{\leadsto} \suchthat \forall i \in I: \ar_\text{for$P$} \ileadstost P > D \lor \ar_\text{non-linear} \ileadstost P > D}$.
	Define $\gPhi = \set{P > D}$.
	Define the theory as $(\gC, \gPhi)$. We see that the claim $\gC$ is indeed falsifiable, as any set of observation including $\alpha^0 \ileadsto D > P$ and $\alpha^1 \ileadsto D > P$ such that $\ar_\text{for$P$} \in \histend^0 \land \ar_\text{non-linear} \in \histend^1$ will falsify the claim.
	
	This example thus defines a theory that anticipates a possible attack using $\ar_0$ against $\ar_\text{for$P$}$, and prepares $\ar_\text{non-linear}$ as an alternative argument in case $\ar_\text{for$P$}$ does not convince an individual.
\end{example}
Refining such descriptive theories to let them articulate better the arguments and counter-arguments, and describing more realistic argumentative strategies, is left for future work.

Given any two disagreeing descriptive theories $(\gC, \gPhi)$ and $(\gC[δ], \gPhi[δ])$ for a given normative theory and any two propositions $\prop, \propbar \in \incompat$, say that these two theories disagree about $(\prop, \propbar)$ iff $[\prop \in \gPhi ⇔ \propbar \notin \gPhi[δ]]$. Say that two theories disagree iff there exists some propositions about which they disagree.

If two theories disagree, then it is possible to prove one of them wrong using a finite set of observations.
\begin{proposition}
	Consider two disagreeing descriptive theories $(\gC, \gPhi)$ and $(\gC[δ], \gPhi[δ])$ for a given normative theory. There exists a finite set of observations that prove $\gC$ false, or that prove $\gC$ false.
\end{proposition}
\begin{proof}
	If they disagree, then $\gC \cap \gC[δ] = \emptyset$, otherwise, pick any $R \in \gC \cap \gC[δ]$, and obtain, using convincingness, the existence of decisive arguments for both $\prop$ and $\propbar$, which is impossible as seen in \cref{th:protcoh}.
	
	As $\gC \cap \gC[δ] = \emptyset$, then at least one of these propositions is true: ${\leadsto} \notin \gC$ or ${\leadsto} \notin \gC[δ]$. In both cases, thanks to falsifiability, the existence is proven.
\end{proof}

\section{Preferences}
\label{sec:preferences}
\subsection{Representing deliberated preferences}
A normative theory of deliberated choice is a normative theory of DJ using propositions of a certain form and a specific protocol of observation.

Let $\allalts$ be a set of alternatives. 

Define $\allprops_\allalts = \set{\prop_{B, \choices} \suchthat \emptyset ≠ B \subseteq \allalts, \choices \subseteq \powersetz{B}}$.

Let $I$ and $\allargs$ be any sets of individuals and arguments.

The observation protocol goes as follows. Given $i \in I$, $\emptyset ≠ B \subseteq \allalts$ and $\hist \in \allhist$, the individual $i$, after having been presented the arguments in the sequence $\hist$, is asked to choose a subset of elements from $B$ among which she is indifferent. Define $\emptyset ≠ c^\hist_i(B) \subseteq B$ as the subset that $i$ chooses after exposure to $\hist$.
Define $\hist \ileadsto \prop_{B, \choices}$ iff $c^\hist_i(B) \in \choices$.

This defines normative theories about $\allalts$. Any empirical theory for a normative theory about $\allalts$ that satisfies $\forall \emptyset ≠ B \subseteq \allalts: \exists ! \emptyset ≠ \choices \subseteq \powersetz{B} \suchthat \prop_{B, \choices} \in \gpropse$ is said to be an empirical theory about $\allalts$. The notation $\exists!$ means: there exists a unique element. The requirement that there is at least one is innocuous, as choosing $\choices = \powersetz{B}$ ensures that $\prop_{B, \choices} \in \iprops$, by definition of $\ileadsto$.

An empirical theory about $\allalts$ thus associates to each possible bundle $B$ a set of possible choices $\choices$, whose semantics is that the theory claims that the deliberated choice of $i$, given the bundle $B$, lies among $\choices$. Let $\gprops(B) \subseteq \powersetz{B}$ denote this set of possible choices, thus $\gprops(B) = \choices  ⇔ \prop_{B, \choices} \in \gpropse$.

Given an empirical theory about $\allalts$ and $a, b \in \allalts$, define $a \gpst b$ iff $\forall \set{a, b} \subseteq B: b \notin \bigcup \gprops(B)$. Thus, $a \gpst b$ iff $\gamma$ claims that $b$ is not deliberately chosen in presence of $a$.
Define $a \gind b$ iff $\forall \set{a, b} \subseteq B: [\forall C \in \choices: a \in C ⇔ b \in C]$.
Define ${\gpeq} = {\gpst} \cup {\gind}$.

If $\gpeq$ is transitive, and if $\gamma$ is valid, $\gpeq$ can be said to represent part of the deliberated preference of $i$, in the sense that choosing from the undominated elements of parts of $\gpeq$ restricts choice in a way that is systematically compatible with $i$’s \ac{DJ}. This is made precise below.

Given $i \in I$ and a bundle $B$, say that a subset $B' \subseteq B$ is a deliberated restricted choice set for $i$ given $B$ iff 
$\exists \prop_{B, \choices} \in \iprops \suchthat \forall C \in \choices: C \cap B' ≠ \emptyset$.
Restricting consideration to deliberated restricted choice sets respects $i$’s \ac{DJ} in the sense that it necessarily includes at least one alternative that is a deliberated choice of $i$.

If $\gpeq$ is transitive and $\gamma$ is valid, the following theorem shows that a choice function can be defined by picking from undominated elements of $\gpeq$ in such a way that only deliberated restricted choice sets result. $\powersetz[\mathit{F}]{\allalts}$ designates the finite non-empty subsets of $\allalts$.

\begin{theorem}[Deliberated preference]
	\label{th:dp}
	Let $\gamma$ be a valid empirical theory for a normatively adequate normative theory about $\allalts$, and such that $\gpeq$ is transitive. Let $r$ be a choice function such that, given any $B \in \powersetz[\mathit{F}]{\allalts}$, $r(B)$ contains exactly one arbitrary argument of each undominated equivalence classes of $B$, where $\gind$ determines the equivalence classes and $\gpst$ determines the undominated elements. Then, $\forall B \in \powersetz[\mathit{F}]{\allalts}$, $r(B)$ is a deliberated restricted choice set for $i$ given $B$.
\end{theorem}
\begin{proof}
	Observe that under these hypothesis, $\gpeq$ is a transitive, reflexive binary relation on $\allalts$, $\gpst$ is its asymmetric part and $\gind$ is its symmetric part. 
	$\gpeq$ is reflexive because $\gind$ is.
	That $\gpst$ is asymmetric is seen by considering, given any $a, b \in \allalts$, the bundle $B = \set{a, b}$: pick any $C \in \gprops(B) \subseteq \powersetz{B}$, thus, $C \in \powersetz{B}$, hence, $a \in C$ or $b \in C$, and therefore, $b \ngpst a$ or $a \ngpst b$.
	
	The phrasing of the theorem only defines $r$ informally. These supplementary notations may be useful to avoid ambiguity. Given $B \subseteq \allalts$, let $B / {\gind} = \set{{\gind}(a) \suchthat a \in B}$ denote the equivalence classes defined by $\gind$ among the elements of $B$. The relation $\gpst$ can be used to discriminate some of these equivalence classes. Define $\gpst^*$, an asymmetric transitive relation on $B / {\gind}$, such that given any two equivalence classes $\alts, \alts' \in B / {\gind}$, $\alts \gpst^* \alts'$ iff $\forall \alt \in \alts, \alt' \in \alts': \alt \gpst \alt'$. Let $\max_{\gpst^*}(B / {\gind}) = \set{\alts \in B / {\gind} \suchthat \forall \alts' \in B / {\gind}: \alts' \ngpst^* \alts}$ denote the undominated equivalence classes of $B / {\gind}$ considering $\gpst^*$.

	In some situations, multiple deliberated restricted choice sets exist. A linear ordering $>$ on $\allalts$ can be used to break ties and obtain a precisely defined choice function. (A linear ordering $>$ on $\allalts$ is a transitive, asymmetric and connected binary relation over $\allalts × \allalts$, connected meaning that $\forall a ≠ b \in \allalts:  [a > b \lor b > a]$.) Note that this tie breaking strategy is adopted for concreteness and conceptual simplicity, but the proof does not depend on this specific choice. It may in general depend on the bundle.

	Define now the choice function $r: \powersetz[\mathit{F}]{\allalts} → \powersetz{\allalts}$ as picking the maximal element (as determined by $>$) of each undominated equivalence class in $B$, thus, $r(B) = \set{\max_>(\alts) \suchthat \alts \in \max_{\gpst^*}(B / {\gind})}$.
	
	Given any bundle $B \in \powersetz[\mathit{F}]{\allalts}$, we now have to show that $B' = r(B)$ satisfies 
	$\exists \prop_{B, \choices} \in \iprops \suchthat \forall C \in \choices: C \cap B' ≠ \emptyset$.
%	$[\forall a \in B \setminus B': \prop_{B, ¬a} \in \iprops] \lor [\exists a \in B' \suchthat \prop_{B, a} \in \iprops]$. 
	Because $\gamma$ is valid, suffices to show that, given $\choices = \gprops(B) \subseteq \powersetz{B}$, 
	$\forall C \in \choices: C \cap B' ≠ \emptyset$.
%	$[\forall a \in B \setminus B': \prop_{B, ¬a} \in \gpropse] \lor [\exists a \in B' \suchthat \prop_{B, a} \in \gpropse]$.
	Pick any $C \in \choices$ and any $a \in C$ (thus, $a \in B$). By definition of $\gpst$, $\nexists b \in B \suchthat b \gpst a$ as otherwise, $a \notin C$.
	Note that, by construction of $B'$, as $a \in B$ is undominated, it belongs to some equivalence class of which a member has been brought in $B'$. In other words, $\exists b \in B' \suchthat b \gind a$. Therefore, by definition of $\gind$, $b \in C$. Thus, $b \in C \cap B'$.
\end{proof}
The phrasing of this theorem considers only finite bundles for simplicity. It more generally applies to bundles that are closed according to $\gpst$. \TODO{Phrase this correctly.}

\begin{proposition}[Completeness implies transitivity]
	\label{th:comptrans}
	If $\gpeq$ is complete ($\forall a, b \in \allalts: a \gpeq b \lor b \gpeq a$) and $\gamma$ is coherent, then it is transitive.
\end{proposition}
\begin{proof}
	Consider any $a, b, c \in \allalts$ with $a \gpeq b \gpeq c$ and let us show that $a \gpeq c$.
	Consider the bundle $B = \set{a, b, c}$ and let $\choices = \gprops(B)$.
	
	Observe first that if $x, y \in B$ and $x \gpeq y$, $y \in \bigcup \choices ⇒ x \in \bigcup \choices$. 
	Indeed, if $x \gpst y$, $y \notin \bigcup \choices$, and if $x \gind y$, $y \in \bigcup \choices ⇔ x \in \bigcup \choices$ as $x$ and $y$ can then be treated interchangeably considering their memberships to elements of $\choices$.

	Because $\bigcup \choices ≠ \emptyset$, $[a \in \bigcup \choices \lor b \in \bigcup \choices \lor c \in \bigcup \choices]$. The above observation leads to $c \in \bigcup \choices ⇒ b \in \bigcup \choices$ and $b \in \bigcup \choices ⇒ a \in \bigcup \choices$. Therefore, $a \in \bigcup \choices$.
	
	Completeness implies that $a \gpeq c \lor c \gpeq a$, or equivalently, $a \gpeq c \lor c \gpst a \lor c \gind a$. Excluding $c \gpst a$ thanks to $a \in \bigcup \choices$, we obtain $a \gpeq c$.
\end{proof}

The following proposition follows from \cref{th:comptrans,th:dp}.
\begin{proposition}[Completeness implies single recommendations]
	If $\gpeq$ is complete and $\gamma$ is valid, then the choice function $r$ defined in \cref{th:dp} associates to each finite bundle a singleton set.
\end{proposition}

Note that $\gpeq$ is independent of $\ileadsto$; thus, any property of $\gpeq$, such as transitiveness or completeness, is a priori (as opposed to empirical), meaning that its satisfaction can be verified without experiments.

When $\gpeq$ is not transitive, choice functions could still be defined that could be said to respect, in some reasonable sense, the deliberated preference of the individuals. However, this claim is more arguable, as it is generally (though by all means not unanimously) considered in the literature that a reflective preference should be transitive. Further explorations of this possibility are left for future articles.

In what follows, the claims of an empirical theory about $\allalts$ will be defined on the basis of a transitive preference ordering $\peq$ on $\allalts$, as follows. Given $\emptyset ≠ B \subseteq \allalts$, define $\gprops(B) = \set{\cup \choices' \suchthat \emptyset ≠ \choices' \subseteq \choices}$, where $\choices$ are the undominated equivalence classes in $B$ according to $\peq$ (thus $\choices = \max_{\pst^*}(B / {\ind})$, using the notations defined in \cref{th:dp}). The check that ${\peq} = {\gpeq}$ is left to the reader. Thus, all the examples that follow have a transitive $\gpeq$ and therefore claim to represent a deliberated preference (when valid) in the sense defined in this section.

\subsection{A theory based on Satisfaction}
Consider a normative theory of preferences about a given set of alternatives $\allalts$. Let $\gamma$ be the following empirical theory for that normative theory.

Define $C$ a set of criteria, functions $c: \allalts → \R$, each function measuring the performances of the alternatives according to some point of view. Let $t \in \R^C$ be a set of thresholds, thus, consisting of one real valued threshold $t_c \in \R$ per criterion. Say that $\alt \in \allalts$ is satisfying iff $\forall c \in C: c(\alt) ≥ t_c$.
 Let $\alts$ denote the satisfying alternatives.
 
The theory $\gamma$ claims, intuitively, that individuals $i \in I$ are satisficers using some defined criteria $C$ and thresholds $t = (t_c)_{c \in C}$. Define $\peq$ as $\alt \peq \altp$ iff $\alt \in \alts \lor \altp \notin \alts$. Note that $\peq$ is complete and contains at most two equivalence classes: $\alts$, and $\allalts \setminus \alts$. This defines the set of claims that $\gamma$ supports, namely, for any $\emptyset ≠ B \subseteq \allalts$, $\gprops(B) = B \cap \alts$.

Given $\alt \in \alts$, define the fragment $\ar_\alt$ as the text: “$\alt$ has performance $c_1(\alt)$ on criterion $c_1$, and it is reasonable to consider that having at least $t_{c_1}$ is satisfying enough from the point of view captured by $c_1$, [… same phrase for the other criteria…], therefore, it is a fully satisfactory alternative.”
Given $\altp \in \allalts \setminus \alts$, define $\ar_\altp$ as the text: “$\altp$ is not satisfactory on the criterion [some criterion such that $c(\altp) < t_{c}$]: the performance $t_c$ should be attained for considering $\altp$ as satisfying on that point of view, but it obtains only $c(\altp)$ on that criterion [… same phrase for the other criteria such that $c(\altp) < t_c$].” 

Given a bundle $B$, define $\ar_B$ as the text obtained by concatenating all fragments in $\set{\ar_\alt \suchthat \alt \in B}$.

%Given $a \in A, b \in \allalts \setminus A$, define $\ar_{a > b}$ as the text: “$a$ has performance $c_1(a)$ on criterion $c_1$, and it is reasonable to consider that having at least $t_{c_1}$ is satisfying enough from the point of view captured by $c_1$, [… same phrase for the other criteria…], therefore, it is a fully satisfactory alternative; whereas $b$ is not satisfactory on criteria [… enumeration of the criteria such that $c(b) < t_{c}$].” 
Define ${\gleadsto} = \set{(\ar_B, \gprops(B)) \suchthat \emptyset ≠ B \subseteq \allalts}$.
Define $\alldargs = {\gbeats} = {\dbeats} = {+} = \emptyset$: the theory does not consider any counter-argument as potentially damaging. 
This defines an empirical theory $(\gleadsto, \gbeats, \alldargs, \dbeats, +)$ about $\allargs$.

A more realistic variant of this empirical theory would refrain from claiming that all satisfactory alternatives are equivalent, and that all non-satisfactory alternatives are equivalent. Indeed, this claims is likely to fail reflecting the deliberated preferences of the individuals: $i$ will probably pick $\alt$ rather than $\set{\alt, \altp}$ given a bundle of two satisfactory alternatives $\set{\alt, \altp}$ with $\alt$ Pareto-dominating $\altp$. 

To this effect, given $\alt, \altp \in \allalts$, define $\alt \peq \altp$ iff $\alt \in \alts \land \altp \notin \alts$ (this also changes the set of claims that $\gamma$ supports) and keep the other definitions intact.
The resulting theory does not claim to capture complete deliberated preferences, as it ignores how any two alternatives that are both satisfactory, or both unsatisfactory, compare. However, whenever there is exactly one satisficing alternative in $B$, it still determines a single deliberated preferred choice for $i$ among $B$.

\subsection{Comparing two lotteries}
%Idea: argue for P > $.
%4 ways:
%- choose
%- rate attractivity: b1 → +; b2 → ++ thus b2 > b1
%- bids to buy (B-bid): I’d pay max $4 to buy and play b1; I’d pay max $6 to play b2 thus b2 > b1
%- bids to sell (S-bid): I’d sell b1 for at least $3; I’d sell b2 for at least $5 thus b2 > b1.
%
%First two: focus on probs. Last two: focus on amounts.
%
%Compare P = (.99, +$4; .01, −$0.1) VS $ = (.33, +$16; .67, −$2). Typically, choose P and bid more for $.
%E[P] ≈ $4; E[$] ≈ 5+1/3 − 4/3 = $4.
%
%XP1. Choice VS S-bid. No gambling. 6 P bets, 7 $ bets. 12 pairs, of which 6 presented in the table: (P, $), choose one, and indicate the strength of your preference (slight to very strong). Then 19 bets using S-big, 6 practice (other bets) then 13 were analyzed.
%
%E[P] and E[$] (of the example above) are similar, but bids “greatly exceeding the expected value of $3.94 are common”.
%
%XP2. Choice VS B-bid.
%
%XP3. With effective gambling and iterated thinking to thing about their preference.
This section presents a theory that claims that $i$ deliberately prefers $P$ to $D$, in the sense of bidding at least as much for $P$ than for $D$. Because the goal of this example is mainly to illustrate the mechanisms of counter-argumentation, this example defines the theory only for these two lotteries. The reader should understand how the mechanisms described here above, together with the principle of argumentation presented here for these two specific lotteries, would permit to define a more general theory about lotteries that argues similarly to this one. \TODO{Think about the right discursive approach here.}

Define $\ard[2]$ as arguing that the goal is to maximise average revenue, and thus, losses should count equally to gains. Define $\ar_3$ as arguing by considering only what is to be gained and not the probabilities, and thus, arguing that $D$ is worth much more. Define $\ard[4]$ as arguing that also the probabilities should be taken into account.
Define $+$ as the simple concatenation of its arguments, which define $\ar_2 = \ar_0 + \set{\ard[2]}$, $\ar_4 = \ar_0 + \set{\ard[4]}$, $\ar_6 = \ar_0 + \set{\ard[2], \ard[4]} = \ar_2 + \set{\ard[4]} = \ar_4 + \set{\ard[2]}$.

Finally, define $\gbeats = \set{(\ar_1, \ar_0), (\ar_3, \ar_0), (\ar_3, \ar_2), (\ar_1, \ar_4)}$, $\dbeats = \set{(\ard[2], \ar_1), (\ard[4], \ar_3)}$.

This defines a normative and an empirical theory about $\allalts = \set{P, D}$.

\section{Conclusion}
\label{sec:conclusion}
(To be written.)

- A theory will never be absolutely convincing in the sense defined here. \citet[p.\ 100]{raiffa_back_1985} (about SEU): “Give therapy to deviators”. \citet[p.\ 108]{raiffa_back_1985}: “In experiments with my students, my ingenious protestations did not always prove to be ingenious enough.
Yes, they were dazzled (and perhaps confused) but many remained unconvinced; if the chips were down, they would behave just as before.”
Existence of a falsification instance does not mean that a theory should be rejected. Instead, comparing theories in the suggested way permits to gather information about their relative convincingness power, which in turn will inform on the appropriateness of using them normatively. Supplementary criteria might be required to decide on normative power. For example, policies that would contradict “very strongly” shallow preferences could be considered as excluded, for practical or for principled reasons, even possibly in situations where it could be shown convincingly that such policies would better satisfy reflective preferences.
Relatedly, it is not claimed here that what society ought to do is exactly what \acp{DJ} indicate, only that reflective preferences provide an appropriate basis for such normative endeavors, in supplement to other criteria. Further discussions are required, which hopefully the formal definitions here can help clarify.

- Theories must be able to adapt their claims to individuals. The mechanism described here that adapts arguments to individuals can be adapted to such goal. This is left for future work.

- Back to the claim that a decision problem may admit several valid responses. The current approach partly permits to develop theories that agree with this claim, because these theories may be incomplete (in the sense of defining only a partial deliberated preference). Taking this view seriously, however, requires to account for meta-preferences (cf. Sen): a (possibly partial) ordering on the possible definitions of one’s preference. As an example, a theory should be able to express that $i$ will adopt either the view that $a > b > c$, or the view that $c > b > a$, but not $a > c > b$.
%Relatedly, akrasia might be considered as well. An individual may be able to declare that she would prefer having a preference rather than another one; or be put in a situation that helps her follow a preference rather than another one. As an example, an individual might be willing to declare 

In order to allow for repeated observations with a given individual without modifying his endowment \commentYM{en quel sens utilises-tu ce terme ici ?}, this example protocol should either be content with purely hypothetical choices, or can make these choices binding by letting the individual know that one of his choices will be picked randomly at the end of the experiment and honored.\commentYM{pas compris le second membre de l'alternative}

\hbadness=10000
\bibliography{simple}

\section{Optional supplements}
It is by no means my intent to claim that descriptive approaches never face validation difficulties. Experimental tests to arbitrate between theories may not be feasible in practice, descriptions are not always solely about the future, falsificationist approaches in turn raise epistemological difficulties, and so on. Correspondingly, the possible confrontation to empirical data of the theories defined here is no magical means to definitively solve every difficulty about validation. My less naïve claim, that should suffice to motivate the present endeavor, is that it is often a good thing for a theory to make empirically testable claims, from the point of view of the possibilities of solving disputes and validating theories.

The important question about how the parameters $t_i$ and $v_i$ of these illustrative theories would be determined is left aside in this article. This article aims at defining and studying theories of \ac{DJ} in general, not at proposing a way of building a specific theory of \ac{DJ}. %For this example, one can assume that these parameters are determined on the basis of observation of the behavior of some population in some similar circumstances. 

\section{Finiteness}
This example permits to ensure that the constraint of bounded depth does not mandate finiteness of the relations $\gbeats$ or $\dbeats$.
\begin{proposition}[Possibility of infinite relations]
	\label{th:infinite}
	An empirical theory may involve infinite relations $\gbeats$ and $\dbeats$.
\end{proposition}
\begin{proof}
	Consider an agenda $\allprops = \set{\prop}$ containing a single proposition, an infinite set of arguments $\allargs = \set{\ar_k \suchthat k \in \N}$ with $\forall j, k \in \N: \ar_j ≠ \ar_k$, and consider any set of individuals $I$ and observation protocol $\allleadsto$. 
	Define $2\N^* = \set{k \in \N^* \suchthat k / 2 \in \N^*}$ as the set of non-zero even integers.
	Define ${\gleadsto} = \set{(\ar_0, \prop)}$.
	Define ${\gbeats} = \set{(\ar_{k - 1}, \ar_0) \suchthat k \in 2\N^*}$ so that all odd numbered arguments are anticipated attacks against $\ar_0$. 
	Consider an infinite set of fragments $\alldargs = \set{\ard[k] \suchthat k \in 2\N^*}$ with $\forall j, k \in 2\N^*: \ard[j] ≠ \ard[k]$ so that fragments correspond to non-zero even integers, and define ${\dbeats} = \set{(\ard[k], \ar_{k - 1}) \suchthat k \in 2\N^*}$ so that each (even numbered) defense fragment attacks exactly one (odd numbered) anticipated attacker of $\ar_0$. 
	Given any coalition of fragments $\dargs$ defending $\ar_0$, consider the set of indices corresponding to the fragments in the coalition, $K_{\dargs} = \set{k \in 2\N^* \suchthat \exists \ard[k] \in \dargs}$; 
	define $j_{\dargs} = \sum_{k \in K_{\dargs}}$ as the sum of these indices;
	and define $\ar_0 + \dargs = \ar_{j_{\dargs}}$. 
%	$\ard[k] \in \alldargs$, define $\ar_0 + \set{\ard[k]} = \ar_k$. 
	Consider an empirical theory $(\gleadsto, \gbeats, \alldargs, \dbeats, +)$ for the normative theory $(\allprops, \allargs, I, {\allleadsto})$. That this is indeed a legal empirical theory can be confirmed by checking that all constraints are satisfied, in particular, $(\dfp)^2(\gleadstoinv(\allprops)) = \dfp(\dfp(\ar_0)) = \dfp(\set{\ar_k \suchthat k \in 2\N^*}) = \emptyset$ because there is no anticipated attack against any argument in $\set{\ar_k \suchthat k \in 2\N^*}$. This theory involves infinite relations $\gbeats$ and $\dbeats$.
\end{proof}

\section{Empirical validity, first}
\label{sec:prooffals}
To verify that the property of Convincingness is purely empirical, it is necessary to show that whenever a theory fails it, it is possible to see it using a set of observations of $\allleadsto$. Conversely, it is also desirable to ensure that only the observations that ensure that Convincingness is false count for falsifying a theory.
In other words, it is desirable to capture exactly which sets of observations falsify the claim of convincingness. This will also provide a proof of cref{th:convfals}.

The following example may help to form an intuition about what the possible falsifying sets of observations are.
\begin{example}[Falsification sequences]
	Let us examine the possible observations that would suffice to show that the theory presented in our running example is not Convincing. 
	
	First, one could exhibit a path $\hist$ that attacks $\ar_0$ and does not trigger the strategic defense of the theory. Those are the paths that satisfy $\ar_0 \in \histend \land \hist \nileadsto D > P \land \forall r \in \R: \ar^r_\text{non-linear} \in \range(\hist)$.
	
	The second possibility is slightly more demanding, and is the only one that is available if indeed $\ar_0$ $i$-defends $D > P$. It requires first to show that $\ar_0 + \ar_\text{linear}$ is a required argument to convince $i$, then to show that this argument fails to convince $i$. For this, two paths are required: $\hist^0 \in \allhist \suchthat \ar_0 \in \histend[0] \land \hist^0 \nileadsto D > P \land \dg(\ar_0, \hist^0) = \ar_0 + \ar_\text{linear}$, and $\hist^1 \in \allhist \suchthat (\ar_0 + \ar_\text{linear}) \in \histend[1] \land \hist^1 \nileadsto D > P$.
\end{example}

Define a falsification sequence for $(i, \ar_0, \phi)$, with $\ar_0 \gleadsto \phi$, thus $\ar_0$ an initial argument for $\phi$, as a finite sequence $(\hist^k)_{k \in \intvl{0, N}}$, for some $N \in \N$, such that it presents successive challenges, which means the following. 
Given $k \in \intvl{0, N}$, assuming $\ar_k$ is defined, define $\ar_{k + 1} = d(\ar_k, \hist^k)$ if $d(\ar_k, \hist^k) ≠ \emptyset$ and say that $\ar_{k + 1}$ is undefined otherwise.
%Define the \emph{defending} sequence of arguments as a finite sequence of arguments $\ar_{k + 1} = d(\ar_k, \hist^k)$, for $k \in \intvl{1, N}$, as long as $d(\ar_k, \hist^k)$ is not empty, thus letting the length of the sequence be the first value of $k$ for which $d(\ar_k, \hist^k) = \emptyset$.
A sequence $(\hist^k)_{k \in \intvl{0, N}}$ presents successive challenges iff [$\ar_{k + 1}$ is undefined exactly when $k = N$] and [$\forall k \in \intvl{0, N}: \ar_k \in \histend[k] \land \hist^k \nileadsto \phi$].

%Given $\args \subseteq \allargs$ and $\hist \in \allhist$, recall that the set $\args \cap \range(\hist)$ denotes the arguments that are both in $\args$ and contained in the sequence $\hist$.

%Define a falsification sequence for $(i, \ar_0, \phi)$, with $\ar_0 \gleadsto \phi$, thus $\ar_0$ an initial argument for $\phi$, as a finite sequence $\hist^1, \hist^3, …, \hist^{N + 1}$, for some $N \in 2\N$, such that it presents successive challenges, which means the following. 
%Given $k \in 2\N \suchthat k ≤ N$, $\ar_k \in \allargs$ and $\hist^{k + 1} \in \allhist$, 
%define $\args_{k + 1} = \gbeatsinv(\ar_k) \cap \range(\hist^{k + 1})$; 
%define $\dargs[k + 2] = \dbeatsinv(\args_{k + 1})$ as those fragments defending against the anticipated attackers $\args_{k + 1}$; 
%and, if $\args_{k + 1} ≠ \emptyset$ (equivalently, if $\dargs[k + 2] ≠ \emptyset$), define $\ar_{k + 2} = \ar_k + \dargs[k + 2]$. 
%The sequence presents successive challenges iff, $\forall k \in 2\N \suchthat k ≤ N$, we have that $\ar_k \in \histend[k + 1]$, $\hist^{k + 1} \nileadsto \phi$, and $[\args_{k + 1} = \emptyset ⇔ k = N]$.

A descriptive theory has no falsification sequence iff $\forall (i, \ar_0, \phi)$, no finite sequence of paths present successive challenges.

 \commentOC{Suffices to prove that any $\ar_k \in (\dgip)^k$ iff $\exists (\ar_0, \hist^0, \ar_1 = \dg(\ar_0, \hist^0), …, \ar_k)$ with $\ar_0 \in \histend \land \hist \nileadsto \phi$.}
\commentOC{We want a descriptive theory to have a property $P$ that implies validity, and that is falsifiable, that is, $P$ implying that no finite sequence of alphas exist in some given set.}
A theory $\gamma$ must define a set $P$ of possible $\ileadsto$ and the theory claims that any effective $\ileadsto$ is in $P$. It must be visible that no, if no, with a finite set of observations of $\ileadsto$.

A claim about $\ileadsto$ is a set $P \subseteq I × \allhist × \Phi$. A claim $P$ is falsifiable iff $\exists O \subseteq I × \allhist × \Phi$, $O$ finite, such that $P \cap \set{{\leadsto} \suchthat {\leadsto} \supseteq O} = \emptyset$.

The theory is falsifiable iff $\exists$ finite $O$ st no completion of it makes the theory convincing.

In other words, such a set $O$ is sufficient to show that the theory is not convincing.

Theorem: the theory is falsifiable iff it says more than nothing.

\section{Empirical validity}
\label{sec:append}
A \emph{coalition of fragments defending $\ar$} is any non-empty subset of $\df(\ar)$ (equivalently, any element of $\powersetz{\df(\ar)}$).
Such a coalition can be combined with the argument it defends to form a new argument.
An empirical theory defines, to that effect, a function $+: \bigcup_{\ar \in \allargs}(\set{\ar} × \powersetz{\df(\ar)}) → \allargs$.
This function permits to add any $\ar \in \allargs$ to any subset $\emptyset ≠ \dargs \subseteq \df(\ar)$

\commentYM{Dans toute cette sous-section, il y a vraiment ENORMEMENT de notations. En conséquence, au moins pour moi, c'est extrêmement difficile à suivre. D'où le fait qu'il y a moins de commentaires de ma part à partir de là : je peine à suivre et du coup j'ai plus de mal à prendre du recul}

Given $i \in I, \prop \in \allprops, \ar \in \allargs$ and $\args \subseteq \gbeatsinv(\ar)$, define $\args \ibeats \ar$ iff $\exists \hist \in \allhist \suchthat \args = \gbeatsinv(\ar) \cap \range(\hist) \land \ar \in \histend \land \hist \nileadstost \prop$. 
Such a set $\args$, when not empty, represents an effective set of anticipated attackers of $\ar$ against $\prop$. 
The notation uses a striked through $\prop$ in superscript to suggest that when $\args \ibeats \ar$, the coalition $\args$ argues successfully against $\prop$ despite the presence of $\ar$, from the point of view of $i$.
I write that $\ar$ \emph{$i$-defends} $\prop$ to mean that $\emptyset \nibeats \ar$. 

Given $i \in I, \prop \in \gpropse, \ar \in \allargs$, define $\di(\ar) \subseteq \powersetz{\alldargs}$ as $\di(\ar) = \set{\dbeatsinv(\args) \suchthat \emptyset ≠ \args \subseteq \gbeatsinv(\ar) \land \args \ibeats \ar}$. It represents the coalitions of fragments defending $\ar$ (when arguing for $\prop$), from the point of view of $i$, against its effective anticipated attackers. 
\Cref{th:indf} shows that these coalitions can be added to $\ar$, and is useful for the later results.
\begin{proposition}
	\label{th:indf}
	$\forall i \in I, \prop \in \gpropse, \ar \in \allargs: \dargs \in \di(\ar) ⇒ \emptyset ≠ \dargs \subseteq \df(\ar)$.
\end{proposition}
\begin{proof}
	I will prove that $\forall \ar \in \allargs: \set{\dbeatsinv(\args) \suchthat \emptyset ≠ \args \subseteq \gbeatsinv(\ar)} = \powersetz{\df(\ar)}$, from which the conclusion follows, and which can be useful to clarify the relationship between $\di(\ar)$ and $\df(\ar)$. Note that by definition of $\df(\ar)$, $\dargs \in \powersetz{\df(\ar)}$ iff $\emptyset ≠ \dargs \subseteq \dbeatsinv(\gbeatsinv(\ar))$. 
	Thus, we have to prove that
	$\forall \dargs \subseteq \allargs$:
	\begin{equation}
		\exists \emptyset ≠ \args \subseteq \gbeatsinv(\ar) \suchthat \dbeatsinv(\args) = \dargs ⇔ \emptyset ≠ \dargs \subseteq \dbeatsinv(\gbeatsinv(\ar)).
	\end{equation}

	First (from left to right), given any $\args \suchthat \emptyset ≠ \args \subseteq \gbeatsinv(\ar_0) \land \dargs = \dbeatsinv(\args)$, let us show that $\dargs ≠ \emptyset$ and that $\dargs \subseteq \dbeatsinv(\gbeatsinv(\ar))$. 
	Indeed, as $\args ≠ \emptyset$, we can pick some $\ar_1 \in \args$, from which follows that $\ar_1 \gbeats \ar_0$, hence, for some $\ard \in \alldargs$, $\ard \dbeats \ar_1$ (as mandated by ${\dbeats}$), thus $\ard \in \dargs$, hence $\dargs ≠ \emptyset$. Also, because $\args \subseteq \gbeatsinv(\ar)$, $\dbeatsinv(\args) \subseteq \dbeatsinv(\gbeatsinv(\ar))$. 
	
	Second (from right to left), let us show that $\emptyset ≠ \dargs \subseteq \dbeatsinv(\gbeatsinv(\ar)) ⇒ \exists \emptyset ≠ \args \subseteq \gbeatsinv(\ar_0) \suchthat \dbeatsinv(\args) = \dargs$. 
	Defining $\args = {\dbeats}(\dargs) \cap \gbeatsinv(\ar_0)$, we see that $\dbeatsinv(\args) = \dargs$: considering any $\ard \in \dargs$, $\ard \dbeats \ar_1 \gbeats \ar_0$ for some $\ar_1 \in \allargs$, thus, $\ar_1 \in \args$, thus, $\ard \in \dbeatsinv(\args)$; and considering any $\ar' \in \dbeatsinv(\args)$, $\ar' \dbeats \ar_1$ for some $\ar_1 \in \args$, thus $\ar_1 \gbeats \ar_0$, thus, $\ar_1 \in {\dbeats}(\dargs)$, thus $\ard \dbeats \ar_1$ for some $\ard \in \dargs$, and by injectivity  of $\dbeats$, $\ar' = \ard$, whence $\ar' \in \dargs$. 
	That $\args ≠ \emptyset$ follows from $\dargs ≠ \emptyset$.
\end{proof}

Given $\ar \in \allargs$, define $\dip(\ar) = \set{(\ar + \dargs) \suchthat \dargs \in \di(\ar)}$, representing the subset of defenders of $\ar$ required for convincing $i$ of $\prop$.
Given $i \in I, \prop \in \gpropse$, define $\gargs \subseteq \allargs$ as the transitive closure of ${\gleadstoinv}(\prop)$ under $\dip$: $\gargs = \cup_{k \in \N} (\dip)^k({\gleadstoinv}(\prop))$.

A \emph{falsification instance} of an empirical theory is a tuple $(i, \prop, \hist, \ar) \in I × \gpropse × \allhist × \gargs$ such that $\gbeatsinv(\ar) \cap \range(\hist) = \emptyset \land \ar \in \histend \land \hist \nileadstost \prop$. 

A \emph{falsification instance} of an empirical theory is a tuple $(i, \prop, \hist, \ar) \in I × \gpropse × \allhist × \gargs$ such that $\ar \in \histend \land \hist \nileadsto \prop \land \dg(\ar, \hist) = \emptyset$. 
I also write that $(i, \prop, \hist, \ar)$ falsifies $\gamma$ to mean that it is a falsification instance of $\gamma$.
We readily see that \cref{th:convnofals} is true.
With a falsification sequence $(\ar_0, \hist^0, \ar_1, \hist^1)$, we see that $\ar_1 \in (\dgip)^1(\ar_0)$, and similarly for longer falsification sequences, $\ar_k \in (\dgip)^k(\ar_0)$.

\begin{lemma}
	\label{th:convnofals}
	An empirical theory is convincing iff it has no falsification instances.
\end{lemma}
\begin{proof}
	For all $i \in I, \prop \in \gpropse, \ar \in \gargs$: $\ar$ does not $i$-defend $\prop$ iff $\exists \hist \in \allhist \suchthat \left[\gbeatsinv(\ar) \cap \range(\hist) = \emptyset \land \ar \in \histend \land \hist \nileadstost \prop\right]$ iff $\exists \hist \in \allhist$ such that $(i, \prop, \hist, \ar)$ falsifies $\gamma$. Therefore, $\gamma$ is not convincing iff $\exists (i, \prop, \ar) \suchthat \ar$ does not $i$-defend $\prop$ iff $\exists \hist \suchthat (i, \prop, \hist, \ar)$ falsifies $\gamma$ iff $\gamma$ has a falsification instance.
\end{proof}

\begin{lemma}
	\label{th:dpiconstr}
	$\forall \ar_0, \ar_2 \in \allargs: \ar_2 \in \dip(\ar_0) ⇔ \exists \hist \in \allhist, \args = \gbeatsinv(\ar_0) \cap \range(\hist), \dargs = \dbeatsinv(\args) \suchthat [\hist \nileadstost \phi
	\land \ar_0 \in \histend
	\land \emptyset ≠ \args \ibeats \ar_0
	\land \ar_2 = \ar_0 + \dargs]$.
\end{lemma}
\begin{proof}
	$\ar_2 \in \dip(\ar_0) ⇔ \exists \dargs \in \di(\ar_0) \suchthat \ar_2 = \ar_0 + \dargs$.
	Also, $\exists \dargs \in \di(\ar_0) ⇔ \exists \args \subseteq \gbeatsinv(\ar_0) \suchthat \dargs = \dbeatsinv(\args) \land \emptyset ≠ \args \ibeats \ar_0$. Equivalently, $\exists \args \subseteq \gbeatsinv(\ar_0) \suchthat \dargs = \dbeatsinv(\args) \land \emptyset ≠ \args  \land \exists \hist \in \allhist \suchthat \args = \gbeatsinv(\ar_0) \cap \range(\hist) \land \ar_0 \in \histend \land \hist \nileadstost \prop$.
\end{proof}

\begin{lemma}
	\label{th:instseq}
	An empirical theory has a falsification instance iff it has a falsification sequence.
\end{lemma}
\begin{proof}
	First, from right to left, consider a falsification sequence $(\hist^{k + 1})_{k \in 2\N, k ≤ N}$ for some $N \in 2\N$, define, as in cref{sec:validity}, $\args_{k + 1} = \gbeatsinv(\ar_k) \cap \range(\hist^{k + 1})$, $\dargs[k + 2] = \dbeatsinv(\args_{k + 1})$ and (for $k ≤ N - 2$) $\ar_{k + 2} = \ar_k + \dargs[k + 2]$, and let us show that $(i, \phi, \hist^{N + 1}, \ar_N)$ is a falsification instance. This amounts to show that $\ar_N \in \gargs$, which can be done by observing that $\ar_0 \in \gargs$, and showing that, $\forall k \in 2\N, k ≤ N - 2$, $\ar_k \in \gargs ⇒ \ar_{k + 2} \in \gargs$.
	Indeed, cref{th:dpiconstr} applies with $\ar_k$ for $\ar_0$ and $\ar_{k + 2}$ for $\ar_2$. As the second member of the equivalence is true by definition of a falsification sequence, $\ar_{k + 2} \in (\dip)(\ar_k)$, and because $\ar_k \in \gargs$, $\ar_{k + 2} \in \gargs$.
	
	Second, from left to right, consider a falsification instance $(i, \phi, \hist^{N + 1}, \ar_N)$. As $\ar_N \in \gargs$, $\ar_N \in (\dip)^N({\gleadstoinv}(\prop))$ (equalizing the indices $N$ wlog). Given $\ar_{k + 2} \in (\dip)^j({\gleadstoinv}(\prop))$, with $j > 0$, thus $\ar_{k + 2} \in \dip(\ar_k)$ for some $\ar_k \in (\dip)^{j - 1}({\gleadstoinv}(\prop))$, apply cref{th:dpiconstr}, obtain $\hist_{k + 1}$. Iterate while $j > 0$. When $j = 0$, the $\hist$ found corresponds to an initial argument.
	
	TODO improve this proof.
\end{proof}

\begin{lemma}
	\label{th:produce}
	Given $\prop \in \gpropse$, $\ar_0 \in \gargs$ that $i$-defends $\prop$, and $\hist \nileadstost \prop$ with $\ar_0 \in \histend$, define $\args =  \gbeatsinv(\ar_0) \cap \range(\hist)$, $\dargs = \dbeatsinv(\args)$, and $\ar_2 = \ar_0 + \dargs$. Then, $\ar_2 \in \gargs \cap \dfp(\ar_0)$.
\end{lemma}
\begin{proof}
	By definition of $\ibeats$, $\args \ibeats \ar_0$.
	Also, because $\ar_0$ $i$-defends $\prop$, $\args ≠ \emptyset$. 
	Therefore, $\dargs \in \di(\ar_0)$.
	It follows that $\dargs \in \df(\ar_0)$, by \cref{th:indf}, and that $\ar_2 \in \dfp(\ar_0)$, by definition of $\dfp$. It also follows that $\ar_2 \in \dip(\ar_0)$, hence, that $\ar_2 \in \gargs$.
\end{proof}

Cref{th:svalid} claims that a normatively adequate empirical theory that has no falsification sequence is valid. TODO prove it.

\begin{theorem}
	\label{th:valid}
	An empirical theory that is normatively adequate and convincing is valid.
\end{theorem}
\begin{proof}
	Given any $i \in I$ and $\prop \in \gpropse$, suffices to show that for some $\ar \in \allargs$, $\ar \ileadstost \prop$; the conclusion then follows thanks to normative adequacy. 
	
	Lemma: $[\ar_0 \in \gargs \land \ar_0 \nileadstost \prop] ⇒ \exists \ar_2 \in \gargs \cap \dfp(\ar_0)$. That is because $\ar_0 \in \gargs$ implies that $\ar_0$ $i$-defends $\prop$, by convincingness, and,
	by definition, $\ar_0 \nileadstost \prop$ means that $\exists \hist \in \allhist \suchthat \ar_0 \in \histend \land \hist \nileadstost \prop$; \cref{th:produce} thus applies.

	To prove the theorem, build a finite sequence of arguments in $\gargs$, starting with any $\ar_0 \in \gleadstoinv(\prop)$. Observe that $\ar_0 \in \gargs$. Given $k \in 2\N$, if $\ar_k \nileadstost \prop$, define $\ar_{k + 2}$ using the above lemma, obtaining $\ar_{k + 2} \in \gargs \cap (\dfp)^\frac{k + 2}{2}(\ar_0)$. Continue applying the lemma and producing further arguments as long as $\ar_k \nileadstost \prop$.
	By definition of an empirical theory, $\exists k \in \N \suchthat (\dfp)^k(\ar_0) = \emptyset$, thus, this process ends with some $\ar$ such that $\ar \ileadstost \prop$.
\end{proof}

\section{Comparing theories}
\label{sec:proc}
This section proves cref{th:proc} constructively, by describing the sequence of tests that must be performed and showing that they must lead to falsifying one of the theories.
To do this, the notion of challenge must be defined and a few simple lemmas are required.
In order to leave the letters $\gamma$ and $\delta$ unbounded for clearer application of the intermediate results to the theorem to be proven, let $\mu$ and $\nu$ designate any two descriptive theories for a given normative theory of \ac{DJ}.
Given $\prop \in \allprops$, $\hist \in \allhist$ and $\ar_\mu \in \gargsmu$, say that $(\hist, \ar_\mu)$ challenges $\prop$ iff $\hist \nileadsto \prop$ and $\ar_\mu \in \histend$.
Given $\prop \in \allprops, \ar_\mu \in \gargsmu, \ar_\nu \in \gargsnu$, say that $(\hist, \ar_\mu, \ar_\nu)$ is a $\prop/\propbar$-challenge iff $(\hist, \ar_\mu)$ challenges $(\mu, \prop)$ or $(\hist, \ar_\nu)$ challenges $(\nu, \propbar)$.

\commentYM{il faudrait tenir la main du lecture : dire pourquoi tu définis ça. Ca se comprends, je sais, mais ça demande au lecteur un gros effort qui n'est pas récompensé par une impression de comprendre où ça va}

\begin{lemma}
	\label{th:endchal}
	Given $\ar_\mu \in \gargs$, $\ar_\nu \in \gargsnu$ and $\hist \in \allhist$ such that $\set{\ar_\mu, \ar_\nu} \subseteq \histend$, $(\hist, \ar_\mu, \ar_\nu)$ is a $\prop/\propbar$-challenge.
\end{lemma}
\begin{proof}
	Either $\hist \nileadsto \prop$, and $(\hist, \ar_\mu)$ challenges $\prop$, or $\hist \ileadsto \prop$, hence, by incompatibility of $\prop$ and $\propbar$, $\hist \nileadsto \propbar$, and $(\hist, \ar_\nu)$ challenges $\propbar$. 
\end{proof}

\begin{lemma}
	\label{th:nextchal}
	Given $(\hist, \ar_\mu)$ challenging $\prop$, with $\ar_\mu \in \gargsmu \cap (\dfp)^k(\gleadstoinv(\prop))$, and $\ar_\nu \in \gargsnu$: either $(i, \prop, \hist, \ar_\mu)$ falsifies $\mu$, or there exists a $\prop/\propbar$-challenge $(\histp, \ar'_\mu, \ar_\nu)$, with $\ar'_\mu \in \gargsmu \cap (\dfp)^{k + 1}(\gleadstoinv(\prop))$.
\end{lemma}
\begin{proof}
	If $\ar_\mu$ does not $i$-defend $\prop$, then $(i, \prop, \hist, \ar_\mu)$ falsifies $\mu$, which ends the proof. 
	Thus, assume that $\ar_\mu$ $i$-defends $\prop$. 

	\Cref{th:produce} applies and yields $\ar'_\mu = \ar_\mu + \dbeatsinv(\gbeatsinv(\ar_\mu)) \in \gargsmu \cap \dfp(\ar_\mu)$, thus $\ar'_\mu \in (\dfp)^{(k + 1)}(\gleadstoinv(\prop))$. 
	Now, define $\histpp = (\hist, \ar'_\mu)$ and test whether $\histpp \nileadstost \prop$. 
	
	If $\histpp \nileadsto \prop$, define $\histp = \histpp$, and observe that $(\histp, \ar'_\mu, \ar_\nu)$ challenges $\prop$, which concludes the proof. 
	Thus, assume that $\histpp \ileadsto \prop$. 

	Let us now define $\histp$ such that $\set{\ar'_\mu, \ar_\nu} \subseteq \histpend$, so that \cref{th:endchal} applies on $(\histp, \ar'_\mu, \ar_\nu)$, which will conclude the proof. 
	
	If $\ar_\nu \in \histppend$, this is done by defining $\histp = \histpp$. Otherwise, define $\histp = (\histpp, \ar_\nu)$ and test whether $\histp \nileadsto \prop$. 
\end{proof}

%Given a normative theory and two empirical theories for that normative theory and $i \in I$, 
%a \emph{test procedure with $i$} is a finite sequence 
%\commentOC{Need the test procedure to be a tree. Atom: $(k, \prop)$ meaning whether ${\hist}^k \ileadstost \prop$. Branch label: atoms linked using or and and operators. From a node $n = (\hist, \ar, \prop)$, let branches depart, such that their labels do not overlap. From a node, if no branch label is true, the node must falsify a theory. In the case of the procedure below, it is not a list because we may query several times the same theory or switch to the other one, depending on the last answer. But if simplifying the theory (single-try), we may then simplify the definition of a test procedure.}

\begin{procedure}
	\label{proc}
	Pick any $\prop, \propbar \in \incompat$ about which $\gamma$ and $\delta$ disagree.
	
	Given $k_\gamma, k_\delta \in \N, \hist \in \allhist, \ar_\gamma \in \gargsgamma, \ar_\delta \in \gargsdelta$, 
	say that 
	$(\hist, \ar_\gamma, \ar_\delta)$ is a challenge of complexity $(k_\gamma, k_\delta)$ iff 
	$(\hist, \ar_\gamma, \ar_\delta)$ is a $\prop/\propbar$-challenge, $\ar_\gamma \in (\dfp[\gamma])^{k_\gamma}(\gleadstoinv[\gamma](\prop))$, and $\ar_\delta \in (\dfp[\delta])^{k_\delta}(\gleadstoinv[\delta](\propbar))$.
	
	Pick any $\ar_{-1} \in \allargs \suchthat \ar_{-1} \gleadsto[\delta] \propbar$. Observe that $\ar_{-1} \in \gargsdelta \cap (\dfp[\delta])^0(\gleadstoinv[\delta](\propbar))$. Test whether $(\ar_{-1}) \nileadsto \prop$. If so, $(i, \prop, (\ar_{-1}), \ar_{-1})$ falsifies $\delta$, which ends the procedure. 
	Thus, assume that $(\ar_{-1}) \ileadsto \prop$. 
	
	Pick any $\ar_0 \in \allargs \suchthat \ar_0 \gleadsto[\gamma] \prop$. Observe that $\ar_0 \in \gargsgamma \cap (\dfp[\gamma])^0(\gleadstoinv[\gamma](\prop))$.  Define $\hist = (\ar_{-1}, \ar_0)$ and test whether $\hist \nileadsto \prop$. By \cref{th:endchal}, $(\hist, \ar_0, \ar_{-1})$ is a $\prop/\propbar$-challenge, thus, $(\hist, \ar_0, \ar_{-1})$ is a challenge of complexity $(0, 0)$.
	
	Suffices now to apply \cref{th:nextchal} repetitively, until finding a falsification instance of $\gamma$ or $\delta$. This is because, given a challenge $(\hist, \ar_\gamma, \ar_\delta)$ of complexity $(k_\gamma, k_\delta)$, \cref{th:nextchal} applies, yielding either a falsification instance, or a new challenge of higher complexity. 
Indeed, if $(\hist, \ar_\gamma)$ challenges $\prop$, apply \cref{th:nextchal} with $\gamma$ for $\mu$, $\delta$ for $\nu$, and obtain either that $\hist$ falsifies $\gamma$, or a new tuple $(\histp, \ar'_\gamma, \ar_\delta)$ that is a challenge of complexity $(k_\gamma + 1, k_\delta)$.
And if $(\hist, \ar_\delta)$ challenges $\propbar$, apply \cref{th:nextchal} with $\propbar$ for $\prop$, $\delta$ for $\mu$, $\gamma$ for $\nu$, and obtain either that $\hist$ falsifies $\delta$, or a new tuple $(\histp, \ar'_\delta, \ar_\gamma)$ such that $(\histp, \ar_\gamma, \ar'_\delta)$ is a challenge of complexity $(k_\gamma, k_\delta + 1)$.
	
	Given that $\exists K_\gamma \suchthat (\dfp[\gamma])^{K_\gamma}(\gleadstoinv[\gamma](\prop)) = \emptyset$ and $\exists K_\delta \suchthat (\dfp[\delta])^{K_\delta}(\gleadstoinv[\delta](\propbar)) = \emptyset$ by definition of empirical theories, this process must end with a falsifying instance of $\gamma$ or $\delta$.
\end{procedure}

\end{document}

\section{Next}
Define a way of saying “we never need this argument”: either the model can prove phi without s0, or if it can’t, then phi does not hold, because I can play s1 against it. 

What about a model for only one situation? Then we need to accept that $q$ can’t be falsified, and accept that it defines the individual (or just assume without proof that it does not).

Assume $i$, given $\ar$ and $\ar'$, picks repeatedly (in various circumstances) nonempty subsets of $\set{\prop, ¬\prop}$. Define $(\ar', \ar) \ileadstoe (\prop, \text{sure})$ iff she sometimes chooses $\set{\prop}$ and $(\ar', \ar) \ileadstoe (\prop, \text{poss})$ iff she sometimes chooses $\set{\prop}$ or $\set{\prop, ¬\prop}$. Then, $(\ar', \ar) \ileadstost (\prop, \text{sure})$ iff she always chooses $\set{\prop}$ and $(\ar', \ar) \ileadstost (\prop, \text{poss})$ iff she always chooses either $\set{\prop}$ or $\set{\prop, ¬\prop}$.

Show that if the claim of the model is of a different form, then the model is not falsifiable, as suggested here below for some form of claims.

Extend the definition of additive single-answer theories to non-additive single-answer theories.

An additive theory need not have freedom for defining anticipated attacks on additive terms: this should be a consequence on the anticipated attacks on defenses and on initial arguments only. This makes the theory easier to define, as only attacks on partial arguments are to be defined.

General theory says that $\ar_1$ is replaceable (suspects that it’s good for some people): $(\ar_1, \ar_0) \ileadstoe ¬\prop ⇒ (\ar'_1, \ar_0) \ileadstoe ¬\prop$. (This will fail if some $i$ thinks $\ar'_1$ is bad arg because of $\ar_0$ but $\ar_1$ is bad because of $\ar_2$.) It may say that $\ar_1$ is bad. General theory prepares potentially good arg $\ar_0$ and claims that if $\ar_0 \ileadstoe \prop$, then $\ar_0 \ileadstost \prop$ (this is Answerability). So that the model only has to prove the weaker existential claim.

It is important to remark that when the query protocol does not admit observation of “both”, thus, when $\ileadstost$ does not distinguish sure from possible, it does \emph{not} follow that $i$ is sure of either $\prop$ or $¬\prop$ for every pair of proposition: the “undecided” case will hold whenever $i$ considers no argument as sufficiently strong to determine a decisive preference (assuming that in that case, $i$ at least sometimes alternates his choice). Admittedly, it might also happen that $i$ is really indifferent between $\prop$ and $¬\prop$ but still systematically chooses, say, $\prop$ over $¬\prop$, in which case it will be (somewhat) erroneously concluded that $\prop \in \iprops$. This is the price to pay for using protocols that do not allow to observe indifference. Observe however that (depending on the application) this error might be considered not harmful, as the obtained model will be faithful to $i$’s behavior after deliberation.

\subsection{Example model}
\begin{itemize}
	\item $\ar_0 = \text{“Veg bec of ethics and health”}$
	\item $\ar_1 = \text{“Ethics is not defined”}$
	\item $\ar_2 = \text{“It is since Aristotle at least”}$
	\item $\ar_1' = \text{“Health isn’t important”}$
	\item $\ar_2' = \text{“Ask my grandmother”}$
	\item $\ar_1 \gbeats \ar_0$, $\ar_1' \gbeats \ar_0$
	\item $\ar_2 \gbeats \ar_1$, $\ar_2' \gbeats \ar_1'$
	\item $\ar_0 + \ar_2 \gbeats \ar_1$
	\item $\ar_1' \gbeats \ar_0 + \ar_2$
	\item $\ar_0 + \ar_2' \gbeats \ar_1'$
	\item $\ar_1 \gbeats \ar_0 + \ar_2'$
	\item $\ar_0 + \ar_2 + \ar_2' \gbeats \ar_1$
	\item $\ar_0 + \ar_2 + \ar_2' \gbeats \ar_1'$
\end{itemize}

\subsection{About comparisons}
Count the number of tests; assume simpler theory.

Indicate sufficient hypothesis for ensuring that if a model resisted every attack, it is valid. Thus, that searching for falsification instances in the tested ones is enough.

\subsection{Interpreting Answerability}
At a given stage in operationalizing the protocol, define $f(\ar_0)$ as the arguments that are considered as having a chance to attack $\ar_0$ (without backtracking). Thus, this will not include $\ar_1$ if indeed $\ar_2$ is considered by $i$ as a satisfying answer to $\ar_1$. We can interpret Answerability in the following ways. Strong answ: $\exists \ar_1 \in f(\ar_0) \suchthat \ar_1 \gbeats \ar_0$. Weak answ: $f(\ar_0) ≠ \emptyset ⇒ \exists \ar_1' \suchthat \ar_1' \gbeats \ar_0$. Or suppress Answ and replace Op. val. by Strong op. val.: $\forall \prop \in \gpropse, \ar_0 \in \gargs[\prop], \ar_1 \in \allargs: (\ar_1, \ar_0) \ileadstoe ¬\prop ⇒ \exists \ar_2 \in {\gbeatsinv}(\ar_1)$.

A model can be allowed to try only once each answer. If $\ar_2 \gbeats \ar_1 \land \ar_1 \in f(\ar_0 + \ar_2)$, the model has lost.

Thus, four variants, given $\ar_2 \gbeats \ar_1$.
\begin{itemize}
	\item Strong answ with single try: $\ar_1 \notin f(\ar_0 + \ar_2) \land \forall \ar_1' \in f(\ar_0 + \ar_2): \ar_1' \gbeats \ar_0 + \ar_2$.
	\item Strong answ with multiple tries: $\forall \ar_1' \in f(\ar_0 + \ar_2): \ar_1' \gbeats \ar_0 + \ar_2$. (Example: $\ar_1 = \text{“for ethical reasons”}$, and $\eta$ knows two c-a to this, from different angles.) 	\item Weak answ with single try: $\ar_1 \notin f(\ar_0 + \ar_2) \land [f(\ar_0 + \ar_2) ≠ \emptyset ⇒ \exists \ar_1' \suchthat \ar_1' \gbeats \ar_0]$.
	\item Weak answ with multiple tries: $f(\ar_0 + \ar_2) ≠ \emptyset ⇒ \exists \ar_1' \suchthat \ar_1' \gbeats \ar_0 + \ar_2$.
\end{itemize}

Goal: leave addition free so as to admit multiple tries, and prove that Strong answ with multiple tries is the right one for a general approach: suitable restriction to the model permits single try (with same conditions); and weak answ actually does not permit more models.

When $f$ is available and is single-valued, question: should we use Strong or weak answ? Right answer: use Strong answ. It is stricly more general. A model that wants weak answ can define its internal attacks suitable (to be proven); and a model that wants to claim more can.

When $f$ is not available, only remains: single try (just give a decisive argument) or multiple tries (try more and more arguments, blindly).

Suppose that a model also defines $f_\eta$, meaning: $f_\eta(\ar_1) = \ar_2$ the argument the model plays against $\ar_1$. Thus $f_\eta$, like $\gbeats$, is a binary relation over $\allargs$, but $f_\eta$ in supplement returns a single argument or none. Also, $\ar_2 = f(\ar_1) ⇒ \ar_2 \gbeats \ar_1$. When a reason $\ar_1$ for rejecting $\ar_0$ is not known, $f_\eta(\ar_0)$ instead gives the argument to be played next. Note that $f_\eta$ is only useful when (in the known c-a variant) for some $\ar_1$ there are multiple candidates $\ar_2$ or when (in the unknown c-a variant) for some $\ar_0$ here are multiple candidates $\ar_2$. In fact, this is more complicated: $f_\eta(\ar_1)$ can give first $\ar_2$, then, if that is not enough, $\ar_4$ (thus it’s rather $f_\eta(\ar_1, \ar_0)$).

Under Strong anws, we ask that $f(\ar_0) \cap {\gbeatsinv}(\ar_0) ≠ \emptyset$, and (inevitably) that $f_\eta(\ar_1) ≠ \emptyset$. Under weak answ, we do not expect anticipation and only demand that $f_\eta(\ar_1) ≠ \emptyset$. But this does not allow supplementary models. TODO prove this.

\subsection{Transitive, complete preference theories}
\NewDocumentCommand{\ppeqab}{}{t_{a \succeq b}}
\NewDocumentCommand{\ppst}{O{a}O{b}}{t_{#1 \succ #2}}
\NewDocumentCommand{\ppstsure}{O{a}O{b}}{(t_{#1 \succ #2}, \mathit{sure})}
\NewDocumentCommand{\ppstbc}{}{t_{b \succ c}}
\NewDocumentCommand{\ppstac}{}{t_{a \succ c}}
\NewDocumentCommand{\ppstba}{}{t_{b \succ a}}
\NewDocumentCommand{\ppeq}{O{a}O{b}}{t_{#1 \succeq #2}}
\NewDocumentCommand{\ppeqsure}{O{a}O{b}}{(t_{#1 \succeq #2}, \mathit{sure})}
\NewDocumentCommand{\domc}{}{\mathscr{B}}

Given a set of alternatives $\allalts$. Define $\allprops_\allalts = \set{\ppst, a ≠ b \in \allalts} \cup \set{\ppstba, a ≠ b \in \allalts}$. Define $¬_\allalts$ such that $\forall a, b \in \allalts: ¬_\allalts\ppst = \ppstba, ¬_\allalts\ppstba = \ppst$. Define $\allprops_\allalts = T_\allalts × \set{\mathit{sure}, \mathit{poss}}$.

Let $I$ represent a given set of individuals; $\allalts$ a set of products; 
and $\allargs$ the set of all strings.
Given $i \in I$, define $\ileadstost$ as follows. 
Given $\hist \in \allhist$ and $\prop \in \allprops_\allalts$, with $\prop \in \set{\ppst} × \set{\mathit{sure}, \mathit{poss}}$ for some $a ≠ b \in \allalts$, the individual is presented the arguments in $\hist$, and has then to choose a product from the set $\set{a, b}$. Define $\hist \ileadstost \prop$ iff $i$ chooses $a$. \commentOC{This is inadequate: if $i$ picks $b$ from $\set{a, b, c}$, the current definition considers it compatible with $\ppstsure \in \iprops$, thus, $\ppst$ does not have a meaningful content. Should rather define $\ppstsure$ so that $a$ is picked rather than $b$ whatever the choice set. This requires to generalize $\ileadstost$.}

Any normative theory of the form $(\allprops_\allalts, ¬_\allalts, \allargs, I, (\ileadsto)_{i \in I})$, for some sets $\allargs$ and $I$, is called a normative theory about $\allalts$.

Given a set $\domc \subseteq \powersetz{\allalts}$, where $\powersetz{\allalts}$ designates the set of subsets of $\allalts$ without the empty set, a function $c: \domc → \allalts$ is a choice function iff $\forall B \in \domc: c(B) \subseteq B$.

Given a normative theory about $\allalts$, an individual $i \in I$, a set $\domc \subseteq \powersetz{\allalts}$ and a choice function $c$, say that $c$ represents a \emph{deliberated choice function} for $i$ iff $\forall B \in \domc: c(B) \subseteq \set{a \in B \suchthat \forall b \in B: \ppstsure[b][a] \notin \iprops}$.

A normative theory about $\allalts$ is transitive iff $\forall a ≠ b ≠ c ≠ a \in \allalts: [\exists \ar_{a \pst b} \in \allargs \suchthat \ar_{a \pst b} \ileadstost \ppstsure] \land [\exists \ar_{b \pst c} \in \allargs \suchthat \ar_{b \pst c} \ileadstost \ppstsure[b][c]] ⇒ [\exists \ar_{a \pst c} \in \allargs \suchthat \ar_{a \pst c} \ileadstost \ppstsure[a][c]]$. 

A set of propositions $P \subseteq \allprops_\allalts$ is transitive iff $\forall a ≠ b ≠ c ≠ a \in \allalts: \ppstsure \in P \land \ppstsure[b][c] \in P ⇒ \ppstsure[a][c] \in P$.

A transitive empirical preference theory about $\allalts$ is an empirical theory for a normative theory about $\allalts$ such that $\gpropse$ is transitive.

Associate to such a theory a preference ordering $\succ$ on $\allalts$ as follows: $a \succ b$ iff $\ppstsure \in \gpropse$.

The deliberated preference of $i$ under a normative theory 
Then, observe $a$ incomp to $z$ using a good argument for $a > z$ and one for $z > a$. Observe also $b$ incomp to $z$. And observe $a > b$. This proves incompleteness as $z$ indifferent to the rest is impossible. BUT this requires to observe possibility.

\subsubsection{Transitive arguments}
Given a normative theory and a function $\tau: \allargs × \allargs → \allargs$, say that $\tau$ is a transitive argument builder iff $\forall a ≠ b ≠ c ≠ a \in \allalts, \ar_{a \pst b}, \ar_{b \pst c} \in \allargs \suchthat \ar_{a \pst b} \ileadstost \ppstsure \land \ar_{b \pst c} \ileadstost \ppstsure[b][c]: \tau(\ar_{a \pst b}, \ar_{b \pst c}) \ileadstost \ppstsure[a][c]$.

Observe that if a normative theory admits a transitive argument builder, then it is transitive.

TODO It is possible to determine whether a given function $\tau$ is a transitive argument builder empiricially. But not to prove empirically that a normative theory is not transitive.

\section{Decision situation}
\NewDocumentCommand{\ileadstoprop}{}{⇝^\prop}
\NewDocumentCommand{\ileadstonprop}{}{⇝^{¬\prop}}
\NewDocumentCommand{\ileadstosts}{}{⇝_*}
\NewDocumentCommand{\ileadstos}{}{\overset{s}{⇝_i}}
\NewDocumentCommand{\ileadstostw}{O{}}{⇝^\mathit{w}_\forall}
\NewDocumentCommand{\ileadstostp}{O{}}{⇝^\mathit{pos}_\forall}
\NewDocumentCommand{\nileadstosts}{}{\not⇝_*}
\NewDocumentCommand{\Ti}{}{T_i}
\NewDocumentCommand{\ibeatse}{O{}}{⊳^{#1}_\exists}
\NewDocumentCommand \ibeatseinv { o }{
	\IfValueTF{#1}{%
		{⊳_\exists^{#1}}^{-1}%
	}{%
		⊳_\exists^{-1}%
	}%
}
\NewDocumentCommand{\mPhi}{}{\allprops_\gamma}
\NewDocumentCommand{\argsd}{}{S^\mathit{d}}
\NewDocumentCommand{\ileadstoall}{O{}}{⇝^{#1}_\forall}
\NewDocumentCommand{\nileadstoall}{O{}}{\not⇝^{#1}_\forall}

The object of study in a given decision situation is the deliberated preferences of a given individual $i$. 
A decision situation concerns a topic $\allprops$, given a set of arguments $\allargs$, given a certain query protocol, and over a certain time frame. The relation $\ileadstoall$, to be defined shortly, models the reaction of $i$ to arguments about the topic. The decision situation will be defined as a triple $(\allprops, \allargs, \ileadstoall)$ satisfying some conditions to be defined after having presented those three fundamental elements.

The decision situation relates to a topic denoted by $\allprops$, which is a set of propositions $\prop \in \allprops$ about which we are interested of knowing the deliberated preferences of $i$. Propositions are not described further, but are supposedly understandable by $i$ (for example, they could be sentences in some natural language). Alternatively, in the case of an analyst helping $i$ to make a decision, $\allprops$ is a set of propositions about which $i$ is interested to know his own deliberated preferences. I assume $\allprops$ is closed under negation and introduce a symbol $¬$ for negating a proposition: if $\prop \in \allprops$, then $¬\prop \in \allprops$, and $¬(¬\prop) = \prop$. 

With the decision situation comes also a set of all arguments $\allargs$. This set contains all the arguments that may possibly be considered relevant by $i$ for his decision problem, and possibly more.
Importantly, the notion of deliberated preferences does not require to constrain a priori $\allargs$ to some set of arguments that would fit some precise notion of relevancy, coherence, or even well-formedness. This permits to avoid introducing inadequate normative principles into the deliberated preferences: only the individual $i$ is then considered legitimate to dictate what is a relevant argument. Under that view, $\allargs$ may contain anything that can possibly be considered as an argument by anyone, under the widest possible conception of an argument. The notion of deliberated judgment can however also be applied when considering a restricted set of arguments, for example, the arguments that have been put forward by some specific set of experts when talking about the topic $\allprops$. 

The set $\allargs$ may be infinite or very large: as we will see, thanks to our falsificationist approach, there is no need to be able to explore it entirely. The set of arguments contains at least the empty argument, denoted by $\zar$.

\begin{example}
	Consider a decision problem where a set of alternatives $\allalts$ is given, containing food products among which $i$ would like to form a deliberated preference (taking into account the effects of food on health, price, pleasure, morality issues related to the production process, and so on). The topic could be defined as $\allprops$ = $\set{\prop_\alt, ¬\prop_\alt, \forall \alt \in \allalts}$, where $\prop_\alt$ is the proposition according to which $\alt$ is one of the best alternatives among $\allalts$ (there is no strictly better alternatives among $\allalts$, from $i$’s point of view), and $¬\prop_\alt$ is the proposition according to which $\alt$ is not one of the best alternatives among $\allalts$ (some $\alt' \in \allalts$ is a strictly better alternative, from $i$’s point of view). 
	The set $\allargs$ is defined as the set of all strings, because in this example it is considered acceptable to restrict arguments to those that can take a textual form.
\end{example}

Define $\allprops = \allprops × \set{\text{possible}, \text{sure}}$. Elements in $\allprops$ will also be called propositions, which should create no ambiguity. 
The relation $\ileadstoall \subseteq (\allargs × \allargs) × \allprops$, pronounced “always leads to”, has the following semantics.
Whenever $(\ar, \ar') \ileadstoall (\prop, \text{sure})$, $i$ always considers that $\prop$ is sure (meaning that $¬\prop$ is excluded), when presented with these two arguments, and whenever $(\ar, \ar') \ileadstoall (\prop, \text{possible})$, $i$ always considers that $\prop$ is “at least” possible (meaning that $¬\prop$ may be possible as well), when presented with these two arguments. In both cases, the term “always” means that the consideration is stable over time, including after having presented other arguments to $i$. 
Two query protocols presented below will illustrate how these semantics might be satisfied. 
However, I voluntarily leave the querying protocol not precisely specified in general, as multiple reasonable choices are possible. The results of this article depend only on assumptions about $\ileadstoall$. 

Define the negation operator $¬$ over $\allprops$ as: $¬(\prop, \text{possible}) = (¬\prop, \text{sure})$ and $¬(\prop, \text{sure}) = (¬\prop, \text{possible})$. It follows that $¬¬\prop = \prop$. Write $(\ar, \ar') \nileadstoall \prop$ for $¬[(\ar, \ar') \ileadstoall \prop]$ and define $(\ar, \ar') \ileadstoe \prop$ as equivalent to $(\ar, \ar') \nileadstoall ¬\prop$. 
The relation $\ileadstoe$ is pronounced “sometimes leads to”.
A proposition $\prop$ is about $\prop$ iff $\prop \in \{\prop, ¬\prop\} × \{\text{possible}, \text{sure}\}$. 

The pair $(\ar, \ar')$ is to be understood as unordered (thus $(\ar, \ar') \ileadstoall \prop ⇔ (\ar', \ar) \ileadstoall \prop$).

\begin{definition}[Decision situation]
	A decision situation is a topic $\allprops$ closed under negation, a set of arguments $\allargs$, an “always leads to” relation $\ileadstoall \subseteq (\allargs × \allargs) × \allprops$ satisfying (A1) and (A2) and using unordered pairs of arguments, $\allprops$ being the set of propositions $\allprops × \set{\text{sure}, \text{possible}}$.
\end{definition}

\subsection{Query protocols}
As vNM aptly summarize, “It is clear that every measurement – or rather every claim of measurability – must ultimately be based on some immediate sensation, which possibly cannot and certainly need not be analyzed any futher.” The fundamental element here is the reaction of $i$ to arguments, which will serve as the informational basis to define $\ileadstoall$. Given a proposition $\prop$ and two arguments $\ar, \ar' \in \allargs$, a query consists in presenting both arguments to $i$ and observing which proposition $i$ considers valid in her current state of mind, thus using both arguments and possibly other arguments she has in mind: is it that $\prop$ holds, that $¬\prop$ holds, or that both are possible? The third possibility allows for the case where no argument appears more decisive than the other one. 

It is also necessary to capture the evolution of the position of $i$ towards arguments over time: a repeated query using a given pair of arguments may yield different answers, for example because the repeated question has been interleaved with another query containing other arguments (which $i$ may still have in mind when answering the second repetition), or for any other known or unknown reason. The important information for us is the \emph{set} of answers that $i$ could give to a query involving two given arguments and a given topic, among the set $\{\prop, ¬\prop, \text{both}\}$. The set designates all possible answers over time and over different order of asking the queries.
Hence, given $\prop \in \allprops$, define a relation $\ileadstoprop \subseteq (\allargs × \allargs) × \powersetz{\{\prop, ¬\prop, \text{both}\}}$, where $\powersetz{B}$ designates the subsets of $B$ excluding the emptyset, with the semantics that $(\ar, \ar') \ileadstoprop C$, with $C \subseteq \set{\prop, ¬\prop, \text{both}}$, iff $\forall c \in C$, it is observable at least once that $i$, when presented with $(\ar, \ar')$, considers $c$ valid in her current state of mind. The querying protocol is supposedly made so that for any $\prop \in \allprops$, $\ileadstoprop = \ileadstonprop$.

This information is only indirectly observable: for example, if $(\ar, \ar') \ileadstoprop \{\prop\}$, a querier will never know more than $(\ar, \ar') \ileadstoprop C$ with $\prop \in C \subseteq \set{\prop, ¬\prop, \text{both}}$. 
Also, it may be only possible to test one ordering of the queries (assuming that $i$ forgets over time might permit to test different orderings, but this assumption might be unrealistic). In some cases, it may even be only possible to test a single query with a given individual.
This limitation of our observation will not be a problem for the results of this article. Intuitively, this bears on the fact that, first, theories apply to multiple individuals, and second, if a theory claims that $(\ar, \ar') \ileadstoprop \{\prop\}$, then either this claim is correct, or it is falsifiable, in the sense that there is a possibility of observing a contradiction to its claim.

To be more concrete, one possible choice of a query protocol, when arguments and propositions are strings, is to read both arguments to $i$, together with $\prop$, and ask $i$ to verbally report the choice he opts for among the three possibilities. Another is to present both arguments and ask $i$ to pick an item among some choice set and make the choice correspond to a validation of $\prop$ or $¬\prop$, possibly even letting the choice engage $i$, for example telling $i$ that she may keep the object of choice (this is illustrated in \cref{ex:pick}). This second possibility is especially interesting as it makes the protocol partially observable in the restricted sense usually appreciated in revealed preferences approaches. Note that depending on the protocol, the choice “both” is not necessarily observable (more about this below).
%In that case, the semantics of $\prop$ and $¬\prop$ relatedly change: instead of meaning that $\prop$ is definitely the 
%Indeed, \citet{tversky_intransitivity_1969} note that individuals “are not perfectly consistent in their choices. When faced with repeated choices between x and y, people often choose x in some instances and y in others. Furthermore, such inconsistencies are observed even in the absence of systematic changes in the decision maker’s taste which might be due to learning or sequential effects. It seems, therefore, that the observed inconsistencies reflect inherent variability or momentary fluctuation in the evaluative process.” This is an accepted fact of experimental psychology \citep{luce_utility_2000}. However, it is reasonable to suspect that fluctuations occur less often, if at all, for some choices. For example, facing a choice between tea or coffee, I would virtually never pick tea (all other things being equal). We are interested in capturing precisely those kind of preferences.

One can also use $\zar$ as one or both of the arguments, in which case the protocol queries the preference of $i$ given only one or given no argument.

\begin{example}[cont.]
	\label{ex:pick}
	Continuing the example, define the querying protocol as, given $\prop_\alt$, presenting two arguments to $i$ and let $i$ choose an item among $\allalts$. $i$ may keep the chosen item. 
	If $i$ chooses $\alt$, it is considered that $i$ has validated $\prop_\alt$, otherwise, $¬\prop_\alt$. The option “both” is not observable.
%	Queries are separated by one week at least, which (as supposed in this example) is enough to ensure that no effect of pleasure for variability enter into play. 
\end{example}

Define $\ileadstoall$ from $\ileadstoprop$ as follows, for any $\prop \in \allprops$: $(\ar, \ar') \ileadstoall (\prop, \text{possible}) ⇔ (\ar, \ar') \ileadstoprop C$ with $C \subseteq \set{\prop, \text{both}}$; and $(\ar, \ar') \ileadstoall (\prop, \text{sure}) ⇔ (\ar, \ar') \ileadstoprop \set{\prop}$. 

Note that (A1) and (A2) hold.

When “both” is not observable, $\ileadstoall$ is defined as: $(\ar, \ar') \ileadstoall (\prop, \text{possible}) ⇔ (\ar, \ar') \ileadstoall (\prop, \text{sure}) ⇔ (\ar, \ar') \ileadstoprop \set{\prop}$. (A1) and (A2) still hold. We say in this case that $\ileadstoall$ does not distinguish sure from possible. Thus, elements $\prop \in \allprops$ may be considered as simply equal to $\prop$ or $¬\prop$, as $(\prop, \text{sure})$ is treated exactly as $(\prop, \text{possible})$.

\subsection{Example}
\begin{example}[(cont.)]
	Returning to the previous example, a theory $T$ could claim that, for all individuals that belong to some given socio-economic situation and that have some given degree (described by the theory in sufficient details that it is possible to determine precisely to which kind of individuals it applies), $(\prop_a, \text{sure})$ is in the deliberated preferences of the individuals, for a given alternative $\alt$ also described by $T$. 
The theory also suggests some argument $\ar$, a text that (according to $T$) will convince any individual (fitting the description) that $\prop_\alt$ holds, thus, that $\alt$ is a “good” food product for him. 
The theory can be put to the test by picking an individual fitting the description, presenting the argument together with any other argument (for example, proposed by another theory), and observing whether $i$ is convinced.
$T$ resists to such a falsification test if $i$ is convinced. According to a previous theorem, $T$ tells the truth if it would resist to any possible falsification test. (This is impossible to definitely make sure of.)
\end{example}
This example illustrates a difference between this proposal and the classical revealed approach: it could be that $T$ tells the truth even though its claim does not correspond to the revealed preference of some individuals, thus that, given no argument, some individuals would consider $a$ as dominated by another alternative in the set considered. It also illustrates a difference between this proposal and persuasion: the goal of $T$ is not merely to convince $i$ (possibly by using its lack of knowledge of any counter-argument to what $T$ says), but to resist any falsification attempt. Thus, $T$ should be confronted to arguments coming from as varied perspectives as possible, in order to give confidence that it tells the truth.

\section{Empirical theories}
\subsection{Simple to complex}
From an additive single-answer empirical theory, we can define a function: $\forall \ar_1 \gbeats \ar_0: f(\ar_1, \ar_0) = \ar_0 + {\gbeatsinv}(\ar_1) = \set{\ar_0 + \ard \suchthat \ard \in {\gbeatsinv}(\ar_1)}$, and $f(\ar_1, \ar_0) = \emptyset$ otherwise. Thus, $f = \set{((\ar_1, \ar_0), \ar_2) \suchthat \exists \ard \in \allargs \suchthat \ard \gbeats \ar_1 \gbeats \ar_0 \land \ar_2 = \ar_0 + \ard}$.
Also, from the previous definition, $g(\ar_0)(\ar_0 + \argsd_2) = dom(f^{-1}(\ar_0 + \argsd_2) \cap (\allargs × \set{\ar_0})) = {\gbeats}(\argsd_2) \cap {\gbeatsinv}(\ar_0)$ because $(\ar_1, \ar_0) \in f^{-1}(\ar_0 + \argsd_2) ⇔ \exists \ar_2 \in \ar_0 + \argsd_2 \suchthat \ar_2 \in f(\ar_1, \ar_0) ⇔ \ar_1 \gbeats \ar_0 \land \exists \ard \in \argsd_2 \suchthat \ard \gbeats \ar_1$.
\commentOC{Could be false.}

Note that $f^{-1}(\ar_0 + \ar_2) ≠ {\gbeats}(\ar_2) × \set{\ar_0}$: if ${\gbeats} = \set{(\ard, \ar_1), (\ar_1, \ar_0), (\ard[6], \ar_5), (\ar_5, \ar_4)}$, and $\ar_0 + \ard = \ar_4 + \ard[6]$, then $f^{-1}(\ar_0 + \ard) = f^{-1}(\ar_4 + \ard[6]) = \set{(\ar_1, \ar_0), (\ar_5, \ar_4)}$ and ${\gbeats}(\ard) × \set{\ar_0} = \set{(\ar_1, \ar_0)}$.
Also, note that $g(\ar_0)(\ar_0 + \argsd_2) ≠ {\gbeats}(\argsd_2) \cap {\gbeatsinv}(\ar_0)$. Consider ${\gbeats} = \set{(\ard, \ar_1), (\ar_1, \ar_0)}, \ar_2 = \ar_0 + \ard = \ar_0 + \ard$. Then, $(\ar_1, \ar_0) \in f^{-1}(\ar_0 + \ard)$, hence, $\ar_1 \in g(\ar_0)(\ar_0 + \ard)$, but ${\gbeats}(\ard) = \emptyset$.

\subsection{Falsifiability}
\commentOC{I should define $\Box(\hist \ileadsto \prop)$ iff it has been observed that $\hist \ileadsto \prop$. Then, axiom Correct observations (that is self-evident): $\Box(\hist \ileadsto \prop) ⇒ \hist \ileadsto \prop$ (what has been observed is). Supplementary axiom Falsification is enough (that is not self-evident): $\not\Box(\hist \ileadsto \prop) ⇒ \hist \nileadsto \prop$ (what has not been observed is not).}

\commentOC{I want: An empirical theory for a normative theory is something like a normative and falsifiable claim. It is also configurable: it must be possible to guarantee that the falsification be easy (if the claim is false) depending on the kind of claim.}

Given $\hist \in \allhist$ and $i \in I$, the claim $\hist \ileadstoe[i] \prop$ may not be verifiable, and is too specific to be falsifiable. However, given $\hist \in \allhist$, a broader claim is falsifiable: that for a given set of individuals $Q \subseteq I, \forall i \in Q, \hist \ileadstoe[i] \prop$ (provided $Q$ is big enough). Indeed, that claim it is equivalent to $\nexists i \in Q \suchthat \hist \nileadstoe[i] \prop$; exhibiting an $i \in Q$ such that $\hist \nileadstoe[i] \prop$ thus falsifies the broader claim.

Given any $(\allprops, ¬, \allargs, I)$, an \emph{observation protocol} is a binary tree whose non-leaf nodes are among $I × \allhist × \allprops$ and whose two edges out of any non leaf node are labeled by “yes” and “no”; the leaf nodes being simply $\emptyset$. The observation protocol is said to assume memory iff for each non-leaf node $n = (i, \hist, \prop)$ about $i$, any node $n' = (i, {\hist}', \prop')$ about $i$ that is further down the tree continues the history, that is, the sequence ${\hist}'$ starts with $\hist$. Given an observation protocol and observables $(\ileadstoe[i])_{i \in I}$, the \emph{observed path} refers to the path in the tree defined recursively as consisting of the root node and, for each non-leaf node $n = (i, \hist, \prop)$ in the path, the child node along the branch “yes” if $\hist \ileadstoe[i] \prop$ and the child node along the branch “no” otherwise. The \emph{observations} refer to the set of nodes along the observed path.

Given any $(\allprops, ¬, \allargs, I)$, a \emph{normative claim} about $(\allprops, ¬, \allargs, I)$ (or about $\allargs$, when the other defined objects are left implicit) is any first-order logic proposition involving the atoms $[(\hist, \prop) \in {\ileadstoe[i]}], \hist \in \allhist, \prop \in \allprops$ which, together with the axioms, permits to deduce that $\prop \in \iprops$ for some $i \in I$ and $\prop \in \allprops$. The claim is said to be \emph{about} the given set $\allargs$. An observation protocol, together with observables $(\ileadstoe[i])_{i \in I}$, \emph{falsifies} a normative claim iff the falsity of the claim is deducible from the axioms and the observations. A normative claim is \emph{falsifiable} iff there exists a finite observation protocol such that for any observables $(\ileadstoe[i])_{i \in I} \in (\allhist × \allprops)^I$, if the claim is false given these observables, then it is possible to deduce it from the axioms and the observations given the observables.

\commentOC{Falsifiable may not be the right word. Perhaps “provable“, meaning “tautological or Popper-falsifiable”. “Falsifiable” suggests that if a claim is not falsifiable, then a weaker claim is not falsifiable either, which does not hold. My falsifiability criterion demands that if a claim is false, we can see it; whether Popper’s demands that the claim be possibly false. Perhaps I should demand that we can pick a set of consequences of the claim and show it wrong it at least one possible state of the universe.}

A normative claim is \emph{trivial} iff it is about a singleton $\allargs = \set{\ar}$, and has the form: $\ar \ileadstosts \prop$, for some $i \in I$. A normative claim is minimal iff it has the form: $\exists \ar \in \allargs \suchthat \ar \ileadstosts \prop$, for some $i \in I$ and $\prop \in \allprops$. Note that given a minimal claim, $\card{\allargs} = 1$ is equivalent to the claim being trivial.

\begin{proposition}[Trivial claims are not falsifiable]
	Given any $(\allprops, ¬, \allargs ≠ \emptyset, I)$, no minimal normative claim is falsifiable.
\end{proposition}
\begin{proof}
	Consider any claim about some $i$ and some $\prop$ and any finite observation protocol. We have to show that for some observables, the claim is false, but it is impossible to observe it (meaning, to deduce it from the axioms and observations).
	
	Given $k \in \N$, define $\ileadstosts[k]$ as $\ar^k \ileadstosts \prop$ and $\ar^{k+1} \ileadstosts ¬\prop$ (where $\ar^k$ designates a finite sequence repeating $k$ times $\ar$, for $\ar \in \allargs$). I claim that for some $k$, the observables that include this relation makes the claim false, but not observably so. This is because the observation protocol is finite, thus, only permits to observe some finite number of instances of the relation $\ileadsto$. Suffices to define $k$ as that number.
\end{proof}

There is another reason for non falsifiability, which does not involve the problem of finiteness.
\begin{proposition}[Minimal non-trivial claims are not falsifiable]
	Given any $(\allprops, ¬, \allargs, I), \card{\allargs} ≥ 2$, no minimal normative claim is falsifiable.
\end{proposition}
\begin{proof}
	Define $\ar_1$ as the first element of the sequence $\hist$ of the root node of the observation protocol (if the observation protocol is empty, suffices to define any observables that make the claim false). Define the observables so that $\forall \hist \in \allhist: \ar_2, \hist \ileadsto ¬\prop$. This proves that no argument is decisive, hence $\prop \notin \Box\Phi$, and the claim is false. But the observables may also be defined so as to satisfy the protocol (defined as: when going through the protocol, the observations can’t fail the claim).
\end{proof}

\subsection{Definition}
A general empirical theory for a normative theory $(\allprops, ¬, \allargs, I, (\ileadstoe[i])_{i \in I})$ is a tuple $({\gleadsto}, f)$, where ${\gleadsto} \subseteq \allargs × \allprops$ and $f \subseteq (\allargs × \allargs) × \allargs$ with $f(\allargs × \allargs)$ being finite.

Define $g(\ar_0) \subseteq \allargs × \allargs$ as $g(\ar_0) = \set{(\ar_2, \ar_1) \suchthat \ar_2 \in f(\ar_1, \ar_0)} = \bigcup_{\ar_1 \in \allargs} (f(\ar_1, \ar_0) × \set{\ar_1})$. The set $g(\ar_0)(\allargs)$ contains the anticipated attacks on $\ar_0$; the set $g(\ar_0)(\ar_2)$ contains the anticipated attacks on $\ar_0$ for which $\ar_2$ is a planned response; and the set $g(\ar_0)^{-1}(\ar_1) = f(\ar_1, \ar_0)$ contains the planned protections of $\ar_0$ against $\ar_1$. 

Given $\ar_0 \in \allargs$ and $\args_2 \subseteq \allargs$, define the set $\args_1 \subseteq \allargs$ of arguments that $\args_2$ protects $\ar_0$ from as $\args_1 = [f^{-1}(\args_2)]^{-1}(\ar_0) = \set{\ar_1 \in \allargs \suchthat f(\ar_1, \ar_0) \cap \args_2 ≠ \emptyset} = g(\ar_0)(\args_2) = dom(f^{-1}(\args_2) \cap \allargs × \set{\ar_0})$.
 
\subsection{A general condition sufficient for validity and (hopefully) falsifiability}
It should be possible to define this condition as a claim that does not involve $f$, so that it can be done before defining an empirical theory. Then, show that a normative and falsifiable claim must have the form of a convincingness claim.

\begin{remark}
	In fact, the model is tasked with finding a stable point after interrogating $i$. The process of trying counter-arguments $\ar_1, …$ is $q(i)$. When done, the model claims that $\ar_0 + \ar_2 + …$ is decisive. The model also claims that another process of interrogation does not lead to different answers. This is falsifiable.
	When claiming that $(\ar_1, \ar_0) \ileadstoall[i] \prop$, the model really claims that $\nexists p, i' \in q^{-1}(q(i)) \suchthat p, \ar_1, \ar_0 \ileadsto[i'] ¬\prop \lor p, \ar_0, \ar_1 \ileadsto[i'] ¬\prop$, where $i' \in q^{-1}(q(i))$ iff $q(i) = q(i')$.
	
	To say that $\prop \in \iprops$, need to not depend on choice of $q$, in the following sense: if with $q$, $\prop$ seems stable, but with $q'$, $¬\prop$ seems stable, then $\prop$ is not proved as being in $\iprops$. But if with $q$, $\prop$ seems stable, but with $q'$, nothing can be shown to be stable, then $\prop$ is proved (temporarily) as being in $\iprops$. More precisely, after $q'$, the model must be able to stabilize $i$ with its $q$.
\end{remark}

$\ar_0$ $i$-defends $\prop$ when ignoring $S_1$ iff $\forall \hist \in \allhist: [\args_1 \cap \range(\hist) = \emptyset] ⇒ (\hist, \ar_0) \ileadstosts \prop$.
$\ar_0$ $i$-defends $\prop$ under protection of $S_2$ iff $\ar_0$ $i$-defends $\prop$ when ignoring $g(\ar_0)(\args_2)$.

\begin{definition}[Convincingness]
	$\forall i \in I, \prop \in \gpropse, \exists k \in 2\N$ and a finite sequence $(\args_j)_{j \in 2\N, j ≤ k} \suchthat \emptyset ≠ \args_0 \subseteq {\gleadstoinv}(\prop)$ and $\forall j \in 2\N, j ≤ k, \forall \ar_j \in S_j: \ar_j$ $i$-defends $\prop$ under protection of $S_{j + 2}$ (defining $S_{k + 2} = \emptyset$) and $S_j \subseteq \cup_{\ar_{j - 2} \in \args_{j - 2}, \ar_{j - 1} \in \allargs} f(\ar_{j - 1}, \ar_{j - 2})$.
\end{definition}
Note that requiring that $\args_0$ be a singleton does not modify the condition, in the sense that any theory satisfying the condition also satisfies its strenghtening. Similarly, requiring that $S_k ≠ \emptyset$ does not modify the condition (because if some $k$ satisfy the condition with $S_k = \emptyset$, then picking $k' = k-2$ with the same sets $S_j$ also satisfy the condition, and because $S_0 ≠ \emptyset$, this recursion must end with a suitable $k$).

\begin{remark}
	The claim must have the form: some structure is adequate for all individuals. Here, the structure is given by $f$. The claim should claim that looking at the attacks planned by $f$ is enough; and looking at the defenses planned by $f$ is enough.

	This may be adequate as if $\ar_0$ seems suitable but some unplanned attacker $\ar_1$ attacks it, then use $\ar_1 \in \hist$ and observe that $¬(\hist, \ar_0) \ileadstoe \prop$ to deny the claim.
	
	Perhaps this could be phrased as follows. Say that $\ar_2$ is replaceable by $\args_2$ iff what $\ar_2$ protects is either protected by $\args_2$ or replaceable (by what?). Say that a set $\args$ is sufficient iff its complement is replaceable by $\args$. Then $g$ claims (among others) that $g(\ar_0)^{-1}(\ar_1)$ is sufficient.
	
	The claim can (hopefully) be given the form: $\forall i \suchthat …, \ar$ decisive. Given $i$, either $\ar_0$ is decisive, or $\ar_1$ attacks it but then $\ar_2$ is decisive, …
\end{remark}

\begin{theorem}
	If an empirical theory is convincing, and its normative theory satisfies the axioms, then it is valid.
\end{theorem}
\begin{proof}
	Consider $i \in I$ and $\prop \in \gpropse$. 
	By hypothesis, $\exists k$ such that some $\ar_k \in \args_k$ defends $\prop$ under protection of $\emptyset$. Therefore, $\ar_k \ileadstosts \prop$. Using \cref{ax:norm}, validity follows.
\end{proof}

\begin{theorem}
	The convincingness claim of any empirical theory is falsifiable: given an ET, there exists a finite obs prot with memory such that for any observables, if the claim is false given these observables, then it is possible to deduce it from the axioms and observations.
\end{theorem}