\RequirePackage[l2tabu, orthodox]{nag}
\documentclass[version=last, pagesize, twoside=off, bibliography=totoc, DIV=calc, fontsize=12pt, a4paper, french, english]{scrartcl}
%Permits to copy eg x ⪰ y ⇔ v(x) ≥ v(y) from PDF to unicode data, and to search. From pdfTeX users manual. See https://tex.stackexchange.com/posts/comments/1203887.
	\input glyphtounicode
	\pdfgentounicode=1
%Latin Modern has more glyphs than Computer Modern, such as diacritical characters. fntguide commands to load the font before fontenc, to prevent default loading of cmr.
	\usepackage{lmodern}
%Encode resulting accented characters correctly in resulting PDF, permits copy from PDF.
	\usepackage[T1]{fontenc}
%UTF8 seems to be the default in recent TeX installations, but not all, see https://tex.stackexchange.com/a/370280.
	\usepackage[utf8]{inputenc}
%Provides \newunicodechar for easy definition of supplementary UTF8 characters such as → or ≤ for use in source code.
	\usepackage{newunicodechar}
%Text Companion fonts, much used together with CM-like fonts. Provides \texteuro and commands for text mode characters such as \textminus, \textrightarrow, \textlbrackdbl.
	\usepackage{textcomp}
%Solves bug in lmodern, https://tex.stackexchange.com/a/261188; probably useful only for unusually big font sizes; and probably better to use exscale instead. Note that the authors of exscale write against this trick.
	%\DeclareFontShape{OMX}{cmex}{m}{n}{
		%<-7.5> cmex7
		%<7.5-8.5> cmex8
		%<8.5-9.5> cmex9
		%<9.5-> cmex10
	%}{}
	%\SetSymbolFont{largesymbols}{normal}{OMX}{cmex}{m}{n}
%More symbols (such as \sum) available in bold version, see https://github.com/latex3/latex2e/issues/71.
	\DeclareFontShape{OMX}{cmex}{bx}{n}{%
	   <->sfixed*cmexb10%
	   }{}
	\SetSymbolFont{largesymbols}{bold}{OMX}{cmex}{bx}{n}
%For small caps also in italics, see https://tex.stackexchange.com/questions/32942/italic-shape-needed-in-small-caps-fonts, https://tex.stackexchange.com/questions/284338/italic-small-caps-not-working.
	\usepackage{slantsc}
	\AtBeginDocument{%
		%“Since nearly no font family will contain real italic small caps variants, the best approach is to substitute them by slanted variants.” -- slantsc doc
		%\DeclareFontShape{T1}{lmr}{m}{scit}{<->ssub*lmr/m/scsl}{}%
		%There’s no bold small caps in Latin Modern, we switch to Computer Modern for bold small caps, see https://tex.stackexchange.com/a/22241
		%\DeclareFontShape{T1}{lmr}{bx}{sc}{<->ssub*cmr/bx/sc}{}%
		%\DeclareFontShape{T1}{lmr}{bx}{scit}{<->ssub*cmr/bx/scsl}{}%
	}
%Warn about missing characters.
	\tracinglostchars=2
%Nicer tables: provides \toprule, \midrule, \bottomrule.
	%\usepackage{booktabs}
%For new column type X which stretches; can be used together with booktabs, see https://tex.stackexchange.com/a/97137. “tabularx modifies the widths of the columns, whereas tabular* modifies the widths of the inter-column spaces.” Loads array.
	%\usepackage{tabularx}
%math-mode version of "l" column type. Requires \usepackage{array}.
	%\usepackage{array}
	%\newcolumntype{L}{>{$}l<{$}}
%Provides \xpretocmd and loads etoolbox which provides \apptocmd, \patchcmd, \newtoggle… Also loads xparse, which provides \NewDocumentCommand and similar commands intended as replacement of \newcommand in LaTeX3 for defining commands (see https://tex.stackexchange.com/q/98152 and https://github.com/latex3/latex2e/issues/89).
	\usepackage{xpatch}
%ntheorem doc says: “empheq provides an enhanced vertical placement of the endmarks”; must be loaded before ntheorem. Loads the mathtools package, which loads and fixes some bugs in amsmath and provides \DeclarePairedDelimiter. amsmath is considered a basic, mandatory package nowadays (Grätzer, More Math Into LaTeX).
	\usepackage[ntheorem]{empheq}
%Package frenchb asks to load natbib before babel-french. Package hyperref asks to load natbib before hyperref.
	\usepackage{natbib}

\newtoggle{LCpres}
	\newtoggle{LCart}
	\newtoggle{LCposter}
	\makeatletter
	\@ifclassloaded{beamer}{
		\toggletrue{LCpres}
		\togglefalse{LCart}
		\togglefalse{LCposter}
		\wlog{Presentation mode}
	}{
		\@ifclassloaded{tikzposter}{
			\toggletrue{LCposter}
			\togglefalse{LCpres}
			\togglefalse{LCart}
			\wlog{Poster mode}
		}{
			\toggletrue{LCart}
			\togglefalse{LCpres}
			\togglefalse{LCposter}
			\wlog{Article mode}
		}
	}
	\makeatother%

%Language options ([french, english]) should be on the document level (last is main); except with tikzposter: put [french, english] options next to \usepackage{babel} to avoid warning. beamer uses the \translate command for the appendix: omitting babel results in a warning, see https://github.com/josephwright/beamer/issues/449. Babel also seems required for \refname.
%	\iftoggle{LCpres}{
		\usepackage{babel}
%	}{
%	}
	%\frenchbsetup{AutoSpacePunctuation=false}
%listings (1.7) does not allow multi-byte encodings. listingsutf8 works around this only for characters that can be represented in a known one-byte encoding and only for \lstinputlisting. Other workarounds: use literate mechanism; or escape to LaTeX (but breaks alignment).
	%\usepackage{listings}
	%\lstset{tabsize=2, basicstyle=\ttfamily, escapechar=§, literate={é}{{\'e}}1}
%I favor acro over acronym because the former is more recently updated (2018 VS 2015 at time of writing); has a longer user manual (about 40 pages VS 6 pages if not counting the example and implementation parts); has a command for capitalization; and acronym suffers a nasty bug when ac used in section, see https://tex.stackexchange.com/q/103483 (though this might be the fault of the silence package and might be solved in more recent versions, I do not know) and from a bug when used with cleveref, see https://tex.stackexchange.com/q/71364. However, loading it makes compilation time (one pass on this template) go from 0.6 to 1.4 seconds, see https://bitbucket.org/cgnieder/acro/issues/115. Option short-format not usable in the package options as it is fragile, see https://tex.stackexchange.com/q/466882.
	%\usepackage[single]{acro}
	%\acsetup{short-format = {\scshape}}
	%\DeclareAcronym{AMCD}{short=AMCD, long={Aide Multicritère à la Décision}}
\DeclareAcronym{AHP}{short=AHP, long={Analytic Hierarchy Process}}
\DeclareAcronym{AR}{short=AR, long={Argumentative Recommender}}
\DeclareAcronym{DA}{short=DA, long={Decision Analysis}}
\DeclareAcronym{DJ}{short=DJ, long={Deliberated Judgment}}
\DeclareAcronym{DM}{short=DM, long={Decision Maker}}
\DeclareAcronym{DP}{short=DP, long={Deliberated Preference}}
\DeclareAcronym{MAVT}{short=MAVT, long={Multiple Attribute Value Theory}}
\DeclareAcronym{MCDA}{short=MCDA, long={Multicriteria Decision Aid}}
\DeclareAcronym{MIP}{short=MIP, long={Mixed Integer Program}}
\DeclareAcronym{SEU}{short=SEU, long={Subjective Expected Utility}}


\iftoggle{LCpres}{
	%I favor fmtcount over nth because it is loaded by datetime anyway; and fmtcount warns about possible conflicts when loaded after nth.
	\usepackage{fmtcount}
	%For nice input of date of presentation. Must be loaded after the babel package. Has possible problems with srcletter: https://golatex.de/verwendung-von-babel-und-datetime-in-scrlttr2-schlaegt-fehlt-t14779.html.
	\usepackage[nodayofweek]{datetime}
}{
}
%For presentations, Beamer implicitely uses the pdfusetitle option. ntheorem doc says to load hyperref “before the first use of \newtheorem”. autonum doc mandates option hypertexnames=false. I want to highlight links only if necessary for the reader to recognize it as a link, to reduce distraction. In presentations, this is already taken care of by beamer (https://tex.stackexchange.com/a/262014). If using colorlinks=true in a presentation, see https://tex.stackexchange.com/q/203056. Crashes the first compilation with tikzposter, just compile again and the problem disappears, see https://tex.stackexchange.com/q/254257.
\makeatletter
\iftoggle{LCpres}{
	\usepackage{hyperref}
}{
	\usepackage[hypertexnames=false, pdfusetitle, linkbordercolor={1 1 1}, citebordercolor={1 1 1}, urlbordercolor={1 1 1}]{hyperref}
	%https://tex.stackexchange.com/a/466235
	\pdfstringdefDisableCommands{%
		\let\thanks\@gobble
	}
}
\makeatother
%urlbordercolor is used both for \url and \doi, which I think shouldn’t be colored, and for \href, thus might want to color manually when required. Requires xcolor.
	\NewDocumentCommand{\hrefblue}{mm}{\textcolor{blue}{\href{#1}{#2}}}
%hyperref doc says: “Package bookmark replaces hyperref’s bookmark organization by a new algorithm (...) Therefore I recommend using this package”.
	\usepackage{bookmark}
%Need to invoke hyperref explicitly to link to line numbers: \hyperlink{lintarget:mylinelabel}{\ref*{lin:mylinelabel}}, with \ref* to disable automatic link. Also see https://tex.stackexchange.com/q/428656 for referencing lines from another document.
	%\usepackage{lineno}
	%\NewDocumentCommand{\llabel}{m}{\hypertarget{lintarget:#1}{}\linelabel{lin:#1}}
	%\setlength\linenumbersep{9mm}
%For complex authors blocks. Seems like authblk wants to be later than hyperref, but sooner than silence. See https://tex.stackexchange.com/q/475513 for the patch to hyperref pdfauthor.
	\ExplSyntaxOn
	\seq_new:N \g_oc_hrauthor_seq
	\NewDocumentCommand{\addhrauthor}{m}{
		\seq_gput_right:Nn \g_oc_hrauthor_seq { #1 }
	}
	%Should be \NewExpandableDocumentCommand, but this is not yet provided by my version of xparse
	\DeclareExpandableDocumentCommand{\hrauthor}{}{
		\seq_use:Nn \g_oc_hrauthor_seq {,~}
	}
	\ExplSyntaxOff
	{
		\catcode`#=11\relax
		\gdef\fixauthor{\xpretocmd{\author}{\addhrauthor{#2}}{}{}}%
	}
	\iftoggle{LCart}{
		\usepackage{authblk}
		\renewcommand\Affilfont{\small}
		\fixauthor
		\AtBeginDocument{
		    \hypersetup{pdfauthor={\hrauthor}}
		}
	}{
	}
%I do not use floatrow, because it requires an ugly hack for proper functioning with KOMA script (see scrhack doc). Instead, the following command centers all floats (using \centering, as the center environment adds space, http://texblog.net/latex-archive/layout/center-centering/), and I manually place my table captions above and figure captions below their contents (https://tex.stackexchange.com/a/3253).
	\makeatletter
	\g@addto@macro\@floatboxreset\centering
	\makeatother
%Permits to customize enumeration display and references
	%\nottoggle{LCpres}{
		%\usepackage{enumitem} %follow list environments by a string to customize enumeration, example: \begin{description}[itemindent=8em, labelwidth=!] or \begin{enumerate}[label=({\roman*}), ref={\roman*}].
	%}{
	%}
%Provides \Cen­ter­ing, \RaggedLeft, and \RaggedRight and en­vi­ron­ments Cen­ter, FlushLeft, and FlushRight, which al­low hy­phen­ation. With tikzposter, seems to cause 1=1 to be printed in the middle of the poster.
	%\usepackage{ragged2e}
%To typeset units by closely following the “official” rules.
	%\usepackage[strict]{siunitx}
%Turns the doi provided by some bibliography styles into URLs. However, uses old-style dx.doi url (see 3.8 DOI system Proxy Server technical details, “Users may resolve DOI names that are structured to use the DOI system Proxy Server (https://doi.org (current, preferred) or earlier syntax http://dx.doi.org).”, https://www.doi.org/doi_handbook/3_Resolution.html). The patch solves this.
	\usepackage{doi}
	\makeatletter
	\patchcmd{\@doi}{http://dx.doi.org}{https://doi.org}{}{}
	\makeatother
%Makes sure upper case greek letters are italic as well.
	\usepackage{fixmath}
%Provides \mathbb; obsoletes latexsym (see http://tug.ctan.org/macros/latex/base/latexsym.dtx). Relatedly, \usepackage{eucal} to change the mathcal font and \usepackage[mathscr]{eucal} (apparently equivalent to \usepackage[mathscr]{euscript}) to supplement \mathcal with \mathscr. This last option is not very useful as both fonts are similar, and the intent of the authors of eucal was to provide a replacement to mathcal (see doc euscript). Also provides \mathfrak for supplementary letters.
	\usepackage{amsfonts}
%Provides a beautiful (IMHO) \mathscr and really different than \mathcal, for supplementary uppercase letters. But there is no bold version. Alternative: mathrsfs (more slanted), but when used with tikzposter, it warns about size substitution, see https://tex.stackexchange.com/q/495167.
	\usepackage[scr]{rsfso}
%Multiple means to produce bold math: \mathbf, \boldmath (defined to be \mathversion{bold}, see fntguide), \pmb, \boldsymbol (all legacy, from LaTeX base and AMS), \bm (the most recommended one), \mathbold from package fixmath (I don’t see its advantage over \boldsymbol).
%“The \boldsymbol command is obtained preferably by using the bm package, which provides a newer, more powerful version than the one provided by the amsmath package. Generally speaking, it is ill-advised to apply \boldsymbol to more than one symbol at a time.” — AMS Short math guide. “If no bold font appears to be available for a particular symbol, \bm will use ‘poor man’s bold’” — bm. It is “best to load the package after any packages that define new symbol fonts” – bm. bm defines \boldsymbol as synonym to \bm. \boldmath accesses the correct font if it exists; it is used by \bm when appropriate. See https://tex.stackexchange.com/a/10643 and https://github.com/latex3/latex2e/issues/71 for some difficulties with \bm.
	\usepackage{bm}
	\nottoggle{LCpres}{
	%https://ctan.org/pkg/amsmath recommends ntheorem, which supersedes amsthm, which corrects the spacing of proclamations and allows for theoremstyle. Option standard loads amssymb and latexsym. Must be loaded after amsmath (from ntheorem doc). From cleveref doc, “ntheorem is fully supported and even recommended”; says to load cleveref after ntheorem. When used with tikzposter, warns about size substitution for the lasy (latexsym) font when using \url, because ntheorem loads latexsym; relatedly (but not directly related to ntheorem), size substitution warning with the cmex font happens when loading amsmath and using \url.
		\usepackage[thmmarks, amsmath, standard, hyperref]{ntheorem}
		%empheq doc says to do this after loading ntheorem
		\usetagform{default}
	%Provides \cref. Unfortunately, cref fails when the language is French and referring to a label whose name contains a colon (https://tex.stackexchange.com/q/83798). Use \cref{sec\string:intro} to work around this. cleveref should go “laster” than hyperref.
		\usepackage{cleveref}
	}{
	}
	\nottoggle{LCposter}{
	%Equations get numbers iff they are referenced. Loading order should be “amsmath → hyperref → cleveref → autonum”, according to autonum doc. Use this in preference to the showonlyrefs option from mathtools, see https://tex.stackexchange.com/q/459918 and autonum doc. See https://tex.stackexchange.com/a/285953 for the etex line. Incompatible with my version of tikzposter (produces “! Improper \prevdepth”).
		\expandafter\def\csname ver@etex.sty\endcsname{3000/12/31}\let\globcount\newcount
		\usepackage{autonum}
	}{
	}
%Also loaded by tikz.
	\usepackage{xcolor}
\iftoggle{LCpres}{
	\usepackage{tikz}
	%\usetikzlibrary{babel, matrix, fit, plotmarks, calc, trees, shapes.geometric, positioning, plothandlers, arrows, shapes.multipart}
}{
}
%Vizualization, on top of TikZ
	%\usepackage{pgfplots}
	%\pgfplotsset{compat=1.14}
\usepackage{graphicx}
	\graphicspath{{graphics/}}

%Provides \print­length{length}, useful for debugging.
	%\usepackage{printlen}
	%\uselengthunit{mm}

\iftoggle{LCpres}{
	\usepackage{appendixnumberbeamer}
	%I have yet to see anyone actually use these navigation symbols; let’s disable them
	\setbeamertemplate{navigation symbols}{} 
	\usepackage{preamble/beamerthemeParisFrance}
	\setcounter{tocdepth}{10}
}{
}

%Do not use the displaymath environment: use equation. Do not use the eqnarray or eqnarray* environments: use align(*). This improves spacing. (See l2tabu or amsldoc.)


\newcommand{\R}{ℝ}
\newcommand{\N}{ℕ}
\newcommand{\Z}{ℤ}
\newcommand{\card}[1]{\lvert{#1}\rvert}
\newcommand{\powerset}[1]{\mathscr{P}(#1)}%\mathscr rather than \mathcal: scr is rounder than cal (at least in XITS Math).
\newcommand{\suchthat}{\;\ifnum\currentgrouptype=16 \middle\fi|\;}
%\newcommand{\Rplus}{\reels^+\xspace}

\AtBeginDocument{%
	\renewcommand{\epsilon}{\varepsilon}
% we want straight form of \phi for mathematics, as recommended in UTR #25: Unicode support for mathematics.
%	\renewcommand{\phi}{\varphi}
}

% with amssymb, but I don’t want to use amssymb just for that.
% \newcommand{\restr}[2]{{#1}_{\restriction #2}}
%\newcommand{\restr}[2]{{#1\upharpoonright}_{#2}}
\newcommand{\restr}[2]{{#1|}_{#2}}%sometimes typed out incorrectly within \set.
%\newcommand{\restr}[2]{{#1}_{\vert #2}}%\vert errors when used within \Set and is typed out incorrectly within \set.
\DeclareMathOperator*{\argmax}{arg\,max}
\DeclareMathOperator*{\argmin}{arg\,min}


\NewDocumentCommand{\range}{}{R}

%Decision Theory (MCDA and SC)
\NewDocumentCommand{\allalts}{}{\mathscr{X}}
\NewDocumentCommand{\allcrits}{}{\mathscr{C}}
\NewDocumentCommand{\alts}{}{X}
\NewDocumentCommand{\alt}{}{x}
\NewDocumentCommand{\altp}{}{y}%alt prime, another alt
\NewDocumentCommand{\dm}{}{i}
\NewDocumentCommand{\allF}{}{\mathscr{F}}
\NewDocumentCommand{\allvoters}{}{\mathscr{N}}
\NewDocumentCommand{\voters}{}{N}
\NewDocumentCommand{\allprofs}{}{\boldsymbol{\mathcal{R}}}
\NewDocumentCommand{\prof}{}{\boldsymbol{R}}
\NewDocumentCommand{\linors}{}{\mathscr{L}(\allalts)}
%Thanks to https://tex.stackexchange.com/q/154549
	%\makeatletter
	%\def\@myRgood@#1#2{\mathrel{R^X_{#2}}}
	%\def\myRgood{\@ifnextchar_{\@myRgood@}{\mathrel{R^X}}}
	%\makeatother
\NewDocumentCommand{\ind}{}{\sim}
\NewDocumentCommand{\peq}{}{\succeq}
\NewDocumentCommand{\pst}{}{\succ}
\NewDocumentCommand{\npeq}{}{\nsucceq}
\NewDocumentCommand{\npst}{}{\nsucc}

%Deliberated Judgment
%%Normative theory
\NewDocumentCommand{\allargs}{}{\mathscr{A}}
\NewDocumentCommand{\args}{}{A}
\NewDocumentCommand{\ard}{O{}}{a^\mathit{d}_{#1}}
\NewDocumentCommand{\ardp}{O{}}{a^{\mathit{d}\prime}_{#1}}
\NewDocumentCommand{\ar}{o}{%
	\IfValueTF{#1}{%
		a^{(#1)}%
	}{%
		a%
	}%
}
\NewDocumentCommand{\zar}{}{\mathbf{0}}%zero, or empty, argument
\NewDocumentCommand{\allhist}{}{\mathscr{A}^*}
\NewDocumentCommand{\hist}{}{α}
\NewDocumentCommand{\histp}{}{α^{\prime}}
\NewDocumentCommand{\histpp}{}{α^{\prime\prime}}
\NewDocumentCommand{\histend}{o}{%
	\IfValueTF{#1}{%
		α^{#1}_\mathit{end}%
	}{%
		α_\mathit{end}%
	}%
}
\NewDocumentCommand{\histpend}{}{α^{\prime}_\mathit{end}}
\NewDocumentCommand{\histppend}{}{α^{\prime\prime}_\mathit{end}}
\NewDocumentCommand{\allprops}{}{\Phi}
\NewDocumentCommand{\prop}{}{φ}
\NewDocumentCommand{\propbar}{}{φ'}%\overline
\NewDocumentCommand{\incompat}{}{\Phi^\mathit{incompat}}
%%Empirical theory
\NewDocumentCommand{\gC}{}{C_γ}
\NewDocumentCommand{\gPhi}{}{\Phi_γ}
\NewDocumentCommand{\gpropse}{O{γ}}{{\hookrightarrow_{#1}}(\allargs)}%e for explicit
\NewDocumentCommand{\gprops}{O{γ}}{\Phi_{#1}}
\NewDocumentCommand{\dargs}{O{}}{A^\mathit{d}_{#1}}
\NewDocumentCommand{\alldargs}{}{\mathscr{A}^d}
\NewDocumentCommand{\gargs}{O{φ}}{A^{#1}_{γ, i}}
\NewDocumentCommand{\gargsmu}{}{A^{φ}_{μ, i}}
\NewDocumentCommand{\gargsnu}{}{A^{φ'}_{ν, i}}
\NewDocumentCommand{\gargsgamma}{}{A^{φ}_{γ, i}}
\NewDocumentCommand{\gargsdelta}{}{A^{φ'}_{δ, i}}
\NewDocumentCommand{\gleadsto}{O{γ}}{\hookrightarrow_{#1}}
\NewDocumentCommand{\gleadstoinv}{O{γ}}{{\hookrightarrow^{-1}_{#1}}}
\NewDocumentCommand{\gbeats}{O{γ}}{⊳^\mathit{t}_{#1}}
\NewDocumentCommand{\gbeatsinv}{O{γ}}{{(⊳^\mathit{t}_{#1})^{-1}}}
\NewDocumentCommand{\ngbeats}{O{γ}}{\not⊳^\mathit{t}_{#1}}
\NewDocumentCommand{\dbeats}{O{γ}}{⊳^\mathit{d}_{#1}}
\NewDocumentCommand{\dbeatsinv}{O{γ}}{{(⊳^\mathit{d}_{#1})^{-1}}}
\NewDocumentCommand{\df}{O{γ}}{\mathit{def}_{#1}}
\NewDocumentCommand{\dfp}{O{γ}}{\mathit{def}_{#1}^+}
\NewDocumentCommand{\dg}{O{γ}}{d_{#1}}
\NewDocumentCommand{\dgip}{O{γ, i}}{d^\phi_{#1}}
%%%DP
\NewDocumentCommand{\choices}{}{\mathscr{C}}
\NewDocumentCommand{\gind}{O{}}{\sim_\gamma^{#1}}
\NewDocumentCommand{\gpeq}{}{\succeq_\gamma}
\NewDocumentCommand{\gpst}{}{\succ_\gamma}
\NewDocumentCommand{\ngpeq}{}{\nsucceq_\gamma}
\NewDocumentCommand{\ngpst}{}{\nsucc_\gamma}

%%i
\NewDocumentCommand{\iprops}{}{\Phi_i}
\NewDocumentCommand{\allleadsto}{}{⇝}%Or \dashrightarrow?
\NewDocumentCommand{\ileadsto}{O{i}}{⇝_{#1}}
\NewDocumentCommand{\nileadsto}{O{i}}{\not⇝_{#1}}
\NewDocumentCommand{\ileadstoe}{O{i}}{⇝_{#1}^\exists}
\NewDocumentCommand{\nileadstoe}{O{i}}{\not⇝_{#1}^\exists}
\NewDocumentCommand{\ileadstost}{}{\hookrightarrow_i}
\NewDocumentCommand{\nileadstost}{}{\not\hookrightarrow_i}
\NewDocumentCommand{\di}{}{c^φ_{γ, i}}
\NewDocumentCommand{\dip}{}{d^{φ +}_{γ, i}}
\NewDocumentCommand{\ibeats}{}{⊳^\text{\sout{\ensuremath{φ}}}_{γ, i}}%Or: \usepackage[normalem]{ulem} \text{\sout{\ensuremath t}}
\NewDocumentCommand{\nibeats}{}{⋫^\text{\sout{\ensuremath{φ}}}_{γ, i}}
%%%Deliberated Preference
\NewDocumentCommand{\ipeq}{}{\succeq_i}
\NewDocumentCommand{\ipst}{}{\succ_i}


\definecolor{darkgreen}{rgb}{0,0.6,0}
\newcommand{\commentOC}[1]{{\small\color{blue}{\selectlanguage{french}$\big[$OC: #1$\big]$}}}
%\newcommand{\commentOC}[1]{{\selectlanguage{french}{\todo{OC : #1}}}}
%Or: \todo[color=green!40]
\newcommand{\innote}[1]{{\scriptsize{#1}}}

%this probably requires outdated float package, see doc KomaScript for an alternative.
% \newfloat{program}{t}{lop}
% \floatname{program}{PM}

%definition, theorem, lemma, example environments, qed trickery are only needed in article mode (not Beamer)
\nottoggle{LCpres}{
%style is plain by default (italic text)
	\newtheorem{definition}{Definition}
	\newtheorem{theorem}{Theorem}
%no italic: expected.
%http://tex.stackexchange.com/questions/144653/italicizing-of-amsthm-package
	\newtheorem{lemma}{Lemma}
%\crefname{axiom}{axiom}{axioms}%might be needed for workaround bug in cref when defining new theorems?

%\ifdefined\theorem\else
%\newtheorem{theorem}{\iflanguage{english}{Theorem}{Théorème}}
%\fi

\theoremstyle{remark}
	\newtheorem{examplex}{Example}
	\newtheorem{remarkx}{Remark}

%trickery allowing use of \qedhere and the like.
\newenvironment{example}{
	\pushQED{\qed}\renewcommand{\qedsymbol}{$\triangle$}\examplex
}{
	\popQED\endexamplex
}
\newenvironment{remark}{
	\pushQED{\qed}\renewcommand{\qedsymbol}{$\triangle$}\remarkx
}{
	\popQED\endremarkx
}
}{
}
\crefname{examplex}{example}{examples}% I wonder why this is unnecessary in case of singular

%which line breaks are chosen: accept worse lines, therefore reducing risk of overfull lines. Default = 200
\tolerance=2000
%accept overfull hbox up to...
\hfuzz=2cm
%reduces verbosity about the bad line breaks
\hbadness 5000
%reduces verbosity about the underful vboxes
\vbadness=1300
%sloppy sets tolerance to 9999
\apptocmd{\sloppy}{\hbadness 10000\relax}{}{}

\bibliographystyle{abbrvnat}
%or \bibliographystyle{apalike} for presentations?

%doi package uses old-style dx.doi url, see 3.8 DOI system Proxy Server technical details, “Users may resolve DOI names that are structured to use the DOI system Proxy Server (http://doi.org (preferred) or http://dx.doi.org).”, https://www.doi.org/doi_handbook/3_Resolution.html
\makeatletter
\patchcmd{\@doi}{dx.doi.org}{doi.org}{}{}
\makeatother

% WRITING
%\newcommand{\ie}{i.e.\@\xspace}%to try
%\newcommand{\eg}{e.g.\@\xspace}
%\newcommand{\etal}{et al.\@\xspace}
\newcommand{\ie}{i.e.\ }
\newcommand{\eg}{e.g.\ }
\newcommand{\mkkOK}{\checkmark}%\color{green}{\checkmark}
\newcommand{\mkkREQ}{\ding{53}}%requires pifont?%\color{green}{\checkmark}
\newcommand{\mkkNO}{}%\text{\color{red}{\textsf{X}}}

\newlength{\xdescwd}
\makeatletter
\NewEnviron{xdesc}{%
  \setbox0=\vbox{\hbadness=\@M \global\xdescwd=0pt
    \def\item[##1]{%
      \settowidth\@tempdima{\textbf{##1}:}%
      \ifdim\@tempdima>\xdescwd \global\xdescwd=\@tempdima\fi}
  \BODY}
  \begin{description}[leftmargin=\dimexpr\xdescwd+.5em\relax,
    labelindent=0pt,labelsep=.5em,
    labelwidth=\xdescwd,align=left]\BODY\end{description}}
\makeatother

\makeatletter
\newcommand{\boldor}[2]{%
	\ifnum\strcmp{\f@series}{bx}=\z@
		#1%
	\else
		#2%
	\fi
}
\newcommand{\textstyleElProm}[1]{\boldor{\MakeUppercase{#1}}{\textsc{#1}}}
\makeatother
\newcommand{\electre}{\textstyleElProm{Électre}\xspace}
\newcommand{\electreIv}{\textstyleElProm{Électre Iv}\xspace}
\newcommand{\electreIV}{\textstyleElProm{Électre IV}\xspace}
\newcommand{\electreIII}{\textstyleElProm{Électre III}\xspace}
\newcommand{\electreTRI}{\textstyleElProm{Électre Tri}\xspace}
% \newcommand{\utadis}{\texorpdfstring{\textstyleElProm{utadis}\xspace}{UTADIS}}
% \newcommand{\utadisI}{\texorpdfstring{\textstyleElProm{utadis i}\xspace}{UTADIS I}}

%TODO
% \newcommand{\textstyleElProm}[1]{{\rmfamily\textsc{#1}}} 


%\NewDocumentCommand{\tikzmark}{m}{%
	\tikz[overlay, remember picture, baseline=(#1.base)] \node (#1) {};%
}

\newlength{\GraphsDNodeSep}
\setlength{\GraphsDNodeSep}{7mm}
\tikzset{/GraphsD/dot/.style={
	shape=circle, fill=black, inner sep=0, minimum size=1mm
}}

% MCDA Drawing Sorting
\newlength{\MCDSCatHeight}
\setlength{\MCDSCatHeight}{6mm}
\newlength{\MCDSAltHeight}
\setlength{\MCDSAltHeight}{4mm}
%separation between two vertical alts
\newlength{\MCDSAltSep}
\setlength{\MCDSAltSep}{2mm}
\newlength{\MCDSCatWidth}
\setlength{\MCDSCatWidth}{3cm}
\newlength{\MCDSAltWidth}
\setlength{\MCDSAltWidth}{2.5cm}
\newlength{\MCDSEvalRowHeight}
\setlength{\MCDSEvalRowHeight}{6mm}
\newlength{\MCDSAltsToCatsSep}
\setlength{\MCDSAltsToCatsSep}{1.5cm}
\newcounter{MCDSNbAlts}
\newcounter{MCDSNbCats}
\newlength{\MCDSArrowDownOffset}
\setlength{\MCDSArrowDownOffset}{0mm}
\tikzset{/MCD/S/alt/.style={
	shape=rectangle, draw=black, inner sep=0, minimum height=\MCDSAltHeight, minimum width=\MCDSAltWidth
}}
\tikzset{/MCD/S/pref/.style={
	shape=ellipse, draw=gray, thick
}}
\tikzset{/MCD/S/cat/.style={
	shape=rectangle, draw=black, inner sep=0, minimum height=\MCDSCatHeight, minimum width=\MCDSCatWidth
}}
\tikzset{/MCD/S/evals matrix/.style={
	matrix, row sep=-\pgflinewidth, column sep=-\pgflinewidth, nodes={shape=rectangle, draw=black, inner sep=0mm, text depth=0.5ex, text height=1em, minimum height=\MCDSEvalRowHeight, minimum width=12mm}, nodes in empty cells, matrix of nodes, inner sep=0mm, outer sep=0mm, row 1/.style={nodes={draw=none, minimum height=0em, text height=, inner ysep=1mm}}
}}

%Git
\newlength{\GitDCommitSep}
\setlength{\GitDCommitSep}{13mm}
\tikzset{/GitD/commit/.style={
	shape=rectangle, draw, minimum width=4em, minimum height=0.6cm
}}
\tikzset{/GitD/branch/.style={
	shape=ellipse, draw, red
}}
\tikzset{/GitD/head/.style={
	shape=ellipse, draw, fill=yellow
}}

%Social Choice
\tikzset{/SCD/profile matrix/.style={
	matrix of math nodes, column sep=3mm, row sep=2mm, nodes={inner sep=0.5mm, anchor=base}
}}
\tikzset{/SCD/rank-profile matrix/.style={
	matrix of math nodes, column sep=3mm, row sep=2mm, nodes={anchor=base}, column 1/.style={nodes={inner sep=0.5mm}}, row 1/.style={nodes={inner sep=0.5mm}}
}}
\tikzset{/SCD/rank-vector/.style={
	draw, rectangle, inner sep=0, outer sep=1mm
}}
\tikzset{/SCD/isolated rank-vector/.style={
	draw, matrix of math nodes, column sep=3mm, inner sep=0, matrix anchor=base, nodes={anchor=base, inner sep=.33em}, ampersand replacement=\&
}}

% GUI
\tikzset{/GUID/button/.style={
	rectangle, very thick, rounded corners, draw=black, fill=black!40%, top color=black!70, bottom color=white
}}

% Logger objects
\tikzset{/loggerD/main/.style={
	shape=rectangle, draw=black, inner sep=1ex, minimum height=7mm
}}
\tikzset{/loggerD/helper/.style={
	shape=rectangle, draw=black, dashed, minimum height=7mm
}}
\tikzset{/loggerD/helper line/.style={
	<->, draw, dotted
}}

% Beliefs
\tikzset{/BeliefsD/attacker/.style={
	shape=rectangle, draw, minimum size=8mm
}}
\tikzset{/BeliefsD/supporter/.style={
	shape=circle, draw
}}


%\DeclareAcronym{AMCD}{short=AMCD, long={Aide Multicritère à la Décision}}
\DeclareAcronym{AHP}{short=AHP, long={Analytic Hierarchy Process}}
\DeclareAcronym{AR}{short=AR, long={Argumentative Recommender}}
\DeclareAcronym{DA}{short=DA, long={Decision Analysis}}
\DeclareAcronym{DJ}{short=DJ, long={Deliberated Judgment}}
\DeclareAcronym{DM}{short=DM, long={Decision Maker}}
\DeclareAcronym{DP}{short=DP, long={Deliberated Preference}}
\DeclareAcronym{MAVT}{short=MAVT, long={Multiple Attribute Value Theory}}
\DeclareAcronym{MCDA}{short=MCDA, long={Multicriteria Decision Aid}}
\DeclareAcronym{MIP}{short=MIP, long={Mixed Integer Program}}
\DeclareAcronym{SEU}{short=SEU, long={Subjective Expected Utility}}


\addtokomafont{labelinglabel}{\sffamily\bfseries}
\renewcommand{\phi}{\varphi}
\DeclareMathAlphabet{\mathup}{OT1}{\familydefault}{m}{n}

%I find these settings useful in draft mode. Should be removed for final versions.
	%Which line breaks are chosen: accept worse lines, therefore reducing risk of overfull lines. Default = 200.
		\tolerance=2000
	%Accept overfull hbox up to...
		\hfuzz=2cm
	%Reduces verbosity about the bad line breaks.
		\hbadness 5000
	%Reduces verbosity about the underful vboxes.
		\vbadness=1300

\begin{document}
\title{Empirical normative theories}
\author{Olivier Cailloux}
\affil{Université Paris-Dauphine, PSL Research University, CNRS, LAMSADE, 75016 PARIS, FRANCE\\
	\href{mailto:olivier.cailloux@dauphine.fr}{olivier.cailloux@dauphine.fr}
}
\makeatletter
	\hypersetup{
		pdfsubject={Epistemology},
		pdfkeywords={Decision aiding, Decision making, Argumentation}
	}
\makeatother
\maketitle

\begin{abstract}
	In decision theory and economy in general, preferences are generally considered as “intuitive”, as originating from an “immediate sensation”, as famously written by \citeauthor{von_neumann_theory_2004}. This conception is insufficient to study the positions that individuals adopt towards difficult decision problems after careful thinking. This article introduces, and defines formally, the concept of deliberated preferences to fill this gap: the deliberated preferences of an individual about a given decision problem represent the position of that individual towards that problem that is stable facing counter-arguments. As defined, deliberated preferences are only indirectly observable. To make it possible to model deliberated preferences, the article also defines formally the notion of a theory of deliberated preferences, together with a protocol of falsification of such a theory. The proposed falsification protocol involves only observations similar to those used in classical theories of revealed preferences. It is shown that a theory of deliberated preference captures effectively a known subset of the deliberated preferences if and only if it is not falsifiable.
\end{abstract}

\section{Motivation}
In decision theory and economy in general, preferences are generally considered as “intuitive”, as originating from an “immediate sensation”, following the terms used by \citet{von_neumann_theory_2004} (vNM) \citep{fishburn_retrospective_1989, cailloux_reasons_2017}. Though appropriate when studying the everyday behavior of individuals, this conception is insufficient to study the positions that individuals adopt towards difficult decision problems after careful thinking. In this article I define the concept of deliberated preferences to fill this gap: the deliberated preferences of an individual about a given decision problem represent the position of that individual towards that problem that is stable facing counter-arguments. 

After having defined this concept formally, I define the notion of a theory pretending to capture the deliberated preferences of a set of individuals about some topic, I propose an empirical procedure to attempt to falsify such a theory, and I show that such a theory tells the truth whenever it is not falsifiable by that procedure.

This article does not attempt to propose a concrete theory of deliberated preferences. It is more preliminarly. It proposes formal definitions for 1) the kind of object that are to be studied for capturing “more reflexive” preferences than considered usually, 2) what theories of such preferences should look like, 3) how to falsify such theories and using which information. Therefore, it proposes a research program rather than a direct solution to the problem of capturing reflexive preferences.

Before delving into the formalism, let me present a bit of motivation for considering deliberated preferences as an interesting object to study.
Being able to capture deliberated preferences can help provide a better understanding of human preferences in the large sense and may improve policy making (though this article does not explore issues of group decision). Besides this very broad goal, here are three example of contexts in which this notion is useful, which may give it a more concrete interest and serve to introduce this concept informally by indicating the kinds of things that it should be able to help understand.

First, decision theory has produced several models of how an individual ought to decide, such as the seminal one of vNM. (The following remarks apply equally well to other normative models such as the one of \citet{savage_foundations_1972}, \citet{keeney_decisions_1993} or more recent ones.) Those models can be used to help individuals take “good” decisions, as opposed to predict their behavior (for which other models, taking into account cognitive limitations, are more appropriate \citep{wakker_prospect_2010}). But it is unclear how to validate recommendations obtained on the basis of such models. Some researchers have opposed the norms on which some of those models are based \citep{ellsberg_risk_1961, allais_so-called_1979}. As \citet{raiffa_back_1985} points out, while it is the case that some sophisticated decision makers may consciously refuse to abide by the vNM axioms (or others), it is also true that some decision makers simply fail to comply with them because of insufficient reflexion. However, no general proposal, as far as I know, has been described yet to test whether, for a given individual and context, those axioms would, after careful thinking, reveal acceptable. Psychologists have tried to test the convincing power of those axioms \citep{slovic_who_1974, maccrimmon_utility_1979}. Their conclusions inevitably depend on how axioms have been presented to the individuals and on whether counter-arguments have been presented, as \citet{slovic_who_1974} themselves point out. But lacking a systematic approach to determine stability facing counter-arguments, those conclusions only shed very partial light on the question of acceptance of specific norms by the general public. The notion of deliberated preferences can help confront different possible norms for normative decision theory by seeing them as constituents of theories of deliberated preferences and using the procedure described in this article to confront theories.

Second, the concept of nudge \citep{thaler_nudge:_2009} has been tremendously popular in the recent past, but has also been vastly criticized: the risk of manipulation is particularly high whenever it is unsure that the individual would indeed have adopted the decision that the nudge suggests after careful thinking.

Third, analytical philosophy has presented very elaborated and diverging views about the concept of justice. \citet{rawls_theory_1999}, in particular, suggests that the appropriate conception of justice is one that individuals would arrive at in a state of reflexive equilibrium (a concept borrowed from \citet{goodman_fact_1983}). \citeauthor{rawls_theory_1999} and \citeauthor{goodman_fact_1983} insist that individuals must not only accept principles, when thinking about difficult notions such as justice, but also think about consequences of those principles: sometimes, principles that appear intuitively acceptable yield consequences that do not appear intuitively acceptable, in which case something must give. The state of reflexive equilibrium is a state where accepted principles yield acceptable consequences only. The concept of deliberated preferences may be considered as defining that state of reflexive equilibrium, or to be a path towards such a goal (an important difference is that it does not explicitly model the difference between principles and consequences). One way to go beyond philosophers disagreement about the right conception of justice, then, is to ask non-philosophers about what they think is right and so use these laymen as arbiters of the competition between views of justice. Similar empirical studies have started to be conducted recently \citep{gaertner_empirical_2012}. But those studies are conducted by asking individuals only which distribution of goods they consider intuitively right in various circumstances. Thus, they do not try to capture their well-considered judgments for justice, where different reasonings in favor of different conceptions of justice would be presented to the individuals together with the distributions. Hence, this does not test appropriately whether laymen would accept, after thinking carefully, the sophisticated conceptions of justice presented by a given philosopher or economist. A more appropriate way of testing the convincing power of elaborated claims about justice may be considered to require, here again, to present his conception of justice to the individual and submit it to counter-arguments (possibly coming from other sophisticated thinkers).

The first and third contexts of usage come together when it is desired to test whether an individual accepts some axiom as representing an acceptable way of reasoning for him. One simple way is to explain axioms individually and ask the individual whether they appear acceptable. But this fails to make sure the individual understands and accepts the consequences of the union of the axioms accepted by such a procedure \citep{meinard_justification_2018}. (Note that probably nobody is able to mentally deduce all consequences of a union of non trivial axioms.) As an example, in a context of trying to determine the voting rule that best represents someone’s notion of justice, the tested individual could accept the axioms of universal domain, independence of irrelevant alternatives, and pareto dominance, without realizing that he has just accepted dictatorship \citep{arrow_social_1963}.

\section{Normative theories}
\subsection{Definition}
A normative theory is a tuple $(\allprops, ¬, \allargs, I, (\ileadstoi)_{i \in I})$, 
with ${\ileadstoi} \subseteq \allhist × \Phi$, 
where $\allhist = \bigcup_{k \in \N} \allargs^{\intvl{1, k}}$ denotes all finite sequences of elements in $\allargs$ (including the empty sequence) and $\Phi = \allprops × \set{\text{possible}, \text{sure}}$. (Throughout the article, $\N$ includes zero, and note that $\intvl{1, 0} = \emptyset$.)  

The symbol $¬$ is a negation operator on $\allprops$, such that $\forall \prop \in \allprops: ¬¬\prop = \prop$. This article also uses this symbol as a negation operator on $\Phi$, defined as follows, $\forall \prop \in \allprops$: $¬(\prop, \text{possible}) = (¬\prop, \text{sure})$ and $¬(\prop, \text{sure}) = (¬\prop, \text{possible})$. Observe that $¬¬\phi = \phi$. (The symbol $¬$ is also used in this article to negate classical mathematical propositions; this should create no confusion.)
As usual with binary relations, $\hist \nileadsto[i] \phi$ means $(\hist, \phi) \notin {\ileadsto[i]}$.

Given $i \in I, \phi \in \Phi, \hist \in \allhist$, say that $\hist$ leads strongly to $\phi$ iff $\hist \ileadstoi \phi \land \hist \nileadstoi ¬\phi$.
Define $\hist \ileadstosti \phi$ iff $[\hist \text{ leads strongly to } \phi] \land [\phi = \propsure ⇒ \hist \text{ leads strongly to } \propposs]$.
Given $\hist \in \allhist$, let $\hist_\mathit{end}$ denote the set containing the last two elements of the sequence $\hist$ if the sequence has at least two elements; the singleton set containing the last element of the sequence if it has exactly one element; and the empty set if $\hist = \emptyset$.
Given $i \in I, \phi \in \Phi, \ar \in \allargs$, define $\ar \ileadstosti \phi$ iff $\forall \hist \in \allhist \suchthat \ar \in \hist_\mathit{end}: \hist \ileadstosti \phi$.

To lighten notations, $\ileadstoi$ and $\ileadstosti$ will generally be denoted by $\ileadsto$ and $\ileadstosts$, these symbols being understood as the result of a function of $i$. 
When the symbol $\ileadsto$ or $\ileadstosts$ is used in a formula without $i$ being bound, the binding $\forall i \in I$ is implied.

\subsection{Classify propositions}
\begin{axiom}[Normative adequacy]
	\label{ax:norm}
	$\forall i \in I, \phi \in \Phi: 
		[(\exists \ar \in \allargs \suchthat \ar \ileadstosti \phi) \land ¬(\exists \ar \in \allargs \suchthat \ar \ileadstosti ¬\phi)] ⇒ \phi \in \iPhi \land ¬\phi \notin \iPhi.$
\end{axiom}
This axiom can be simplified, thanks to the following proposition.
\begin{proposition}[Protocol coherence]
	\label{prop:protcoh}
	$\forall i \in I, \phi \in \Phi$:
	\begin{equation}
		[\exists \ar \in \allargs \suchthat \ar \ileadstosti \phi] ⇒ [\nexists \ar \in \allargs \suchthat \ar \ileadstosti ¬\phi].
	\end{equation}
\end{proposition}
\begin{proof}
	Protocol coherence is satisfied by definition of $\ileadstosti$. Indeed, if for some $\ar \in \allargs$, $\ar \ileadstosti \phi$, then $\forall \ar_1 \in \allargs: (\ar_1, \ar) \nileadstoi ¬\phi$; thus, for any $\ar' \in \allargs$, $\ar' \ileadstosti ¬\phi$ is false, because it requires that $\forall \ar_1 \in \allargs: (\ar', \ar_1) \ileadstoi ¬\phi$, and thus in particular, that $(\ar', \ar) \ileadstoi ¬\phi$.
\end{proof}

Therefore, \cref{ax:norm} is actually equivalent to the following property.
\begin{property}[Simple normative adequacy]
	\label{ax:snorm}
	$\forall i \in I, \phi \in \Phi: 
		[\exists \ar \in \allargs \suchthat \ar \ileadstosti \phi] ⇒ \phi \in \iPhi \land ¬\phi \notin \iPhi.$
\end{property}

To further analyze this and related matters, given a normative theory and $i \in I$, say that a proposition $\phi \in \Phi$ is \emph{$i$-decidable} iff $\exists \ar \in \allargs \suchthat \ar \ileadstosti \phi \lor \ar \ileadstosti ¬\phi$.
The $i$-decidable propositions are those on which the normative theory permits to take a position. It will generally be unknown in practical applications, and may be empty.
The next property ensures an unambiguous interpretation of the content of $\iPhi$ for all $i$-decidable propositions. In this definition, $\oplus$ denotes the exclusive disjunction; thus $\phi \in \iPhi \oplus ¬\phi \in \iPhi$ is equivalent to: $\phi \in \iPhi ⇔ ¬\phi \notin \iPhi$. It can be seen to hold by recalling that \cref{ax:snorm} is equivalent to \cref{ax:norm}.

\begin{property}[Restricted interpretability]
	\label{def:restrinterpr}
	$\forall i \in I, \phi \in \Phi: [\exists \ar \in \allargs \suchthat \ar \ileadstosti \phi \lor \ar \ileadstosti ¬\phi] ⇒ [\phi \in \iPhi \oplus ¬\phi \in \iPhi]$.
\end{property}

\subsection{Non-completeness}
\begin{definition}[Theory completeness]
	$\forall \phi \in \Phi: [\exists \ar \in \allargs \suchthat \ar \ileadstoi \phi] \lor [\exists \ar \in \allargs \suchthat \ar \ileadstoi ¬\phi]$.
\end{definition}
\begin{definition}[Unrestricted interpretability]
	$\forall \phi \in \Phi: \phi \in \iPhi \oplus ¬\phi \in \iPhi$.
\end{definition}
\begin{definition}[Deliberation completeness]
	$\forall \phi \in \Phi: \phi \in \iPhi \lor ¬\phi \in \iPhi$.
\end{definition}
The following proposition directly follows from the definitions, together with \cref{prop:protcoh}.
\begin{proposition}
	1) Under Normative adequacy, Theory completeness implies Deliberation completeness. 2) Unrestricted interpretability implies Deliberation completeness.
\end{proposition}
Theories as defined here do not mandate Theory completeness. This is good for two reasons. First, even the weaker Deliberation completeness can be considered too strong. Second, it is dubious that, for non trivial decision problems, theories can be found for which both Normative adequacy is normatively compelling and Theory completeness holds.

\section{Empirical theories}
\subsection{Falsifiability}
\commentOC{I want: An empirical theory for a normative theory is something like a normative and falsifiable claim. It is also configurable: it must be possible to guarantee that the falsification be easy (if the claim is false) depending on the kind of claim.}

Given $\hist \in \allhist$ and $i \in I$, the claim $\hist \ileadstoe[i] \phi$ may not be verifiable, and is too specific to be falsifiable. However, given $\hist \in \allhist$, a broader claim is falsifiable: that for a given set of individuals $Q \subseteq I, \forall i \in Q, \hist \ileadstoe[i] \phi$ (provided $Q$ is big enough). Indeed, that claim it is equivalent to $\nexists i \in Q \suchthat \hist \nileadstoe[i] \phi$; exhibiting an $i \in Q$ such that $\hist \nileadstoe[i] \phi$ thus falsifies the broader claim.

Given any $(\alltopic, ¬, \allargs, I)$, an \emph{observation protocol} is a binary tree whose non-leaf nodes are among $I × \allhist × \Phi$ and whose two edges out of any non leaf node are labeled by “yes” and “no”; the leaf nodes being simply $\emptyset$. The observation protocol is said to assume memory iff for each non-leaf node $n = (i, \hist, \phi)$ about $i$, any node $n' = (i, {\hist}', \phi')$ about $i$ that is further down the tree continues the history, that is, the sequence ${\hist}'$ starts with $\hist$. Given an observation protocol and observables $(\ileadstoe[i])_{i \in I}$, the \emph{observed path} refers to the path in the tree defined recursively as consisting of the root node and, for each non-leaf node $n = (i, \hist, \phi)$ in the path, the child node along the branch “yes” if $\hist \ileadstoe[i] \phi$ and the child node along the branch “no” otherwise. The \emph{observations} refer to the set of nodes along the observed path.

Given any $(\alltopic, ¬, \allargs, I)$, a \emph{normative claim} about $(\alltopic, ¬, \allargs, I)$ (or about $\allargs$, when the other defined objects are left implicit) is any first-order logic proposition involving the atoms $[(\hist, \phi) \in {\ileadstoe[i]}], \hist \in \allhist, \phi \in \Phi$ which, together with \cref{ax:nocontr,ax:norm}, permits to deduce that $\phi \in \iPhi$ for some $i \in I$ and $\phi \in \Phi$. The claim is said to be \emph{about} the given set $\allargs$. An observation protocol, together with observables $(\ileadstoe[i])_{i \in I}$, \emph{falsifies} a normative claim iff the falsity of the claim is deducible from \cref{ax:nocontr,ax:norm} and the observations. A normative claim is \emph{falsifiable} iff there exists a finite observation protocol such that for any observables $(\ileadstoe[i])_{i \in I} \in (\allhist × \Phi)^I$, if the claim is false given these observables, then it is possible to deduce it from \cref{ax:nocontr,ax:norm} and the observations given the observables.

\commentOC{Falsifiable may not be the right word. Perhaps “provable“, meaning “tautological or Popper-falsifiable”. “Falsifiable” suggests that if a claim is not falsifiable, then a weaker claim is not falsifiable either, which does not hold. My falsifiability criterion demands that if a claim is false, we can see it; whether Popper’s demands that the claim be possibly false. Perhaps I should demand that we can pick a set of consequences of the claim and show it wrong it at least one possible state of the universe.}

A normative claim is \emph{trivial} iff it is about a singleton $\allargs = \set{\ar}$, and has the form: $\ar \ileadstosts \phi$, for some $i \in I$. A normative claim is minimal iff it has the form: $\exists \ar \in \allargs \suchthat \ar \ileadstosts \phi$, for some $i \in I$ and $\phi \in \Phi$. Note that given a minimal claim, $\card{\allargs} = 1$ is equivalent to the claim being trivial.

\begin{proposition}[Trivial claims are not falsifiable]
	Given any $(\alltopic, ¬, \allargs ≠ \emptyset, I)$, no minimal normative claim is falsifiable.
\end{proposition}
\begin{proof}
	Consider any claim about some $i$ and some $\phi$ and any finite observation protocol. We have to show that for some observables, the claim is false, but it is impossible to observe it (meaning, to deduce it from \cref{ax:nocontr,ax:norm} and observations).
	
	Given $k \in \N$, define $\ileadstosts[k]$ as $\ar^k \ileadstosts \phi$ and $\ar^{k+1} \ileadstosts ¬\phi$ (where $\ar^k$ designates a finite sequence repeating $k$ times $\ar$, for $\ar \in \allargs$). I claim that for some $k$, the observables that include this relation makes the claim false, but not observably so. This is because the observation protocol is finite, thus, only permits to observe some finite number of instances of the relation $\ileadsto$. Suffices to define $k$ as that number.
\end{proof}

There is another reason for non falsifiability, which does not involve the problem of finiteness.
\begin{proposition}[Minimal non-trivial claims are not falsifiable]
	Given any $(\alltopic, ¬, \allargs, I), \card{\allargs} ≥ 2$, no minimal normative claim is falsifiable.
\end{proposition}
\begin{proof}
	Define $\ar_1$ as the first element of the sequence $\hist$ of the root node of the observation protocol (if the observation protocol is empty, suffices to define any observables that make the claim false). Define the observables so that $\forall \hist \in \allhist: \ar_2, \hist \ileadsto ¬\phi$. This proves that no argument is decisive, hence $\phi \notin \Box\Phi$, and the claim is false. But the observables may also be defined so as to satisfy the protocol (defined as: when going through the protocol, the observations can’t fail the claim).
\end{proof}

\subsection{Definition}
A general empirical theory for a normative theory $(\allprops, ¬, \allargs, I, (\ileadstoe[i])_{i \in I})$ is a tuple $({\gleadsto}, f)$, where ${\gleadsto} \subseteq \allargs × \Phi$ and $f \subseteq (\allargs × \allargs) × \allargs$ with $f(\allargs × \allargs)$ being finite.

Define $g(\ar_0) \subseteq \allargs × \allargs$ as $g(\ar_0) = \set{(\ar_2, \ar_1) \suchthat \ar_2 \in f(\ar_1, \ar_0)} = \bigcup_{\ar_1 \in \allargs} (f(\ar_1, \ar_0) × \set{\ar_1})$. The set $g(\ar_0)(\allargs)$ contains the planned attacks on $\ar_0$; the set $g(\ar_0)(\ar_2)$ contains the planned attacks on $\ar_0$ for which $\ar_2$ is a planned response; and the set $g(\ar_0)^{-1}(\ar_1) = f(\ar_1, \ar_0)$ contains the planned protections of $\ar_0$ against $\ar_1$. 

Given $\ar_0 \in \allargs$ and $\args_2 \subseteq \allargs$, define the set $\args_1 \subseteq \allargs$ of arguments that $\args_2$ protects $\ar_0$ from as $\args_1 = [f^{-1}(\args_2)]^{-1}(\ar_0) = \set{\ar_1 \in \allargs \suchthat f(\ar_1, \ar_0) \cap \args_2 ≠ \emptyset} = g(\ar_0)(\args_2) = dom(f^{-1}(\args_2) \cap \allargs × \set{\ar_0})$.
 
\subsection{A general condition sufficient for validity and (hopefully) falsifiability}
It should be possible to define this condition as a claim that does not involve $f$, so that it can be done before defining an empirical theory. Then, show that a normative and falsifiable claim must have the form of a convincingness claim.

\begin{remark}
	In fact, the model is tasked with finding a stable point after interrogating $i$. The process of trying counter-arguments $\ar_1, …$ is $q(i)$. When done, the model claims that $\ar_0 + \ar_2 + …$ is decisive. The model also claims that another process of interrogation does not lead to different answers. This is falsifiable.
	When claiming that $(\ar_1, \ar_0) \ileadstost[i] \phi$, the model really claims that $\nexists p, i' \in q^{-1}(q(i)) \suchthat p, \ar_1, \ar_0 \ileadsto[i'] ¬\phi \lor p, \ar_0, \ar_1 \ileadsto[i'] ¬\phi$, where $i' \in q^{-1}(q(i))$ iff $q(i) = q(i')$.
	
	To say that $\phi \in \iPhi$, need to not depend on choice of $q$, in the following sense: if with $q$, $\phi$ seems stable, but with $q'$, $¬\phi$ seems stable, then $\phi$ is not proved as being in $\iPhi$. But if with $q$, $\phi$ seems stable, but with $q'$, nothing can be shown to be stable, then $\phi$ is proved (temporarily) as being in $\iPhi$. More precisely, after $q'$, the model must be able to stabilize $i$ with its $q$.
\end{remark}

$\ar_0$ $i$-defends $\phi$ when ignoring $S_1$ iff $\forall \hist \in \allhist: [\args_1 \cap \hist = \emptyset] ⇒ (\hist, \ar_0) \ileadstosts \phi$.
$\ar_0$ $i$-defends $\phi$ under protection of $S_2$ iff $\ar_0$ $i$-defends $\phi$ when ignoring $g(\ar_0)(\args_2)$.

\begin{definition}[Convincingness]
	$\forall i \in I, \phi \in {\gleadsto}(\allargs), \exists k \in 2\N$ and a finite sequence $(\args_j)_{j \in 2\N, j ≤ k} \suchthat \emptyset ≠ \args_0 \subseteq {\gleadstoinv}(\phi)$ and $\forall j \in 2\N, j ≤ k, \forall \ar_j \in S_j: \ar_j$ $i$-defends $\phi$ under protection of $S_{j + 2}$ (defining $S_{k + 2} = \emptyset$) and $S_j \subseteq \cup_{\ar_{j - 2} \in \args_{j - 2}, \ar_{j - 1} \in \allargs} f(\ar_{j - 1}, \ar_{j - 2})$.
\end{definition}
Note that requiring that $\args_0$ be a singleton does not modify the condition, in the sense that any theory satisfying the condition also satisfies its strenghtening. Similarly, requiring that $S_k ≠ \emptyset$ does not modify the condition (because if some $k$ satisfy the condition with $S_k = \emptyset$, then picking $k' = k-2$ with the same sets $S_j$ also satisfy the condition, and because $S_0 ≠ \emptyset$, this recursion must end with a suitable $k$).

\begin{remark}
	The claim must have the form: some structure is adequate for all individuals. Here, the structure is given by $f$. The claim should claim that looking at the attacks planned by $f$ is enough; and looking at the defenses planned by $f$ is enough.

	This may be adequate as if $\ar_0$ seems suitable but some unplanned attacker $\ar_1$ attacks it, then use $\ar_1 \in \hist$ and observe that $¬(\hist, \ar_0) \ileadstoe \phi$ to deny the claim.
	
	Perhaps this could be phrased as follows. Say that $\ar_2$ is replaceable by $\args_2$ iff what $\ar_2$ protects is either protected by $\args_2$ or replaceable (by what?). Say that a set $\args$ is sufficient iff its complement is replaceable by $\args$. Then $g$ claims (among others) that $g(\ar_0)^{-1}(\ar_1)$ is sufficient.
	
	The claim can (hopefully) be given the form: $\forall i \suchthat …, \ar$ decisive. Given $i$, either $\ar_0$ is decisive, or $\ar_1$ attacks it but then $\ar_2$ is decisive, …
\end{remark}

\begin{theorem}
	If an empirical theory is convincing, and its normative theory satisfies \cref{ax:nocontr,ax:norm}, then it is valid.
\end{theorem}
\begin{proof}
	Consider $i \in I$ and $\phi \in {\gleadsto}(\allargs)$. 
	By hypothesis, $\exists k$ such that some $\ar_k \in \args_k$ defends $\phi$ under protection of $\emptyset$. Therefore, $\ar_k \ileadstosts \phi$. Using \cref{ax:norm}, this yields that $\phi \in \iPhi$, and validity follows from \cref{ax:nocontr}.
\end{proof}

\begin{theorem}
	The convincingness claim of any empirical theory is falsifiable: given an ET, there exists a finite obs prot with memory such that for any observables, if the claim is false given these observables, then it is possible to deduce it from the axioms and observations.
\end{theorem}
\begin{proof}
	
\end{proof}

\section{Additive empirical theories}
TODO: 3) define single-answer models. 4) phrase conditions below for history and single-answer models. 5) show that these conditions can be verified or falsified. 6) show that if the claim of the model does not include these conditions, then the model is not falsifiable, as suggested here below for some form of claims.

\subsection{Definition}
An empirical theory for a normative theory is a tuple $(q, (\gleadsto[q(i)])_{q(i) \in q(I)}, \gbeats, +)$,
with ${\gleadsto[q(i)]} \subseteq \allargs × \Phi$, ${\gbeats} \subseteq \allargs × \allargs$,
${\gbeatsinv}(\allargs)$ finite,
and $+$ a commutative associative internal binary operation on $\allargs$. Given a finite $\args \subseteq \allargs$, the notation $\sum \args = \sum_{\ar \in \args} \ar$ thus designates a well-defined argument. 
\commentOC{Check about finiteness.}
\commentOC{${\gbeatsinv}$ need not be finite, only those even ones: the transitive closure of ${\gleadstoinv}(\Phi)$ through ${\gbeatsinv}^2$, that is, $({\gbeatsinv})^2[{\gleadstoinv}(\Phi)] = \cup_{k \in 2\N}({\gbeatsinv}^{k}({\gleadstoinv}(\Phi)))$.}
\commentOC{I should only define $\ar \in \allargs$ plus $\ard$.}

This defines a function: $\forall \ar_1 \gbeats \ar_0: f(\ar_1, \ar_0) = \ar_0 + {\gbeatsinv}(\ar_1) = \set{\ar_0 + \ard_2 \suchthat \ard_2 \in {\gbeatsinv}(\ar_1)}$, and $f(\ar_1, \ar_0) = \emptyset$ otherwise. Thus, $f = \set{((\ar_1, \ar_0), \ar_2) \suchthat \exists \ard_2 \in \allargs \suchthat \ard_2 \gbeats \ar_1 \gbeats \ar_0 \land \ar_2 = \ar_0 + \ard_2}$.
Also, from the previous definition, $g(\ar_0)(\ar_0 + \argsd_2) = dom(f^{-1}(\ar_0 + \argsd_2) \cap (\allargs × \set{\ar_0})) = {\gbeats}(\argsd_2) \cap {\gbeatsinv}(\ar_0)$ because $(\ar_1, \ar_0) \in f^{-1}(\ar_0 + \argsd_2) ⇔ \exists \ar_2 \in \ar_0 + \argsd_2 \suchthat \ar_2 \in f(\ar_1, \ar_0) ⇔ \ar_1 \gbeats \ar_0 \land \exists \ard \in \argsd_2 \suchthat \ard \gbeats \ar_1$.
\commentOC{Could be false. Define }

Note that $f^{-1}(\ar_0 + \ar_2) ≠ {\gbeats}(\ar_2) × \set{\ar_0}$: if ${\gbeats} = \set{(\ard_2, \ar_1), (\ar_1, \ar_0), (\ard_6, \ar_5), (\ar_5, \ar_4)}$, and $\ar_0 + \ard_2 = \ar_4 + \ard_6$, then $f^{-1}(\ar_0 + \ard_2) = f^{-1}(\ar_4 + \ard_6) = \set{(\ar_1, \ar_0), (\ar_5, \ar_4)}$ and ${\gbeats}(\ard_2) × \set{\ar_0} = \set{(\ar_1, \ar_0)}$.
Also, note that $g(\ar_0)(\ar_0 + \argsd_2) ≠ {\gbeats}(\argsd_2) \cap {\gbeatsinv}(\ar_0)$. Consider ${\gbeats} = \set{(\ard_2, \ar_1), (\ar_1, \ar_0)}, \ar_2 = \ar_0 + \ard = \ar_0 + \ard_2$. Then, $(\ar_1, \ar_0) \in f^{-1}(\ar_0 + \ard)$, hence, $\ar_1 \in g(\ar_0)(\ar_0 + \ard)$, but ${\gbeats}(\ard) = \emptyset$.

\subsection{Falsifiability}
\begin{proposition}
	An empirical theory has only finite sets $\gargs[i, \phi]$, and an upper bound on their size exists that is independent of ${\ileadsto}$.
\end{proposition}
\begin{proof}
	Given $i, \phi$. Observe that $\forall \ard: d^{i, \phi}(\ard) \subseteq {\gbeatsmt}(\ard)$. Thus, $d^{i, \phi}$ is finite. Also, the transitive closure through $\sigma^i$ changes a finite number of time, otherwise it is possible to build an infinite chain in ${\gbeatsmt}$.
\end{proof}

Given $i \in I, \phi \in \Phi, \ar_1, \ar_0 \in \allargs$, define $\ar_1 \ibeatse[!\phi] \ar_0$ iff $\exists \hist \in \allhist \suchthat \hist \text{ ends with } \set{\ar_1, \ar_0} \land \hist \nileadstosts \phi$.

\begin{proposition}
	$\forall i \in I, \phi \in \Phi, \ar_0 \in \allargs$:
\begin{equation}
	\label{eq:st-in-terms-of-e}
	\ar_0 \nileadstosts \phi ⇔ [\exists \ar_1 \in \allargs \suchthat \ar_1 \ibeatse[!\phi] \ar_0].
\end{equation}
\end{proposition}
\begin{proof}
	By definition of ${\ileadstosts}$, $\ar_0 \ileadstosts \phi ⇔ \forall \hist \in \allhist \suchthat [\exists \ar_1 \in \allargs \suchthat \hist \text{ ends with } \set{\ar_1, \ar_0}]: \hist \ileadstosts \phi.$ Equivalently, $\ar_0 \nileadstosts \phi ⇔ \exists \hist \in \allhist \suchthat [\exists \ar_1 \suchthat \hist \text{ ends with } \set{\ar_1, \ar_0}] \land \hist \nileadstosts \phi$. This is also equivalent to $\ar_0 \nileadstosts \phi ⇔ \exists \hist \in \allhist, \ar_1 \in \allargs \suchthat \hist \text{ ends with } \set{\ar_1, \ar_0} \land \hist \nileadstosts \phi$. By definition of ${\ibeatse}$, this is the proposition to be proven.
\end{proof}

$\ar_0$ $i$-defends $\phi$ when ignoring $\args_1$ iff $\forall \hist \in \allhist: [\args_1 \cap \hist = \emptyset] ⇒ (\hist, \ar_0) \ileadstosts \phi$.
$\ar_0$ $i$-defends $\phi$ under protection of $\args_2$ iff $\ar_0$ $i$-defends $\phi$ when ignoring $g(\ar_0)(\args_2)$.
$\ar_0$ $i$-defends $\phi$ under d-protection of $\argsd_2$ iff $\ar_0$ $i$-defends $\phi$ when ignoring ${\gbeats}(\argsd_2) \cap {\gbeatsinv}(\ar_0)$.

Given $i \in I, \phi \in {\gleadsto}(\allargs)$, define the $\phi$-protectors of $\ard$ as $d^{i, \phi}(\ard) = [{\gbeatsinv} \circ ({\gbeatsinv} \cap {\ibeatseinv[!\phi]})](\ard)$.
Define the successors of $(\phi, \ar, \ard)$ as $\sigma^i(\phi, \ar, \ard) = \set{\phi} × \set{\ar + \ard} × d^{i, \phi}(\ard)$. Given $i \in I, \phi \in {\gleadsto}(\allargs)$, define $\gargs[i, \phi]$ as the transitive closure of $(\phi, \emptyset, \ar)$ under $\sigma^i$, with $\ar \gleadsto \phi$.
\commentOC{Assuming only one $\ar$ leads to that $\phi$.}
Define $\gargs[i, \phi] = \bigcup_{k \in 2\N} \gargs[i, \phi, k]$, and $\gargs = \cup_{i \in I, \phi \in {\gleadsto}(\allargs)} \gargs[i, \phi]$. 

\begin{definition}[Conv. single-try]
	$\forall i \in I, \phi \in {\gleadsto}(\allargs), (\phi, \ar, \ard) \in \gargs[i, \phi], \ar_0 = \ar + \ard: \ar_0$ $i$-defends $\phi$ under d-protection of $d^{i, \phi}(\ard)$.
\end{definition}

1. Checkpath: this is completely general.
2. Only using model that counter-argues: requires a sort of path independence. (arguments can be condensed: s5, s4, s1 => s1big)

I also write that $\hist \ibeatse[!\phi] \ar_0$ iff $\exists \ar_1 \in \allargs \suchthat (\hist \text{ ends with } \set{\ar_1, \ar_0}) \land (\ar_1 \ibeatse[!\phi] \ar_0)$.

A falsification instance of an empirical theory is a tuple $(i, \phi, \ar, \ard, \hist, \ar_1)$ such that $(\phi, \ar, \ard) \in \gargs[i, \phi]$ and, defining $\ar_0 = \ar + \ard$: $[{\gbeatsinv}(\ard) \cap \hist = \emptyset] \land [\hist \text{ ends with } \set{\ar_1, \ar_0}] \land \ar_1 \ibeatse[!\phi] \ar_0]$ (the theory says that this history should defend $\phi$ under prot of something and that is not in the history).

\begin{theorem}
	An empirical theory is conv.\ single-try iff it has no falsification instances.
\end{theorem}
\begin{proof}
	Given $i, \phi, (\phi, \ar, \ard) \in \gargs[i, \phi]$. Define $\args_1 = {\gbeats}(d^{i, \phi}(\ard)) \cap {\gbeatsinv}(\ar)$. Now $\ar$ $i$-defends $\phi$ ignoring $\args_1$ is equivalent to: $\forall \hist: \hist \cap \args_1 = \emptyset ⇒ (\hist, \ar) \ileadstosts \phi$. Negation: $\exists \hist \suchthat \hist \cap \args_1 = \emptyset \land (\hist, \ar) \nileadstosts \phi$. TODO prove that equivalent to no falsification.
\end{proof}

\subsection{Vrac from here}
An empirical theory is defined only given some normative theory. In the following, any reference to an empirical theory should always be understood as an empirical theory for some (sometimes implicitly defined) normative theory.

As above,  $\gleadsto$ will be used instead of $\gleadsto[q(i)]$ when it creates no ambiguity, with the understanding that this symbol result of applying a function of $q(i)$; and the binding $\forall i \in I$ will generally be omitted.

Given $\phi \in {\gleadsto}(\allargs)$, define $\gargs[i, \phi, 0] \subseteq \allargs$ as $\gargs[i, \phi, 0] = {\gleadstoinv}(\phi)$; and, for $k \in 2\N$, define $\gargs[i, \phi, k + 2] = \gargs[i, \phi, k] + [{\gbeatsinv} \circ ({\gbeatsinv} \cap {\ibeatseinv[!\phi]})](\gargs[i, \phi, k]) = \set{\ar_k + \ar \suchthat \ar_k \in \gargs[i, \phi, k], \ar  \in [{\gbeatsinv} \circ ({\gbeatsinv} \cap {\ibeatseinv[!\phi]})](\gargs[i, \phi, k])}$.
Define $\gargs[i, \phi] = \bigcup_{k \in 2\N} \gargs[i, \phi, k]$, and $\gargs = \cup_{i \in I, \phi \in {\gleadsto}(\allargs)} \gargs[i, \phi]$. 

Given $\ar \in \gargs$, let $c(\ar) = \bigcup \set{\args \subseteq \gargs \suchthat \ar = \sum \args} \subseteq \allargs$ denote the \emph{components} of $\ar$.

Given $\ar_2 \in \gargs, \ar_1 \in \allargs$, $\ar_2 \mbeats \ar_1$ has the semantic that $\gamma$ claims that $\ar_2$ trumps $\ar_1$ if necessary.
Given $\ar_1 \in \allargs, \ar_0 \in \gargs$, $\ar_1 \mbeats \ar_0$ has the semantic that $\gamma$ claims that it knows how to resist an attack from $\ar_1$ coming in response to using $\ar_0$, if it does come.

\subsection{Validity}
\begin{definition}[Validity]
	An empirical theory is \emph{valid} iff $\forall \phi \in {\gleadsto}(\allargs): [\phi \in \iPhi \land ¬\phi \notin \iPhi]$.
\end{definition}

\subsection{Single-try empirical models}
NB these two conditions should be phrased as properties of a claim. A claim has bl iff for some $k$, … 

What we search for here are conditions of the form: $\forall i$.

For now, assume that $\ar_1$ attacks a single argument, so that there is no problem with replying with $\ar_2$ independently of which argument $\ar_1$ attacks.

\begin{definition}[Conv.\ SA]
	$\forall i \in I, \phi \in {\gleadsto}(\allargs), k \in 2\N, \ar \in \gargs[i, \phi, k]: \ar$ $i$-defends $\phi$ ignoring ${\gbeats}(\gargs[i, \phi, k + 2])$.
\end{definition}

\begin{definition}[Length bounded a priori]
	$\exists m \in \N^* \suchthat ({\gbeats})^m(\allargs) = \emptyset$.
\end{definition}

\begin{theorem}
	If an empirical theory is Conv.\ SA and has a priori bounded length, and its normative theory satisfies \cref{ax:nocontr,ax:norm}, then it is valid.
\end{theorem}
\begin{proof}
	Consider $i \in I$ and $\phi \in {\gleadsto}(\allargs)$. 

	Lemma: $\forall \ar_0 \in \gargs[i, \phi], \ar_1 \in \allargs: \ar_1 \ibeatse \ar_0 ⇒ [\exists \ar_2, \ar_1' \in \allargs \suchthat \ar_2 \gbeats \ar_1' \gbeats \ar_0]$. 
	Consider $\ar_0 \in \gargs[i, \phi], \ar_1 \in \allargs \suchthat \ar_1 \ibeatse \ar_0$.
	Define $\args^1 = {\gbeats}(\gargs[i, \phi, k + 2])$. Thus, $\ar_0$ $i$-defends $\phi$ ignoring $\args^1$. Now, $\args^1 ≠ \emptyset$, because otherwise, by definition of $i$-defense, $\ar_0 \ileadstosts \phi$, which is incompatible with $\ar_1 \ibeatse \ar_0$, by \cref{eq:st-in-terms-of-e}. 
	Pick $\ar \in \args^1$ and observe that, by definition of $\args^1$, $\exists \ar_2 \in \gargs[i, \phi, k + 2] \suchthat \ar_2 \gbeats \ar$, and by definition of $\gargs[i, \phi, k + 2]$,  $\exists \ar_1' \suchthat \ar_2 \gbeats \ar_1' \gbeats \ar_0$. \commentOC{NO. I should be protected only from anticipated attacks. Currently, $\ar_1$ need not be anticipated (another $\ar_1'$ may be anticipated and then $\ar_2$ attacking both would suffice). Perhaps I need to define $\gargs[2]$ with only requiring that $\ar_1 \ibeatse \ar_0$, not $\ar_1 \ibeatse \ar_0$ and $\ar_1 \gbeats \ar_0$, because otherwise no requirements on attackers of attackers of $\ar_0$. I need to observe that if $\ar_1 \ibeatse \ar_0$, and $\ar_0$ defends $\phi$, then $\ar_1$ is ignored. Thus, it is anticipated (because I can ignore only those I anticipate).}
	
	But $\ar_0$ i-def $\phi$, if not ignoring $\ar_1$, implies that $\forall \hist \in \allhist: (\hist, \ar_1, \ar_0) \ileadstosts \phi$. Thus, $\ar_1$ is ignored.

	Thus, $\gargs[2] \subseteq ({\gbeatsinv})^2(\ar_0)$.
	Thus, some $\gargs[k] = \emptyset$ (by bounded length).
	Thus, some $\gargs[k] ≠ \emptyset$ that nobody attacks. This is our decisive argument.
	Therefore, $\ar_k \ileadstosts \phi$. Using \cref{ax:norm}, this yields that $\phi \in \iPhi$, and validity follows from \cref{ax:nocontr}.
\end{proof}

\subsection{Conditions without history}
To verify or falsify these conditions, whether $(\ar_1, \ar_0) \ileadstoe \phi$ must be observable, but one may suspect that once $i$ has seen some $\ar$, then no more $(\ar_1, \ar_0) \ileadstoe \phi$ but that this does not imply that this would hold without $\ar$ for $i$. The general falsifiable claim is not clear.

Given $\phi \in {\gleadsto}(\allargs)$, define ${\ibeatse[!\phi]} \subseteq \allargs × \allargs$ such that $\ar' \ibeatse[!\phi] \ar$ iff $(\ar', \ar) \nileadstost \phi$. 

The following condition mandates that if the arguments used by the theory are not decisive, it knows a reason.
\begin{definition}[Answerability]
	$\forall \phi \in {\gleadsto}(\allargs), \ar_0 \in \gargs[\phi], \ar_1 \in \allargs: (\ar_1, \ar_0) \nileadstost \phi ⇒ [\exists \ar_1' \in {\gbeatsinv}(\ar_0)]$.
\end{definition}

\begin{definition}[Bounded length]
	$\gargs$ must be finite.
\end{definition}

\begin{remark}
	Bounded length does not imply finiteness of arguments used as counter-arguments by the model: model is able to counter infinitely many arguments with infinitely many counter-args, but needs to declare that an (unknown) finite subset of those will suffice to convince $i$.
\end{remark}

\begin{definition}[No repetition]
	$\nexists \ar_0 \in \gargs[\phi], \ar_1 \in \allargs, \ar_2 \in c(\ar_0) \suchthat \ar_2 \gbeats \ar_1 \gbeats \ar_0$.
\end{definition}

\begin{definition}[Operational validity]
	$\forall \phi \in {\gleadsto}(\allargs), \ar_0 \in \gargs[\phi], \ar_1 \in \allargs: [\ar_1 \gbeats \ar_0 \land (\ar_1, \ar_0) \nileadstost \phi] ⇒ \exists \ar_2 \in {\gbeatsinv}(\ar_1)$.
\end{definition}

$\ar_0 + \ar_2$ anticipates attacks from those attacking either, minus those attacked by any. (Actually any attacked may still attack the addition, but that’s because of some other not-yet-attacked, anticipated, attack.)

\begin{remark}
	I currently conceive these conditions in the following way. The model must anticipate in its initial arguments the counter-arguments that $i$ may have in mind. Thus, $\ar_1$ trumps $\ar_0$ is observed when $i$ has been given $\ar_1$. When the model answers with $\ar_2$, if $i$ is unconvinced (of $\ar_2 + \ar_0$ trumping $\ar_1$), the procedure stops (the model can’t argue that yes-but-maybe-$i$-has-$\ar_1'$-in-mind; in particular, it wouldn’t be feasible to interrogate $i$ if the arguments are given thanks to labeling in a supermarket). However, if $i$ is given $\ar_1'$, the model must be given a chance to play its counter-argument, and may not be falsified because $\ar_2 + \ar_0$ is not enough: once $i$ is given $\ar_1'$, it is no more claimed by the model that $\ar_2 + \ar_0$ suffices any longer.
	
	Relatedly, Operational validity is (almost) void when $\ar_1$ trumps $\ar_0$ and $\ar_0 \gbeats \ar_1$ (and has already observed to be successful, thus $\ar_0$ trumps $\ar_1$): in that case, suffices to play $\ar_2 = \ar_0$ and define $\ar_0 + \ar_0 = \ar_0$, so $\ar_2 + \ar_0$ trumps $\ar_1$.
	
	In summary: when $\ar_1$ trumps $\ar_0$, and the model has not already answered this argument using $\ar_0$ (thus $¬ \ar_0 \gbeats \ar_1$), Op. val. demands an answer. When $\ar_1$ trumps $\ar_0$, and the model has answered already using $\ar_0$, and we have checked that indeed $\ar_0$ trumps $\ar_1$, Answerability demands that the model can explain this instability by the fact that someone must have given another argument $\ar_1'$ to $i$ (and this is to be checked using means outside the formal model).
\end{remark}

I want an empirical theory to: 1) declare some $P \subseteq \Phi$ and it being valid; 2) prove its claims (otherwise, uncheckable and unfalsifiable and thus not empirical) by producing decisive arguments; 3) permit reuse… Focusing on valid theories is thus justified by the fact that if no valid theory for some $P$, then any theory will not be able to prove its claims (as there is no decisive argument).

\subsection{Theorem}
\begin{theorem}
	If a theory is answerable, has bounded length, does not repeat, and is operationally valid, it is valid.
\end{theorem}
\begin{proof}
	Lemma: given $\phi \in {\gleadsto}(\allargs)$ and $\ar[k] \in \gargs[\phi]$, if $\ar[k] \nileadstost \phi$, then: $\exists \ar[k + 1] \in \gargs[\phi] \suchthat \card{c(\ar[k + 1])} ≥ \card{c(\ar[k])} + 1$.
	To prove the lemma, let us pick any $\phi \in {\gleadsto}(\allargs)$ and $\ar[k] \in \gargs[\phi]$ such that $\ar[k] \nileadstost \phi$. 
	We know that $\exists \ar_1 \in \allargs \suchthat (\ar_1, \ar[k]) \nileadstost \phi$. 
	Therefore, by Answerability, $\exists \ar_1' \in \allargs \suchthat \ar_1' \gbeats \ar[k]$, and, by Operational validity, $\exists \ar_2 \in \gargs[\phi] \suchthat \ar_2 \gbeats \ar'_1$. 
	Define $\ar[k + 1] = \ar[k] + \ar_2$. This is the element we look for.
	First, $\ar[k + 1] \in \gargs[\phi]$ because $\ar[k + 1] \gbeats \ar_1' \ibeatse[!\phi] \ar[k]$.
	Second, by No repetition, $\ar_2 \notin c(\ar[k])$,
	thus, $\card{c(\ar[k + 1])} ≥ \card{c(\ar[k])} + 1$ (observing that, by definition of $c$, for any $\ar, \ar'$: $c(\ar) \subseteq c(\ar + \ar')$).
	
	To prove the theorem, given $\phi \in {\gleadsto}(\allargs)$, we need to exhibit an argument $\ar$ such that $\ar \ileadstost \phi$.
	Define a finite sequence $(\ar[0], \ar[1], \ar[2], …)$, starting from $\ar[0] = {\gleadstoinv}(\phi)$, using the lemma, until the condition $\ar[k] \nileadstost \phi$ no longer holds. This condition must turn false after some finite number of steps, because $\card{c(\ar[k])}$ grows at each step, $c(\ar[k]) \subseteq \gargs$ by definition of $c(.)$, and using Bounded length. The last element of the sequence is the argument we look for.
\end{proof}
\commentOC{If playing $\ar_2$ at some stage did not move $i$ and $\ar_1$ still attacks $\ar_0 + \ar_2$, we can say that the model missed answering that step, thus either we choose to give it a second chance (maybe it chose the wrong $\ar_2$ to counter the $\ar_1'$ it believed was relevant, or it needed both $\ar_2$ and $\ar_4$), or we give up in fear of losing our time. Actually, the protocol is that when applying Answerability, we are supposed to pick the $\ar_1'$ that $i$ just was given by another model, thus, if that’s still the same $\ar_1$ as the step before, the model has lost (it can’t map what it’s told to an argument it knows how to answer).}

\subsection{Multiple tries}
Should generalize the building of possible arguments attempted by the model. A model defines $m(\ar_0, \ar_1) \subseteq \allargs$ giving its possible arguments for answering $\ar_1$. Define $\gargs[\phi, 0] = \mleadsto(\allargs)$. If $\ar_k$ has lost (thus the model can’t answer some counter-argument to it), there must be another candidate argument.

Mandate that $\exists \ar_0 \in \gargs[\phi, 0]$ such that $\exists \ar_1 \in \allargs \suchthat \ar_1 \ibeatse[!\phi] \ar \land f(\ar_0, \ar_1) ≠ \emptyset$. 

\commentOC{TODO 1. Justify the fundamental restriction: mandates that $\exists \ar_0 \in \gargs[\phi] \suchthat \ar_0$ decisive. 2. Rephrase condition as: must not be that at some point, for all $\ar_0$ in the current set of possibilities, $\nexists \ar_1 \ibeatse \ar_0 \suchthat \ar_2 \gbeats \ar_1$ (or rather $m(\ar_0, \ar_1) \subseteq C$ with $C$ the current set of possibilities).}

First point is not restrictive: if there is a decisive argument for $\phi$, there is a valid model which says that $\phi$ (and if there is a valid model for $\phi$, then $\phi \in \iPhi$).

First point bars models that want to fully explore $\ibeatse$ and then choose a decisive argument, for example. This would not be desirable because if model can’t be a priori expressed as a restricted model, then we have no means to check that really $\ar \ileadsto \phi$ as another path can’t be adopted (unless $i$ forgets). Also, what if $i$ gets tired after 10 arguments?

\subsection{Current question}
Should I leave open the uplifting of $+$ for more generality? Is this really sufficiently general to encompass both interpretations of Answerability, and more? Or should I stick to the current, “single-try”, approach and interpretation?

\subsection{Interpreting Answerability}
At a given stage in operationalizing the protocol, define $f(\ar_0)$ as the arguments that are considered as having a chance to attack $\ar_0$ (without backtracking). Thus, this will not include $\ar_1$ if indeed $\ar_2$ is considered by $i$ as a satisfying answer to $\ar_1$. We can interpret Answerability in the following ways. Strong answ: $\exists \ar_1 \in f(\ar_0) \suchthat \ar_1 \gbeats \ar_0$. Weak answ: $f(\ar_0) ≠ \emptyset ⇒ \exists \ar_1' \suchthat \ar_1' \gbeats \ar_0$. Or suppress Answ and replace Op. val. by Strong op. val.: $\forall \phi \in {\gleadsto}(\allargs), \ar_0 \in \gargs[\phi], \ar_1 \in \allargs: (\ar_1, \ar_0) \ileadstoe ¬\phi ⇒ \exists \ar_2 \in {\gbeatsinv}(\ar_1)$.

A model can be allowed to try only once each answer. If $\ar_2 \gbeats \ar_1 \land \ar_1 \in f(\ar_0 + \ar_2)$, the model has lost.

Thus, four variants, given $\ar_2 \gbeats \ar_1$.
\begin{itemize}
	\item Strong answ with single try: $\ar_1 \notin f(\ar_0 + \ar_2) \land \forall \ar_1' \in f(\ar_0 + \ar_2): \ar_1' \gbeats \ar_0 + \ar_2$.
	\item Strong answ with multiple tries: $\forall \ar_1' \in f(\ar_0 + \ar_2): \ar_1' \gbeats \ar_0 + \ar_2$. (Example: $\ar_1 = \text{“for ethical reasons”}$, and $\eta$ knows two c-a to this, from different angles.) 	\item Weak answ with single try: $\ar_1 \notin f(\ar_0 + \ar_2) \land [f(\ar_0 + \ar_2) ≠ \emptyset ⇒ \exists \ar_1' \suchthat \ar_1' \gbeats \ar_0]$.
	\item Weak answ with multiple tries: $f(\ar_0 + \ar_2) ≠ \emptyset ⇒ \exists \ar_1' \suchthat \ar_1' \gbeats \ar_0 + \ar_2$.
\end{itemize}

Goal: leave addition free so as to admit multiple tries, and prove that Strong answ with multiple tries is the right one for a general approach: suitable restriction to the model permits single try (with same conditions); and weak answ actually does not permit more models.

When $f$ is available and is single-valued, question: should we use Strong or weak answ? Right answer: use Strong answ. It is stricly more general. A model that wants weak answ can define its internal attacks suitable (to be proven); and a model that wants to claim more can.

When $f$ is not available, only remains: single try (just give a decisive argument) or multiple tries (try more and more arguments, blindly).

Suppose that a model also defines $f_\eta$, meaning: $f_\eta(\ar_1) = \ar_2$ the argument the model plays against $\ar_1$. Thus $f_\eta$, like $\gbeats$, is a binary relation over $\allargs$, but $f_\eta$ in supplement returns a single argument or none. Also, $\ar_2 = f(\ar_1) ⇒ \ar_2 \gbeats \ar_1$. When a reason $\ar_1$ for rejecting $\ar_0$ is not known, $f_\eta(\ar_0)$ instead gives the argument to be played next. Note that $f_\eta$ is only useful when (in the known c-a variant) for some $\ar_1$ there are multiple candidates $\ar_2$ or when (in the unknown c-a variant) for some $\ar_0$ here are multiple candidates $\ar_2$. In fact, this is more complicated: $f_\eta(\ar_1)$ can give first $\ar_2$, then, if that is not enough, $\ar_4$ (thus it’s rather $f_\eta(\ar_1, \ar_0)$).

Under Strong anws, we ask that $f(\ar_0) \cap {\gbeatsinv}(\ar_0) ≠ \emptyset$, and (inevitably) that $f_\eta(\ar_1) ≠ \emptyset$. Under weak answ, we do not expect anticipation and only demand that $f_\eta(\ar_1) ≠ \emptyset$. But this does not allow supplementary models. TODO prove this.

\subsection{Single-try models}
The above models can define several $\ar_2$ for one $\ar_1$. Even when interpreting Answerability in the strong sense (mandating that exactly the $\ar_1'$ we search for be defined, not just any $\ar_1'$), this gives a lot of flexibility to the model, and correspondingly, makes it very difficult to falsify: we only know that is unconclusive when all possible subset of answers to a $\ar_1$ has been tried

\commentOC{This is incorrect. The current approach already mandates single-try. If $\ar_2 \gbeats \ar_1$, then $\ar_1 \ibeatse \ar_0 + \ar_2$, in the sense that $\ar_1$ would be the only such argument, is forbidden because this would require $\ar_1 \gbeats \ar_0 + \ar_2$ and thus yields a repetition.}

\subsection{Example model}
\begin{itemize}
	\item $\ar_0 = \text{“Veg bec of ethics and health”}$
	\item $\ar_1 = \text{“Ethics is not defined”}$
	\item $\ar_2 = \text{“It is since Aristotle at least”}$
	\item $\ar_1' = \text{“Health isn’t important”}$
	\item $\ar_2' = \text{“Ask my grandmother”}$
	\item $\ar_1 \mbeats \ar_0$, $\ar_1' \mbeats \ar_0$
	\item $\ar_2 \mbeats \ar_1$, $\ar_2' \mbeats \ar_1'$
	\item $\ar_0 + \ar_2 \mbeats \ar_1$
	\item $\ar_1' \mbeats \ar_0 + \ar_2$
	\item $\ar_0 + \ar_2' \mbeats \ar_1'$
	\item $\ar_1 \mbeats \ar_0 + \ar_2'$
	\item $\ar_0 + \ar_2 + \ar_2' \mbeats \ar_1$
	\item $\ar_0 + \ar_2 + \ar_2' \mbeats \ar_1'$
\end{itemize}

\subsection{The rest}
Change conditions. I should be able to prove the theorem using reinst + bounded width instead of bounded length. Reinst should serve to discard: $(\ar_2, \ar_1) \ileadstost \phi$, $(\ar'_2, \ar'_1) \ileadstost \phi$, $(\ar'_1, \ar_2) \ileadstoe ¬\phi$, $(\ar_1, \ar'_2) \ileadstoe ¬\phi$.

Clarification. General theory says that $\ar_1$ is replaceable (suspects that it’s good for some people): $(\ar_1, \ar_0) \ileadstoe ¬\phi ⇒ (\ar'_1, \ar_0) \ileadstoe ¬\phi$. (This will fail if some $i$ thinks $\ar'_1$ is bad arg because of $\ar_0$ but $\ar_1$ is bad because of $\ar_2$.) It may say that $\ar_1$ is bad. General theory prepares potentially good arg $\ar_0$ and claims that if $\ar_0 \ileadstoe \phi$, then $\ar_0 \ileadstost \phi$ (this is Answerability). So that the model only has to prove the weaker existential claim.

Model: for a given $i$. Defends $\prop$. $\exists \ar_1 \in \args_\eta \suchthat \ar_1 \ileadstoe \prop$. Then, $\forall \ar_2 \in \gargs \suchthat (\ar_2, \ar_1) \ileadstoe ¬\prop: \exists \ar_3 \in \args_\eta \suchthat (\ar_3, \ar_2) \ileadstoe \prop.$

TODO: theorist proposes theory $\gamma$. Contains set $\allargs$. The theory must exhibit a limited number of candidates $s_c$ when $s'$ sometimes trumps $s$ and $s'$ sometimes does not trump $s$ (valid for all $i$): that’s falsifiable, and then permits validation of $\eta$. Also, $\gamma$ must exhibit a replacer $s$ of $s_3, s_1$ against $s_2$, or a set of candidate replacers. The replacer (or candidates) may depend on $s_3$ and $s_1$, but not on $i$. This is falsifiable again. And $\gamma$ must fix bounds of width and breadth. And $\gamma$ must indicate how an $s \notin \args_\gamma$ is unnecessary, by exhibiting resistant [not trumped by any decisive argument in $\args_\gamma$] candidates from $\args_\gamma$ such that at least one trump $s$ or replace $s$.

A counter-theory may propose $\args^\text{ext}$ with some arguments not in $\allargs$ (or, more generally, may propose to enlarge the conditions for considering that $s', s \ileadsto t$: if DM has eaten a pancake or has eaten no pancake, must be stable). Then they either can exhibit a plausible argument in $\args^\text{ext}$ not in $\allargs$ (and indeed $\allargs$ is rejected), or not, and then both theories are equivalent, better keep the simpler one.

Thus, $\gamma$ takes positions only on those things true for all $i$.

Then a theorist can propose a mapping $i$ to $\eta_i$. If $\gamma$ is valid, then it is easy to validate $\eta_i$.

More generally, I need a set of possible situations, and an operation defined by the theory $\gamma$, that brings a situation to another situation, that is normatively better for taking a decision (or innocuous?). The operation is: adding an argument. This operation corresponds to forming the pair $(s', s)$: from a situation $s'$, we add the argument $s$ and bring $i$ in the situation $(s', s)$ where she has heard two arguments.

Then the normative criterion is: $t$ in DJ of $i$ when $\exists s \suchthat \forall s': (s', s) \ileadstost t$ (the theory provides a part of this guarantee under falsification, claiming for all $i$, the model makes the rest under validation for this $i$).

\commentOC{Say that a theory supposedly draws on general principles, which justifies that it talks about multiple $\phi$, even though at this level, there is no connection between them.}

\section{Decision situation}
The object of study in a given decision situation is the deliberated preferences of a given individual $i$. 
A decision situation concerns a topic $\alltopic$, given a set of arguments $\allargs$, given a certain query protocol, and over a certain time frame. The relation $\ileadstost$, to be defined shortly, models the reaction of $i$ to arguments about the topic. The decision situation will be defined as a triple $(\alltopic, \allargs, \ileadstost)$ satisfying some conditions to be defined after having presented those three fundamental elements.

The decision situation relates to a topic denoted by $\alltopic$, which is a set of propositions $\prop \in \alltopic$ about which we are interested of knowing the deliberated preferences of $i$. Propositions are not described further, but are supposedly understandable by $i$ (for example, they could be sentences in some natural language). Alternatively, in the case of an analyst helping $i$ to make a decision, $\alltopic$ is a set of propositions about which $i$ is interested to know his own deliberated preferences. I assume $\alltopic$ is closed under negation and introduce a symbol $¬$ for negating a proposition: if $\prop \in \alltopic$, then $¬\prop \in \alltopic$, and $¬(¬\prop) = \prop$. 

With the decision situation comes also a set of all arguments $\allargs$. This set contains all the arguments that may possibly be considered relevant by $i$ for his decision problem, and possibly more.
Importantly, the notion of deliberated preferences does not require to constrain a priori $\allargs$ to some set of arguments that would fit some precise notion of relevancy, coherence, or even well-formedness. This permits to avoid introducing inadequate normative principles into the deliberated preferences: only the individual $i$ is then considered legitimate to dictate what is a relevant argument. Under that view, $\allargs$ may contain anything that can possibly be considered as an argument by anyone, under the widest possible conception of an argument. The notion of deliberated judgment can however also be applied when considering a restricted set of arguments, for example, the arguments that have been put forward by some specific set of experts when talking about the topic $\alltopic$. 

The set $\allargs$ may be infinite or very large: as we will see, thanks to our falsificationist approach, there is no need to be able to explore it entirely. The set of arguments contains at least the empty argument, denoted by $\zar$.

\begin{example}
	Consider a decision problem where a set of alternatives $\allalts$ is given, containing food products among which $i$ would like to form a deliberated preference (taking into account the effects of food on health, price, pleasure, morality issues related to the production process, and so on). The topic could be defined as $\alltopic$ = $\set{\prop_\alt, ¬\prop_\alt, \forall \alt \in \allalts}$, where $\prop_\alt$ is the proposition according to which $\alt$ is one of the best alternatives among $\allalts$ (there is no strictly better alternatives among $\allalts$, from $i$’s point of view), and $¬\prop_\alt$ is the proposition according to which $\alt$ is not one of the best alternatives among $\allalts$ (some $\alt' \in \allalts$ is a strictly better alternative, from $i$’s point of view). 
	The set $\allargs$ is defined as the set of all strings, because in this example it is considered acceptable to restrict arguments to those that can take a textual form.
\end{example}

Define $\Phi = \alltopic × \set{\text{possible}, \text{sure}}$. Elements in $\Phi$ will also be called propositions, which should create no ambiguity. 
The relation $\ileadstost \subseteq (\allargs × \allargs) × \Phi$, pronounced “always leads to”, has the following semantics.
Whenever $(\ar, \ar') \ileadstost (\prop, \text{sure})$, $i$ always considers that $\prop$ is sure (meaning that $¬\prop$ is excluded), when presented with these two arguments, and whenever $(\ar, \ar') \ileadstost (\prop, \text{possible})$, $i$ always considers that $\prop$ is “at least” possible (meaning that $¬\prop$ may be possible as well), when presented with these two arguments. In both cases, the term “always” means that the consideration is stable over time, including after having presented other arguments to $i$. 
Two query protocols presented below will illustrate how these semantics might be satisfied. 
However, I voluntarily leave the querying protocol not precisely specified in general, as multiple reasonable choices are possible. The results of this article depend only on assumptions about $\ileadstost$. 

Define the negation operator $¬$ over $\Phi$ as: $¬(\prop, \text{possible}) = (¬\prop, \text{sure})$ and $¬(\prop, \text{sure}) = (¬\prop, \text{possible})$. It follows that $¬¬\phi = \phi$. Write $(\ar, \ar') \nileadstost \phi$ for $¬[(\ar, \ar') \ileadstost \phi]$ and define $(\ar, \ar') \ileadstoe \phi$ as equivalent to $(\ar, \ar') \nileadstost ¬\phi$. 
The relation $\ileadstoe$ is pronounced “sometimes leads to”.
A proposition $\phi$ is about $\prop$ iff $\phi \in \{\prop, ¬\prop\} × \{\text{possible}, \text{sure}\}$. 

The pair $(\ar, \ar')$ is to be understood as unordered (thus $(\ar, \ar') \ileadstost \prop ⇔ (\ar', \ar) \ileadstost \prop$).

\begin{definition}[Decision situation]
	A decision situation is a topic $\alltopic$ closed under negation, a set of arguments $\allargs$, an “always leads to” relation $\ileadstost \subseteq (\allargs × \allargs) × \Phi$ satisfying (A1) and (A2) and using unordered pairs of arguments, $\Phi$ being the set of propositions $\alltopic × \set{\text{sure}, \text{possible}}$.
\end{definition}

\subsection{Query protocols}
As vNM aptly summarize, “It is clear that every measurement – or rather every claim of measurability – must ultimately be based on some immediate sensation, which possibly cannot and certainly need not be analyzed any futher.” The fundamental element here is the reaction of $i$ to arguments, which will serve as the informational basis to define $\ileadstost$. Given a proposition $\prop$ and two arguments $\ar, \ar' \in \allargs$, a query consists in presenting both arguments to $i$ and observing which proposition $i$ considers valid in her current state of mind, thus using both arguments and possibly other arguments she has in mind: is it that $\prop$ holds, that $¬\prop$ holds, or that both are possible? The third possibility allows for the case where no argument appears more decisive than the other one. 

It is also necessary to capture the evolution of the position of $i$ towards arguments over time: a repeated query using a given pair of arguments may yield different answers, for example because the repeated question has been interleaved with another query containing other arguments (which $i$ may still have in mind when answering the second repetition), or for any other known or unknown reason. The important information for us is the \emph{set} of answers that $i$ could give to a query involving two given arguments and a given topic, among the set $\{\prop, ¬\prop, \text{both}\}$. The set designates all possible answers over time and over different order of asking the queries.
Hence, given $\prop \in \alltopic$, define a relation $\ileadsto^\prop \subseteq (\allargs × \allargs) × \powersetz{\{\prop, ¬\prop, \text{both}\}}$, where $\powersetz{B}$ designates the subsets of $B$ excluding the emptyset, with the semantics that $(\ar, \ar') \ileadsto^\prop C$, with $C \subseteq \set{\prop, ¬\prop, \text{both}}$, iff $\forall c \in C$, it is observable at least once that $i$, when presented with $(\ar, \ar')$, considers $c$ valid in her current state of mind. The querying protocol is supposedly made so that for any $\prop \in \alltopic$, $\ileadsto^\prop = \ileadsto^{¬\prop}$.

This information is only indirectly observable: for example, if $(\ar, \ar') \ileadsto^\prop \{\prop\}$, a querier will never know more than $(\ar, \ar') \ileadsto^\prop C$ with $\prop \in C \subseteq \set{\prop, ¬\prop, \text{both}}$. 
Also, it may be only possible to test one ordering of the queries (assuming that $i$ forgets over time might permit to test different orderings, but this assumption might be unrealistic). In some cases, it may even be only possible to test a single query with a given individual.
This limitation of our observation will not be a problem for the results of this article. Intuitively, this bears on the fact that, first, theories apply to multiple individuals, and second, if a theory claims that $(\ar, \ar') \ileadsto^\prop \{\prop\}$, then either this claim is correct, or it is falsifiable, in the sense that there is a possibility of observing a contradiction to its claim.

To be more concrete, one possible choice of a query protocol, when arguments and propositions are strings, is to read both arguments to $i$, together with $\prop$, and ask $i$ to verbally report the choice he opts for among the three possibilities. Another is to present both arguments and ask $i$ to pick an item among some choice set and make the choice correspond to a validation of $\prop$ or $¬\prop$, possibly even letting the choice engage $i$, for example telling $i$ that she may keep the object of choice (this is illustrated in \cref{ex:pick}). This second possibility is especially interesting as it makes the protocol partially observable in the restricted sense usually appreciated in revealed preferences approaches. Note that depending on the protocol, the choice “both” is not necessarily observable (more about this below).
%In that case, the semantics of $\prop$ and $¬\prop$ relatedly change: instead of meaning that $\prop$ is definitely the 
%Indeed, \citet{tversky_intransitivity_1969} note that individuals “are not perfectly consistent in their choices. When faced with repeated choices between x and y, people often choose x in some instances and y in others. Furthermore, such inconsistencies are observed even in the absence of systematic changes in the decision maker’s taste which might be due to learning or sequential effects. It seems, therefore, that the observed inconsistencies reflect inherent variability or momentary fluctuation in the evaluative process.” This is an accepted fact of experimental psychology \citep{luce_utility_2000}. However, it is reasonable to suspect that fluctuations occur less often, if at all, for some choices. For example, facing a choice between tea or coffee, I would virtually never pick tea (all other things being equal). We are interested in capturing precisely those kind of preferences.

One can also use $\zar$ as one or both of the arguments, in which case the protocol queries the preference of $i$ given only one or given no argument.

\begin{example}[cont.]
	\label{ex:pick}
	Continuing the example, define the querying protocol as, given $\prop_\alt$, presenting two arguments to $i$ and let $i$ choose an item among $\allalts$. $i$ may keep the chosen item. 
	If $i$ chooses $\alt$, it is considered that $i$ has validated $\prop_\alt$, otherwise, $¬\prop_\alt$. The option “both” is not observable.
%	Queries are separated by one week at least, which (as supposed in this example) is enough to ensure that no effect of pleasure for variability enter into play. 
\end{example}

Define $\ileadstost$ from $\ileadsto^\prop$ as follows, for any $\prop \in \alltopic$: $(\ar, \ar') \ileadstost (\prop, \text{possible}) ⇔ (\ar, \ar') \ileadsto^\prop C$ with $C \subseteq \set{\prop, \text{both}}$; and $(\ar, \ar') \ileadstost (\prop, \text{sure}) ⇔ (\ar, \ar') \ileadsto^\prop \set{\prop}$. 

Note that (A1) and (A2) hold.

When “both” is not observable, $\ileadstost$ is defined as: $(\ar, \ar') \ileadstost (\prop, \text{possible}) ⇔ (\ar, \ar') \ileadstost (\prop, \text{sure}) ⇔ (\ar, \ar') \ileadsto^\prop \set{\prop}$. (A1) and (A2) still hold. We say in this case that $\ileadstost$ does not distinguish sure from possible. Thus, elements $\phi \in \Phi$ may be considered as simply equal to $\prop$ or $¬\prop$, as $(\prop, \text{sure})$ is treated exactly as $(\prop, \text{possible})$.

\subsection{Deliberated preference}
The following important definition is required in order to define the deliberated preference of $i$. An argument $\ar$ is decisive for $\phi$, in short, $\phi\text{-dec}$ iff $\forall \ar' \in \allargs: (\ar', \ar) \ileadstost \phi$. 
\footnote{Equivalently: $\ar \ \phidec ⇔ [\ileadstostinv(\phi)](\ar) = \allargs$, and $\exists \ar \ \phidec ⇔ [\ileadstostinv(\phi)]^{-1}(\allargs) ≠ \emptyset$.}
I also say that $\ar$ is $\propsuredec$, $\proppossdec$, or $¬\proppossdec$, when $\phi=(\prop, \text{sure})$, $\phi=(\prop, \text{possible})$, or $\phi=(¬\prop, \text{possible})$, respectively.
\footnote{$\propsuredec ⇔ \forall \ar': (\ar', \ar) \ileadstosst \prop ⇔ \forall \ar': ¬[(\ar', \ar) \ileadstope ¬\prop]$, $\proppossdec ⇔ \forall \ar': (\ar', \ar) \ileadstopst \prop$. Note that “both” ≠ “we donno”: in case “both”, we know how to prove that both are reasonable options, in the sense that neither is certain. We have excluded some possibilities (and some other model could disagree with the “both” conclusion).}

\begin{proposition}
	\label{thm:nocontrdec}
	For any decision situation, a) $\exists \ar \ \phidec ⇒ \nexists \ar' \ ¬\phidec$ and b) $\exists \ar \ \propsuredec ⇒ \nexists \ar' \ ¬\propsuredec$.
\end{proposition}
\begin{proof}
	About a), when $\ar$ is $\phidec$, given any $\ar'$, $(\ar, \ar') \ileadstost \phi$ hence $(\ar, \ar') \ileadstoe \phi$ (using (A1)), hence $(\ar, \ar') \nileadstost ¬\phi$. b) obtains similarly using (A2).
\end{proof}

The deliberated preference of $i$ maps each pair $(\prop, ¬\prop)$ to one among four possibilities, depending on the existence of decisive arguments supporting the propositions about $\prop$. The mapping is defined as follows.
\begin{description}
	\item[$\prop$ is sure] iff $\exists \ar \ \propsuredec$
	\item[$¬\prop$ is sure] iff $\exists \ar \ ¬\propsuredec$
	\item[both $\prop$ and $¬\prop$ are possible] iff $\exists \ar \ \proppossdec$ and $\exists \ar \ ¬\proppossdec$
	\item[$(\prop, ¬\prop)$ is undecided] iff neither $\prop$ is sure, nor $¬\prop$ is sure, nor both are possible.
\end{description}
\Cref{thm:nocontrdec} shows that the cases are separate.
\footnote{Example of $(\prop, ¬\prop)\text{-undecided}$: $\forall (\ar, \ar'), (\ar, \ar') \ileadstoe (¬\prop, \text{sure}), (\ar, \ar') \ileadstoe (\prop, \text{sure})$.}
\footnote{Example of $\overline{¬\prop\text{-CC}} ∧ \prop \in \iposs$ (thus with $\prop\text{-CC}$ and no $(\prop, ¬\prop)$-unstability): $\forall (\ar, \ar'), (\ar, \ar') \ileadstose \prop, (\ar, \ar') \ileadstope ¬\prop, (\ar, \ar') \ileadstopst \prop$. In such a situation, it is unclear whether $\prop$ holds rather than $¬\prop$ or both $\prop$ and $¬\prop$.}
\footnote{Something like Axiom JU should be sufficient for CC. We say $(\ar, \ar')$ is $\prop$-unstable iff $(\ar, \ar') \ileadstose \prop ∧ (\ar, \ar') \ileadstose ¬\prop$. JU: If $(\ar, \ar')$ is $\prop$-unstable, then $\exists \ar_2 \in \allargs$ such that for $(\ar_1, \ar_0) = (\ar, \ar')$ or for $(\ar_1, \ar_0) = (\ar', \ar)$, $\ar_2$ attacks decisively $\ar_1$, where we mean by this that $\ar_2$ combined with $\ar_0$ is decisive. Means that if not CC, then not JU. Hence, infinite discussion happens, or irreducible unstability. Or, the memory buffer of $i$ is too short to hold adequate reinstated arguments, in which case our approach based on decisive arguments is not adequate.}

It should be clear that the validity of the conclusions obtained from any theory of deliberated preference rests on the acceptance of this just given normative definition, namely, it is to be considered meaningful that the deliberated preference of $i$ be defined in such a way, given the query protocol used for defining $\ileadstost$ in a particular application.

\commentOC{TODO}
We want to leave “undecided” as an option: the model takes no position, does not know about $(t, ¬t)$. But we want to exclude $t$ possible and $¬t$ unknown. Thus $t$ possible should imply $¬t$ possible or $t$ sure. If this holds, we can simplify the status of the propositions: only determine whether $t$ is possible for all $t \in \alltopic$, and from there, we know everything. This result can be obtained with: assume $s$ decisive for $t$ poss and no $s$ is decisive for $¬t$ poss, then, for some $(s_1, s)$ leading to $t$ sure and $(s_2, s)$ leading to $\set{t, ¬t}$, it is possible to form $(s_2, s_1, s)$, and if this leads to $t$ sure, it should imply that $s + s_1$ is decisive for $t$ sure; and if this leads to $¬t$ possible, then this should imply that $s+s_2$ is decisive for $¬t$ possible.

This information is summarized by defining a subset $\iPhi \subseteq \Phi$ of propositions that represent $i$’s deliberated preference, defined as follows.
\begin{description}
	\item[$\prop$ is sure] ⇒ $(\prop, \text{possible}) \in \iPhi ∧ (\prop, \text{sure}) \in \iPhi ∧ (¬\prop, \text{possible}) \notin \iPhi ∧ (¬\prop, \text{sure}) \notin \iPhi$
	\item[both $\prop$ and $¬\prop$ are possible] ⇒ $(\prop, \text{possible}) \in \iPhi ∧ (¬\prop, \text{possible}) \in \iPhi ∧ (\prop, \text{sure}) \notin \iPhi ∧ (¬\prop, \text{sure}) \notin \iPhi$
	\item[$(\prop, ¬\prop)$ is undecided] ⇒ $\phi \notin \iPhi$ for all $\phi$ about $\prop$.
\end{description}
Because this defines unequivocally the membership of every proposition about $\prop$ for every $\prop \in \alltopic$, this suffices to define $\iPhi$; and it follows that the reverse implications hold as well.

The following propositions can help get convinced that the proposed definition of deliberated preference makes sense.
\begin{proposition}
	a) $\phi \in \iPhi ⇒$ some $\ar$ is $\phidec$. b) $(\prop, \text{possible}) \in \iPhi$ iff $\exists \ar \ \proppossdec ∧ (\exists \ar \ \propsuredec ∨ \exists \ar \ ¬\proppossdec)$. c) $(\prop, \text{sure}) \in \iPhi ⇒ (\prop, \text{possible}) \in \iPhi$. d) $\phi \in \iPhi ⇒ ¬\phi \notin \iPhi$. e) $(\prop, ¬\prop)$ undecided $⇔ \phi \notin \iPhi ∧ ¬\phi \notin \iPhi$ for some $\phi$ about $\prop$. f) If $\phi \in \iPhi$ for some $\phi$ about $\prop$, then $\phi \in \iPhi ⇔ ¬\phi \notin \iPhi$ for all $\phi$ about $\prop$. g) If $\ileadstost$ does not distinguish sure from possible, no pair of proposition is mapped to “both are possible”.
\end{proposition}
\begin{proof}
	a) is a consequence of A2. b) and c) follow from the definition of $\iPhi$. d) is obtained using a), \cref{thm:nocontrdec}, and the contrapositive of a). About e), we must show that $\phi \notin \iPhi ∧ ¬\phi \notin \iPhi$ for some $\phi$ about $\prop$ implies $\phi' \notin \iPhi$ for all $\phi'$ about $\prop$. Assume $\phi=(l, \text{possible})$ and $¬\phi = (¬l, \text{sure})$ are not in $\iPhi$, with $l = t$ or $l = ¬t$. We need to prove that $(l, \text{sure})$ and $(¬l, \text{possible})$ are not in $\iPhi$ either. We obtain $(l, \text{sure}) \notin \iPhi$ using the contrapositive of c). And $(¬l, \text{possible}) \notin \iPhi$ because $(¬l, \text{possible}) \in \iPhi$ would require another $\phi'$ about $\prop$ to be present in $\iPhi$, as can be checked from the definition of $\iPhi$. f) follows from d) together with the fact that $(\prop, ¬\prop)$ is (by definition) not undecided when $\phi \in \iPhi$ for some $\phi$ about $\prop$, hence, from e), $\phi \in \iPhi ∨ ¬\phi \in \iPhi$. Finally, g) is because in that case, $\exists \ar \ \proppossdec ⇔ \exists \ar \ \propsuredec$, thus using \cref{thm:nocontrdec}, $\exists \ar \ \proppossdec ⇒ \nexists \ar' \ ¬\proppossdec$.
\end{proof}

It is important to remark that when the query protocol does not admit observation of “both”, thus, when $\ileadstost$ does not distinguish sure from possible, it does \emph{not} follow that $i$ is sure of either $\prop$ or $¬\prop$ for every pair of proposition: the “undecided” case will hold whenever $i$ considers no argument as sufficiently strong to determine a decisive preference (assuming that in that case, $i$ at least sometimes alternates his choice). Admittedly, it might also happen that $i$ is really indifferent between $\prop$ and $¬\prop$ but still systematically chooses, say, $\prop$ over $¬\prop$, in which case it will be (somewhat) erroneously concluded that $\prop \in \iPhi$. This is the price to pay for using protocols that do not allow to observe indifference. Observe however that (depending on the application) this error might be considered not harmful, as the obtained model will be faithful to $i$’s behavior after deliberation.

\section{Next}
Define a way of saying “we never need this argument”: either the model can prove phi without s0, or if it can’t, then phi does not hold, because I can play s1 against it. 

What about a model for only one situation? Then we need to accept that $q$ can’t be falsified, and accept that it defines the individual (or just assume without proof that it does not).

Assume $i$, given $\ar$ and $\ar'$, picks repeatedly (in various circumstances) nonempty subsets of $\set{\prop, ¬\prop}$. Define $(\ar', \ar) \ileadstoe (\prop, \text{sure})$ iff she sometimes chooses $\set{\prop}$ and $(\ar', \ar) \ileadstoe (\prop, \text{poss})$ iff she sometimes chooses $\set{\prop}$ or $\set{\prop, ¬\prop}$. Then, $(\ar', \ar) \ileadstost (\prop, \text{sure})$ iff she always chooses $\set{\prop}$ and $(\ar', \ar) \ileadstost (\prop, \text{poss})$ iff she always chooses either $\set{\prop}$ or $\set{\prop, ¬\prop}$.

Discuss variant with an empirical claim: $\hist \ileadstoi \phi ⇒ \hist \nileadstoi ¬\phi$; and change \cref{ax:norm}: $[\forall \hist, \hist \ileadstoi \propposs] ⇒ \propposs \in \iPhi$, … This is essentially what will be done in the following, I suppose.

If we do not have $\propsure$ implies $\propposs$, then we can have $\propsure$ and $\notpropsure$ both accepted. But all propositions of the normative analysis would still hold, it seems. We could have a property that says: $\propsure \in \iPhi ⇒ \propposs \in \iPhi \land \notpropsure \notin \iPhi \land \notpropposs \notin \iPhi$.

\section{Theories proposing decisive arguments}
In most situations, there are too many arguments to attempt to exhaustively test whether any $\ar$ is a decisive argument for any $\phi$. And most importantly, it is impossible to confirm definitely that $(\ar, \ar') \ileadstost \phi$, whenever $(\ar, \ar') \ileadstost \phi$. This is why falsifiable theories are needed.

A theory $T$ proposing decisive arguments describes a set $D_T$ of decision situations to which it applies, and proposes, for any situation $d = (\Phi, \allargs, \ileadstost) \in D_T$, a subset $\Phi_{d, T} \subseteq \Phi$ and a mapping $f_{d, T}: \Phi_{d, T} → \allargs$. The theory claims that it knows the deliberated preference of $i$ (the individual described by $\ileadstost$ in $d$) about the propositions in $\Phi_{d, T}$, and claims that $f_{d, T}(\phi)$ is a decisive argument for $\phi$. The theory $T$ is said to have a position about $\prop$ if $(\prop, \text{sure}) \in \Phi_{d, T} ∨ (¬\prop, \text{possible}) \in \Phi_{d, T}$. 
In order for the claims of the theory to be interpretable, it is requested that if $T$ has a position about $\prop$, then it must have a position about $¬\prop$. This forces $T$, for each pair $(\prop, ¬\prop)$, to either claim that one is sure, claim that both are possible, or have no claim about the pair.

A falsification instance of $T$ is a triplet $(d, \ar', \phi) \in D_T × \allargs × \Phi_{d, T}$ such that $(f_{d, T}(\phi), \ar') \ileadstoe ¬\phi$.
A theory $T$ is said to be falsified iff such a triplet is found. 

As is usual in science, falsification procedures should not be considered as ways of definitely validating a theory, but rather as ways of confronting theories.
Two theories $T, T'$ are said to disagree on $\phi$ iff $\phi \in \Phi_{d, T}$ and $¬\phi \in \Phi_{d, T'}$ in some situation $d \in D_T ∪ D_{T'}$. $T$ and $T'$ are said to disagree iff they disagree on some $\phi \in \Phi$.
To confront a theory $T$ to a disagreeing theory $T'$, suffice to test, for any situation $d \in D_T ∪ D_{T'}$ and $\phi$ such that $\phi \in \Phi_{d, T}, ¬\phi \in \Phi_{d, T'}$, whether $(f_{d, T}(\phi), f_{d, T'}(¬\phi)) \ileadstoe \phi$.
Such a test necessarily falsifies one of the theories. 

The following proposition shows that this notion of falsification adequately tests the claims of the theory.
\begin{proposition}
	\label{thm:fals}
	A theory $T$ proposing decisive arguments has no falsification instances iff $\forall d \in D_T, \forall \phi \in \Phi_{d, T}: f_{d, T}(\phi)$ is $\phidec$. Furthermore, if a theory $T$ proposing decisive arguments has no falsification instances, then $\Phi_{d, T} \subseteq \iPhi$, $\forall d \in D_T$.
\end{proposition}

\begin{example}[(cont.)]
	Returning to the previous example, a theory $T$ could claim that, for all individuals that belong to some given socio-economic situation and that have some given degree (described by the theory in sufficient details that it is possible to determine precisely to which kind of individuals it applies), $(\prop_a, \text{sure})$ is in the deliberated preferences of the individuals, for a given alternative $\alt$ also described by $T$. 
The theory also suggests some argument $\ar$, a text that (according to $T$) will convince any individual (fitting the description) that $\prop_\alt$ holds, thus, that $\alt$ is a “good” food product for him. 
The theory can be put to the test by picking an individual fitting the description, presenting the argument together with any other argument (for example, proposed by another theory), and observing whether $i$ is convinced.
$T$ resists to such a falsification test if $i$ is convinced. According to \cref{thm:fals}, $T$ tells the truth if it would resist to any possible falsification test. (This is impossible to definitely make sure of.)
\end{example}
This example illustrates a difference between this proposal and the classical revealed approach: it could be that $T$ tells the truth even though its claim does not correspond to the revealed preference of some individuals, thus that, given no argument, some individuals would consider $a$ as dominated by another alternative in the set considered. It also illustrates a difference between this proposal and persuasion: the goal of $T$ is not merely to convince $i$ (possibly by using its lack of knowledge of any counter-argument to what $T$ says), but to resist any falsification attempt. Thus, $T$ should be confronted to arguments coming from as varied perspectives as possible, in order to give confidence that it tells the truth.

\bibliography{simple}
\end{document}

\section{Models}
A model is a triple $(\mbeats, \mleadsto, +)$ defined as follows and satisfying the following constraints.

$\mbeats$ a binary relation over $\allargs$. $\mleadsto \subseteq \allargs × \alltopic$. Define $\clargs \subseteq \allargs$ as the set of arguments used in $\mbeats ∪ \mleadsto$. Let $+$ be defined over arguments used in the model: $\ar_3 + \ar_1 = \ar'$ for some $\ar' \in \clargs$, for any $\ar_3, \ar_1 \in \clargs$. 

Requirements. The maximum length of a path in $\mbeats$ is finite (which implies that $\mbeats$ is acyclic).
$\ar_3 \mbeats \ar_2 \mbeats \ar_1 ⇒ \ar_2 \mbeats \ar_3 + \ar_1$.\footnote{Necessary for definition of $\ar_3 \wibeatse[\phi] \ar_2$. TODO remove this condition, define $\wibeatse[\phi]$ more largely and never valid when not in $\mbeats$.}

Notation. Let $\mleadstoinv(\alltopic) \subseteq \clargs$ denote the subset of arguments supporting propositions. 
$\mbeats(\args)$: arguments that $\args$ attacks, $\ar_1 \in \mbeats(\args) ⇔ \exists \ar_2 \in \args \suchthat \ar_2 \mbeats \ar_1$; $\mbeatsinv(\ar)$: arguments attacking $\ar$, $\ar_2 \in \mbeatsinv(\ar) ⇔ \ar_2 \mbeats \ar$.  We write $\ar \mleadsto (\prop, \text{possible})$ for $\ar \mleadsto \prop$ and $\ar \mleadsto (\prop, \text{sure})$ for $\ar \mleadsto \prop ∧ (¬\prop \notin \mleadsto(\allargs))$.

Given a decision situation, we define the relations $\wibeatse[\phi], \nwibeatse[\phi] \subseteq \allargs × \allargs$ in order to obtain the following result, and with $\wibeatse[\phi], \nwibeatse[\phi] \subseteq \mbeats$.

Given $\ar_3 \mbeats \ar_2, \phi \in \Phi$: $\ar_3 \wibeatse[\phi] \ar_2$ iff $[\exists \ar_1 \in \mbeats(\ar_2) \suchthat (\ar_2 \wibeatse[\phi] \ar_1 ∧ \ar_2 \nwibeatse[\phi] \ar_3 + \ar_1)] ∨ [\ar_2 \mleadsto \phi ∧ (\ar_3, \ar_2) \ileadstoe ¬\phi]$.

Given $\ar_3 \mbeats \ar_2, \phi \in \Phi$: $\ar_3 \nwibeatse[\phi] \ar_2$ iff $[\exists \ar_1 \in \mbeats(\ar_2) \suchthat (\ar_2 \wibeatse[\phi] \ar_1 ∧ \ar_2 \wibeatse[\phi] \ar_3 + \ar_1)] ∨ [\ar_2 \mleadsto \phi ∧ (\ar_3, \ar_2) \ileadstoe \phi]$.

This is done with the following construction.
Given a model. 
\begin{itemize}
	\item Define a root as an argument that $\mbeats$-attacks nothing. 
	\item Associate to each $\ar \in \clargs$ its depth $d(\ar)$, the distance to the farthest root: $d(\ar) = 0$ iff $\ar$ is a root and $d(\ar) = k+1$ iff $\ar$ $\mbeats$-attacks some $\ar'$ of depth $k$ and $\mbeats$-attacks no $\ar'$ of depth superior to $k$. \footnote{Thus $d(\ar) = 1$ iff $\ar$ attacks some and only root nodes; $d(\ar) = 2$ iff $\ar$ attacks some node of depth 1 and only nodes of depth 1 or 0; and so on.} 
	\item Observe that if $\ar_3 \mbeats \ar_2$ and $d(\ar_3) = k+1$, then $d(\ar_2) ≤ k$. 
	\item Define $\wibeatse[\phi]_{(0)} = \nwibeatse[\phi]_{(0)} = \emptyset$. 
	\item Define $\wibeatse[\phi]_{(k+1)} \subseteq \mbeats$, $k ≥ 0$, as $\ar_3 \wibeatse[\phi]_{(k+1)} \ar_2$ iff $\ar_3 \mbeats \ar_2$, $d(\ar_3) = k+1$, and: $[\exists \ar_1 \in \mbeats(\ar_2) \suchthat (\ar_2 \wibeatse[\phi]_{(d(\ar_2))} \ar_1 ∧ \ar_2 \nwibeatse[\phi]_{(d(\ar_2))} \ar_3 + \ar_1)] ∨ [\ar_2 \mleadsto \phi ∧ (\ar_3, \ar_2) \ileadstoe ¬\phi]$. Proceed similarly for $\nwibeatse[\phi]_{(k+1)}$.
	\item Define $\wibeatse[\phi] = \bigcup_{k ≥ 0} \wibeatse[\phi]_{(k)}$, and similarly for $\nwibeatse[\phi]$.
\end{itemize}

Given $\ar_3 \in \clargs, \ar_2 \in \clargs$, with $\exists \ar_1 \in \mbeats(\ar_2) \suchthat \ar_2 \wibeatse[\phi] \ar_1$, we have: $\ar_3 \wibeatse[\phi] \ar_2 ∨ \ar_3 \nwibeatse[\phi] \ar_2$.

Define $\ar_2 \wibeatse \ar_1 ⇔ \exists \phi \in \alltopic \suchthat \ar_2 \wibeatse[\phi] \ar_1$.

Hence, given $\ar_3 \in \clargs, \ar_2 \in \clargs, \ar_2 \notin \mleadstoinv(\alltopic)$: $¬ (\ar_3 \nwibeatse \ar_2)$ iff $\forall \ar_1 \in \mbeats(\ar_2) ∩ \wibeatse(\ar_2): ¬ (\ar_2 \wibeatse \ar_3 + \ar_1)$.

\section{Conditions}
All these conditions assume that a decision situation $(\allalts, \allargs, …)$ and a model $\eta = (\mbeats, \mleadsto, +)$ are given.

Define $\argsdec^\phi = \clargs \setminus \text{im}(\wibeatse[\phi])$ the decisive arguments according to $\wibeatse[\phi]$, or $\wibeatse[\phi]$-decisive arguments for short: $\ar \in \argsdec^\phi ⇔ \wibeatseinv[\phi](\ar) = \emptyset$.

We say that $\ar$ replaces $\ar_1$ iff $\mbeats(\ar_1) \subseteq \mbeats(\ar)$.

\begin{definition}[Reinstatement]
	Given $\ar_3 \wibeatse[\phi] \ar_2 \wibeatse[\phi] \ar_1, \ar_3 \in \argsdec^\phi$: $\mbeats(\ar_1) \subseteq \mbeats(\ar_3 + \ar_1)  ∧ \mbeatsinv(\ar_3 + \ar_1) \subseteq \mbeatsinv(\ar_1) \setminus \mbeats(\ar_3)$.
	\footnote{TODO the condition must be $\wibeatseinv(\ar_3 + \ar_1) \subseteq \mbeatsinv(\ar_1) \setminus \mbeats(\ar_3)$ to allow $\ar_5 \mbeats \ar_4 \mbeats \ar_3 \mbeats \ar_2 \mbeats \ar_1$ and $\ar_5 \mbeats \ar_4 \mbeats \ar_3 + \ar_1$, considering that possibly $\ar_3$ is $\wibeatse$-decisive. This should not invalidate the conditions, but it does currently. But it’s not a problem: the model would actually not be built this way. In this scenario the argument $\ar_3 + \ar_1$ is useful only in case $\ar_3$ is decisive, thus $\ar_4 \mbeats \ar_3 + \ar_1$ must not be planned. Rather $\ar_5 + \ar_3$ decisive, then $(\ar_5 + \ar_3) + \ar_1$. Alternatively, also $\ar_4 \mbeats \ar_1$ and then no problem as well.}
	\footnote{The stronger condition mandating $\mbeatsinv(\ar_3 + \ar_1) \subseteq \wibeatseinv(\ar_1) \setminus \mbeats(\ar_3)$ would be more difficult to check: when some $\ar_2 \wibeatse \ar_3 + \ar_1$, we’d need to check not only that $\ar_2 \mbeats \ar_1$ but also $\ar_2 \wibeatse \ar_1$.}
	\footnote{We do not mandate that $\ar_3 + \ar_1 \ileadsto \phi$, so that the model can afford not resisting to the counter-attacks to $\ar_3 + \ar_1$ (resistance to c-a to $\ar_1$ suffice).}
	\footnote{Is Reinf equivalent to Justifiable unstability? Define $\ar_3 \wibeatsst[\phi] \ar_2 ⇔ \ar_3 \wibeatse[\phi] \ar_2 ∧ ¬(\ar_3 \nwibeatse[\phi] \ar_2)$. Prove that $\ar_3 \wibeatsst[\phi] \ar_2 \wibeatse[\phi] \ar_2 ⇒ ¬(\ar_2 \wibeatse[\phi] \ar_3 + \ar_1)$.} 
\end{definition}

\begin{definition}[Justifiable unstability]
	$\forall \ar_2 \mbeats \ar_1 \suchthat \ar_2 \wibeatse[\phi] \ar_1, \ar_2 \nwibeatse[\phi] \ar_1: \exists \ar_3 \mbeats \ar_2 \suchthat \ar_3 \wibeatse[\phi] \ar_2$.
\end{definition}

$\ar$ is $\phi$-defended iff $\wibeatseinv[\phi](\ar) \subseteq \wibeatse[\phi](\argsdec^\phi)$ (its $\wibeatse[\phi]$-attackers are $\wibeatse[\phi]$-attacked by $\wibeatse[\phi]$-decisive arguments).

\begin{definition}[Finite defense]
	If $\wibeatseinv[\phi](\ar) \subseteq \wibeatse[\phi](\argsdec^\phi)$, then $\exists \args \subseteq \argsdec, \card{\args} ≤ j \suchthat \mbeatsinv(\ar) \subseteq \wibeatse[\phi](\args)$.
\footnote{To satisfy Finite defense, in presence of the other conditions, suffice to limit the width of the model (TODO check). But it may be interesting to not limit it and declare that the model has specific replies to any counter-argument, but promises to use only a few rebuttals and that afterwards, the dm will stop using those kind of arguments (but we don’t know in advance which ones will be chosen).}
\end{definition}
Thus, if the attackers of $\ar$ are attacked by decisive arguments, then $j$ defenders are enough to defend $\ar$.

Define $R^\phi$, the reinstates relation, as follows: $(\ar_3 + \ar_1) R^\phi \ar_1$ iff $\ar_3 \wibeatse[\phi] \ar_2 \wibeatse[\phi] \ar_1$ (for some $\ar_2, \phi$), $\ar_3 \in \argsdec^\phi$. Define $\gdargs^\phi$ as the transitive closure of $\mleadstoinv(\Phi)$ under $R^\phi$. Given $\phi, \ar \in \gdargs^\phi, \ar' \in \allargs$, define $\ar' \ibeatse[\phi] \ar ⇔ (\ar', \ar) \ileadstoe ¬\phi$.
\begin{definition}[Covering]
	$\forall \phi, \ar \in \gdargs^\phi, \ar' \in \allargs: \ar' \ibeatse[\phi] \ar ⇒ \ar' \wibeatse[\phi] \ar$. 
\end{definition}

\begin{definition}[Observable validity]
	$\forall \ar_2 \mbeats \ar_1 \mleadsto \phi: ¬(\ar_2 \wibeatse[\phi] \ar_1) ∨ \exists \ar_3 \mbeats \ar_2 \suchthat \ar_3 \wibeatse[\phi] \ar_2$.
	\footnote{If the model claims $¬\prop \in \isure$, this requires clear-cut (for that prop), so we must mandate it (hopefully A3 or an equivalent such as Justifiable unstability fits). Thus we only need to prove $\prop \notin \iposs$, for which $(\ar_1, \ar) \ileadstose ¬\prop$, for each $\ar$, suffices. In fact, from $\ar_1$ in the model arguing for $\phi$ and $(\ar_1, \ar) \ileadstoe \phi \forall \ar \in \allargs$, we should probably be able, using the argument in the theorem about finite chains, to deduce that this is enough to conclude that $\ar_1$ is $\phidec$.}
\end{definition}

\section{Theorem}
\begin{theorem}[Validity]
	Given a decision situation and a model $\eta$, if all our conditions are satisfied, $\mleadsto(\clargs) \subseteq \iPhi$.
\end{theorem}
\begin{proof}
First, we want to prove that $\ar_1$ $\phi$-defended implies $\ar_1$ replaceable by some $\wibeatse[\phi]$-decisive $\ar$, and if $\ar_1 \in \gdargs^\phi$, then its replacer $\ar$ is in $\gdargs^\phi$ as well.

Because $\ar_1$ is $\phi$-defended, $\exists \args \subseteq \argsdec^\phi \suchthat \mbeatsinv(\ar_1) \subseteq \wibeatse[\phi](\args)$, $\args$ finite [Finite defense]. 

Pick any $\ar_{3, 1} \in \args$ such that $\ar_{3, 1} \wibeatse[\phi] \ar_2 \wibeatse[\phi] \ar_1$ (if there’s none, $\ar_1 \in \argsdec^\phi$ and we’re done). $\ar_{3, 1} + \ar_1$ replaces $\ar_1$, and $\mbeatsinv(\ar_{3, 1} + \ar_1) \subseteq \mbeatsinv(\ar_1) \setminus \mbeats(\ar_{3, 1})$ [Reinstatement]. Hence, $\mbeatsinv(\ar_{3, 1} + \ar_1) \subseteq \wibeatse[\phi](\args) \setminus \mbeats(\ar_{3, 1})$. Iterate by picking any $\ar_{3, 2} \in \args$ such that $\ar_{3, 2} \wibeatse[\phi] \ar_2 \wibeatse[\phi] \ar_{3, 1} + \ar_1$ (if there’s none, $\ar_{3, 1} + \ar_1 \in \argsdec^\phi$ and we’re done) and obtaining $\ar_{3, 2} + (\ar_{3, 1} + \ar_1)$ replacing $\ar_{3, 1} + \ar_1$ (hence, replacing $\ar_1$) with $\mbeatsinv(\ar_{3, 2} + (\ar_{3, 1} + \ar_1)) \subseteq \mbeatsinv(\ar_{3, 1} + \ar_1) \setminus \mbeats(\ar_{3, 2})$. Hence, $\mbeatsinv(\ar_{3, 2} + (\ar_{3, 1} + \ar_1)) \subseteq \wibeatse[\phi](\args) \setminus \mbeats(\ar_{3, 1}) \setminus \mbeats(\ar_{3, 2})$. Iterating in such a way over the finite set $\args$ will finally yield an element that is $\wibeatse[\phi]$-decisive. The last point, $\ar_1 \in \gdargs^\phi ⇒ \ar \in \gdargs^\phi$, follows from the definition of $\gdargs^\phi$.

Second, we want to prove that if $\ar_1$ is not $\phi$-defended and has no decisive $\wibeatse[\phi]$-attackers (meaning that $\wibeatseinv[\phi](\ar_1) \subseteq \overline{\argsdec^\phi}$), then $\ar_1$ is $\wibeatse[\phi]$-attacked by some $\ar_2$ that is not $\phi$-defended and has no decisive $\wibeatse[\phi]$-attacker.

Consider $\ar_1$ not $\phi$-defended and having no decisive $\wibeatse[\phi]$-attackers. Because $\ar_1$ is not $\phi$-defended, by definition, it is $\wibeatse[\phi]$-attacked by some $\ar_2$ that has no decisive $\wibeatse[\phi]$-attacker. Because $\ar_1$ has no decisive attacker, $\ar_2$ is not decisive. If $\ar_2$ was defended, by the first part of this proof, it would be replaceable by a decisive argument, and $\ar_1$ would have a decisive attacker. Thus, $\ar_2$ is not defended.

We conclude from this second point that if $\wibeatseinv[\phi](\ar_1) \subseteq \overline{\argsdec^\phi}$, then $\ar_1$ is $\phi$-defended. Otherwise, it would be attacked by some $\ar_2$, on which the condition would apply in turn, thus which would be attacked by some $\ar_3$, on which the condition would apply, and iterating in this way would build an infinite chain in $\wibeatse[\phi]$, a contradiction.

Third, consider an argument $\ar_1 \in \mleadstoinv(\Phi)$. If some $\ar_2 \wibeatse[\phi] \ar_1$ then $\ar_3 \wibeatse[\phi] \ar_2$ [Obs val]. Thus, $\wibeatseinv[\phi](\ar_1) \subseteq \overline{\argsdec^\phi}$. Thus, $\ar_1$ is $\phi$-defended, by our second point. Hence, by our first point in this proof, $\ar_1$ is replaceable by some $\wibeatse[\phi]$-decisive $\ar \in \gdargs$. We obtain that $\forall \ar' \in \allargs: ¬(\ar' \wibeatse[\phi] \ar)$. As $\ar \in \gdargs$, $\ar' \ibeatse[\phi] \ar ⇒ \ar' \wibeatse[\phi] \ar$ [Covering], thus we obtain $¬(\ar' \ibeatse[\phi] \ar)$, hence $\ar$ is $\phidec$ and $\phi \in \iPhi$.
\end{proof}

\appendix
\section{Todo}
We define $(\ar, \ar') \ileadstopst \prop ⇔ \ar' \nibeatsst[\prop] \ar ⇔ \ileadsto(\ar, \ar') \subseteq \{\propsure, b\}$, and $(\ar, \ar') \ileadstosst ¬\prop ⇔ \ar \nibeatsst[¬\propsure] \ar' ⇔ \ileadsto(\ar, \ar') = \{¬\propsure\}$.
Hence, $(\ar, \ar') \ileadstose ¬\prop ⇔ \ar' \ibeatse[\prop] \ar ⇔ ¬(\ar' \nibeatsst[\prop] \ar) ⇔ \ileadsto(\ar, \ar') \supseteq \{¬\propsure\}$, and $(\ar, \ar') \ileadstope \prop ⇔ \ar \ibeatse[¬\propsure] \ar' ⇔ ¬(\ar \nibeatsst[¬\propsure] \ar') ⇔ \ileadsto(\ar, \ar') ≠ \{¬\propsure\}$.

\begin{itemize}
	\item If $i$ does not consider $\ar$ as supporting $\prop$, it also works: if $\prop$ is not weakly acceptable by default, then any $\ar'$ is considered by $i$ as a better argument than $\ar$ in favor of certain $¬\prop$, and so on. In fact, whether $\emptyset \ibeatse[\prop] \emptyset$ determines whether $\prop$ is weakly supported by default.
	\item I should define $\ar' (□\ibeatse[\prop]) \ar$ as an observable: “Assuming $\ar'$ would survive, do you consider $\ar'$ as leading to certainty of $¬\prop$, even when considering $\ar$?”. It distinguishes our knowledge and the truth: $\ar' (□\ibeatse[\prop]) \ar ⇒ \ar' \ibeatse[\prop] \ar$, thus, implies $¬ (\ar' \nibeatsst[\prop] \ar)$. But out of $¬ (\ar' (□\ibeatse[\prop]) \ar)$, nothing.
	\item Partition (objectively) $\allargs$ (or $\allargs × \alltopic$) into arguments in favor of $\prop, \text{sure}$, $¬\prop, \text{sure}$, and similarly for possible. Use only one rel $\ibeatse$, defined on contradictory arguments only, instead of $\ibeatse[\prop, \text{sure}]$ and others. Define $\ar' \ibeatse[\prop] \ar$ equals no when $¬(\ar' \ileadsto ¬\prop, \text{sure})$, equals $\ibeatse$ for adequate arguments, and equals yes when $¬(\ar \ileadsto \prop, \text{possible})$ and $\ar' \ileadsto ¬\prop, \text{sure}$, with probably some complications needed for the argument $\emptyset$ (and related default attitude towards $\prop$).
\end{itemize}

Questions:
Q1. Relationship with $\ar \ibeatse[\prop] \ar$?

We want to exclude: $s$ supports $p$ perhaps, attacked by $s2$ (supporting $¬p$ sure), but then $s2$ is attacked by $s$. Exclude $\ar' \ibeatse[\prop] \ar$ and $\ar \ibeatse[¬\prop, \text{sure}] \ar'$. Require to assume that this situation implies another argument $\ar_3$ “attacking” $\ar'$, thus, such that $\ar_3 + \ar$ is no more attacked by $\ar_2$. 

\subsection{Attack rel}
Given $\prop$, define a (symmetric) strong attack rel, $Q_\prop$, as $\ar Q_\prop \ar'$ iff $(\ar, \ar') \ileadstosst ¬\prop$.

Define a (symmetric) weak attack rel, $R_\prop$, as $\ar R_\prop \ar'$ iff $(\ar, \ar') \ileadstopst ¬\prop$.

Given $\prop$, we want to check the arguments that defend it. We want to build an asymmetric attack relation.
$\ar_1 M \ar_0$, and not the converse, iff $\ar_1$ attacks $\ar_0$ and, would $\ar_1$ be attacked, $\ar_0$ reinstated would defend $\prop$.
$\ar_1$ should be able to support a sub-claim $\prop'$, and then $\ar_2$ attack $\ar_1$ on that sub-claim precisely. $\ar_2$ decisive on this sub-claim and $\ar_0 + \ar_2$ decisive implies $\ar_1$ attacks $\ar_0$ and not conversely.

Or simply. $\ar_1$ decisive for $¬\prop$ means $\ar_1$ attacks $\ar_0$ and not conversely.

Then define $i$’s DJ in terms of argumentation semantics.

\section{To think}
Propositions weakly self-supported $T \subseteq \alltopic$: weakly accepted if no arg is given. Examples: $m$ = “eat miam”; $¬b$ = “beurk is to exclude”; or, in a problem where there’s no particularly good aliments, both $a$ = “eat this” and $¬a$.

When given $(\ar, \prop)$, $i$ may say: $\ar$ does not survive; or: assuming $\ar$ survives, then $\ar$ supports $\prop$, or, assuming $\ar$ survives, then $\ar$ does not support $\prop$ anyway.

When given $\ar'$ against $\ar$, $i$ may say: $\ar'$ does not survive, or: assuming $\ar'$ survives, then $\ar'$ supports $¬\prop$, …

Given $(\ar_2, \prop), (\ar_1, ¬\prop) \in D$, define $¬(\ar_2 \wibeatse^\text{neg}_{¬\prop} \ar_1)$ iff for some $(\ar, \prop) \in D$, where $\ar_1 \ibeatse[\prop] \ar$: $\ar_1 \ibeatse[\prop] \ar + \ar_2$. Equivalently: $\ar_2 \wibeatse^\text{neg}_{¬\prop} \ar_1$ iff for all $\ar$, where $\ar_1 \ibeatse[\prop] \ar$: $¬(\ar_1 \ibeatse[\prop] \ar + \ar_2)$. (This does not seem right: if given $\ar_3$ attacking $\ar_2$, and not given $\ar_4$ which would convincingly rebut $\ar_3$, then temporarily it may hold again that $\ar_1 \ibeatse[\prop] \ar + \ar2$ (in the sense that $\ar_1 + \ar_3 \ibeatse[\prop] \ar + \ar_2$).)

$\ar_2 \wibeatse_{¬\prop} \ar_1$ can perhaps be queried directly by asking (in the context of some $\ar_1 \ibeatse[\prop] \ar$): “assume $\ar_2$ survives, then does $\ar_2$ counter $\ar_1$?” (In the sense that $\ar_2$ is sufficiently convincing that $\prop$ holds perhaps, to cancel the argument $\ar_1$ according to which $¬\prop$ surely holds.)

\section{Certainties}
Looking for certainties. Those propositions that are in the reflexive preferences in a demanding sense: there is a strong enough reason to prefer it than its contrary.
\begin{itemize}
	\item $\ar' \wibeatse \ar$: weak attack; $\ar'$ renders $\ar$ invalid (can’t be used to say that $t$ holds for sure) (assuming $\ar'$ survives)
	\item Propositions strongly self-supported: strongly accepted if no arg is given. Examples: $m$ = “eat miam”; $¬b$ = “beurk is to exclude”. We might have neither $c$ nor $¬c$ in that set.
\end{itemize}

\begin{definition}[Sure acceptance]
	Define a situation $(\allalts, \allargs, \set{\ibeatse[\prop]})$. A proposition $\prop \in \alltopic$ is accepted as sure iff $\exists \ar' \in \allargs \suchthat \forall \ar \in \allargs: \ar \nibeatsst[\prop, \text{sure}] \ar'$.
\end{definition}

Assume we use rather: if $p$ is not sure, then $¬p$ is weakly accepted (by def). Then we have never problems of inconsistency! But we could be in a situation where $p$ is not accepted as sure but nobody can tell why because it is fundamentally unstable (sometimes $p$ being accepted, sometimes not).

\section{Example about model instanciation}
The general conditions are Reinstatement, Justifiable unstability, Finite defense and Covering.
A general model is a model that claims it satisfies the general conditions.

TODO give up general models. In this example, $\ar_1$ would need to be planned as attacking sometimes $\ar_2$. Better consider an instanciation mechanism. An instantiated model is particular, and can be tested (especially against another one).
\begin{example}
	$\ar_3 \mbeats \ar_2 \mbeats \ar_1 \mleadsto \prop$, $\ar_2 \mleadsto ¬\prop$; $\ar_3 + \ar_1 \mleadsto \prop$.
\end{example}
This model is compatible (meaning that it satisfies the general conditions) with the following decision situations. We describe $\ibeatse$ fully (no attack iff not mentioned).
\begin{itemize}
	\item Sure of $\prop$: $\ar_1 \mleadsto \prop$; $\ar_3 \ibeatse \ar_2$ (the rest is implied, for example $\forall \ar_4 \in \allargs: \ar_1 \ibeatse[¬\prop] \ar_4$ because of covering).
	\item Sure of $\prop$ with reinstatement: $\ar_3 \ibeatse[\prop] \ar_2 \ibeatse[\prop] \ar_1 \mleadsto \prop$; $\ar_1 + \ar_3 \mleadsto \prop$; $\ar_3 \ibeatse[¬\prop] \ar_2$
	\item Sure of $¬\prop$: $\ar_2 \mleadsto ¬\prop$; $\ar_2 \ibeatse \ar_1$
	\item Both: $¬(\ar_2 \ibeatse \ar_1), \ar_1 \mleadsto \prop$, $¬(\ar_3 \ibeatse \ar_2), \ar_2 \mleadsto ¬\prop$
\end{itemize}
This situation falsifies the model. $\ar_4 \ibeatse \ar_1$, $\ar_4$ not attacked.

\section{Model certainties}
Assume we define $\ar_1 \wibeatse[¬\prop, \text{sure}] \ar_2 ⇔ \ar_2 \nwibeatse[\prop] \ar_1$. Then, indeed, given $\ar_1 \mleadsto \prop$, $\ar_2 \wibeatse[¬\prop, \text{sure}] \ar_1 ⇔ \ar_2 \ibeatse[¬\prop, \text{sure}] \ar_1$. But it gives the wrong conclusion. For $\ar_3 \mbeats \ar_2 \mbeats \ar_1 \mleadsto t: \ar_3 \wibeatse[\prop, \text{sure}] \ar_2$ iff $\ar_2 \nwibeatse[¬\prop] \ar_3$ iff $\exists \ar_1 \in \mbeats(\ar_3) \suchthat \ar_3 \wibeatse[¬\prop] \ar_1 ∧ \ar_3 \wibeatse[¬\prop] \ar_2 + \ar_1$.

Given $\ar_1 \mleadsto \prop$, define $\ar_2 \wibeatse[\prop, \text{sure}] \ar_1$ iff $\ar_2 \ibeatse[\prop, \text{sure}] \ar_1$.

Given $\ar_1 \mleadsto \prop$, define $\ar_2 \nwibeatse[\prop, \text{sure}] \ar_1$ iff $\ar_1 \ibeatse[¬\prop] \ar_2$.

Given $\ar_3 \in \clargs, \ar_2 \in \clargs, \ar_2 \notin \mleadstoinv(\alltopic), \prop \in T$: $\ar_3 \wibeatse[\prop, \text{sure}] \ar_2$ iff $\exists \ar_1 \in \mbeats(\ar_2) \suchthat \ar_2 \wibeatse[\prop, \text{sure}] \ar_1 ∧ \ar_2 \nwibeatse[\prop, \text{sure}] \ar_3 + \ar_1$.

Given $\ar_3 \in \clargs, \ar_2 \in \clargs, \ar_2 \notin \mleadstoinv(\alltopic), \prop \in T$: $\ar_3 \nwibeatse[\prop, \text{sure}] \ar_2$ iff $\exists \ar_1 \in \mbeats(\ar_2) \suchthat \ar_2 \wibeatse[\prop, \text{sure}] \ar_1 ∧ \ar_2 \wibeatse[\prop, \text{sure}] \ar_3 + \ar_1$.

\section{Example about default arguments}
s2 argues in favor of p against s1: s "le monde n’est pas fiable". s1 "le monde est fiable, bhl l’a dit". s2 "bhl est un clown, il s’est planté sur l’Irak". s3 "il avait raison sur l’Irak : l’Irak a des ADM". s4 "l’Irak n’a pas d’ADM, Bush l’a reconnu".
Does s4 attack s3?
"bhl est un clown, il s’est planté sur l’irak" + "l’irak n’a pas d’ADM, Bush l’a reconnu" VS "il avait raison sur l’Irak : l’Irak a des ADM" !

Measure problem?

\section{Alternative definitions of finite defense}
Define Finite defense-$\wibeatse$-$\wibeatse$-$\mbeats$-dec as: $\wibeatseinv(\ar) \subseteq \wibeatse(\clargs \setminus \text{im}(\mbeats)) ⇒ \wibeatseinv(\ar) \subseteq \wibeatse(\args)$. Finite defense-$\wibeatse$-$\wibeatse$-$\mbeats$-dec is insufficient to provide $T_\eta = \iposs$. Define $\ar' \mbeats^\text{fail} \ar$ iff $\ar' \mbeats \ar ∧ ¬(\ar' \wibeatse \ar)$. Consider $\ar_3 \wibeatse \ar_2 \wibeatse \ar_1$, $\ar_3' \wibeatse \ar_2' \wibeatse \ar_1$, and so on, and $\ar_4 \mbeats^\text{fail} \{\ar_3, \ar_3', …\}$. Then I really need infinitely many arguments to defend $\ar_1$ but Finite defense-$\wibeatse$-$\wibeatse$-$\mbeats$-dec is artificially satisfied because the antecedent fails to trigger.

Define Finite defense-$\wibeatse$-$\wibeatse$-subsets as: $\wibeatseinv(\ar) \subseteq \wibeatse(\args) ⇒ \wibeatseinv(\ar) \subseteq \wibeatse(\args')$. Finite defense-$\wibeatse$-$\wibeatse$-subsets is insufficient to provide $T_\eta = \iposs$. This is because Reinstatement allows for new attacks in $\wibeatse$ (it only forbids new attacks in $\mbeats$), thus we can forever transform previously failing attacks to new attacks, hence always satisfying Finite defense (always finite cover of $\wibeatse$, but infinite cover of $\mbeats$) but still not converging. Consider $\ar_3 \wibeatse \ar_2 \wibeatse \ar_1, \ar_3' \wibeatse \ar_2' \mbeats^\text{fail} \ar_1$, and so on; and $\ar_3' \wibeatse \ar_2' \wibeatse \ar_3 + \ar_1, \ar_3'' \wibeatse \ar_2'' \mbeats^\text{fail} \ar_3 + \ar_1$, and so on.

Define Finite defense-$\mbeats$-$\wibeatse$-startdec as: $\mbeatsinv(\ar) \subseteq \wibeatse(\argsdec) ⇒ \mbeatsinv(\ar) \subseteq \wibeatse(\args)$. Finite defense-$\mbeats$-$\wibeatse$-startdec is insufficient to provide $T_\eta = \iposs$. Consider $\ar_3 \wibeatse \ar_2 \wibeatse \ar_1, \ar_3' \wibeatse \ar_2' \wibeatse \ar_1$, and so on, and $\ar_5 \mbeats^\text{fail} \ar_4 \mbeats^\text{fail} \ar_1$. Then I really need infinitely many arguments to defend $\ar_1$ but Finite defense-$\mbeats$-$\wibeatse$-startdec is satisfied as there is no cover of the $\mbeats$ attacks to $\ar_1$.

Define Finite defense-$\mbeats$-$\mbeats$-startdec as: $\mbeatsinv(\ar) \subseteq \mbeats(\argsdec) ⇒ \mbeatsinv(\ar) \subseteq \mbeats(\args)$. Finite defense-$\mbeats$-$\mbeats$-startdec is insufficient to provide $T_\eta = \iposs$. Consider $\ar_3 \wibeatse \ar_2 \wibeatse \ar_1$, $\ar_3' \wibeatse \ar_2' \wibeatse \ar_1$, and so on, and $\ar \mbeats^\text{fail} \{\ar_2, \ar_2', …\}$. Then I really need infinitely many arguments to defend $\ar_1$ but Finite defense-$\mbeats$-$\mbeats$-startdec is artificially satisfied because of $\ar$. Define Finite defense-$\mbeats$-$\mbeats$-subsets as: $\mbeatsinv(\ar) \subseteq \mbeats(\args) ⇒ \mbeatsinv(\ar) \subseteq \mbeats(\args')$ (for any $\args \subseteq \argsdec$). Finite defense-$\mbeats$-$\mbeats$-subsets is (rightly) non satisfied in this example.

Define Finite defense-$\wibeatse$-$\wibeatse$-subsets as: $\wibeatseinv(\ar) \subseteq \wibeatse(\args) ⇒ \wibeatseinv(\ar) \subseteq \wibeatse(\args')$. Finite defense-$\mbeats$-$\wibeatse$-subsets $⇏$ Finite defense-$\wibeatse$-$\wibeatse$-subsets. Consider $\ar_2 \mbeats^\text{fail} \ar_1$ (to be continued…)
\end{document}
