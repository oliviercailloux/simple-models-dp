\RequirePackage[l2tabu, orthodox]{nag}
\RequirePackage{silence}\WarningFilter{fmtcount}{\ordinal already defined use \FCordinal instead}
\documentclass[version=last, pagesize, twoside=semi, DIV=calc, bibliography=totoc, 12pt, a4paper, french, english]{scrartcl}
\input{preamble/packages}
\input{preamble/math_basics}
%Decision Theory (MCDA and SC)
\NewDocumentCommand{\allalts}{}{\mathscr{A}}
\NewDocumentCommand{\allcrits}{}{\mathscr{C}}
\NewDocumentCommand{\alts}{}{A}
\NewDocumentCommand{\alt}{}{a}
\NewDocumentCommand{\dm}{}{i}
\NewDocumentCommand{\allF}{}{\mathscr{F}}
\NewDocumentCommand{\allvoters}{}{\mathscr{N}}
\NewDocumentCommand{\voters}{}{N}
\NewDocumentCommand{\allprofs}{}{\boldsymbol{\mathcal{R}}}
\NewDocumentCommand{\prof}{}{\boldsymbol{R}}
\NewDocumentCommand{\linors}{}{\mathscr{L}(\allalts)}
%Thanks to https://tex.stackexchange.com/q/154549
	%\makeatletter
	%\def\@myRgood@#1#2{\mathrel{R^X_{#2}}}
	%\def\myRgood{\@ifnextchar_{\@myRgood@}{\mathrel{R^X}}}
	%\makeatother
\NewDocumentCommand{\ind}{}{\sim}
\NewDocumentCommand{\peq}{}{\succeq}
\NewDocumentCommand{\pst}{}{\succ}
\NewDocumentCommand{\npeq}{}{\nsucceq}
\NewDocumentCommand{\npst}{}{\nsucc}

%Deliberated Judgment
%%Normative theory
\NewDocumentCommand{\allargs}{}{\mathscr{S}}
\NewDocumentCommand{\args}{}{S}
\NewDocumentCommand{\ard}{O{}}{s^\mathit{d}_{#1}}
\NewDocumentCommand{\ardp}{O{}}{s^{\mathit{d}\prime}_{#1}}
\NewDocumentCommand{\ar}{o}{%
	\IfValueTF{#1}{%
		s^{(#1)}%
	}{%
		s%
	}%
}
\NewDocumentCommand{\zar}{}{\mathbf{0}}%zero, or empty, argument
\NewDocumentCommand{\allhist}{}{\mathscr{S}^*}
\NewDocumentCommand{\hist}{}{p}
\NewDocumentCommand{\histp}{}{p^{\prime}}
\NewDocumentCommand{\histpp}{}{p^{\prime\prime}}
\NewDocumentCommand{\histend}{}{p_\mathit{end}}
\NewDocumentCommand{\histpend}{}{p^{\prime}_\mathit{end}}
\NewDocumentCommand{\histppend}{}{p^{\prime\prime}_\mathit{end}}
\NewDocumentCommand{\allprops}{}{\mathscr{T}}
\NewDocumentCommand{\props}{}{T}
\NewDocumentCommand{\prop}{}{t}
\NewDocumentCommand{\propsure}{}{(t, \mathit{sure})}
\NewDocumentCommand{\propposs}{}{(t, \mathit{poss})}
\NewDocumentCommand{\notpropsure}{}{(¬t, \mathit{sure})}
\NewDocumentCommand{\notpropposs}{}{(¬t, \mathit{poss})}
%%Empirical theory
\NewDocumentCommand{\gPhi}{O{γ}}{{\hookrightarrow_{#1}}(\allargs)}
\NewDocumentCommand{\dargs}{}{S^\mathit{d}}
\NewDocumentCommand{\alldargs}{}{\mathscr{S}^d}
\NewDocumentCommand{\gargs}{O{\phi}}{S^{#1}_{γ, i}}
\NewDocumentCommand{\gargsalpha}{}{S^{\phi}_{α, i}}
\NewDocumentCommand{\gargsbeta}{}{S^{¬\phi}_{β, i}}
\NewDocumentCommand{\gargsdelta}{}{S^{¬\phi}_{δ, i}}
\NewDocumentCommand{\gleadsto}{O{γ}}{\hookrightarrow_{#1}}
\NewDocumentCommand{\gleadstoinv}{O{γ}}{{\hookrightarrow^{-1}_{#1}}}
\NewDocumentCommand{\gbeats}{O{γ}}{⊳^\mathit{a}_{#1}}
\NewDocumentCommand{\gbeatsinv}{O{γ}}{{(⊳^\mathit{a}_{#1})^{-1}}}
\NewDocumentCommand{\ngbeats}{O{γ}}{\not⊳^\mathit{a}_{#1}}
\NewDocumentCommand{\dbeats}{O{γ}}{⊳^\mathit{d}_{#1}}
\NewDocumentCommand{\dbeatsinv}{O{γ}}{{(⊳^\mathit{d}_{#1})^{-1}}}
\NewDocumentCommand{\df}{O{γ}}{\mathit{def}_{#1}}
\NewDocumentCommand{\dfp}{O{γ}}{\mathit{def}_{#1}^+}
%%i
\NewDocumentCommand{\iPhi}{}{\Phi_i}
\NewDocumentCommand{\allleadsto}{}{⇝}%Or \dashrightarrow?
\NewDocumentCommand{\ileadsto}{O{i}}{⇝_{#1}}
\NewDocumentCommand{\nileadsto}{O{i}}{\not⇝_{#1}}
\NewDocumentCommand{\ileadstoe}{O{i}}{⇝_{#1}^\exists}
\NewDocumentCommand{\nileadstoe}{O{i}}{\not⇝_{#1}^\exists}
\NewDocumentCommand{\ileadstost}{}{\hookrightarrow_i}
\NewDocumentCommand{\nileadstost}{}{\not\hookrightarrow_i}
\NewDocumentCommand{\di}{}{c^\phi_{γ, i}}
\NewDocumentCommand{\dip}{}{d^{\phi +}_{γ, i}}
\NewDocumentCommand{\ibeats}{}{⊳^\text{\sout{\ensuremath\phi}}_{γ, i}}
\NewDocumentCommand{\nibeats}{}{⋫^\text{\sout{\ensuremath\phi}}_{γ, i}}

\NewDocumentCommand{\gind}{O{}}{\sim_\gamma^{#1}}
\NewDocumentCommand{\gpeq}{}{\succeq_\gamma}
\NewDocumentCommand{\gpst}{}{\succ_\gamma}
\NewDocumentCommand{\ngpeq}{}{\nsucceq_\gamma}
\NewDocumentCommand{\ngpst}{}{\nsucc_\gamma}

%Deliberated Preference
\NewDocumentCommand{\ppeqab}{}{t_{a \succeq b}}
\NewDocumentCommand{\ppst}{O{a}O{b}}{t_{#1 \succ #2}}
\NewDocumentCommand{\ppstsure}{O{a}O{b}}{(t_{#1 \succ #2}, \mathit{sure})}
\NewDocumentCommand{\ppstbc}{}{t_{b \succ c}}
\NewDocumentCommand{\ppstac}{}{t_{a \succ c}}
\NewDocumentCommand{\ppstba}{}{t_{b \succ a}}
\NewDocumentCommand{\ppeq}{O{a}O{b}}{t_{#1 \succeq #2}}
\NewDocumentCommand{\ppeqsure}{O{a}O{b}}{(t_{#1 \succeq #2}, \mathit{sure})}
\NewDocumentCommand{\domc}{}{\mathscr{B}}
\NewDocumentCommand{\ipeq}{}{\succeq_i}
\NewDocumentCommand{\ipst}{}{\succ_i}


\input{preamble/redac}
\input{preamble/draw}
\DeclareAcronym{AMCD}{short=amcd, long={Aide Multicritère à la Décision}}
\DeclareAcronym{AHP}{short=AHP, long={Analytic Hierarchy Process}}
\DeclareAcronym{AR}{short=ar, long={Argumentative Recommender}}
\DeclareAcronym{DA}{short=da, long={Decision Analysis}}
\DeclareAcronym{DJ}{short=DJ, long={Deliberated Judgment}}
\DeclareAcronym{DM}{short=dm, long={Decision Maker}}
\DeclareAcronym{DP}{short=DP, long={Deliberated Preference}}
\DeclareAcronym{MAVT}{short=mavt, long={Multiple Attribute Value Theory}}
\DeclareAcronym{MCDA}{short=mcda, long={Multicriteria Decision Aid}}
\DeclareAcronym{MIP}{short=mip, long={Mixed Integer Program}}
\DeclareAcronym{SEU}{short=SEU, long={Subjective Expected Utility}}



\begin{document}
\title{The title}
\author{Olivier Cailloux}
\makeatletter
	\hypersetup{
		pdfsubject={Epistemology},
		pdfkeywords={Decision aiding, Decision making, Argumentation}
	}
\makeatother
\maketitle

\section{Overview}
Looking for possibilities (weak acceptance). Those propositions that are in the reflexive preferences in a large sense: there is no strong enough reason to reject those propositions, though their contrary may hold as well.
\begin{itemize}
	\item All alternatives $\allalts$.
	\item Topic $\alltopic$ = $\Set{\prop_\alt, \alt \in \allalts} ∪ \Set{\prop_{¬\alt}, \alt \in \allalts}$. Denoted simply $\alt$ and $¬\alt$. We define $¬\prop$, with $\prop = \prop_\alt$, as equal to $\prop_{¬\alt}$ and $¬\prop$, with $\prop = \prop_{¬\alt}$, as equal to $\prop_{\alt}$.
	\item All possible arguments: $\allargs$, the set of all strings. It can be chosen differently, but must be non empty as it contains at least $\emptyset$, the empty argument.
	\item $\ar' \nibeatsst[\prop] \ar$: absence, all the time, of strong rejection attack; $\ar'$ does not render $\ar$ invalid; it is sure that $\ar'$ has no impact on $\ar$, even assuming that $\ar'$ would in turn resist all counter-arguments to it. This is an hypothesized relation, unobservable, used to define the deliberated preferences of $i$ (it is clear we can’t observe it, thus, as we can’t observe the DP). With $\ar$ supporting $\prop$, $\ar'$ claims that $¬\prop$ is a certainty, thus $\ar'$ supports that $\prop$ is not weakly accepted, thus $\ar$, if rebutted, can’t be used even to say that $\prop$ is a possibility. Thus we have only attacks between contradictory propositions. Suffices that the attack occurs at least once over the considered time frame and unstability factors (such as submitting $i$ to other counter-arguments) to negate $\ar' \nibeatsst[\prop] \ar$. Here we do not condition on $\ar'$ surviving: $\ar'$ is declared incorrect, with no necessity of pursuing the debate and no hope of reinstatement. Example: $\ar'$ has already been taken into account and countered in $\ar$; or $\ar'$ does not talk about $\ar$ at all; or is not understood by $i$. Note that when $\ar' \nibeatsst[\prop] \ar$, further attacks to $\ar'$ have no chance to change that fact (assuming some properties over the way $i$ reason). The negation of this should read: $\ar'$ may attack $\ar$ in at least some cases. For example, $¬ (\ar' \nibeatsst[\prop] \ar)$ ($\ar'$ may attack $\ar$) in case $i$ suspects that $\ar'$ is invalid (because of some counter-argument $\ar_2$ to $\ar'$ that $i$ has in mind) but wants to leave the door open to reinstatement of $\ar'$.
	\item Define $¬ (\ar' \nibeatsst[\prop] \ar)$ iff $\ar' \ibeatse[\prop] \ar$.
	\item Define $\ar \ibeatse[¬\prop, \text{sure}] \ar'$ iff $\ar$ may attack $\ar'$, where $\ar$ has a weak claim ($\prop$ is possible) and $\ar'$ has a strong claim ($¬\prop$ is sure). Similarly, $\ar \nibeatsst[¬\prop, \text{sure}] \ar'$ is a constant absence of attack.
	\item Require (as an axiom) that $\ar' \ibeatse[\prop] \ar ∨ \ar \ibeatse[¬\prop, \text{sure}] \ar'$. Equivalently: $¬ (\ar \nibeatsst[¬\prop, \text{sure}] \ar' ∧ \ar' \nibeatsst[\prop] \ar)$; $¬(\ar' \ibeatsst[\prop] \ar ∧ \ar' \nibeatsst[\prop] \ar)$; $\ar' \ibeatsst[\prop] \ar ⇒ \ar' \ibeatse[\prop] \ar$.
\end{itemize}
It seems that knowledge of the poss rels is insufficient to know the sure rel: assume $¬(\ar' \ibeatse[\prop] \ar)$ and $\ar \ibeatse[¬\prop] \ar'$. Thus, 1. $\ar$ is sufficient for $\prop$ possible, $\ar'$ not sufficient for $¬\prop$ sure; and 2. $\ar$ may be sufficient for $\prop$ sure, $\ar'$ insuff for $\prop$. But is $\ar$ sufficient for $\prop$ sure? Thus, $\ar' \ibeatse[\prop, \text{sure}] \ar$? Equivalently: $\ar \nibeatse[¬\prop] \ar'$? To know this we need to know whether $¬\prop \in T_i$. Or can we use $¬(\ar' \ibeatse[\prop] \ar) ⇒ \ar' \nibeatse[\prop] \ar$?

Previously:
\begin{itemize}
	\item $¬ (\ar' \ibeatsst[\prop] \ar) ⇔ \ar' \nibeatse[\prop] \ar ⇔ \ar \ibeatse[¬\prop, \text{sure}] \ar' ⇔ ¬ (\ar \nibeatsst[¬\prop, \text{sure}] \ar')$.
	\item Require (as an axiom) that $\ar' \ibeatse[\prop] \ar ∨ \ar' \nibeatse[\prop] \ar$. This yields $¬ (\ar' \ibeatsst[\prop] \ar ∧ \ar' \nibeatsst[\prop] \ar)$.
\end{itemize}

Define $T_i \subseteq \alltopic$ as the set of propositions that are weakly accepted.
\begin{definition}[Weak acceptance]
	Define a situation $(\allalts, \allargs, \set{\ibeatse[\prop]})$. A proposition $\prop \in \alltopic$ is weakly accepted, $\prop \in T_i$, iff $\exists \ar \in \allargs \suchthat \forall \ar' \in \allargs: \ar' \nibeatsst[\prop] \ar$.
\end{definition}
It follows that $¬\prop \notin T_i$ iff $\forall \ar: \exists \ar' \suchthat \ar' \ibeatse[¬\prop] \ar$.

\begin{definition}[Strong rejection]
	Define a situation $(\allalts, \allargs, \set{\ibeatse[\prop]})$. A proposition $¬\prop \in \alltopic$ is strongly rejected iff $\forall \ar \in \allargs, \exists \ar' \in \allargs \suchthat \ar \nibeatsst[\prop, \text{sure}] \ar'$.
\end{definition}
Thus, strong rejection mandates $\ar'$ such that $\ar' \ibeatsst[¬\prop] \ar$.

If $¬\prop$ is strongly rejected, then it is not weakly accepted ($¬\prop \notin T_i$) as $\ar' \ibeatsst[¬\prop] \ar ⇒ \ar' \ibeatse[¬\prop] \ar$.

\begin{definition}[Clear-cut]
	A situation is clear-cut iff $¬\prop \notin T_i ⇒ ¬\prop$ is strongly rejected.
\end{definition}

First, we want that if situation is CC, 1) exists a valid model, 2) any valid model has the right $T_i$.
A model is valid if: a) can correctly get reinstatement; b) i is justifiably unstable; c) finite arguments; d) model is convincing. 

Second, we want that given a situation, if there exists a valid model, then the situation is CC.

Third, perhaps we get as a bonus that there exists a valid model ⇔ the situation is CC3. CC1: ¬p not in wa ⇔ ¬p s rej. CC2: ¬p s rej ⇒ p w acc. CC3: ¬p s rej ⇒ p is sure.

\section{Models}
$\mbeats$ a (possibly infinite!) DAG over $(s, p)$ pairs, $+$ defined over arguments in $\mbeats$: $(\ar_1, \prop) + (\ar_3, \prop) = (\ar', \prop)$ for some $(\ar', \prop) \in \mbeats$. Define $D' \subseteq \clargs$ as a subset of arguments (those supporting propositions). Define $\clargs \subseteq \allargs$ as the set of arguments used in $\mbeats$. Assume it can be represented by $≥$, a weak order defined on the basis of $\mbeats$, defining at most $k$ equivalence classes, so that our recursive definitions make sense. (TODO make this correct and precise.)

$\mbeats(\ar_2)$: arguments that $\ar_2$ attacks, $\ar_1 \in \mbeats(\ar_2) ⇔ \ar_2 \mbeats \ar_1$.

Given $\ar_1 \in D'$, define $\ar_2 \wibeatse \ar_1$ iff $\ar_2 \ibeatse \ar_1$.

Given $\ar_1 \in D'$, define $\ar_2 \nwibeatse \ar_1$ iff $\ar_1 \ibeatse[\text{sure}] \ar_2$.

Define $\ar_1 \wibeatse[\text{sure}] \ar_2$ ⇔ $\ar_2 \nwibeatse \ar_1$.

$\ar_1$ root in $\mbeats$ $⇒$ $\ar_1 \in D'$. $\ar_1 \in D' ⇒ \ar_3 + \ar_1 \in D'$.

Given $\ar_3 \in \clargs, \ar_2 \in \clargs, \ar_2 \notin D'$: $\ar_3 \wibeatse \ar_2$ iff $\exists \ar_1 \in \mbeats(\ar_2) \suchthat \ar_2 \wibeatse \ar_1 ∧ \ar_2 \nwibeatse \ar_3 + \ar_1$.

Given $\ar_3 \in \clargs, \ar_2 \in \clargs, \ar_2 \notin D'$: $\ar_3 \nwibeatse \ar_2$ iff $\exists \ar_1 \in \mbeats(\ar_2) \suchthat \ar_2 \wibeatse \ar_1 ∧ \ar_2 \wibeatse \ar_3 + \ar_1$.

Check: For $\ar_4, \ar_3, \ar_3 \notin D'$: $\ar_4 \wibeatse \ar_3$ iff $\exists \ar_2 \suchthat  \ar_3 \wibeatse \ar_2 ∧ \ar_3 \nwibeatse \ar_4 + \ar_2$. Thus, $\exists \ar_2 \suchthat  \ar_3 \wibeatse \ar_2 ∧ \exists \ar_1 \suchthat (\ar_4 + \ar_2) \wibeatse \ar_1 ∧ (\ar_4 + \ar_2) \wibeatse \ar_3 + \ar_1$.

Given $\ar_3 \in \clargs, \ar_2 \in \clargs$, with $\exists \ar_1 \in \mbeats(\ar_2) \suchthat \ar_2 \wibeatse \ar_1$, we have: $\ar_3 \wibeatse \ar_2 ∨ \ar_3 \nwibeatse \ar_2$.

Define $\wibeatsst = ¬\nwibeatse$. Hence, $¬(\ar \wibeatse[\text{sure}] \ar') ⇔ \ar \nwibeatsst[\text{sure}] \ar' ⇔ ¬(\ar' \nwibeatse \ar) ⇔ \ar' \wibeatsst \ar$.

Define $\nwibeatsst = ¬\wibeatse$. 

Hence, given $\ar_3 \in \clargs, \ar_2 \in \clargs, \ar_2 \notin D'$: $\ar_3 \wibeatsst \ar_2$ iff $\forall \ar_1 \in \mbeats(\ar_2) ∩ \wibeatse(\ar_2): \ar_2 \nwibeatsst \ar_3 + \ar_1$.

\section{Conditions}
\begin{definition}[Reinstatement]
	Given a model $\mu = (\mbeats, D')$ and a decision situation $(\allalts, \allargs, \ibeatse)$, given $\ar_1 \in D', \ar_2 \mbeats \ar_1, \ar_2 \wibeatse \ar_1, \ar_3 \wibeatse \ar_2, \mbeatsinv(\ar_3) = \emptyset$: $\mbeats(\ar_1) \subseteq \mbeats(\ar_1 + \ar_3) ∧ \mbeatsinv(\ar_1 + \ar_3) \subseteq \mbeatsinv(\ar_1) \setminus \mbeats(\ar_3)$.
\end{definition}

\begin{definition}[Observable validity]
	Given as above, the model is observably valid iff $\forall \ar_2 \mbeats \ar: ¬(\ar_2 \wibeatse \ar) ∨ \exists \ar_3 \mbeats \ar_2 \suchthat \ar_3 \wibeatse \ar_2$.
\end{definition}

\begin{definition}[Justifiable unstability]
	Given as above…
\end{definition}

\appendix
\section{Todo}
Road map.
\begin{itemize}
	\item P1: $p_a$ is w-a or $p_{¬a}$ is w-a
	\item Define $p_{a}$ is strongly accepted; $p_{¬a}$ is strongly rejected, so that they are equivalent.
	\item P2: $p_a$ is w-a or $p_{¬a}$ is strongly accepted.
\end{itemize}

Other todos.
\begin{itemize}
	\item If $i$ does not consider $\ar$ as supporting $\prop$, it also works: if $\prop$ is not weakly acceptable by default, then any $\ar'$ is considered by $i$ as a better argument than $\ar$ in favor of certain $¬\prop$, and so on. In fact, whether $\emptyset \ibeatse[\prop] \emptyset$ determines whether $\prop$ is weakly supported by default.
	\item I should define $\ar' (□\ibeatse[\prop]) \ar$ as an observable: “Assuming $\ar'$ would survive, do you consider $\ar'$ as leading to certainty of $¬\prop$, even when considering $\ar$?”. It distinguishes our knowledge and the truth: $\ar' (□\ibeatse[\prop]) \ar ⇒ \ar' \ibeatse[\prop] \ar$, thus, implies $¬ (\ar' \nibeatsst[\prop] \ar)$. But out of $¬ (\ar' (□\ibeatse[\prop]) \ar)$, nothing.
	\item Partition (objectively) $\allargs$ (or $\allargs × \alltopic$) into arguments in favor of $\prop, \text{sure}$, $¬\prop, \text{sure}$, and similarly for possible. Use only one rel $\ibeatse$, defined on contradictory arguments only, instead of $\ibeatse[\prop, \text{sure}]$ and others. Define $\ar' \ibeatse[\prop] \ar$ equals no when $¬(\ar' \ileadsto ¬\prop, \text{sure})$, equals $\ibeatse$ for adequate arguments, and equals yes when $¬(\ar \ileadsto \prop, \text{possible})$ and $\ar' \ileadsto ¬\prop, \text{sure}$, with probably some complications needed for the argument $\emptyset$ (and related default attitude towards $\prop$).
\end{itemize}

Questions:
Q1. Relationship with $\ar \ibeatse[\prop] \ar$?

We want to exclude: $s$ supports $p$ perhaps, attacked by $s2$ (supporting $¬p$ sure), but then $s2$ is attacked by $s$. Exclude $\ar' \ibeatse[\prop] \ar$ and $\ar \ibeatse[¬\prop, \text{sure}] \ar'$. Require to assume that this situation implies another argument $\ar_3$ “attacking” $\ar'$, thus, such that $\ar_3 + \ar$ is no more attacked by $\ar_2$. 

\section{To think}
Propositions weakly self-supported $\topic \subseteq \alltopic$: weakly accepted if no arg is given. Examples: $m$ = “eat miam”; $¬b$ = “beurk is to exclude”; or, in a problem where there’s no particularly good aliments, both $a$ = “eat this” and $¬a$.

When given $(\ar, \prop)$, $i$ may say: $\ar$ does not survive; or: assuming $\ar$ survives, then $\ar$ supports $\prop$, or, assuming $\ar$ survives, then $\ar$ does not support $\prop$ anyway.

When given $\ar'$ against $\ar$, $i$ may say: $\ar'$ does not survive, or: assuming $\ar'$ survives, then $\ar'$ supports $¬\prop$, …

Given $(\ar_2, \prop), (\ar_1, ¬\prop) \in D$, define $¬(\ar_2 \wibeatse^\text{neg}_{¬\prop} \ar_1)$ iff for some $(\ar, \prop) \in D$, where $\ar_1 \ibeatse[\prop] \ar$: $\ar_1 \ibeatse[\prop] \ar + \ar_2$. Equivalently: $\ar_2 \wibeatse^\text{neg}_{¬\prop} \ar_1$ iff for all $\ar$, where $\ar_1 \ibeatse[\prop] \ar$: $¬(\ar_1 \ibeatse[\prop] \ar + \ar_2)$. (This does not seem right: if given $\ar_3$ attacking $\ar_2$, and not given $\ar_4$ which would convincingly rebut $\ar_3$, then temporarily it may hold again that $\ar_1 \ibeatse[\prop] \ar + \ar2$ (in the sense that $\ar_1 + \ar_3 \ibeatse[\prop] \ar + \ar_2$).)

$\ar_2 \wibeatse_{¬\prop} \ar_1$ can perhaps be queried directly by asking (in the context of some $\ar_1 \ibeatse[\prop] \ar$): “assume $\ar_2$ survives, then does $\ar_2$ counter $\ar_1$?” (In the sense that $\ar_2$ is sufficiently convincing that $\prop$ holds perhaps, to cancel the argument $\ar_1$ according to which $¬\prop$ surely holds.)

\section{Certainties}
Looking for certainties. Those propositions that are in the reflexive preferences in a demanding sense: there is a strong enough reason to prefer it than its contrary.
\begin{itemize}
	\item $\ar' \wibeatse \ar$: weak attack; $\ar'$ renders $\ar$ invalid (can’t be used to say that $t$ holds for sure) (assuming $\ar'$ survives)
	\item Propositions strongly self-supported: strongly accepted if no arg is given. Examples: $m$ = “eat miam”; $¬b$ = “beurk is to exclude”. We might have neither $c$ nor $¬c$ in that set.
\end{itemize}

\begin{definition}[Sure acceptance]
	Define a situation $(\allalts, \allargs, \set{\ibeatse[\prop]})$. A proposition $\prop \in \alltopic$ is accepted as sure iff $\exists \ar' \in \allargs \suchthat \forall \ar \in \allargs: \ar \nibeatsst[\prop, \text{sure}] \ar'$.
\end{definition}

Assume we use rather: if $p$ is not sure, then $¬p$ is weakly accepted (by def). Then we have never problems of inconsistency! But we could be in a situation where $p$ is not accepted as sure but nobody can tell why because it is fundamentally unstable (sometimes $p$ being accepted, sometimes not).

\section{Example}
s2 argues in favor of p against s1: s "le monde n’est pas fiable". s1 "le monde est fiable, bhl l’a dit". s2 "bhl est un clown, il s’est planté sur l’Irak". s3 "il avait raison sur l’Irak : l’Irak a des ADM". s4 "l’Irak n’a pas d’ADM, Bush l’a reconnu".
Does s4 attack s3?
"bhl est un clown, il s’est planté sur l’irak" + "l’irak n’a pas d’ADM, Bush l’a reconnu" VS "il avait raison sur l’Irak : l’Irak a des ADM" !

Measure problem?

%\bibliography{mybib}

\end{document}

