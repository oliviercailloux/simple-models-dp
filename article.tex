\RequirePackage[l2tabu, orthodox]{nag}
\RequirePackage{silence}\WarningFilter{fmtcount}{\ordinal already defined use \FCordinal instead}
\documentclass[version=last, pagesize, twoside=semi, DIV=calc, bibliography=totoc, 12pt, a4paper, french, english]{scrartcl}
\input{preamble/packages}
\input{preamble/math_basics}
%Decision Theory (MCDA and SC)
\NewDocumentCommand{\allalts}{}{\mathscr{A}}
\NewDocumentCommand{\allcrits}{}{\mathscr{C}}
\NewDocumentCommand{\alts}{}{A}
\NewDocumentCommand{\alt}{}{a}
\NewDocumentCommand{\dm}{}{i}
\NewDocumentCommand{\allF}{}{\mathscr{F}}
\NewDocumentCommand{\allvoters}{}{\mathscr{N}}
\NewDocumentCommand{\voters}{}{N}
\NewDocumentCommand{\allprofs}{}{\boldsymbol{\mathcal{R}}}
\NewDocumentCommand{\prof}{}{\boldsymbol{R}}
\NewDocumentCommand{\linors}{}{\mathscr{L}(\allalts)}
%Thanks to https://tex.stackexchange.com/q/154549
	%\makeatletter
	%\def\@myRgood@#1#2{\mathrel{R^X_{#2}}}
	%\def\myRgood{\@ifnextchar_{\@myRgood@}{\mathrel{R^X}}}
	%\makeatother
\NewDocumentCommand{\ind}{}{\sim}
\NewDocumentCommand{\peq}{}{\succeq}
\NewDocumentCommand{\pst}{}{\succ}
\NewDocumentCommand{\npeq}{}{\nsucceq}
\NewDocumentCommand{\npst}{}{\nsucc}

%Deliberated Judgment
%%Normative theory
\NewDocumentCommand{\allargs}{}{\mathscr{S}}
\NewDocumentCommand{\args}{}{S}
\NewDocumentCommand{\ard}{O{}}{s^\mathit{d}_{#1}}
\NewDocumentCommand{\ardp}{O{}}{s^{\mathit{d}\prime}_{#1}}
\NewDocumentCommand{\ar}{o}{%
	\IfValueTF{#1}{%
		s^{(#1)}%
	}{%
		s%
	}%
}
\NewDocumentCommand{\zar}{}{\mathbf{0}}%zero, or empty, argument
\NewDocumentCommand{\allhist}{}{\mathscr{S}^*}
\NewDocumentCommand{\hist}{}{p}
\NewDocumentCommand{\histp}{}{p^{\prime}}
\NewDocumentCommand{\histpp}{}{p^{\prime\prime}}
\NewDocumentCommand{\histend}{}{p_\mathit{end}}
\NewDocumentCommand{\histpend}{}{p^{\prime}_\mathit{end}}
\NewDocumentCommand{\histppend}{}{p^{\prime\prime}_\mathit{end}}
\NewDocumentCommand{\allprops}{}{\mathscr{T}}
\NewDocumentCommand{\props}{}{T}
\NewDocumentCommand{\prop}{}{t}
\NewDocumentCommand{\propsure}{}{(t, \mathit{sure})}
\NewDocumentCommand{\propposs}{}{(t, \mathit{poss})}
\NewDocumentCommand{\notpropsure}{}{(¬t, \mathit{sure})}
\NewDocumentCommand{\notpropposs}{}{(¬t, \mathit{poss})}
%%Empirical theory
\NewDocumentCommand{\gPhi}{O{γ}}{{\hookrightarrow_{#1}}(\allargs)}
\NewDocumentCommand{\dargs}{}{S^\mathit{d}}
\NewDocumentCommand{\alldargs}{}{\mathscr{S}^d}
\NewDocumentCommand{\gargs}{O{\phi}}{S^{#1}_{γ, i}}
\NewDocumentCommand{\gargsalpha}{}{S^{\phi}_{α, i}}
\NewDocumentCommand{\gargsbeta}{}{S^{¬\phi}_{β, i}}
\NewDocumentCommand{\gargsdelta}{}{S^{¬\phi}_{δ, i}}
\NewDocumentCommand{\gleadsto}{O{γ}}{\hookrightarrow_{#1}}
\NewDocumentCommand{\gleadstoinv}{O{γ}}{{\hookrightarrow^{-1}_{#1}}}
\NewDocumentCommand{\gbeats}{O{γ}}{⊳^\mathit{a}_{#1}}
\NewDocumentCommand{\gbeatsinv}{O{γ}}{{(⊳^\mathit{a}_{#1})^{-1}}}
\NewDocumentCommand{\ngbeats}{O{γ}}{\not⊳^\mathit{a}_{#1}}
\NewDocumentCommand{\dbeats}{O{γ}}{⊳^\mathit{d}_{#1}}
\NewDocumentCommand{\dbeatsinv}{O{γ}}{{(⊳^\mathit{d}_{#1})^{-1}}}
\NewDocumentCommand{\df}{O{γ}}{\mathit{def}_{#1}}
\NewDocumentCommand{\dfp}{O{γ}}{\mathit{def}_{#1}^+}
%%i
\NewDocumentCommand{\iPhi}{}{\Phi_i}
\NewDocumentCommand{\allleadsto}{}{⇝}%Or \dashrightarrow?
\NewDocumentCommand{\ileadsto}{O{i}}{⇝_{#1}}
\NewDocumentCommand{\nileadsto}{O{i}}{\not⇝_{#1}}
\NewDocumentCommand{\ileadstoe}{O{i}}{⇝_{#1}^\exists}
\NewDocumentCommand{\nileadstoe}{O{i}}{\not⇝_{#1}^\exists}
\NewDocumentCommand{\ileadstost}{}{\hookrightarrow_i}
\NewDocumentCommand{\nileadstost}{}{\not\hookrightarrow_i}
\NewDocumentCommand{\di}{}{c^\phi_{γ, i}}
\NewDocumentCommand{\dip}{}{d^{\phi +}_{γ, i}}
\NewDocumentCommand{\ibeats}{}{⊳^\text{\sout{\ensuremath\phi}}_{γ, i}}
\NewDocumentCommand{\nibeats}{}{⋫^\text{\sout{\ensuremath\phi}}_{γ, i}}

\NewDocumentCommand{\gind}{O{}}{\sim_\gamma^{#1}}
\NewDocumentCommand{\gpeq}{}{\succeq_\gamma}
\NewDocumentCommand{\gpst}{}{\succ_\gamma}
\NewDocumentCommand{\ngpeq}{}{\nsucceq_\gamma}
\NewDocumentCommand{\ngpst}{}{\nsucc_\gamma}

%Deliberated Preference
\NewDocumentCommand{\ppeqab}{}{t_{a \succeq b}}
\NewDocumentCommand{\ppst}{O{a}O{b}}{t_{#1 \succ #2}}
\NewDocumentCommand{\ppstsure}{O{a}O{b}}{(t_{#1 \succ #2}, \mathit{sure})}
\NewDocumentCommand{\ppstbc}{}{t_{b \succ c}}
\NewDocumentCommand{\ppstac}{}{t_{a \succ c}}
\NewDocumentCommand{\ppstba}{}{t_{b \succ a}}
\NewDocumentCommand{\ppeq}{O{a}O{b}}{t_{#1 \succeq #2}}
\NewDocumentCommand{\ppeqsure}{O{a}O{b}}{(t_{#1 \succeq #2}, \mathit{sure})}
\NewDocumentCommand{\domc}{}{\mathscr{B}}
\NewDocumentCommand{\ipeq}{}{\succeq_i}
\NewDocumentCommand{\ipst}{}{\succ_i}


\input{preamble/redac}
\input{preamble/draw}
\DeclareAcronym{AMCD}{short=amcd, long={Aide Multicritère à la Décision}}
\DeclareAcronym{AHP}{short=AHP, long={Analytic Hierarchy Process}}
\DeclareAcronym{AR}{short=ar, long={Argumentative Recommender}}
\DeclareAcronym{DA}{short=da, long={Decision Analysis}}
\DeclareAcronym{DJ}{short=DJ, long={Deliberated Judgment}}
\DeclareAcronym{DM}{short=dm, long={Decision Maker}}
\DeclareAcronym{DP}{short=DP, long={Deliberated Preference}}
\DeclareAcronym{MAVT}{short=mavt, long={Multiple Attribute Value Theory}}
\DeclareAcronym{MCDA}{short=mcda, long={Multicriteria Decision Aid}}
\DeclareAcronym{MIP}{short=mip, long={Mixed Integer Program}}
\DeclareAcronym{SEU}{short=SEU, long={Subjective Expected Utility}}



\begin{document}
\title{The title}
\author{Olivier Cailloux}
\makeatletter
	\hypersetup{
		pdfsubject={Epistemology},
		pdfkeywords={Decision aiding, Decision making, Argumentation}
	}
\makeatother
\maketitle

\section{Overview}
Looking for possibilities (weak acceptance). Those propositions that are in the reflexive preferences in a large sense: there is no strong enough reason to reject those propositions, though their contrary may hold as well.
\begin{itemize}
	\item All alternatives $\allalts$.
	\item Topic $\alltopic$ = $\Set{\prop_\alt, \alt \in \allalts} ∪ \Set{\prop_{¬\alt}, \alt \in \allalts}$. Denoted simply $\alt$ and $¬\alt$. We define $¬\prop$, with $\prop = \prop_\alt$, as equal to $\prop_{¬\alt}$ and $¬\prop$, with $\prop = \prop_{¬\alt}$, as equal to $\prop_{\alt}$.
	\item All possible arguments: $\allargs$, the set of all strings.
	\item $\ar' \ibeatsed_\prop \ar$: strong rejection attack; $\ar'$ renders $\ar$ invalid (with $\ar$ supporting $\prop$, $\ar'$ claims that $¬\prop$ is a certainty, thus $\ar$ can’t be used even to say that $\prop$ is a possibility, thus $\ar'$ supports that $\prop$ is not weakly accepted). Thus we have only attacks between contradictory propositions. Suffices that the attack occurs at least once over the considered time frame and unstability factors (such as submitting $i$ to other counter-arguments). Here we do not condition on $\ar'$ surviving: $\ar'$ is declared incorrect, with no necessity of pursuing the debate and no hope of reinstatement. Example: $\ar'$ has already been taken into account and countered in $\ar$. (TODO exploit the fact that we probably only need the negation, $¬(\ar' \ibeatsed_\prop \ar)$, which is pretty clear as further attacks to $\ar'$ have no impact.)
	\item Define $\ileadsto$, decisive support, as $\ar \ileadsto \prop$ iff $i$ declares that $\ar$ is a definitive argument in that weakly supports $\prop$: there are no $\ar'$ that changes her position (once she has heard about $\ar$), in the sense of weak acceptancy, thus, no $\ar'$ attacking $\ar$ in the sense of strong rejection attack.
\end{itemize}

\begin{definition}[Clear-cut]
	Define a situation $(\alltopic, \allargs, \ileadsto)$. It is clear-cut iff $\forall \alt \in \allalts: \ileadstoinv(\{\prop_\alt, \prop_{¬\alt}\}) ≠ \emptyset$.
\end{definition}

We want to prove that, under suitable assumptions (justifiable unstability, and so on): situation is clear cut iff a model (with no cycles) exist.

\section{Models}
$D$ a DAG over $(s, p)$ pairs.

Given $(\ar_2, \prop), (\ar_1, ¬\prop) \in D$, define $\ar_2 \wibeatse_{¬\prop} \ar_1$ iff $\forall \ar \suchthat (\ar, \prop) \in D: [¬(\ar_1 \ibeatsed_\prop \ar) ∨ ¬(\ar_1 \ibeatsed_\prop (\ar + \ar_2)) ∨ \ar_2 \text{ is itself attacked}]$.
Can perhaps be queried directly by asking (in the context of some $\ar_1 \ibeatsed_\prop \ar$): “assume $\ar_2$ survives, then does $\ar_2$ counter $\ar_1$?” (In the sense that $\ar_2$ is sufficiently convincing that $\prop$ holds perhaps, to cancel the argument $\ar_1$ according to which $¬\prop$ surely holds.)

Given $(\ar_3, ¬\prop), (\ar_2, \prop) \in D$, define $\ar_3 \ibeatse_\prop \ar_2$ iff $\forall (\ar_1, ¬\prop) \in D: [¬(\ar_2 \wibeatse^\text{dec}_{¬\prop} \ar_1) ∨ ¬(\ar_2 \wibeatse^\text{dec}_{¬\prop} (\ar_1 + \ar_3)) ∨ \ar_3 \text{ is itself attacked}]$.

Define $¬(\ar_2 \wibeatse^\text{dec}_{¬\prop} \ar_1) = ¬(\ar_2 \wibeatse_{¬\prop} \ar_1)$. Then,
$(\ar_3, ¬\prop) \ibeatse (\ar_2, \prop)$ iff $\forall (\ar_1, ¬\prop): [(\ar_1 \ibeatsed_\prop \ar_2) ∨ ((\ar_1 + \ar_3) \ibeatsed_\prop \ar_2) ∨ \ar_3 \text{ is itself attacked}]$.

\subsection{Perhaps equivalent and better}
Let $D$ be a DAG over $(\ar, \prop)$ pairs, such that $\forall \ar: (\ar, \prop) \notin D ∨ (\ar, ¬\prop) \notin D$, thus, a given argument is never used both with $\prop$ and with $¬\prop$.

Define $\ibeats^D_\prop$ as follows.
\begin{itemize}
	\item $\ar_2 \ibeats^D_\prop \ar_1$ iff $(\ar_2 \ibeatsed_\prop \ar_1)$ when $(\ar_1, \prop), (\ar_2, ¬\prop) \in D$,
	\item $\ar_2 \ibeats^D_\prop \ar_1$ iff $¬(\ar_1 \ibeatsed_\prop \ar_2)$ when $(\ar_2, \prop), (\ar_1, ¬\prop) \in D$, 
	\item $\ar_2 \ibeats^D_\prop \ar_1$ not defined otherwise.
\end{itemize}
Thus, $\ar_2 \ibeats^D_{¬\prop} \ar_1$ iff $¬(\ar_1 \ibeatsed_{¬\prop} \ar_2)$ when $(\ar_1, \prop), (\ar_2, ¬\prop) \in D$.

Given $(\ar_3, ¬\prop), (\ar_2, \prop) \in D$, define $\ar_3 \ibeatse_\prop \ar_2$ iff $\forall (\ar_1, ¬\prop) \in D: [¬(\ar_2 \ibeats^D_\prop \ar_1) ∨ ¬(\ar_2 \ibeats^D_\prop (\ar_1 + \ar_3)) ∨ \ar_3 \text{ is itself attacked}]$.

Given $(\ar_3, \prop), (\ar_2, ¬\prop) \in D$, define $\ar_3 \wibeatse_{¬\prop} \ar_2$ iff $\forall (\ar_1, \prop) \in D: [¬(\ar_2 \ibeatsed_\prop \ar_1) ∨ ¬(\ar_2 \ibeatsed_\prop (\ar_1 + \ar_3)) ∨ \ar_3 \text{ is itself attacked}]$.

\subsection{Newer try}
$s' ntepd s$ (for not triangle exists p d) iff $¬(s' tepd s)$, iff it is sure that $s'$ has no impact on $s$, even assuming that $s'$ would in turn resist all counter-arguments to it.

$s2 naenp s1$ iff it is sure that $s2$ has no sufficient impact on $s1$, even assuming that $s2$ survives, more precisely, iff for some $s$, where $¬(s1 ntepd s)$: $¬(s1 ntepd (s+s2))$.

\section{Models - try}
Additionally.
\begin{itemize}
	\item Propositions weakly self-supported $\topic \subseteq \alltopic$: weakly accepted if no arg is given. Examples: $m$ = “eat miam”; $¬b$ = “beurk is to exclude”; or, in a problem where there’s no particularly good aliments, both $a$ = “eat this” and $¬a$.
%	\item $\ar' \nibeatse_\prop \ar$: existence of an absence of strong rejection attack.
	\item $\ar' \wibeatse_\prop \ar$: weak attack; $\ar'$ renders $\ar$ invalid because too strong (with $\ar$ supporting $\prop$, $\ar'$ claims that $¬\prop$ is a possibility, thus $\ar$ can’t be used to say that $\prop$ is a certainty, thus $\ar'$ supports that $¬\prop$ is weakly accepted). Assuming $\ar'$ survives. This is a relation on the arguments that defend $\prop$ times the arguments that defend $¬\prop$, union the converse, union on all propositions. Thus we have only attacks between contradictory propositions.
\end{itemize}

When given $(\ar, \prop)$, $i$ may say: $\ar$ does not survive; or: assuming $\ar$ survives, then $\ar$ supports $\prop$, or, assuming $\ar$ survives, then $\ar$ does not support $\prop$ anyway.

When given $\ar'$ against $\ar$, $i$ may say: $\ar'$ does not survive, or: assuming $\ar'$ survives, then $\ar'$ supports $¬\prop$, …

We might use the following hyp. (We can dispense of it?) (TODO what does that mean?)
\begin{definition}[Completeness of $\topic$]
	At least one of $\alt$ and $¬\alt$ is in $\topic$, for each alternative.
\end{definition}
If none, we should assume the DM means that both are (which can be done if falsifying against observed choices).

I think we need a primitive definition of $(s, p)$ attacked by $s1$.

Then we can define: $s_{cc}$ attacks $s_c$ iff [$(p + s_{cc}, p)$ not attacked by $s_c$, or $s_{cc}$ not decisive].

Except this is maybe not well defined in case of circularities! (But this should not be a problem as we will not need this definition in that case.)

\begin{definition}
	Given 
\end{definition}

\appendix
\section{Certainties}
Looking for certainties. Those propositions that are in the reflexive preferences in a demanding sense: there is a strong enough reason to prefer it than its contrary.
\begin{itemize}
	\item $\ar' \wibeatse \ar$: weak attack; $\ar'$ renders $\ar$ invalid (can’t be used to say that $t$ holds for sure) (assuming $\ar'$ survives)
	\item Propositions strongly self-supported: strongly accepted if no arg is given. Examples: $m$ = “eat miam”; $¬b$ = “beurk is to exclude”. We might have neither $c$ nor $¬c$ in that set.
\end{itemize}

%\bibliography{mybib}

\end{document}

