\RequirePackage[l2tabu, orthodox]{nag}
\RequirePackage{silence}\WarningFilter{fmtcount}{\ordinal already defined use \FCordinal instead}
\documentclass[version=last, pagesize, twoside=semi, DIV=calc, bibliography=totoc, 12pt, a4paper, french, english]{scrartcl}
%Permits to copy eg x ⪰ y ⇔ v(x) ≥ v(y) from PDF to unicode data, and to search. From pdfTeX users manual. See https://tex.stackexchange.com/posts/comments/1203887.
	\input glyphtounicode
	\pdfgentounicode=1
%Latin Modern has more glyphs than Computer Modern, such as diacritical characters. fntguide commands to load the font before fontenc, to prevent default loading of cmr.
	\usepackage{lmodern}
%Encode resulting accented characters correctly in resulting PDF, permits copy from PDF.
	\usepackage[T1]{fontenc}
%UTF8 seems to be the default in recent TeX installations, but not all, see https://tex.stackexchange.com/a/370280.
	\usepackage[utf8]{inputenc}
%Provides \newunicodechar for easy definition of supplementary UTF8 characters such as → or ≤ for use in source code.
	\usepackage{newunicodechar}
%Text Companion fonts, much used together with CM-like fonts. Provides \texteuro and commands for text mode characters such as \textminus, \textrightarrow, \textlbrackdbl.
	\usepackage{textcomp}
%Solves bug in lmodern, https://tex.stackexchange.com/a/261188; probably useful only for unusually big font sizes; and probably better to use exscale instead. Note that the authors of exscale write against this trick.
	%\DeclareFontShape{OMX}{cmex}{m}{n}{
		%<-7.5> cmex7
		%<7.5-8.5> cmex8
		%<8.5-9.5> cmex9
		%<9.5-> cmex10
	%}{}
	%\SetSymbolFont{largesymbols}{normal}{OMX}{cmex}{m}{n}
%More symbols (such as \sum) available in bold version, see https://github.com/latex3/latex2e/issues/71.
	\DeclareFontShape{OMX}{cmex}{bx}{n}{%
	   <->sfixed*cmexb10%
	   }{}
	\SetSymbolFont{largesymbols}{bold}{OMX}{cmex}{bx}{n}
%For small caps also in italics, see https://tex.stackexchange.com/questions/32942/italic-shape-needed-in-small-caps-fonts, https://tex.stackexchange.com/questions/284338/italic-small-caps-not-working.
	\usepackage{slantsc}
	\AtBeginDocument{%
		%“Since nearly no font family will contain real italic small caps variants, the best approach is to substitute them by slanted variants.” -- slantsc doc
		%\DeclareFontShape{T1}{lmr}{m}{scit}{<->ssub*lmr/m/scsl}{}%
		%There’s no bold small caps in Latin Modern, we switch to Computer Modern for bold small caps, see https://tex.stackexchange.com/a/22241
		%\DeclareFontShape{T1}{lmr}{bx}{sc}{<->ssub*cmr/bx/sc}{}%
		%\DeclareFontShape{T1}{lmr}{bx}{scit}{<->ssub*cmr/bx/scsl}{}%
	}
%Warn about missing characters.
	\tracinglostchars=2
%Nicer tables: provides \toprule, \midrule, \bottomrule.
	%\usepackage{booktabs}
%For new column type X which stretches; can be used together with booktabs, see https://tex.stackexchange.com/a/97137. “tabularx modifies the widths of the columns, whereas tabular* modifies the widths of the inter-column spaces.” Loads array.
	%\usepackage{tabularx}
%math-mode version of "l" column type. Requires \usepackage{array}.
	%\usepackage{array}
	%\newcolumntype{L}{>{$}l<{$}}
%Provides \xpretocmd and loads etoolbox which provides \apptocmd, \patchcmd, \newtoggle… Also loads xparse, which provides \NewDocumentCommand and similar commands intended as replacement of \newcommand in LaTeX3 for defining commands (see https://tex.stackexchange.com/q/98152 and https://github.com/latex3/latex2e/issues/89).
	\usepackage{xpatch}
%ntheorem doc says: “empheq provides an enhanced vertical placement of the endmarks”; must be loaded before ntheorem. Loads the mathtools package, which loads and fixes some bugs in amsmath and provides \DeclarePairedDelimiter. amsmath is considered a basic, mandatory package nowadays (Grätzer, More Math Into LaTeX).
	\usepackage[ntheorem]{empheq}
%Package frenchb asks to load natbib before babel-french. Package hyperref asks to load natbib before hyperref.
	\usepackage{natbib}

\newtoggle{LCpres}
	\newtoggle{LCart}
	\newtoggle{LCposter}
	\makeatletter
	\@ifclassloaded{beamer}{
		\toggletrue{LCpres}
		\togglefalse{LCart}
		\togglefalse{LCposter}
		\wlog{Presentation mode}
	}{
		\@ifclassloaded{tikzposter}{
			\toggletrue{LCposter}
			\togglefalse{LCpres}
			\togglefalse{LCart}
			\wlog{Poster mode}
		}{
			\toggletrue{LCart}
			\togglefalse{LCpres}
			\togglefalse{LCposter}
			\wlog{Article mode}
		}
	}
	\makeatother%

%Language options ([french, english]) should be on the document level (last is main); except with tikzposter: put [french, english] options next to \usepackage{babel} to avoid warning. beamer uses the \translate command for the appendix: omitting babel results in a warning, see https://github.com/josephwright/beamer/issues/449. Babel also seems required for \refname.
%	\iftoggle{LCpres}{
		\usepackage{babel}
%	}{
%	}
	%\frenchbsetup{AutoSpacePunctuation=false}
%listings (1.7) does not allow multi-byte encodings. listingsutf8 works around this only for characters that can be represented in a known one-byte encoding and only for \lstinputlisting. Other workarounds: use literate mechanism; or escape to LaTeX (but breaks alignment).
	%\usepackage{listings}
	%\lstset{tabsize=2, basicstyle=\ttfamily, escapechar=§, literate={é}{{\'e}}1}
%I favor acro over acronym because the former is more recently updated (2018 VS 2015 at time of writing); has a longer user manual (about 40 pages VS 6 pages if not counting the example and implementation parts); has a command for capitalization; and acronym suffers a nasty bug when ac used in section, see https://tex.stackexchange.com/q/103483 (though this might be the fault of the silence package and might be solved in more recent versions, I do not know) and from a bug when used with cleveref, see https://tex.stackexchange.com/q/71364. However, loading it makes compilation time (one pass on this template) go from 0.6 to 1.4 seconds, see https://bitbucket.org/cgnieder/acro/issues/115. Option short-format not usable in the package options as it is fragile, see https://tex.stackexchange.com/q/466882.
	%\usepackage[single]{acro}
	%\acsetup{short-format = {\scshape}}
	%\DeclareAcronym{AMCD}{short=AMCD, long={Aide Multicritère à la Décision}}
\DeclareAcronym{AHP}{short=AHP, long={Analytic Hierarchy Process}}
\DeclareAcronym{AR}{short=AR, long={Argumentative Recommender}}
\DeclareAcronym{DA}{short=DA, long={Decision Analysis}}
\DeclareAcronym{DJ}{short=DJ, long={Deliberated Judgment}}
\DeclareAcronym{DM}{short=DM, long={Decision Maker}}
\DeclareAcronym{DP}{short=DP, long={Deliberated Preference}}
\DeclareAcronym{MAVT}{short=MAVT, long={Multiple Attribute Value Theory}}
\DeclareAcronym{MCDA}{short=MCDA, long={Multicriteria Decision Aid}}
\DeclareAcronym{MIP}{short=MIP, long={Mixed Integer Program}}
\DeclareAcronym{SEU}{short=SEU, long={Subjective Expected Utility}}


\iftoggle{LCpres}{
	%I favor fmtcount over nth because it is loaded by datetime anyway; and fmtcount warns about possible conflicts when loaded after nth.
	\usepackage{fmtcount}
	%For nice input of date of presentation. Must be loaded after the babel package. Has possible problems with srcletter: https://golatex.de/verwendung-von-babel-und-datetime-in-scrlttr2-schlaegt-fehlt-t14779.html.
	\usepackage[nodayofweek]{datetime}
}{
}
%For presentations, Beamer implicitely uses the pdfusetitle option. ntheorem doc says to load hyperref “before the first use of \newtheorem”. autonum doc mandates option hypertexnames=false. I want to highlight links only if necessary for the reader to recognize it as a link, to reduce distraction. In presentations, this is already taken care of by beamer (https://tex.stackexchange.com/a/262014). If using colorlinks=true in a presentation, see https://tex.stackexchange.com/q/203056. Crashes the first compilation with tikzposter, just compile again and the problem disappears, see https://tex.stackexchange.com/q/254257.
\makeatletter
\iftoggle{LCpres}{
	\usepackage{hyperref}
}{
	\usepackage[hypertexnames=false, pdfusetitle, linkbordercolor={1 1 1}, citebordercolor={1 1 1}, urlbordercolor={1 1 1}]{hyperref}
	%https://tex.stackexchange.com/a/466235
	\pdfstringdefDisableCommands{%
		\let\thanks\@gobble
	}
}
\makeatother
%urlbordercolor is used both for \url and \doi, which I think shouldn’t be colored, and for \href, thus might want to color manually when required. Requires xcolor.
	\NewDocumentCommand{\hrefblue}{mm}{\textcolor{blue}{\href{#1}{#2}}}
%hyperref doc says: “Package bookmark replaces hyperref’s bookmark organization by a new algorithm (...) Therefore I recommend using this package”.
	\usepackage{bookmark}
%Need to invoke hyperref explicitly to link to line numbers: \hyperlink{lintarget:mylinelabel}{\ref*{lin:mylinelabel}}, with \ref* to disable automatic link. Also see https://tex.stackexchange.com/q/428656 for referencing lines from another document.
	%\usepackage{lineno}
	%\NewDocumentCommand{\llabel}{m}{\hypertarget{lintarget:#1}{}\linelabel{lin:#1}}
	%\setlength\linenumbersep{9mm}
%For complex authors blocks. Seems like authblk wants to be later than hyperref, but sooner than silence. See https://tex.stackexchange.com/q/475513 for the patch to hyperref pdfauthor.
	\ExplSyntaxOn
	\seq_new:N \g_oc_hrauthor_seq
	\NewDocumentCommand{\addhrauthor}{m}{
		\seq_gput_right:Nn \g_oc_hrauthor_seq { #1 }
	}
	%Should be \NewExpandableDocumentCommand, but this is not yet provided by my version of xparse
	\DeclareExpandableDocumentCommand{\hrauthor}{}{
		\seq_use:Nn \g_oc_hrauthor_seq {,~}
	}
	\ExplSyntaxOff
	{
		\catcode`#=11\relax
		\gdef\fixauthor{\xpretocmd{\author}{\addhrauthor{#2}}{}{}}%
	}
	\iftoggle{LCart}{
		\usepackage{authblk}
		\renewcommand\Affilfont{\small}
		\fixauthor
		\AtBeginDocument{
		    \hypersetup{pdfauthor={\hrauthor}}
		}
	}{
	}
%I do not use floatrow, because it requires an ugly hack for proper functioning with KOMA script (see scrhack doc). Instead, the following command centers all floats (using \centering, as the center environment adds space, http://texblog.net/latex-archive/layout/center-centering/), and I manually place my table captions above and figure captions below their contents (https://tex.stackexchange.com/a/3253).
	\makeatletter
	\g@addto@macro\@floatboxreset\centering
	\makeatother
%Permits to customize enumeration display and references
	%\nottoggle{LCpres}{
		%\usepackage{enumitem} %follow list environments by a string to customize enumeration, example: \begin{description}[itemindent=8em, labelwidth=!] or \begin{enumerate}[label=({\roman*}), ref={\roman*}].
	%}{
	%}
%Provides \Cen­ter­ing, \RaggedLeft, and \RaggedRight and en­vi­ron­ments Cen­ter, FlushLeft, and FlushRight, which al­low hy­phen­ation. With tikzposter, seems to cause 1=1 to be printed in the middle of the poster.
	%\usepackage{ragged2e}
%To typeset units by closely following the “official” rules.
	%\usepackage[strict]{siunitx}
%Turns the doi provided by some bibliography styles into URLs. However, uses old-style dx.doi url (see 3.8 DOI system Proxy Server technical details, “Users may resolve DOI names that are structured to use the DOI system Proxy Server (https://doi.org (current, preferred) or earlier syntax http://dx.doi.org).”, https://www.doi.org/doi_handbook/3_Resolution.html). The patch solves this.
	\usepackage{doi}
	\makeatletter
	\patchcmd{\@doi}{http://dx.doi.org}{https://doi.org}{}{}
	\makeatother
%Makes sure upper case greek letters are italic as well.
	\usepackage{fixmath}
%Provides \mathbb; obsoletes latexsym (see http://tug.ctan.org/macros/latex/base/latexsym.dtx). Relatedly, \usepackage{eucal} to change the mathcal font and \usepackage[mathscr]{eucal} (apparently equivalent to \usepackage[mathscr]{euscript}) to supplement \mathcal with \mathscr. This last option is not very useful as both fonts are similar, and the intent of the authors of eucal was to provide a replacement to mathcal (see doc euscript). Also provides \mathfrak for supplementary letters.
	\usepackage{amsfonts}
%Provides a beautiful (IMHO) \mathscr and really different than \mathcal, for supplementary uppercase letters. But there is no bold version. Alternative: mathrsfs (more slanted), but when used with tikzposter, it warns about size substitution, see https://tex.stackexchange.com/q/495167.
	\usepackage[scr]{rsfso}
%Multiple means to produce bold math: \mathbf, \boldmath (defined to be \mathversion{bold}, see fntguide), \pmb, \boldsymbol (all legacy, from LaTeX base and AMS), \bm (the most recommended one), \mathbold from package fixmath (I don’t see its advantage over \boldsymbol).
%“The \boldsymbol command is obtained preferably by using the bm package, which provides a newer, more powerful version than the one provided by the amsmath package. Generally speaking, it is ill-advised to apply \boldsymbol to more than one symbol at a time.” — AMS Short math guide. “If no bold font appears to be available for a particular symbol, \bm will use ‘poor man’s bold’” — bm. It is “best to load the package after any packages that define new symbol fonts” – bm. bm defines \boldsymbol as synonym to \bm. \boldmath accesses the correct font if it exists; it is used by \bm when appropriate. See https://tex.stackexchange.com/a/10643 and https://github.com/latex3/latex2e/issues/71 for some difficulties with \bm.
	\usepackage{bm}
	\nottoggle{LCpres}{
	%https://ctan.org/pkg/amsmath recommends ntheorem, which supersedes amsthm, which corrects the spacing of proclamations and allows for theoremstyle. Option standard loads amssymb and latexsym. Must be loaded after amsmath (from ntheorem doc). From cleveref doc, “ntheorem is fully supported and even recommended”; says to load cleveref after ntheorem. When used with tikzposter, warns about size substitution for the lasy (latexsym) font when using \url, because ntheorem loads latexsym; relatedly (but not directly related to ntheorem), size substitution warning with the cmex font happens when loading amsmath and using \url.
		\usepackage[thmmarks, amsmath, standard, hyperref]{ntheorem}
		%empheq doc says to do this after loading ntheorem
		\usetagform{default}
	%Provides \cref. Unfortunately, cref fails when the language is French and referring to a label whose name contains a colon (https://tex.stackexchange.com/q/83798). Use \cref{sec\string:intro} to work around this. cleveref should go “laster” than hyperref.
		\usepackage{cleveref}
	}{
	}
	\nottoggle{LCposter}{
	%Equations get numbers iff they are referenced. Loading order should be “amsmath → hyperref → cleveref → autonum”, according to autonum doc. Use this in preference to the showonlyrefs option from mathtools, see https://tex.stackexchange.com/q/459918 and autonum doc. See https://tex.stackexchange.com/a/285953 for the etex line. Incompatible with my version of tikzposter (produces “! Improper \prevdepth”).
		\expandafter\def\csname ver@etex.sty\endcsname{3000/12/31}\let\globcount\newcount
		\usepackage{autonum}
	}{
	}
%Also loaded by tikz.
	\usepackage{xcolor}
\iftoggle{LCpres}{
	\usepackage{tikz}
	%\usetikzlibrary{babel, matrix, fit, plotmarks, calc, trees, shapes.geometric, positioning, plothandlers, arrows, shapes.multipart}
}{
}
%Vizualization, on top of TikZ
	%\usepackage{pgfplots}
	%\pgfplotsset{compat=1.14}
\usepackage{graphicx}
	\graphicspath{{graphics/}}

%Provides \print­length{length}, useful for debugging.
	%\usepackage{printlen}
	%\uselengthunit{mm}

\iftoggle{LCpres}{
	\usepackage{appendixnumberbeamer}
	%I have yet to see anyone actually use these navigation symbols; let’s disable them
	\setbeamertemplate{navigation symbols}{} 
	\usepackage{preamble/beamerthemeParisFrance}
	\setcounter{tocdepth}{10}
}{
}

%Do not use the displaymath environment: use equation. Do not use the eqnarray or eqnarray* environments: use align(*). This improves spacing. (See l2tabu or amsldoc.)


\newcommand{\R}{ℝ}
\newcommand{\N}{ℕ}
\newcommand{\Z}{ℤ}
\newcommand{\card}[1]{\lvert{#1}\rvert}
\newcommand{\powerset}[1]{\mathscr{P}(#1)}%\mathscr rather than \mathcal: scr is rounder than cal (at least in XITS Math).
\newcommand{\suchthat}{\;\ifnum\currentgrouptype=16 \middle\fi|\;}
%\newcommand{\Rplus}{\reels^+\xspace}

\AtBeginDocument{%
	\renewcommand{\epsilon}{\varepsilon}
% we want straight form of \phi for mathematics, as recommended in UTR #25: Unicode support for mathematics.
%	\renewcommand{\phi}{\varphi}
}

% with amssymb, but I don’t want to use amssymb just for that.
% \newcommand{\restr}[2]{{#1}_{\restriction #2}}
%\newcommand{\restr}[2]{{#1\upharpoonright}_{#2}}
\newcommand{\restr}[2]{{#1|}_{#2}}%sometimes typed out incorrectly within \set.
%\newcommand{\restr}[2]{{#1}_{\vert #2}}%\vert errors when used within \Set and is typed out incorrectly within \set.
\DeclareMathOperator*{\argmax}{arg\,max}
\DeclareMathOperator*{\argmin}{arg\,min}


\NewDocumentCommand{\range}{}{R}

%Decision Theory (MCDA and SC)
\NewDocumentCommand{\allalts}{}{\mathscr{X}}
\NewDocumentCommand{\allcrits}{}{\mathscr{C}}
\NewDocumentCommand{\alts}{}{X}
\NewDocumentCommand{\alt}{}{x}
\NewDocumentCommand{\altp}{}{y}%alt prime, another alt
\NewDocumentCommand{\dm}{}{i}
\NewDocumentCommand{\allF}{}{\mathscr{F}}
\NewDocumentCommand{\allvoters}{}{\mathscr{N}}
\NewDocumentCommand{\voters}{}{N}
\NewDocumentCommand{\allprofs}{}{\boldsymbol{\mathcal{R}}}
\NewDocumentCommand{\prof}{}{\boldsymbol{R}}
\NewDocumentCommand{\linors}{}{\mathscr{L}(\allalts)}
%Thanks to https://tex.stackexchange.com/q/154549
	%\makeatletter
	%\def\@myRgood@#1#2{\mathrel{R^X_{#2}}}
	%\def\myRgood{\@ifnextchar_{\@myRgood@}{\mathrel{R^X}}}
	%\makeatother
\NewDocumentCommand{\ind}{}{\sim}
\NewDocumentCommand{\peq}{}{\succeq}
\NewDocumentCommand{\pst}{}{\succ}
\NewDocumentCommand{\npeq}{}{\nsucceq}
\NewDocumentCommand{\npst}{}{\nsucc}

%Deliberated Judgment
%%Normative theory
\NewDocumentCommand{\allargs}{}{\mathscr{A}}
\NewDocumentCommand{\args}{}{A}
\NewDocumentCommand{\ard}{O{}}{a^\mathit{d}_{#1}}
\NewDocumentCommand{\ardp}{O{}}{a^{\mathit{d}\prime}_{#1}}
\NewDocumentCommand{\ar}{o}{%
	\IfValueTF{#1}{%
		a^{(#1)}%
	}{%
		a%
	}%
}
\NewDocumentCommand{\zar}{}{\mathbf{0}}%zero, or empty, argument
\NewDocumentCommand{\allhist}{}{\mathscr{A}^*}
\NewDocumentCommand{\hist}{}{α}
\NewDocumentCommand{\histp}{}{α^{\prime}}
\NewDocumentCommand{\histpp}{}{α^{\prime\prime}}
\NewDocumentCommand{\histend}{o}{%
	\IfValueTF{#1}{%
		α^{#1}_\mathit{end}%
	}{%
		α_\mathit{end}%
	}%
}
\NewDocumentCommand{\histpend}{}{α^{\prime}_\mathit{end}}
\NewDocumentCommand{\histppend}{}{α^{\prime\prime}_\mathit{end}}
\NewDocumentCommand{\allprops}{}{\Phi}
\NewDocumentCommand{\prop}{}{φ}
\NewDocumentCommand{\propbar}{}{φ'}%\overline
\NewDocumentCommand{\incompat}{}{\Phi^\mathit{incompat}}
%%Empirical theory
\NewDocumentCommand{\gC}{}{C_γ}
\NewDocumentCommand{\gPhi}{}{\Phi_γ}
\NewDocumentCommand{\gpropse}{O{γ}}{{\hookrightarrow_{#1}}(\allargs)}%e for explicit
\NewDocumentCommand{\gprops}{O{γ}}{\Phi_{#1}}
\NewDocumentCommand{\dargs}{O{}}{A^\mathit{d}_{#1}}
\NewDocumentCommand{\alldargs}{}{\mathscr{A}^d}
\NewDocumentCommand{\gargs}{O{φ}}{A^{#1}_{γ, i}}
\NewDocumentCommand{\gargsmu}{}{A^{φ}_{μ, i}}
\NewDocumentCommand{\gargsnu}{}{A^{φ'}_{ν, i}}
\NewDocumentCommand{\gargsgamma}{}{A^{φ}_{γ, i}}
\NewDocumentCommand{\gargsdelta}{}{A^{φ'}_{δ, i}}
\NewDocumentCommand{\gleadsto}{O{γ}}{\hookrightarrow_{#1}}
\NewDocumentCommand{\gleadstoinv}{O{γ}}{{\hookrightarrow^{-1}_{#1}}}
\NewDocumentCommand{\gbeats}{O{γ}}{⊳^\mathit{t}_{#1}}
\NewDocumentCommand{\gbeatsinv}{O{γ}}{{(⊳^\mathit{t}_{#1})^{-1}}}
\NewDocumentCommand{\ngbeats}{O{γ}}{\not⊳^\mathit{t}_{#1}}
\NewDocumentCommand{\dbeats}{O{γ}}{⊳^\mathit{d}_{#1}}
\NewDocumentCommand{\dbeatsinv}{O{γ}}{{(⊳^\mathit{d}_{#1})^{-1}}}
\NewDocumentCommand{\df}{O{γ}}{\mathit{def}_{#1}}
\NewDocumentCommand{\dfp}{O{γ}}{\mathit{def}_{#1}^+}
\NewDocumentCommand{\dg}{O{γ}}{d_{#1}}
\NewDocumentCommand{\dgip}{O{γ, i}}{d^\phi_{#1}}
%%%DP
\NewDocumentCommand{\choices}{}{\mathscr{C}}
\NewDocumentCommand{\gind}{O{}}{\sim_\gamma^{#1}}
\NewDocumentCommand{\gpeq}{}{\succeq_\gamma}
\NewDocumentCommand{\gpst}{}{\succ_\gamma}
\NewDocumentCommand{\ngpeq}{}{\nsucceq_\gamma}
\NewDocumentCommand{\ngpst}{}{\nsucc_\gamma}

%%i
\NewDocumentCommand{\iprops}{}{\Phi_i}
\NewDocumentCommand{\allleadsto}{}{⇝}%Or \dashrightarrow?
\NewDocumentCommand{\ileadsto}{O{i}}{⇝_{#1}}
\NewDocumentCommand{\nileadsto}{O{i}}{\not⇝_{#1}}
\NewDocumentCommand{\ileadstoe}{O{i}}{⇝_{#1}^\exists}
\NewDocumentCommand{\nileadstoe}{O{i}}{\not⇝_{#1}^\exists}
\NewDocumentCommand{\ileadstost}{}{\hookrightarrow_i}
\NewDocumentCommand{\nileadstost}{}{\not\hookrightarrow_i}
\NewDocumentCommand{\di}{}{c^φ_{γ, i}}
\NewDocumentCommand{\dip}{}{d^{φ +}_{γ, i}}
\NewDocumentCommand{\ibeats}{}{⊳^\text{\sout{\ensuremath{φ}}}_{γ, i}}%Or: \usepackage[normalem]{ulem} \text{\sout{\ensuremath t}}
\NewDocumentCommand{\nibeats}{}{⋫^\text{\sout{\ensuremath{φ}}}_{γ, i}}
%%%Deliberated Preference
\NewDocumentCommand{\ipeq}{}{\succeq_i}
\NewDocumentCommand{\ipst}{}{\succ_i}


\definecolor{darkgreen}{rgb}{0,0.6,0}
\newcommand{\commentOC}[1]{{\small\color{blue}{\selectlanguage{french}$\big[$OC: #1$\big]$}}}
%\newcommand{\commentOC}[1]{{\selectlanguage{french}{\todo{OC : #1}}}}
%Or: \todo[color=green!40]
\newcommand{\innote}[1]{{\scriptsize{#1}}}

%this probably requires outdated float package, see doc KomaScript for an alternative.
% \newfloat{program}{t}{lop}
% \floatname{program}{PM}

%definition, theorem, lemma, example environments, qed trickery are only needed in article mode (not Beamer)
\nottoggle{LCpres}{
%style is plain by default (italic text)
	\newtheorem{definition}{Definition}
	\newtheorem{theorem}{Theorem}
%no italic: expected.
%http://tex.stackexchange.com/questions/144653/italicizing-of-amsthm-package
	\newtheorem{lemma}{Lemma}
%\crefname{axiom}{axiom}{axioms}%might be needed for workaround bug in cref when defining new theorems?

%\ifdefined\theorem\else
%\newtheorem{theorem}{\iflanguage{english}{Theorem}{Théorème}}
%\fi

\theoremstyle{remark}
	\newtheorem{examplex}{Example}
	\newtheorem{remarkx}{Remark}

%trickery allowing use of \qedhere and the like.
\newenvironment{example}{
	\pushQED{\qed}\renewcommand{\qedsymbol}{$\triangle$}\examplex
}{
	\popQED\endexamplex
}
\newenvironment{remark}{
	\pushQED{\qed}\renewcommand{\qedsymbol}{$\triangle$}\remarkx
}{
	\popQED\endremarkx
}
}{
}
\crefname{examplex}{example}{examples}% I wonder why this is unnecessary in case of singular

%which line breaks are chosen: accept worse lines, therefore reducing risk of overfull lines. Default = 200
\tolerance=2000
%accept overfull hbox up to...
\hfuzz=2cm
%reduces verbosity about the bad line breaks
\hbadness 5000
%reduces verbosity about the underful vboxes
\vbadness=1300
%sloppy sets tolerance to 9999
\apptocmd{\sloppy}{\hbadness 10000\relax}{}{}

\bibliographystyle{abbrvnat}
%or \bibliographystyle{apalike} for presentations?

%doi package uses old-style dx.doi url, see 3.8 DOI system Proxy Server technical details, “Users may resolve DOI names that are structured to use the DOI system Proxy Server (http://doi.org (preferred) or http://dx.doi.org).”, https://www.doi.org/doi_handbook/3_Resolution.html
\makeatletter
\patchcmd{\@doi}{dx.doi.org}{doi.org}{}{}
\makeatother

% WRITING
%\newcommand{\ie}{i.e.\@\xspace}%to try
%\newcommand{\eg}{e.g.\@\xspace}
%\newcommand{\etal}{et al.\@\xspace}
\newcommand{\ie}{i.e.\ }
\newcommand{\eg}{e.g.\ }
\newcommand{\mkkOK}{\checkmark}%\color{green}{\checkmark}
\newcommand{\mkkREQ}{\ding{53}}%requires pifont?%\color{green}{\checkmark}
\newcommand{\mkkNO}{}%\text{\color{red}{\textsf{X}}}

\newlength{\xdescwd}
\makeatletter
\NewEnviron{xdesc}{%
  \setbox0=\vbox{\hbadness=\@M \global\xdescwd=0pt
    \def\item[##1]{%
      \settowidth\@tempdima{\textbf{##1}:}%
      \ifdim\@tempdima>\xdescwd \global\xdescwd=\@tempdima\fi}
  \BODY}
  \begin{description}[leftmargin=\dimexpr\xdescwd+.5em\relax,
    labelindent=0pt,labelsep=.5em,
    labelwidth=\xdescwd,align=left]\BODY\end{description}}
\makeatother

\makeatletter
\newcommand{\boldor}[2]{%
	\ifnum\strcmp{\f@series}{bx}=\z@
		#1%
	\else
		#2%
	\fi
}
\newcommand{\textstyleElProm}[1]{\boldor{\MakeUppercase{#1}}{\textsc{#1}}}
\makeatother
\newcommand{\electre}{\textstyleElProm{Électre}\xspace}
\newcommand{\electreIv}{\textstyleElProm{Électre Iv}\xspace}
\newcommand{\electreIV}{\textstyleElProm{Électre IV}\xspace}
\newcommand{\electreIII}{\textstyleElProm{Électre III}\xspace}
\newcommand{\electreTRI}{\textstyleElProm{Électre Tri}\xspace}
% \newcommand{\utadis}{\texorpdfstring{\textstyleElProm{utadis}\xspace}{UTADIS}}
% \newcommand{\utadisI}{\texorpdfstring{\textstyleElProm{utadis i}\xspace}{UTADIS I}}

%TODO
% \newcommand{\textstyleElProm}[1]{{\rmfamily\textsc{#1}}} 


\NewDocumentCommand{\tikzmark}{m}{%
	\tikz[overlay, remember picture, baseline=(#1.base)] \node (#1) {};%
}

\newlength{\GraphsDNodeSep}
\setlength{\GraphsDNodeSep}{7mm}
\tikzset{/GraphsD/dot/.style={
	shape=circle, fill=black, inner sep=0, minimum size=1mm
}}

% MCDA Drawing Sorting
\newlength{\MCDSCatHeight}
\setlength{\MCDSCatHeight}{6mm}
\newlength{\MCDSAltHeight}
\setlength{\MCDSAltHeight}{4mm}
%separation between two vertical alts
\newlength{\MCDSAltSep}
\setlength{\MCDSAltSep}{2mm}
\newlength{\MCDSCatWidth}
\setlength{\MCDSCatWidth}{3cm}
\newlength{\MCDSAltWidth}
\setlength{\MCDSAltWidth}{2.5cm}
\newlength{\MCDSEvalRowHeight}
\setlength{\MCDSEvalRowHeight}{6mm}
\newlength{\MCDSAltsToCatsSep}
\setlength{\MCDSAltsToCatsSep}{1.5cm}
\newcounter{MCDSNbAlts}
\newcounter{MCDSNbCats}
\newlength{\MCDSArrowDownOffset}
\setlength{\MCDSArrowDownOffset}{0mm}
\tikzset{/MCD/S/alt/.style={
	shape=rectangle, draw=black, inner sep=0, minimum height=\MCDSAltHeight, minimum width=\MCDSAltWidth
}}
\tikzset{/MCD/S/pref/.style={
	shape=ellipse, draw=gray, thick
}}
\tikzset{/MCD/S/cat/.style={
	shape=rectangle, draw=black, inner sep=0, minimum height=\MCDSCatHeight, minimum width=\MCDSCatWidth
}}
\tikzset{/MCD/S/evals matrix/.style={
	matrix, row sep=-\pgflinewidth, column sep=-\pgflinewidth, nodes={shape=rectangle, draw=black, inner sep=0mm, text depth=0.5ex, text height=1em, minimum height=\MCDSEvalRowHeight, minimum width=12mm}, nodes in empty cells, matrix of nodes, inner sep=0mm, outer sep=0mm, row 1/.style={nodes={draw=none, minimum height=0em, text height=, inner ysep=1mm}}
}}

%Git
\newlength{\GitDCommitSep}
\setlength{\GitDCommitSep}{13mm}
\tikzset{/GitD/commit/.style={
	shape=rectangle, draw, minimum width=4em, minimum height=0.6cm
}}
\tikzset{/GitD/branch/.style={
	shape=ellipse, draw, red
}}
\tikzset{/GitD/head/.style={
	shape=ellipse, draw, fill=yellow
}}

%Social Choice
\tikzset{/SCD/profile matrix/.style={
	matrix of math nodes, column sep=3mm, row sep=2mm, nodes={inner sep=0.5mm, anchor=base}
}}
\tikzset{/SCD/rank-profile matrix/.style={
	matrix of math nodes, column sep=3mm, row sep=2mm, nodes={anchor=base}, column 1/.style={nodes={inner sep=0.5mm}}, row 1/.style={nodes={inner sep=0.5mm}}
}}
\tikzset{/SCD/rank-vector/.style={
	draw, rectangle, inner sep=0, outer sep=1mm
}}
\tikzset{/SCD/isolated rank-vector/.style={
	draw, matrix of math nodes, column sep=3mm, inner sep=0, matrix anchor=base, nodes={anchor=base, inner sep=.33em}, ampersand replacement=\&
}}

% GUI
\tikzset{/GUID/button/.style={
	rectangle, very thick, rounded corners, draw=black, fill=black!40%, top color=black!70, bottom color=white
}}

% Logger objects
\tikzset{/loggerD/main/.style={
	shape=rectangle, draw=black, inner sep=1ex, minimum height=7mm
}}
\tikzset{/loggerD/helper/.style={
	shape=rectangle, draw=black, dashed, minimum height=7mm
}}
\tikzset{/loggerD/helper line/.style={
	<->, draw, dotted
}}

% Beliefs
\tikzset{/BeliefsD/attacker/.style={
	shape=rectangle, draw, minimum size=8mm
}}
\tikzset{/BeliefsD/supporter/.style={
	shape=circle, draw
}}


\DeclareAcronym{AMCD}{short=AMCD, long={Aide Multicritère à la Décision}}
\DeclareAcronym{AHP}{short=AHP, long={Analytic Hierarchy Process}}
\DeclareAcronym{AR}{short=AR, long={Argumentative Recommender}}
\DeclareAcronym{DA}{short=DA, long={Decision Analysis}}
\DeclareAcronym{DJ}{short=DJ, long={Deliberated Judgment}}
\DeclareAcronym{DM}{short=DM, long={Decision Maker}}
\DeclareAcronym{DP}{short=DP, long={Deliberated Preference}}
\DeclareAcronym{MAVT}{short=MAVT, long={Multiple Attribute Value Theory}}
\DeclareAcronym{MCDA}{short=MCDA, long={Multicriteria Decision Aid}}
\DeclareAcronym{MIP}{short=MIP, long={Mixed Integer Program}}
\DeclareAcronym{SEU}{short=SEU, long={Subjective Expected Utility}}



\begin{document}
\title{The title}
\author{Olivier Cailloux}
\makeatletter
	\hypersetup{
		pdfsubject={Epistemology},
		pdfkeywords={Decision aiding, Decision making, Argumentation}
	}
\makeatother
\maketitle

\section{Overview}
Looking for possibilities (weak acceptance). Those propositions that are in the reflexive preferences in a large sense: there is no strong enough reason to reject those propositions, though their contrary may hold as well.
\begin{itemize}
	\item All alternatives $\allalts$.
	\item Topic $\alltopic$ = $\Set{\prop_\alt, \alt \in \allalts} ∪ \Set{\prop_{¬\alt}, \alt \in \allalts}$. Denoted simply $\alt$ and $¬\alt$. We define $¬\prop$, with $\prop = \prop_\alt$, as equal to $\prop_{¬\alt}$ and $¬\prop$, with $\prop = \prop_{¬\alt}$, as equal to $\prop_{\alt}$.
	\item All possible arguments: $\allargs$, the set of all strings. It can be chosen differently, but must be non empty as it contains at least $\emptyset$, the empty argument.
	\item $\ar' \nibeatsst[\prop] \ar$: absence, all the time, of strong rejection attack; $\ar'$ does not render $\ar$ invalid; it is sure that $\ar'$ has no impact on $\ar$, even assuming that $\ar'$ would in turn resist all counter-arguments to it. This is an hypothesized relation, unobservable, used to define the deliberated preferences of $i$ (it is clear we can’t observe it, thus, as we can’t observe the DP). With $\ar$ supporting $\prop$, $\ar'$ claims that $¬\prop$ is a certainty, thus $\ar'$ supports that $\prop$ is not weakly accepted, thus $\ar$, if rebutted, can’t be used even to say that $\prop$ is a possibility. Thus we have only attacks between contradictory propositions. Suffices that the attack occurs at least once over the considered time frame and unstability factors (such as submitting $i$ to other counter-arguments) to negate $\ar' \nibeatsst[\prop] \ar$. Here we do not condition on $\ar'$ surviving: $\ar'$ is declared incorrect, with no necessity of pursuing the debate and no hope of reinstatement. Example: $\ar'$ has already been taken into account and countered in $\ar$; or $\ar'$ does not talk about $\ar$ at all; or is not understood by $i$. Note that when $\ar' \nibeatsst[\prop] \ar$, further attacks to $\ar'$ have no chance to change that fact (assuming some properties over the way $i$ reason). The negation of this should read: $\ar'$ may attack $\ar$ in at least some cases. For example, $¬ (\ar' \nibeatsst[\prop] \ar)$ ($\ar'$ may attack $\ar$) in case $i$ suspects that $\ar'$ is invalid (because of some counter-argument $\ar_2$ to $\ar'$ that $i$ has in mind) but wants to leave the door open to reinstatement of $\ar'$.
	\item Define $¬ (\ar' \nibeatsst[\prop] \ar)$ iff $\ar' \ibeatse[\prop] \ar$.
	\item Define $\args^\prop$ as the decisive arguments in favor of $\prop$: $\ar \in \args^\prop ⇔ \nexists \ar' \ibeatse[\prop] \ar ⇔ \ar' \nibeatsst[\prop] \ar$.
	\item Define $\ar \ibeatse[¬\prop, \text{sure}] \ar'$ iff $\ar$ may attack $\ar'$, where $\ar$ has a weak claim ($\prop$ is possible) and $\ar'$ has a strong claim ($¬\prop$ is sure). Similarly, $\ar \nibeatsst[¬\prop, \text{sure}] \ar'$ is a constant absence of attack.
	\item Define $\args^{¬\prop, \text{sure}}$ as the decisive arguments in favor of $¬\prop, \text{sure}$: $\ar' \in \args^{¬\prop, \text{sure}} ⇔ \nexists \ar \ibeatse[¬\prop, \text{sure}] \ar' ⇔ \ar \nibeatsst[¬\prop, \text{sure}] \ar'$.
	\item Require (axiom A1) that $\ar' \ibeatse[\prop] \ar ∨ \ar \ibeatse[¬\prop, \text{sure}] \ar'$. Equivalently: $¬ (\ar \nibeatsst[¬\prop, \text{sure}] \ar' ∧ \ar' \nibeatsst[\prop] \ar)$; $\ar \nibeatsst[¬\prop, \text{sure}] \ar' ⇒ \ar' \ibeatse[\prop] \ar$.
	\item Axiom A2: $\ar \nibeatsst[¬\prop, \text{sure}] \ar' ⇒ \ar \nibeatsst[¬\prop] \ar'$. Equivalently: $\ar \ibeatse[¬\prop] \ar' ⇒ \ar \ibeatse[¬\prop, \text{sure}] \ar'$.
	\footnote{A3: $[\exists \ar \suchthat (\ar' \ibeatse[\prop] \ar ∧ \ar \ibeatse[¬\prop, \text{sure}] \ar')] ⇒ \exists \ar_2 \suchthat \ar_2 \ibeatse[¬\prop] \ar'$. Equivalently: $\ar_2 \nibeatsst[¬\prop] \ar', \forall \ar_2$ implies that for all $\ar$, $(\ar' \nibeatsst[\prop] \ar ∨ \ar \nibeatsst[¬\prop, \text{sure}] \ar')$. A1, A2 and A3 are equivalent to: $\ar \nibeatsst[¬\prop, \text{sure}] \ar' ⇔ \ar \nibeatsst[¬\prop] \ar' ∧ \ar' \ibeatse[\prop] \ar$.}
	\footnote{Having A1 and A2, can we have $\ar \nibeatsst[¬\prop] \ar' ∧ \ar' \ibeatse[\prop] \ar ∧ \ar \ibeatse[¬\prop, \text{sure}] \ar'$? Consider $(\ar', ¬\prop) \succeq (\ar, \prop) \sim (\ar', ¬\prop, \text{sure}) \succeq (\ar, \prop, \text{sure})$.}
\end{itemize}

It seems that knowledge of the poss rels is insufficient to know the sure rel: assume $¬(\ar' \ibeatse[\prop] \ar)$ and $\ar \ibeatse[¬\prop] \ar'$. Thus, 1. $\ar$ is sufficient for $\prop$ possible, $\ar'$ not sufficient for $¬\prop$ sure; and 2. $\ar$ may be sufficient for $\prop$ sure, $\ar'$ insuff for $\prop$. But is $\ar$ sufficient for $\prop$ sure? Thus, $\ar' \ibeatse[\prop, \text{sure}] \ar$? Equivalently: $\ar \nibeatse[¬\prop] \ar'$? To know this we need to know whether $¬\prop \in \iposs$. Or can we use $¬(\ar' \ibeatse[\prop] \ar) ⇒ \ar' \nibeatse[\prop] \ar$?

Previously:
\begin{itemize}
	\item $¬ (\ar' \ibeatsst[\prop] \ar) ⇔ \ar' \nibeatse[\prop] \ar ⇔ \ar \ibeatse[¬\prop, \text{sure}] \ar' ⇔ ¬ (\ar \nibeatsst[¬\prop, \text{sure}] \ar')$.
\end{itemize}

Define $\iposs \subseteq \alltopic$ as the set of propositions that are weakly accepted.
\begin{definition}[Weak acceptance]
	Define a situation $(\allalts, \allargs, \set{\ibeatse[\prop]})$. A proposition $\prop \in \alltopic$ is weakly accepted, $\prop \in \iposs$, iff $\exists \ar \in \allargs \suchthat \forall \ar' \in \allargs: \ar' \nibeatsst[\prop] \ar$.
\end{definition}
It follows that $¬\prop \notin \iposs$ iff $\forall \ar: \exists \ar' \suchthat \ar' \ibeatse[¬\prop] \ar$. Equivalently, $¬\prop \notin \iposs ⇔ \ibeatse[¬\prop](\allargs) = \allargs$.

\begin{definition}[Sure acceptance]
	$¬\prop \in \isure$ iff $\exists \ar' \in \allargs \suchthat \forall \ar \in \allargs: \ar \nibeatsst[¬\prop, \text{sure}] \ar'$.\footnote{Instead of accepting $¬\prop$ for sure, it is tempting to define strong rejection of $t$ as follows. A proposition $\prop \in \alltopic$ is strongly rejected iff $\forall \ar_0 \in \allargs, \exists \ar' \in \allargs \suchthat \ar_0 \nibeatsst[¬\prop, \text{sure}] \ar'$. But this is too weak: we want $\ar'$ to be also decisive, thus $\ar \nibeatsst[¬\prop, \text{sure}] \ar', \forall \ar$. Hence the definition becomes $\exists \ar' \in \allargs \suchthat \ar \nibeatsst[¬\prop, \text{sure}] \ar'$.}
\end{definition}
If $¬\prop \in \isure$, then $\prop \notin \iposs$: with $\ar' \in \args^{¬\prop, \text{sure}}$, given any $\ar$, $\ar' \ibeatse[\prop] \ar$ (because $\ar \nibeatsst[¬\prop, \text{sure}] \ar'$, see A1).

If $¬\prop \in \isure$, then $¬\prop \in \iposs$: from $\ar \nibeatsst[¬\prop, \text{sure}] \ar'$ we obtain $\ar \nibeatsst[¬\prop] \ar'$ using A2.

\begin{definition}[Clear-cut]
	A situation is clear-cut iff $\prop \notin \iposs ⇒ ¬\prop \in \isure$.
\end{definition}
Here is an example of a non clear-cut situation. $\allargs = \{\emptyset\}, \alltopic = \{\prop\}, \emptyset \ibeatse[\prop] \emptyset, \emptyset \ibeatse[¬\prop, \text{sure}] \emptyset$.

\section{Models}
$\mbeats$ an acyclic binary relation over $\allargs$ (by which we mean that its transitive closure is irreflexive). $\mleadsto \subseteq \allargs × \alltopic$. Define $\clargs \subseteq \allargs$ as the set of arguments used in $\mbeats ∪ \mleadsto$. Let $+$ be defined over arguments used in the model: $\ar_3 + \ar_1 = \ar'$ for some $\ar' \in \clargs$, for any $\ar_3, \ar_1 \in \clargs$. 

Requirements. The maximum length of a path in $\mbeats$ is finite.
$\ar_3 \mbeats \ar_2 \mbeats \ar_1 ⇒ \ar_2 \mbeats \ar_3 + \ar_1$.\footnote{Necessary for definition of $\ar_3 \wibeatse \ar_2$.}

Notation. Let $\mleadstoinv(\alltopic) \subseteq \clargs$ denote the subset of arguments supporting propositions. 
$\mbeats(\ar_2)$: arguments that $\ar_2$ attacks, $\ar_1 \in \mbeats(\ar_2) ⇔ \ar_2 \mbeats \ar_1$. We write $\args \mbeats \ar$ to mean that $\forall \ar' \in \args: \ar' \mbeats \ar$, and similarly for other binary relations.

Given a decision situation, define $\wibeatse \subseteq \mbeats$ as follows.

Given $\ar_3 \mbeats \ar_2, \prop \in \alltopic$: $\ar_3 \wibeatse[\prop] \ar_2$ iff $[\exists \ar_1 \in \mbeats(\ar_2) \suchthat (\ar_2 \wibeatse[\prop] \ar_1 ∧ \ar_2 \nwibeatse[\prop] \ar_3 + \ar_1)] ∨ [\ar_2 \ileadsto \prop ∧ \ar_3 \ibeatse[\prop] \ar_2]$. \footnote{Or $\ar_2 \ileadsto \prop ∧ ¬\prop \notin \mleadsto(\allargs) ∧ \ar_3 \ibeatse[\prop, \text{sure}] \ar_2$.}

Given $\ar_3 \mbeats \ar_2, \prop \in \alltopic$: $\ar_3 \nwibeatse[\prop] \ar_2$ iff $[\exists \ar_1 \in \mbeats(\ar_2) \suchthat (\ar_2 \wibeatse[\prop] \ar_1 ∧ \ar_2 \wibeatse[\prop] \ar_3 + \ar_1)] ∨ [\ar_2 \mleadsto \prop ∧ \ar_2 \ibeatse[¬\prop, \text{sure}] \ar_3]$.
\footnote{Check: Given $\ar_4 \mbeats \ar_3, \ar_3 \notin \mleadstoinv(\alltopic)$, with $\ar_2 \in \mbeats(\ar_3)$ ⇒ ($\ar_2$ and $\ar_4 + \ar_2$ $\mbeats$-attack only root nodes and do not support any proposition): $\ar_4 \wibeatse[\prop] \ar_3$ iff
\begin{itemize}
	\item $\exists \ar_2 \in \mbeats(\ar_3) \suchthat [(\ar_3 \wibeatse[\prop] \ar_2) ∧ (\ar_3 \nwibeatse[\prop] \ar_4 + \ar_2)]$ iff 
	\item $\exists \ar_2 \in \mbeats(\ar_3) \suchthat [(\exists \ar_1 \in \mbeats(\ar_2) \suchthat \ar_2 \wibeatse[\prop] \ar_1 ∧ \ar_2 \nwibeatse[\prop] \ar_3 + \ar_1) ∧ (\exists \ar_1 \in \mbeats(\ar_4 + \ar_2) \suchthat \ar_4 + \ar_2 \wibeatse[\prop] \ar_1 ∧ \ar_4 + \ar_2 \wibeatse[\prop] \ar_3 + \ar_1)]$ iff 
	\item $\exists \ar_2 \in \mbeats(\ar_3) \suchthat [(\exists \ar_1 \in \mbeats(\ar_2) \suchthat \ar_1 \ileadsto t ∧ \ar_2 \ibeatse[\prop] \ar_1 ∧ \ar_3 + \ar_1 \ileadsto \prop ∧ \ar_3 + \ar_1 \ibeatse[¬\prop, \text{sure}] \ar_2) ∧ (\exists \ar_1 \in \mbeats(\ar_4 + \ar_2) \suchthat \ar_1 \mleadsto \prop ∧ \ar_4 + \ar_2 \ibeatse[\prop] \ar_1 ∧ \ar_3 + \ar_1 \ileadsto \prop ∧ \ar_4 + \ar_2 \ibeatse[\prop] \ar_3 + \ar_1)]$.
\end{itemize}
}

TODO this is well-defined because associate to each $\ar \in \clargs$ $d(\ar)$, the distance to the farthest root (a root is an argument that $\mbeats$-attacks nobody). Then $\wibeatse$ is defined for all attacks from $d(.)=1$ nodes (because those nodes attack only $d(.)=0$ nodes), and thus is defined for all nodes $2$, I suppose, …

Given $\ar_3 \in \clargs, \ar_2 \in \clargs$, with $\exists \ar_1 \in \mbeats(\ar_2) \suchthat \ar_2 \wibeatse[\prop] \ar_1$, we have: $\ar_3 \wibeatse[\prop] \ar_2 ∨ \ar_3 \nwibeatse[\prop] \ar_2$.

Define $\ar_2 \wibeatse \ar_1 ⇔ \exists \prop \in \alltopic \suchthat \ar_2 \wibeatse[\prop] \ar_1$.

Define $\wibeatsst = ¬\nwibeatse$. Define $\nwibeatsst = ¬\wibeatse$. 

Hence, given $\ar_3 \in \clargs, \ar_2 \in \clargs, \ar_2 \notin \mleadstoinv(\alltopic)$: $\ar_3 \wibeatsst \ar_2$ iff $\forall \ar_1 \in \mbeats(\ar_2) ∩ \wibeatse(\ar_2): \ar_2 \nwibeatsst \ar_3 + \ar_1$.

\subsection{Certainties}
Assume we define $\ar_1 \wibeatse[¬\prop, \text{sure}] \ar_2$ ⇔ $\ar_2 \nwibeatse[\prop] \ar_1$. Then, indeed, given $\ar_1 \mleadsto \prop$, $\ar_2 \wibeatse[¬\prop, \text{sure}] \ar_1$ ⇔ $\ar_2 \ibeatse[¬\prop, \text{sure}] \ar_1$. But it gives the wrong conclusion. For $\ar_3 \mbeats \ar_2 \mbeats \ar_1 \mleadsto t: \ar_3 \wibeatse[\prop, \text{sure}] \ar_2$ iff $\ar_2 \nwibeatse[¬\prop] \ar_3$ iff $\exists \ar_1 \in \mbeats(\ar_3) \suchthat \ar_3 \wibeatse[¬\prop] \ar_1 ∧ \ar_3 \wibeatse[¬\prop] \ar_2 + \ar_1$.

Given $\ar_1 \mleadsto \prop$, define $\ar_2 \wibeatse[\prop, \text{sure}] \ar_1$ iff $\ar_2 \ibeatse[\prop, \text{sure}] \ar_1$.

Given $\ar_1 \mleadsto \prop$, define $\ar_2 \nwibeatse[\prop, \text{sure}] \ar_1$ iff $\ar_1 \ibeatse[¬\prop] \ar_2$.

Given $\ar_3 \in \clargs, \ar_2 \in \clargs, \ar_2 \notin \mleadstoinv(\alltopic), \prop \in T$: $\ar_3 \wibeatse[\prop, \text{sure}] \ar_2$ iff $\exists \ar_1 \in \mbeats(\ar_2) \suchthat \ar_2 \wibeatse[\prop, \text{sure}] \ar_1 ∧ \ar_2 \nwibeatse[\prop, \text{sure}] \ar_3 + \ar_1$.

Given $\ar_3 \in \clargs, \ar_2 \in \clargs, \ar_2 \notin \mleadstoinv(\alltopic), \prop \in T$: $\ar_3 \nwibeatse[\prop, \text{sure}] \ar_2$ iff $\exists \ar_1 \in \mbeats(\ar_2) \suchthat \ar_2 \wibeatse[\prop, \text{sure}] \ar_1 ∧ \ar_2 \wibeatse[\prop, \text{sure}] \ar_3 + \ar_1$.


\section{Conditions}
All these conditions assume that a decision situation $(\allalts, \allargs, \{\ibeatse[\prop]\})$ and a model $\eta = (\mbeats, \mleadsto, +)$ are given.

Define $\argsdec = \clargs \setminus \text{im}(\wibeatse)$ the decisive arguments according to $\wibeatse$, or $\wibeatse$-decisive arguments for short: $\ar \in \argsdec ⇔ \wibeatseinv(\ar) = \emptyset$.

\begin{definition}[Reinstatement]
	Given $\ar_3 \wibeatse \ar_2 \wibeatse \ar_1, \ar_3 \in \argsdec$: $\mbeats(\ar_1) \subseteq \mbeats(\ar_3 + \ar_1)  ∧ \mbeatsinv(\ar_3 + \ar_1) \subseteq \mbeatsinv(\ar_1) \setminus \mbeats(\ar_3)$.
	\footnote{TODO the condition must be $\wibeatseinv(\ar_3 + \ar_1) \subseteq \mbeatsinv(\ar_1) \setminus \mbeats(\ar_3)$ to allow $\ar_5 \mbeats \ar_4 \mbeats \ar_3 \mbeats \ar_2 \mbeats \ar_1$ and $\ar_5 \mbeats \ar_4 \mbeats \ar_3 + \ar_1$, considering that possibly $\ar_3$ is $\wibeatse$-decisive. This should not invalidate the conditions, but it does currently. But it’s not a problem: the model would actually not be built this way. In this scenario the argument $\ar_3 + \ar_1$ is useful only in case $\ar_3$ is decisive, thus $\ar_4 \mbeats \ar_3 + \ar_1$ must not be planned. Rather $\ar_5 + \ar_3$ decisive, then $(\ar_5 + \ar_3) + \ar_1$. Alternatively, also $\ar_4 \mbeats \ar_1$ and then no problem as well.}
\footnote{The stronger condition mandating $\mbeatsinv(\ar_3 + \ar_1) \subseteq \wibeatseinv(\ar_1) \setminus \mbeats(\ar_3)$ would be more difficult to check: when some $\ar_2 \wibeatse \ar_3 + \ar_1$, we’d need to check not only that $\ar_2 \mbeats \ar_1$ but also $\ar_2 \wibeatse \ar_1$.}
\end{definition}

\begin{definition}[Justifiable unstability]
	$\forall \ar_2 \mbeats \ar_1 \suchthat \ar_2 \wibeatse \ar_1, \ar_2 \nwibeatse \ar_1: \exists \ar_3 \mbeats \ar_2 \suchthat \ar_3 \wibeatse \ar_2$.
\end{definition}

\begin{definition}[Finite defense]
	If $\wibeatseinv(\ar) \subseteq \wibeatse(\argsdec)$, then $\exists \args \subseteq \argsdec, \card{\args} ≤ j \suchthat \mbeatsinv(\ar) \subseteq \wibeatse(\args)$.
\footnote{Define Finite defense-$\wibeatse$-$\wibeatse$-$\mbeats$-dec as: $\wibeatseinv(\ar) \subseteq \wibeatse(\clargs \setminus \text{im}(\mbeats)) ⇒ \wibeatseinv(\ar) \subseteq \wibeatse(\args)$. Finite defense-$\wibeatse$-$\wibeatse$-$\mbeats$-dec is insufficient to provide $T_\eta = \iposs$. Define $\ar' \mbeats^\text{fail} \ar$ iff $\ar' \mbeats \ar ∧ ¬(\ar' \wibeatse \ar)$. Consider $\ar_3 \wibeatse \ar_2 \wibeatse \ar_1$, $\ar_3' \wibeatse \ar_2' \wibeatse \ar_1$, and so on, and $\ar_4 \mbeats^\text{fail} \{\ar_3, \ar_3', …\}$. Then I really need infinitely many arguments to defend $\ar_1$ but Finite defense-$\wibeatse$-$\wibeatse$-$\mbeats$-dec is artificially satisfied because the antecedent fails to trigger.}
\footnote{Define Finite defense-$\wibeatse$-$\wibeatse$-subsets as: $\wibeatseinv(\ar) \subseteq \wibeatse(\args) ⇒ \wibeatseinv(\ar) \subseteq \wibeatse(\args')$. Finite defense-$\wibeatse$-$\wibeatse$-subsets is insufficient to provide $T_\eta = \iposs$. This is because Reinstatement allows for new attacks in $\wibeatse$ (it only forbids new attacks in $\mbeats$), thus we can forever transform previously failing attacks to new attacks, hence always satisfying Finite defense (always finite cover of $\wibeatse$, but infinite cover of $\mbeats$) but still not converging. Consider $\ar_3 \wibeatse \ar_2 \wibeatse \ar_1, \ar_3' \wibeatse \ar_2' \mbeats^\text{fail} \ar_1$, and so on; and $\ar_3' \wibeatse \ar_2' \wibeatse \ar_3 + \ar_1, \ar_3'' \wibeatse \ar_2'' \mbeats^\text{fail} \ar_3 + \ar_1$, and so on.}
\footnote{Define Finite defense-$\mbeats$-$\wibeatse$-startdec as: $\mbeatsinv(\ar) \subseteq \wibeatse(\argsdec) ⇒ \mbeatsinv(\ar) \subseteq \wibeatse(\args)$. Finite defense-$\mbeats$-$\wibeatse$-startdec is insufficient to provide $T_\eta = \iposs$. Consider $\ar_3 \wibeatse \ar_2 \wibeatse \ar_1, \ar_3' \wibeatse \ar_2' \wibeatse \ar_1$, and so on, and $\ar_5 \mbeats^\text{fail} \ar_4 \mbeats^\text{fail} \ar_1$. Then I really need infinitely many arguments to defend $\ar_1$ but Finite defense-$\mbeats$-$\wibeatse$-startdec is satisfied as there is no cover of the $\mbeats$ attacks to $\ar_1$.}
\footnote{Define Finite defense-$\mbeats$-$\mbeats$-startdec as: $\mbeatsinv(\ar) \subseteq \mbeats(\argsdec) ⇒ \mbeatsinv(\ar) \subseteq \mbeats(\args)$. Finite defense-$\mbeats$-$\mbeats$-startdec is insufficient to provide $T_\eta = \iposs$. Consider $\ar_3 \wibeatse \ar_2 \wibeatse \ar_1$, $\ar_3' \wibeatse \ar_2' \wibeatse \ar_1$, and so on, and $\ar \mbeats^\text{fail} \{\ar_2, \ar_2', …\}$. Then I really need infinitely many arguments to defend $\ar_1$ but Finite defense-$\mbeats$-$\mbeats$-startdec is artificially satisfied because of $\ar$. Define Finite defense-$\mbeats$-$\mbeats$-subsets as: $\mbeatsinv(\ar) \subseteq \mbeats(\args) ⇒ \mbeatsinv(\ar) \subseteq \mbeats(\args')$ (for any $\args \subseteq \argsdec$). Finite defense-$\mbeats$-$\mbeats$-subsets is (rightly) non satisfied in this example.}
\footnote{To satisfy Finite defense, in presence of the other conditions, suffice to limit the width of the model (TODO check). But it may be interesting to not limit it and declare that the model has specific replies to any counter-argument, but promises to use only a few rebuttals and that afterwards, the dm will stop using those kind of arguments (but we don’t know in advance which ones will be chosen).}
\footnote{Define Finite defense-$\wibeatse$-$\wibeatse$-subsets as: $\wibeatseinv(\ar) \subseteq \wibeatse(\args) ⇒ \wibeatseinv(\ar) \subseteq \wibeatse(\args')$. Finite defense-$\mbeats$-$\wibeatse$-subsets $⇏$ Finite defense-$\wibeatse$-$\wibeatse$-subsets. Consider $\ar_2 \mbeats^\text{fail} \ar_1$ (to be continued…)}
\end{definition}
Thus, if the attackers of $\ar$ are attacked by decisive arguments, then $j$ defenders are enough to defend $\ar$.

Define $R$, the reinstates relation, as follows: $\ar_3 R \ar_1$ iff $\ar_3 \wibeatse \ar_2 \wibeatse \ar_1$ (for some $\ar_2$), $\ar_3 \in \argsdec$. Define $\gdargs$ as the transitive closure of $\mleadstoinv(\alltopic)$ under $R$.
\begin{definition}[Covering]
	$\forall \ar \in \gdargs, \ar' \in \allargs: \ar' \ibeatse \ar ⇒ \ar' \mbeats \ar$. \footnote{Specify $\ibeatse$.}
\end{definition}

\begin{definition}[Observable validity]
	$\forall \ar_2 \mbeats \ar_1 \mleadsto \prop: ¬(\ar_2 \wibeatse[\prop] \ar_1) ∨ \exists \ar_3 \mbeats \ar_2 \suchthat \ar_3 \wibeatse[\prop] \ar_2$. Furthermore, if $¬(\clargs \mleadsto \prop), \forall \ar_1 \mleadsto ¬\prop, \ar \in \allargs: \ar_1 \ibeatse[\prop] \ar$.
	\footnote{If the model claims $¬\prop \in \isure$, this requires clear-cut (for that prop), so we must mandate it (hopefully A3 or an equivalent such as Justifiable unstability fits). Thus we only need to prove $\prop \notin \iposs$, for which $\ar_1 \ibeatse[\prop] \ar$ suffices.}
\end{definition}

\section{Theorem}
\begin{theorem}[Validity]
	Given a decision situation and a model $\eta$, if all our conditions are satisfied, $\mleadsto(\clargs) \subseteq \iposs$. Furthermore, if $¬\prop \in \mleadsto(\clargs) ∧ \prop \notin \mleadstoinv(\clargs)$, $¬\prop \in \isure$.
\end{theorem}
\begin{proof}
$\ar$ is defended iff its $\wibeatse$-attackers are $\wibeatse$-attacked by $\wibeatse$-decisive arguments.

First, we want to prove that $\ar_1$ defended implies $\ar_1$ replaceable by some $\wibeatse$-decisive $\ar$, and if $\ar_1 \in \gdargs$, then its replacer $\ar$ is in $\gdargs$ as well.

By hypothesis, $\wibeatseinv(\ar_1) \subseteq \wibeatse(\argsdec)$. Thus, $\exists \args \subseteq \argsdec \suchthat \mbeatsinv(\ar_1) \subseteq \wibeatse(\args)$, $\args$ finite [Finite defense]. 

Pick any $\ar_{3, 1} \in \args$ such that $\ar_{3, 1} \wibeatse \ar_2 \wibeatse \ar_1$ (if there’s none, $\ar_1 \in \argsdec$ and we’re done). $\ar_{3, 1} + \ar_1$ replaces $\ar_1$, and $\mbeatsinv(\ar_{3, 1} + \ar_1) \subseteq \mbeatsinv(\ar_1) \setminus \mbeats(\ar_{3, 1})$ [Reinstatement]. Hence, $\mbeatsinv(\ar_{3, 1} + \ar_1) \subseteq \wibeatse(\args) \setminus \mbeats(\ar_{3, 1})$. Iterate by picking any $\ar_{3, 2} \in \args$ such that $\ar_{3, 2} \wibeatse \ar_2 \wibeatse \ar_{3, 1} + \ar_1$ (if there’s none, $\ar_{3, 1} + \ar_1 \in \argsdec$ and we’re done) and obtaining $\ar_{3, 2} + (\ar_{3, 1} + \ar_1)$ replacing $\ar_{3, 1} + \ar_1$ (hence, replacing $\ar_1$) with $\mbeatsinv(\ar_{3, 2} + (\ar_{3, 1} + \ar_1)) \subseteq \mbeatsinv(\ar_{3, 1} + \ar_1) \setminus \mbeats(\ar_{3, 2})$. Hence, $\mbeatsinv(\ar_{3, 2} + (\ar_{3, 1} + \ar_1)) \subseteq \wibeatse(\args) \setminus \mbeats(\ar_{3, 1}) \setminus \mbeats(\ar_{3, 2})$. Iterating in such a way over the finite set $\args$ will finally yield an element that is $\wibeatse$-decisive. The last point, $\ar_1 \in \gdargs ⇒ \ar \in \gdargs$, follows from the definition of $\gdargs$.

Second, we want to prove that if $\ar_1$ not defended and has no decisive $\wibeatse$-attackers (meaning that $\wibeatseinv(\ar_1) \subseteq \overline{\argsdec}$), then $\ar_1$ is $\wibeatse$-attacked by some $\ar_2$ that is not defended and has no decisive $\wibeatse$-attacker.

Consider $\ar_1$ not defended and having no decisive $\wibeatse$-attackers. Because $\ar_1$ is not defended, by definition, it is $\wibeatse$-attacked by some $\ar_2$ that has no decisive $\wibeatse$-attacker. Because $\ar_1$ has no decisive attacker, $\ar_2$ is not decisive. If $\ar_2$ was defended, by the first part of this proof, it would be replaceable by a decisive argument, and $\ar_1$ would have a decisive attacker. Thus, $\ar_2$ is not defended.

TODO do not mandate that $\ar_3 + \ar_1 \ileadsto \prop$, so that the model can afford not resisting to the counter-attacks to $\ar_3 + \ar_1$ (resistance to c-a to $\ar_1$ suffice). We obtain that some argument exists that support $\prop$, but not necessarily one that the model claims supports it. We need: Obs applies to restricted supports (one per prop decided by model); Covering applies to extended supports (restricted supports plus those obtained by reinstatement). Replacement-1 applies to all and requires attack at least as large; Replacement-2 applies to restricted supports and requires no new $\ibeatse$-attacks.

Third, consider an argument $\ar_1 \in \mleadstoinv(\alltopic)$. It has no decisive $\ibeatse$-attacker: as $\ar_1 \in \gdargs$, any $\ibeatse$-attack is a $\wibeatse$-attack [Covering], and $\ar_1$ has no decisive $\wibeatse$-attacker [Obs val]. Also, $\ar_1$ is defended: assume it is not, then by our second point in this proof some $\ar_2 \wibeatse \ar_1$, with $\ar_2$ not defended and with no decisive $\wibeatse$-attacker, and iterating and using finiteness of $\wibeatse$ leads to a contradiction. Hence, by our first point in this proof, $\ar_1$ is replaceable by some $\wibeatse$-decisive $\ar \in \gdargs$. As $\ar \in \gdargs$, any $\ibeatse$-attack is a $\wibeatse$-attack [Covering], thus $\ar$ is $\ibeatse$-decisive.
\end{proof}

\appendix
\section{Todo}
Road map.
\begin{itemize}
	\item P1: $p_a$ is w-a or $p_{¬a}$ is w-a
	\item Define $p_{a}$ is strongly accepted; $p_{¬a}$ is strongly rejected, so that they are equivalent.
	\item P2: $p_a$ is w-a or $p_{¬a}$ is strongly accepted.
\end{itemize}

Other todos.
\begin{itemize}
	\item If $i$ does not consider $\ar$ as supporting $\prop$, it also works: if $\prop$ is not weakly acceptable by default, then any $\ar'$ is considered by $i$ as a better argument than $\ar$ in favor of certain $¬\prop$, and so on. In fact, whether $\emptyset \ibeatse[\prop] \emptyset$ determines whether $\prop$ is weakly supported by default.
	\item I should define $\ar' (□\ibeatse[\prop]) \ar$ as an observable: “Assuming $\ar'$ would survive, do you consider $\ar'$ as leading to certainty of $¬\prop$, even when considering $\ar$?”. It distinguishes our knowledge and the truth: $\ar' (□\ibeatse[\prop]) \ar ⇒ \ar' \ibeatse[\prop] \ar$, thus, implies $¬ (\ar' \nibeatsst[\prop] \ar)$. But out of $¬ (\ar' (□\ibeatse[\prop]) \ar)$, nothing.
	\item Partition (objectively) $\allargs$ (or $\allargs × \alltopic$) into arguments in favor of $\prop, \text{sure}$, $¬\prop, \text{sure}$, and similarly for possible. Use only one rel $\ibeatse$, defined on contradictory arguments only, instead of $\ibeatse[\prop, \text{sure}]$ and others. Define $\ar' \ibeatse[\prop] \ar$ equals no when $¬(\ar' \ileadsto ¬\prop, \text{sure})$, equals $\ibeatse$ for adequate arguments, and equals yes when $¬(\ar \ileadsto \prop, \text{possible})$ and $\ar' \ileadsto ¬\prop, \text{sure}$, with probably some complications needed for the argument $\emptyset$ (and related default attitude towards $\prop$).
\end{itemize}

Questions:
Q1. Relationship with $\ar \ibeatse[\prop] \ar$?

We want to exclude: $s$ supports $p$ perhaps, attacked by $s2$ (supporting $¬p$ sure), but then $s2$ is attacked by $s$. Exclude $\ar' \ibeatse[\prop] \ar$ and $\ar \ibeatse[¬\prop, \text{sure}] \ar'$. Require to assume that this situation implies another argument $\ar_3$ “attacking” $\ar'$, thus, such that $\ar_3 + \ar$ is no more attacked by $\ar_2$. 

\section{To think}
Propositions weakly self-supported $T \subseteq \alltopic$: weakly accepted if no arg is given. Examples: $m$ = “eat miam”; $¬b$ = “beurk is to exclude”; or, in a problem where there’s no particularly good aliments, both $a$ = “eat this” and $¬a$.

When given $(\ar, \prop)$, $i$ may say: $\ar$ does not survive; or: assuming $\ar$ survives, then $\ar$ supports $\prop$, or, assuming $\ar$ survives, then $\ar$ does not support $\prop$ anyway.

When given $\ar'$ against $\ar$, $i$ may say: $\ar'$ does not survive, or: assuming $\ar'$ survives, then $\ar'$ supports $¬\prop$, …

Given $(\ar_2, \prop), (\ar_1, ¬\prop) \in D$, define $¬(\ar_2 \wibeatse^\text{neg}_{¬\prop} \ar_1)$ iff for some $(\ar, \prop) \in D$, where $\ar_1 \ibeatse[\prop] \ar$: $\ar_1 \ibeatse[\prop] \ar + \ar_2$. Equivalently: $\ar_2 \wibeatse^\text{neg}_{¬\prop} \ar_1$ iff for all $\ar$, where $\ar_1 \ibeatse[\prop] \ar$: $¬(\ar_1 \ibeatse[\prop] \ar + \ar_2)$. (This does not seem right: if given $\ar_3$ attacking $\ar_2$, and not given $\ar_4$ which would convincingly rebut $\ar_3$, then temporarily it may hold again that $\ar_1 \ibeatse[\prop] \ar + \ar2$ (in the sense that $\ar_1 + \ar_3 \ibeatse[\prop] \ar + \ar_2$).)

$\ar_2 \wibeatse_{¬\prop} \ar_1$ can perhaps be queried directly by asking (in the context of some $\ar_1 \ibeatse[\prop] \ar$): “assume $\ar_2$ survives, then does $\ar_2$ counter $\ar_1$?” (In the sense that $\ar_2$ is sufficiently convincing that $\prop$ holds perhaps, to cancel the argument $\ar_1$ according to which $¬\prop$ surely holds.)

\section{Certainties}
Looking for certainties. Those propositions that are in the reflexive preferences in a demanding sense: there is a strong enough reason to prefer it than its contrary.
\begin{itemize}
	\item $\ar' \wibeatse \ar$: weak attack; $\ar'$ renders $\ar$ invalid (can’t be used to say that $t$ holds for sure) (assuming $\ar'$ survives)
	\item Propositions strongly self-supported: strongly accepted if no arg is given. Examples: $m$ = “eat miam”; $¬b$ = “beurk is to exclude”. We might have neither $c$ nor $¬c$ in that set.
\end{itemize}

\begin{definition}[Sure acceptance]
	Define a situation $(\allalts, \allargs, \set{\ibeatse[\prop]})$. A proposition $\prop \in \alltopic$ is accepted as sure iff $\exists \ar' \in \allargs \suchthat \forall \ar \in \allargs: \ar \nibeatsst[\prop, \text{sure}] \ar'$.
\end{definition}

Assume we use rather: if $p$ is not sure, then $¬p$ is weakly accepted (by def). Then we have never problems of inconsistency! But we could be in a situation where $p$ is not accepted as sure but nobody can tell why because it is fundamentally unstable (sometimes $p$ being accepted, sometimes not).

\section{Example about model instanciation}
The general conditions are Reinstatement, Justifiable unstability, Finite defense and Covering.
A general model is a model that claims it satisfies the general conditions.

TODO give up general models. In this example, $\ar_1$ would need to be planned as attacking sometimes $\ar_2$. Better consider an instanciation mechanism. An instantiated model is particular, and can be tested (especially against another one).
\begin{example}
	$\ar_3 \mbeats \ar_2 \mbeats \ar_1 \mleadsto \prop$, $\ar_2 \mleadsto ¬\prop$; $\ar_3 + \ar_1 \mleadsto \prop$.
\end{example}
This model is compatible (meaning that it satisfies the general conditions) with the following decision situations. We describe $\ibeatse$ fully (no attack iff not mentioned).
\begin{itemize}
	\item Sure of $\prop$: $\ar_1 \mleadsto \prop$; $\ar_3 \ibeatse \ar_2$ (the rest is implied, for example $\forall \ar_4 \in \allargs: \ar_1 \ibeatse[¬\prop] \ar_4$ because of covering).
	\item Sure of $\prop$ with reinstatement: $\ar_3 \ibeatse[\prop] \ar_2 \ibeatse[\prop] \ar_1 \mleadsto \prop$; $\ar_1 + \ar_3 \mleadsto \prop$; $\ar_3 \ibeatse[¬\prop] \ar_2$
	\item Sure of $¬\prop$: $\ar_2 \mleadsto ¬\prop$; $\ar_2 \ibeatse \ar_1$
	\item Both: $¬(\ar_2 \ibeatse \ar_1), \ar_1 \mleadsto \prop$, $¬(\ar_3 \ibeatse \ar_2), \ar_2 \mleadsto ¬\prop$
\end{itemize}
This situation falsifies the model. $\ar_4 \ibeatse \ar_1$, $\ar_4$ not attacked.

\section{Example about default arguments}
s2 argues in favor of p against s1: s "le monde n’est pas fiable". s1 "le monde est fiable, bhl l’a dit". s2 "bhl est un clown, il s’est planté sur l’Irak". s3 "il avait raison sur l’Irak : l’Irak a des ADM". s4 "l’Irak n’a pas d’ADM, Bush l’a reconnu".
Does s4 attack s3?
"bhl est un clown, il s’est planté sur l’irak" + "l’irak n’a pas d’ADM, Bush l’a reconnu" VS "il avait raison sur l’Irak : l’Irak a des ADM" !

Measure problem?

%\bibliography{mybib}

\end{document}

